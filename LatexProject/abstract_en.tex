Sound field synthesis aims at the the reproduction of the physical properties of an arbitrary wavefield over an extended listening region by driving a densely spaced loudspeaker ensemble, termed the secondary source distribution.
The loudspeakers are fed with properly chosen driving signals , or driving function so that the sum of the fields of the individual loudspeakers would equal to the desired virtual/target sound field.
Two basic types of solutions exist in order to arrive at the required driving functions:
Explicit solution solves the integral equation describing the synthesized sound field directly for the driving function by performing its spectral decomposition.
On the other hand, the implicit solution aims at the formulation of the synthesized field in terms of a contour integral which would contain the driving function implicitly.
The implicit approach is generally termed as Wave Field Synthesis.

The present dissertation revisits the theoretical foundations of Wave Field Synthesis in order to overcome the limitations of the previous approaches.
As an important result, loudspeaker driving functions are presented for a freely chosen secondary source contour in order to synthesize an arbitrary virtual field and optimizing the synthesis to an arbitrary reference curve at which amplitude correct synthesis is achieved.

Afterwards, a high frequency, asymptotic spatial domain formulation is given for the explicit solution which is shown to be equivalent with the presented Wave Field Synthesis approach.
Hence, the general asymptotic relationship between the two approaches is established for the first time.
Furthermore, a simple asymptotic anti-aliasing strategy is proposed in order to suppress aliasing wave emerging in case of the application of a practical discrete secondary source distribution, instead of the theoretical continuous one.

Finally, as a complex application example the synthesis of the field generated by moving virtual sources is discussed.
It is shown that all the presented results can be extended for the inclusion of this dynamical case: 
Both Wave Field Synthesis and explicit driving functions are given for the reproduction of moving sources---with also highlighting their high frequency equivalence---before extending the introduced anti-aliasing strategy for the synthesis of time-variant wavefields.


%\fscom{check if/how 'application examples' could be more emphasized within TOC or marked/boxed for easier finding within the manuscript}
%
%\fscom{since for plane waves the SPA and SDM not holds in strictly sense, you might show the limiting case for a point source very far away, reducing to no curvarture?? this should end in plane wave drinfing functions, or? I've checked this at once numerically once with the parallel / ref point referencing }

