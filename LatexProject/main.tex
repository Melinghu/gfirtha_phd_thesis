\documentclass[12pt,a4paper]{report}
%
\usepackage{layouts}
\usepackage{amsmath}
\usepackage{a4wide}
\usepackage[T1]{fontenc}
\usepackage[utf8]{inputenc}
\usepackage{xcolor}
\usepackage{listings}
\usepackage{graphicx,overpic,subfigure}
\usepackage{tikz}
\usetikzlibrary{positioning,arrows}
\usepackage{booktabs} 			% Nice tables
\usepackage{csquotes}			% Quotation
\usepackage{multirow} 			% Multirow cells in tables
\usepackage{rotating}
\usepackage{pdflscape}
\usepackage[small,bf]{caption}
\usepackage{ae,aecompl}
\usepackage{url}
\usepackage[american]{babel}
\usepackage{hyperref}
\usepackage{nomencl}
\usepackage[toc,page]{appendix}
\usepackage{amssymb}
\usepackage{steinmetz}
\usepackage{palatino}
\usepackage{array}
\usepackage{booktabs}
\usepackage{footnote}
\usepackage{multicol}
\usepackage[export]{adjustbox}
\usepackage{caption}
\usepackage{placeins}
%\usepackage{subcaption}
 \usepackage{mathtools}
\usepackage{sidecap}
\usepackage{transparent}

 \makesavenoteenv{tabular}
%
\renewcommand{\floatpagefraction}{.99}

\newcount\posveccount
\newcommand*\posvec[1]{
        \global\posveccount#1
        [
        \posvecnext
}
\def\posvecnext#1{
        #1
        \global\advance\posveccount-1
        \ifnum\posveccount>0
                ,\
                \expandafter\posvecnext
        \else
                ]^{\mathrm{T}}
        \fi
}

\newcount\colveccount
\newcommand*\colvec[1]{
        \global\colveccount#1
        \begin{bmatrix}
        \colvecnext
}
\def\colvecnext#1{
        #1
        \global\advance\colveccount-1
        \ifnum\colveccount>0
                \\[5pt]
                \expandafter\colvecnext
        \else
                \end{bmatrix}
        \fi
}

%\setcounter{secnumdepth}{2}

\newcommand{\dint}{\int\!\!\!\!\!\int}
\newcommand{\tint}{\int\!\!\!\!\int\!\!\!\!\int}
\newcommand{\qint}{\int\!\!\!\!\int\!\!\!\!\int\!\!\!\!\int}
\newcommand{\td}{\mathrm{d}}
\newcommand{\te}{\mathrm{e}}
\newcommand{\ti}{\mathrm{j}}
\newcommand{\sinfi}{\sin\varphi}
\newcommand{\cosfi}{\cos\varphi}
\newcommand{\sinteta}{\sin\theta}
\newcommand{\costeta}{\cos\theta}
\newcommand{\yref}{y_{\mathrm{ref}}}
\newcommand{\dref}{d_{\mathrm{ref}}}
\newcommand{\vx}{\mathbf{x}}
\newcommand{\vn}{\mathbf{n}}
\newcommand{\vxo}{\mathbf{x}_0}
\newcommand{\vni}{\mathbf{n}_{\mathrm{in}}}
\newcommand{\vno}{ \mathbf{n}_{\mathrm{out}} }
\newcommand{\vxs}{\mathbf{x}_{\mathrm{s}}}
\newcommand{\vxref}{\mathbf{x}_{\mathrm{ref}}}
\newcommand{\vk}{\mathbf{k}}
\newcommand{\vhk}{\hat{\mathbf{k}}}
\newcommand{\kn}{k_\mathrm{n}}
\newcommand{\Oi}{\Omega_{\mathrm{i}}}
\newcommand{\Oe}{\Omega_{\mathrm{e}}}
\newcommand{\dO}{\partial \Omega}
\newcommand{\Div}{\mathrm{div}}
\newcommand{\Dx}{\nabla_{\!\!\vx}\,}
\newcommand{\Dxo}{\nabla_{\!\!\vxo}\,}
\newcommand{\Lx}{\nabla^2_{\!\!\vx}}

\newcommand{\fpcom}[1]{{\color{blue}#1}}
\newcommand{\fgcom}[1]{{\color{red}#1}}

\newcommand{\phix}{\phi'_{x}}
\newcommand{\phixx}{\phi''_{xx}}

\newcommand{\phiy}{\phi'_{y}}
\newcommand{\phiyy}{\phi''_{yy}}

\newcommand{\phiz}{\phi'_{z}}
\newcommand{\phizz}{\phi''_{zz}}

\newcommand{\phiPxx}{\phi^{P''}_{xx}}
\newcommand{\phiGxx}{\phi^{G''}_{xx}}

\newcommand{\phiPyy}{\phi^{P''}_{yy}}
\newcommand{\phiGyy}{\phi^{G''}_{yy}}

\newcommand{\phiPzz}{\phi^{P''}_{zz}}
\newcommand{\phiGzz}{\phi^{G''}_{zz}}

\newcommand{\Phikk}{\Phi''_{k_x k_x}}


%
\renewcommand{\arraystretch}{1}

\title{A Unified Wave Field Synthesis Framework \\
		\large with Application for Moving Virtual Sources}
\date{\today \\
Budapest University of Technology and Economics, \\ Dept. of Networked Systems and Services, \\ Laboratory of Acoustics and Studio Technologies}
\author{Gergely Firtha}
\makenomenclature

\begin{document}
\pagenumbering{roman}

\maketitle
\tableofcontents
\printnomenclature
%
\vspace{1cm}
\paragraph{Temporal Fourier transforms:}
The forward and inverse temporal Fourier transform is defined as 
\begin{equation}
\label{eq:temporal_fourier_transform_def}
F(\omega) = \mathcal{F}_t \left\{ f(t) \right\} = \int\limits_ {-\infty}^{\infty} f(t) \te^{-\ti \omega t} \td t,
\end{equation}
\begin{equation}
\label{eq:temporal_inverse_fourier_transform_def}
f(t) = \mathcal{F}_{\omega}^{-1} \left\{ F(\omega) \right\} = \frac{1}{2\pi} \int_ {-\infty}^{\infty} F(\omega) \te^{ \ti \omega t} \td \omega.
\end{equation}
%
\paragraph{Spatial Fourier transforms:}
Following the convention as given in e.g. \cite{Williams1999} the spatial Fourier transformed is defined with reversed exponential for the sake of a consequent physical interpretation, when applied for planar or outgoing spherical waves:
\begin{equation}
\label{eq:spatial_fourier_transform_def}
F(k_x) = \mathcal{F}_x \left\{ f(x) \right\} = \int_ {-\infty}^{\infty} f(x) \te^{\ti k_x x} \td x,
\end{equation}
\begin{equation}
\label{eq:spatial_inverse_fourier_transform_def}
f(x) = \mathcal{F}_{k_x}^{-1} \left\{ F(k_x) \right\} = \frac{1}{2\pi} \int_ {-\infty}^{\infty} F(k_x) \te^{ -\ti k_x x} \td k_x,
\end{equation}

\paragraph{Spatio-temporal Fourier transforms:}
\paragraph{Fourier transform properties}

\paragraph{Properties of Dirac delta:}
 
\paragraph{Basic differentiation properties:}

%
%\chapter{Introduction}
%
\chapter{Theory of wave propagation and radiation problems}
\pagenumbering{arabic}
\label{sec:general_wave_theory}
In this chapter the theoretical basis of sound radiation is introduced. 
The section starts with discussing the physics of sound propagation and radiation by deriving the formulation and solution of the governing homogeneous and inhomogeneous wave equations. 
Various integral representations of sound fields are presented including spectral and boundary integral representations.
%

\section{The wave equation}

Sound is a mechanical disturbance propagating in an elastic fluid, causing an alternation in the fluid's density and pressure, as well as displacement of the medium's particles.
The propagation of the disturbance is described by the acoustic wave equation. 
For a detailed treatise on the derivation refer to \cite{Beranek1993, Morse1968, Williams1999, Blackstock2000}.

Consider a homogeneous, elastic fluid, modeled as an ideal gas with no viscosity. 
In the aspect of the present thesis it is appropriate to restrict the investigations to sound propagation solely in air at room temperature.

The domain of investigation $\Omega \in \mathbb{R}^n$, where sound waves propagate is termed \emph{sound field} hereinafter.
Within this thesis usually 3 dimensional problems are investigated ($n = 3$).
The acoustical quantities of the the sound field are described by \emph{dynamic field variables} in each point $\vx \in \Omega$, at each time instant $t$: the vector variable \emph{particle velocity} $\mathbf{v}(\vx,t)$ and the scalar \emph{instantaneous sound pressure} $p(\vx,t)$ superimposed onto the static pressure $P_0 \approx 10^5~\mathrm{Pa}$.
The medium is quiescent, meaning that on average each particle is at rest with zero particle displacement (thus zero particle velocity) at the static pressure $P_0$. 
The presence of sound causes incremental change in the instantaneous pressure and the particle velocity.

In order to apply a linear model for sound propagation two assumptions are made.
Since the traveling speed of thermal diffusion is small compared to the speed of sound, it is feasible to assume that heat exchange in the wave due to compression and expansion is negligible: the state changes are modeled as adiabatic.
Furthermore the alternation of the instantaneous sound pressure is small compared to the static pressure, so the non-linear adiabatic state-change characteristics can be linearized around $P_0$.
This later assumption is fulfilled for pressure magnitudes below the threshold of pain of the human auditory system \cite{Gumerov2004, Ahrens2012}.
First the homogeneous wave equation is presented, valid for a source free domain, describing purely the propagation characteristics of acoustic waves.

\subsection{The homogeneous wave equation}

First, the homogeneous wave equation is introduced, which is valid for \emph{source-free} domains.
The linear homogeneous wave equation may be derived by utilizing two fundamental physical principles.
\begin{itemize}
\item \emph{The equation of motion:} By applying Newton's second law for an infinitesimal small volume of gas we obtain the connection between the particle velocity vector and the pressure field at each point at each time instant. 
The resulting \emph{Euler's equation} states that the force acting on the volume due to variation in the pressure distribution causes an acceleration of the volume:
\begin{equation}
\nabla p(\vx,t) = -\rho_0 \frac{\partial}{\partial t} \mathbf{v}(\vx,t),
\label{Eq:Theory:Eulers_equation}
\end{equation}
\nomenclature[2]{$\nabla$}{Gradient operator. In Descartes-coordinates it is given by $\nabla = \frac{\partial}{\partial x} \mathbf{e}_x + \frac{\partial}{\partial y} \mathbf{e}_y + \frac{\partial}{\partial z} \mathbf{e}_z$}
where $\nabla$ is the gradient operator and $\rho_0$ is the fluid's ambient density. 
In room temperature for the above given static pressure $\rho_0 = 1.18~\mathrm{kg}/\mathrm{m}^3$.

\item \emph{The gas law:} For adiabatic processes the change of state is governed by the relation
\begin{equation}
P V^{\gamma} = \mathrm{constant},
\label{Eq:Theory:Adiabatic_change}
\end{equation}
where $\gamma = C_P/C_V$ is the ratio of specific heat of the fluid with constant pressure and with constant volume.
For air $\gamma = 1.4$.
Linearization of \eqref{Eq:Theory:Adiabatic_change} around the static state $P_0, V_0$ yields
\begin{equation}
\td P = p(\vx,t) = -\gamma P_0 \frac{\td V}{V_0},
\end{equation}
where $V_0$ is the undisturbed volume. 
The relative change of volume $\td V / V_0$ may be expressed by the divergence of the particle displacement $\nabla \cdot \mathbf{u}$. 
Applying the definition of divergence and expressing the equation in terms of particle velocity yields
\begin{equation}
\frac{\partial}{\partial t} p(\vx,t) = -\gamma P_0 \, \nabla \cdot \mathbf{v}(\vx,t),
\label{Eq:Theory:continuity_eq}
\end{equation}
where $\nabla \cdot$ is the \emph{divergence operator}.
\nomenclature[4]{$\nabla \cdot$}{Divergence operator. In Descartes-coordinates: $\nabla \cdot =  \frac{\partial}{\partial x} + \frac{\partial}{\partial y} + \frac{\partial}{\partial z}$}
This \emph{continuity equation} states that the net flow of the fluid out of an infinitezimal volume results in decreased density and pressure inside the volume \cite{Arfken2005}.
\end{itemize}
%
%
Taking the time derivative of equation \eqref{Eq:Theory:continuity_eq} and the divergence of equation \eqref{Eq:Theory:Eulers_equation} the particle velocity may be eliminated. By using the \emph{Laplacian-operator} $\nabla \cdot \nabla = \nabla^2$ the scalar linear homogeneous wave equation is obtained for the sound pressure
\begin{equation}
\nabla^2 p(\vx,t) - \frac{1}{c^2} \frac{\partial^2}{\partial t^2} p(\vx,t) = 0,
\label{Eq:Theory:Scalar_wave_equation}
\end{equation}
\nomenclature[1]{$c$}{Speed of sound}%
\nomenclature[3]{$\nabla^2$}{Laplacian operator. 
In Descartes-coordinates: $\nabla^2 = \frac{\partial^2}{\partial x^2} + \frac{\partial^2}{\partial y^2} +  \frac{\partial^2}{\partial z^2}$}%
where $c \equiv \sqrt{ \frac{\gamma P_0}{\rho_0} }$ is the speed of sound in the medium. 
For air in room temperature it is given as $c = 343.1 ~ \mathrm{m}/\mathrm{s}$.
The instantaneous pressure may also be eliminated in a similar manner, resulting in the vector wave equation for each component of the particle velocity
\begin{equation}
\nabla^2 \mathbf{v}(\vx,t) - \frac{1}{c^2} \frac{\partial^2}{\partial t^2} \mathbf{v}(\vx,t) = \mathbf{0},
\label{Eq:Theory:Vector_wave_equation}
\end{equation}
valid in curl-free media, where $\nabla \left( \nabla \cdot \right) = \nabla^2$ holds.
%
Besides the pressure and the velocity, acoustic fields are often expressed via the scalar \emph{velocity potential} $\varphi(\vx,t)$, for which the acoustic wave equation also holds, and which is related to the other field variables as 
\begin{equation}
\mathbf{v}(\vx,t) = \nabla \varphi(\vx,t), \hspace{7mm} p(\vx,t) = -\rho_0 \frac{\partial}{\partial t} \varphi(\vx,t).
\label{eq:theory:velocity_potential_definition}
\end{equation}
%
The wave equations fully describe the properties of acoustic wave propagation as long as the above made assumptions are fulfilled.

\vspace{3mm}
%
Equations \eqref{Eq:Theory:Eulers_equation} and \eqref{Eq:Theory:Scalar_wave_equation} may be transformed into the angular frequency domain by performing a temporal Fourier transform  \eqref{eq:temporal_fourier_transform_def}.
Applying the differentiation property of the Fourier-transform to \eqref{Eq:Theory:Eulers_equation} yields the frequency domain Euler's equation,
\begin{equation}
\nabla P(\vx,\omega) = -\ti \omega \rho_0 \mathbf{V}(\vx,\omega)
\label{Eq:Theory:Freq_Eulers_equation}
\end{equation}
relating the pressure distribution of a time-harmonic sound field to the harmonic velocity vector field.
By taking the Fourier transform of the wave equation \eqref{Eq:Theory:Scalar_wave_equation}, the \emph{homogeneous Helmholtz-equation} is obtained:
\begin{equation}
\nabla^2 P(\vx,\omega) + k^2 P(\vx,\omega) = 0,
\label{Eq:Theory:Homog_Helmholtz}
\end{equation}
where $k$ is the \emph{acoustic wavenumber}, which is related to the temporal frequency through the \emph{dispersion relation}:
\begin{equation}
k = \frac{\omega}{c}.
\end{equation}
%
Equation \eqref{Eq:Theory:Homog_Helmholtz} must hold for every physically possible \emph{steady-state} wave form with harmonic time-dependence for a source-free volume. 
In the aspect of the present thesis the time-domain wave equation is rarely solved, therefore the general solution of the Helmholtz-equation is presented in the followings.

\subsection{The inhomogeneous wave equation}

So far wave propagation in source-free volumes was investigated.
Simple scalar disturbance of the pressure field may be included into the wave equation resulting in the time domain \emph{inhomogeneous wave equation} %\fpcom{Why is it the disturbane of the pressure field? Don't understand.}
\begin{equation}
\nabla^2 p(\vx,t) -\frac{1}{c^2}\frac{\partial^2}{\partial t^2}p(\vx,t) = -s(\vx,t),
\label{Eq:Theory:Inhomogene_wave_eq_time_domain}
\end{equation}
and by transforming wrt. time in the \emph{inhomogeneous Helmholtz equation}
\begin{equation}
(\nabla^2 + k^2 ) P(\vx,\omega ) = -S(\vx,\omega).
\end{equation}
Term $s(\vx,t)$ is referred to as the \emph{load term}, and it describes the spatial extension and time history of the excitation.

%It should be noted that pressure excitation is hardly realizable in practice. \fpcom{Why is $s$ a pressure excitation?}
To involve more physical source excitation models, additional force source terms may be added to the equation of motion \eqref{Eq:Theory:Eulers_equation}, or injected mass/volume terms may be included in the continuity equation \eqref{Eq:Theory:continuity_eq}.
This results in the \emph{general inhomogeneous wave equations} \cite{Howe2007, Kinsler2000, Pierce1991}
\begin{equation}
\nabla^2 p(\vx,t) -\frac{1}{c^2}\frac{\partial^2}{\partial t^2}p(\vx,t) = - \rho_0 \frac{\partial}{\partial t} q(\vx,t) + \nabla \cdot \mathbf{f}(\vx,t),
\label{Eq:Theory:Inhomogene_wave_eq_time_domain}
\end{equation}
and
\begin{equation}
(\nabla^2 + k^2 ) P(\vx,\omega ) = - \ti \omega \rho_0 Q(\vx,\omega) + \nabla \cdot \mathbf{F}(\vx,\omega).
\label{Eq:Theory:Inhomogene_wave_eq_freq_domain}
\end{equation}
in the time and angular frequency domains, respectively, where $q(\vx,t)$ describes the rate of increase of fluid volume per unit volume 
\footnote{The volume injection term $q(\vx,t)$ can be modeled as a simple disturbance in the velocity potential i.e. satisfies equation $\nabla^2 \varphi(\vx,t) -\frac{1}{c^2}\frac{\partial^2}{\partial t^2}\varphi(\vx,t) = -q(\vx,t)$ \cite{Jensen2007}.}
and $\mathbf{f}(\vx,t)$ represent body force excitation.
The first term is generated by sources that change the fluid volume, e.g. a pulsating sphere or a baffled dynamic loudspeaker.
The latter force term is produced by sources moving through the fluid without any change in volume e.g unbaffled loudspeakers.
A further third type of excitation term, as introduced by Lighthill, accounts for sounds produced by turbulence resulting in quadrupole sound fields \cite[p. 141]{Kinsler2000}. 
This third term is not investigated in the present thesis.

\subsection{Boundary conditions}
\label{Section:Theory:Boundary_conditions}

So far we considered wave propagation in free-field, i.e. no boundaries were present.
% In order to obtain a particular solution of the wave equation boundary conditions must be known.
%As initial conditions through the present thesis we suppose \emph{homogeneous Cauchy initial conditions} by setting $p(\vx,0) = 0$, $\frac{\partial}{\partial t}p(\vx,t)|_{t=0} = 0$, ensuring, that no sound waves are present in the domain of investigation, that would case the non-uniqueness of the particular solution.
In order to obtain a particular solution of the wave equation the wave field must satisfy prescribed boundary conditions.
The general geometry is depicted in Figure \ref{Fig:Theory:bounday_condition}.
If the domain of interest is the exterior of the enclosing boundary, while the sources are inside the volume---or it is the vibrating boundary surface itself---the problem to be solved is termed an \emph{exterior radiation problem}. 
On the other hand, if the aim is to determine the sound field inside a source-free volume---or the reflected field of a sound source inside a cavity---an \emph{interior problem} must be solved.

The boundary conditions are typically prescribed pressure or particle velocity. 
By assuming zero pressure or velocity on the boundary surface \emph{homogeneous boundary conditions} are considered. 
Non-zero field variables on the other hand represent a vibrating surface and are termed \emph{inhomogeneous bondary conditions}.

In the aspect of this thesis two important types of boundary conditions are of interest:
\begin{itemize}
\item \emph{Dirichlet boundary condition} prescribes the pressure measured on the boundary surface. 
The homogeneous Dirichlet boundary conditions are thus
\begin{equation}
P(\vx,\omega) = 0, \hspace{3mm} \forall \hspace{3mm} \vx \in \dO.
\end{equation}
These types of boundaries are called \emph{sound-soft} or \emph{pressure release} boundaries, and are used to model e.g. the surface of the ocean for a wave propagating in the water \cite{Blackstock2000, Ziomek1995}.

The inhomogeneous Dirichlet boundary condition assumes a prescribed pressure value on the boundary surface:
\begin{equation}
P(\vx,\omega) = f_D(\vx,\omega), \hspace{3mm} \forall \hspace{3mm} \vx \in \dO.
\end{equation}
\begin{figure}
	\centering
	\begin{overpic}[width = .5\columnwidth]{Figures/Basic_acoustics/boundary_conditions.png}
	\small
	\put(27,37){$\mathbf{n}_{\mathrm{in}}$}
	\put(40,47){$\Oi$}
	\put(50,82){$\Oe$}
	\put(11,48){$\dO$}	
	\put(79,71){$r$}	
	\put(84,85){$\dO_{\infty}$}
	\end{overpic}
	\caption{Geometry for the boundary conditions in general interior and exterior radiation problems, and the infinite boundary for the Sommerfeld radiation condition}
	\label{Fig:Theory:bounday_condition}
\end{figure}

\item \emph{Neumann boundary condition} gives the normal derivative of the pressure on the boundary surface, i.e. prescribes the normal velocity of the surface. %For the sake of simplicity the normal derivative taken on the surface uses the following notation and definition
%\begin{equation}
%\frac{\partial}{\partial n} f(\vx)\equiv \left. \frac{\partial}{\partial \mathbf{n}(\vx)} f(\vx) \right|_{\dO} \equiv \left. \langle \nabla f(\vx), %\mathbf{n}(\vx) \rangle \right|_{\dO},
%\end{equation}
%where $ \mathbf{n}(\vx) $ is the normal vector of the boundary surface. For interior problems the inward pointing normal is used.%
%
Homogeneous Neumann boundary condition are
\begin{equation}
\left. \frac{\partial}{\partial \mathbf{n}(\vx)} f(\vx) \right|_{\vx \in \dO}= 0,
\end{equation}
where $ \mathbf{n}(\vx) $ is the normal vector of the boundary surface.
These type of boundaries are termed \emph{sound hard}, or \emph{rigid} boundaries, ensuring that no incident wave can mobilize the boundary surface.

Inhomogeneous Neumann boundary conditions are given by
\begin{equation}
\left. \frac{\partial}{\partial \mathbf{n}(\vx)} f(\vx) \right|_{\vx \in \dO}= f_N(\vx,\omega).
\end{equation}
Vibrating surfaces---e.g. mounted loudspeakers, or baffled pistons---are most often modeled using these type of boundary conditions.
\end{itemize}

For radiation problems it is feasible to assume free field conditions, i.e. only outgoing waves are present in the sound field. 
This is ensured by the \emph{Sommerfeld radiation condition} that excludes the non-physical solutions of the wave equation emerging from infinity.
Mathematically it can be formulated by implying boundary condition on $\dO_{\infty}$, with $r$ increased to infinity, as shown in Figure \ref{Fig:Theory:bounday_condition} \cite{Schot1992:Eighty_years, Williams1999}:
\begin{equation}
\lim_{r \rightarrow \infty} r \left( \left. \frac{\partial}{\partial r}P(\vx,\omega)\right|_{\vx \in \dO_{\infty}} +\ti \frac{\omega}{c}P(\vx,\omega) \right) = 0, \hspace{3mm} \forall \hspace{3mm} \vx \in \dO_{\infty}.
\label{Eq:Theory:Sommerfeld_radiation_condition}
\end{equation}
%The condition stems from the inclusion of the surface $\dO_{\infty}$ to the Kirchhoff-Helmholtz integral for the general exterior problem (see in the following section), and by deriving boundary conditions that ensure zero contribution of this surface in its limiting value.


\section{Solution of the homogeneous wave equation}

\subsection{Plane wave theory}
Now the general solution of the homogeneous wave equation is considered in Cartesian coordinate systems, leading to the plane wave theory.
The Descartes coordinate form of the Laplace-operator is given by
\begin{equation}
\nabla^2 = \frac{\partial^2}{\partial x^2} + \frac{\partial^2}{\partial y^2} +  \frac{\partial^2}{\partial z^2}.
\end{equation}
A common method for obtaining the general solution of the Helmholtz-equation is the separation of variables \cite{Devaney2012}: 
it is supposed that the solution of \eqref{Eq:Theory:Homog_Helmholtz} can be written in the form of the product
\begin{equation}
P(\vx,\omega) = \hat{P}(\omega) X(x)Y(y)Z(z).
\label{Eq:Theory:Seperated_variables}
\end{equation}
Substituting it into \eqref{Eq:Theory:Homog_Helmholtz} and dividing both sides by $\hat{P}(\omega) X(x)Y(y)Z(z)$ yields
\begin{equation}
\underbrace{\frac{\td^2 X(x)}{\td x^2}\frac{1}{X(x)}}_{-k_x^2} + 
\underbrace{\frac{\td^2 Y(y)}{\td y^2}\frac{1}{Y(y)}}_{-k_y^2} + 
\underbrace{\frac{\td^2 Z(z)}{\td z^2}\frac{1}{Z(z)}}_{-k_z^2}
= - k^2.
\label{Eq:Theory:Seperated_variables_expanded}
\end{equation}
Since each term contains a total derivative---independent from any other variable-- equality may hold only if each term is constant. 
These constant are denoted by $k_x^2, k_y^2, k_z^2$. 
Consequently each part of the equation leads to a simple eigenvalue problem, for which the eigenfunction solution is given by exponentials. 
Written e.g. for the $x$-variable:
\begin{equation}
\frac{\partial^2 X(x)}{\partial x^2} = -k_x^2 X(x) \hspace{5mm} \rightarrow \hspace{5mm} X(x) = A_1 \te^{-\ti k_x x} + A_2 \te^{\ti k_x x}.
\end{equation}
The solutions may be substituted back into equation \eqref{Eq:Theory:Seperated_variables}. 
In order to include every possible solution the general solution for the free-field homogeneous Helmholtz-equation is yielded by summation over all possible values of $k_x-k_y-k_z$ weighted by arbitrary constants. 
However, the variables are not independent, for a fixed temporal frequency they are related according to the dispersion relation, resulting from \eqref{Eq:Theory:Seperated_variables_expanded}:
\begin{equation}
k^2 = \left( \frac{\omega}{c} \right)^2 = k_x^2 + k_y^2 + k_z^2.
\end{equation}
As a dependent variable we will use $k_y$ through this treatise so that
\begin{equation}
k_y^2 = k^2 - k_x^2 - k_z^2.
\end{equation}
With all the foregoing and by denoting the arbitrary weighting constant by $\tilde{P}(k_x,k_z, \omega)$, the general solution of the 3D Helmholtz equation reads
\begin{equation}
P(\vx,\omega) = \frac{1}{(2\pi)^2}\iint_{-\infty}^{\infty} \tilde{P}(k_x,k_z, \omega)  \te^{- \ti \left( k_x x + k_y y + k_z z \right) }
\td k_x\td k_z.
\label{Eq:Theory:Helmholtz_Inverse_Fourier}
\end{equation}
Constant $\frac{1}{(2\pi)^2}$ is introduced as a Fourier-transform normalization term. 
The general solution---describing the inverse Fourier transform of $\tilde{P}(k_x,k_z, \omega)  \te^{- \ti k_y y }$---therefore is obtained in the form of a spectral integral, similarly to the case of the temporal solution.
Obviously, the spectral coefficients are obtained via a suitable forward Fourier transform, as explained in the next section.
%\footnote{\color{red} {The same result can be obtained by performing a forward Fourier transform to the Helmholtz equation along all spatial dimensions resulting in $\left(-k_x^2 -k_y^2 -k_z^2 + \left(\frac{\omega}{c}\right)^2 \right)\tilde{P}(k_x,k_y,k_z,\omega) = 0$, which is satisfied for any $\tilde{P}$ as long as the dispersion relation holds.
%In the inverse transform for $P(\vx,\omega)$ the latter can be taken into consideration by multiplication with $\delta(k_y - \sqrt{k^2-k_x^2-k_z^2})$, and by exploiting the sifting property of the Dirac function integration with respect to $k_y$ can be carried out.}}.

\vspace{3mm}
One separated solution from the integral is in the form of \cite{Williams1999}
\begin{equation}
P(\vx,\omega) = \hat{P}(\omega) \te^{-\ti \left( k_x x + k_y y + k_z z \right) } =  \hat{P}(\omega) \te^{-\ti \left< \vk \cdot \vx \right> },
\end{equation}
where $\mathbf{k} = [k_x,\ k_y,\ k_z]^{\mathrm{T}}$ is the \emph{wavenumber vector}, with its length equaling the acoustic wavenumber $k = | \mathbf{k}|$ and pointing into the direction of the maximum phase advance, given by the gradient of the phase function  \nomenclature[5]{$\left< \mathbf{a} \cdot \mathbf{b} \right>$}{Inner product of vector $\mathbf{a}$ and $\mathbf{b}$, given by $\mathbf{a}^{\mathrm{T}} \mathbf{b}$}.
The solution represents a \emph{plane wave} component with the acoustic wavelength $\lambda = 2\pi/k$, traveling in the direction
\begin{equation}
\vk = - \nabla \phi_P(\vx,\omega)
\label{Eq:Theory:PW_wavenumber_vec}
\end{equation}
where $\phi_P$ denotes the phase of function $P$  \nomenclature[5]{$\phi_f$}{Phase of the complex valued function $f(x) \in \mathbb{C}$, written in a general polar form $f(x) = A_f(x) \te^{\ti \phi_f(x)}$}.
The terminology indicates that the surface of constant phase points are lying along an infinite plane, perpendicular to $\mathbf{k}$. Refer to Figure \ref{Fig:Theory:plane_wave} (a) for the illustration of a traveling plane wave.

\begin{figure}%[!h]
	\centering
	\begin{overpic}[width = .85\columnwidth ]{Figures/Basic_acoustics/plane_wave_illustration.png}
	\put(2,1){(a)}
	\put(52,1){(b)}
	\end{overpic}
\caption{Illustration of a traveling plane wave (a) and an evanescent wave (b) with $\omega = 2\pi \cdot 1000 ~\mathrm{rad/s}$. 
In the present case the plane wave travels along the $xy$-plane, with $k_z = 0$. 
Variables $k_x = k \cos \varphi, \hspace{2mm} k_y = k \sin \varphi$ give the wavenumber components along the $x$ and $y$ directions. 
For the case of the evanescent wave $k_x > \frac{\omega}{c}$, resulting in exponential decay along the $y$-coordinate.
In a source free region propagating and evanescent waves form a complete, orthonormal basis for the solution of the helmholtz equation.}
	\label{Fig:Theory:plane_wave}
\end{figure}
\vspace{2mm}
%As it is indicated in the figure $k_x-k_y-k_z$ variables are the $x-y-z$ directional components of the wavenumber vector. For the sake of simplicity assume that $k_z = 0$, thus the propagation direction of the plane wave is parallel with the $z=0$ plane. In this case the wavenumber components are expressed as
%\begin{eqnarray}
%k_x = k \sin \theta , \\
%k_y = k \cos \theta .
%\end{eqnarray}
%  
%\subsubsection{Evanescent waves}

Since there is no constraint on the values of $k_x$ and $k_z$, the plane wave equation is satisfied also when $k_x^2 + k_z^2 > k^2$. Resulting from the dispersion relation in these cases $k_y$ becomes complex.
In order to ignore the non-physical exponentially increasing solution in the followings we define $k_y$ as 
\begin{equation}
k_y = \begin{cases}
                       \sqrt{\left(\frac{\omega}{c}\right)^2 - k_x^2 - k_z^2}  & \text{if} \hspace{3mm} k_x^2 + k_z^2 \leq \left(\frac{\omega}{c}\right)^2\\
                      -\ti \sqrt{k_x^2+k_z^2 - \left(\frac{\omega}{c}\right)^2 } = -\ti k_y' &  \text{if} \hspace{3mm} k_x^2 + k_z^2 > \left(\frac{\omega}{c}\right)^2,
                 \end{cases}
\label{eq:theory:k_y_definition}
\end{equation}
Solutions with $k_y'$ given as
 \begin{equation}
 P(\vx,\omega) = \hat{P}(\omega) \te^{-k_y' y} \te^{-\ti \left( k_x x + k_z z \right) }
 \end{equation}
describe plane waves, propagating perpendicular to the $y$-axis, and exhibiting an exponentially decaying amplitude along the $y$-direction (see Figure \ref{Fig:Theory:plane_wave} (b)):
In those cases when one wavelength component is shorter than the acoustic wavelength, the wave can not propagate from the $y = 0$ surface, but an exponentially decaying radiation phenomena occurs.
These type of waves are termed \emph{evanescent waves}, opposed to \emph{propagating waves}, when all wavenumber components are real valued.
 
%
Evanescent waves are often the results of the difference between the speed of sound in different materials: in solids the propagation speed of flexural bending waves is proportional to the square root of their temporal frequency. 
As a consequence in case of e.g. a vibrating plate, higher-order modes will not be radiated into the free-space, since the bending wave's wavelength on the surface may become shorter than the acoustic wavelength would be in air. 
In these cases air above the surface acts as a hydrodynamic short-circuit.
An other example may be the radiation from a cold surface into a warmer half space with a continuous temperature profile, resulting in a continuous sound speed profile.
Due to the variation in the speed of sound the wavenumber components of a plane wave also alter continuously resulting in refraction phenomena.
As soon as one wavenumber component reaches the evanescent region total internal refraction occurs and only rapidly decaying evanescent waves are present in the further part of the half space.

%
The evanescent contribution is of central importance in the field of \emph{Nearfield Acoustic Holography}---when one needs a high-resolution image from the velocity distribution on the vibrating object's surface---, however their contribution is often neglected in the field of sound field synthesis, when the listener is relatively far from the secondary loudspeaker array, and loudspeaker spacing is higher than the evanescent wavelengths.

%
\subsection{The angular spectrum representation}

It could be seen that any source-free sound field may be expressed in terms of a double inverse Fourier-transform, given by \eqref{Eq:Theory:Helmholtz_Inverse_Fourier}.
This formulation is termed the \emph{angular spectrum representation} \cite{Ahrens2010phd, Ahrens2012, Williams1999} or the \emph{plane wave expansion} \cite{Spors2005} of the sound field.
The meaning of the spectral weighting components $\tilde{P}(k_x,k_z,\omega)$ is obtained by expressing the pressure at the infinite plane $y=0$: it is revealed, that $P(x,0,z,\omega) = \mathcal{F}_x^{-1}\mathcal{F}_z^{-1} \left\{\tilde{P}(k_x,k_z, \omega)\right\}$,
and the \emph{angular spectrum}, or \emph{plane wave expansion coefficients} $\tilde{P}(k_x,k_z, \omega)$ can be therefore expressed as the corresponding forward Fourier-transform of the pressure distribution at $y=0$.
%: $\hat{P}(k_x,k_y, \omega) = \mathcal{F}_x\mathcal{F}_z \left\{  P(x,0,z,\omega) \right\}$.
In the followings, the domain characterized by $k_x$, $k_z$ is termed the \emph{wavenumber domain}.

Equation \eqref{Eq:Theory:Helmholtz_Inverse_Fourier} therefore relates the pressure distribution of an arbitrary sound field measured on the plane $y=0$ to its pressure distribution on an arbitrary parallel plane. 
In the wavenumber domain the relation reads
\footnote{This formulation is the direct consequence of the Fourier shift theorem applied in the $k_x, k_y, k_z$ space for the $y$-coordinate}
\begin{equation}
\mathcal{F}_x\mathcal{F}_z \left\{ P(\vx,\omega) \right\} = \tilde{P}(k_x,y,k_z,\omega) = \tilde{P}(k_x,0,k_z,\omega) \te^{-\ti k_y y},
\label{Eq:Theory:Wave_field_extrapolation}
\end{equation}
with $k_y$ given by \eqref{eq:theory:k_y_definition}.
Note, that wave propagation is determined by the phase change of the plane wave expansion's $y$-component, therefore generally speaking the following equation holds:
\begin{equation}
\tilde{P}(k_x,y,k_z,\omega) = \tilde{P}(k_x,y_0,k_z,\omega) \te^{-\ti k_y ( y - y_0 ) }.
\label{Eq:Theory:Wave_field_extrapolation_2}
\end{equation}

\vspace{3mm}
This statement leads to two important formulations:
the above equation written in the spatial domain yields
\begin{equation}
P(\vx,\omega) = \frac{1}{4\pi^2}\iint_{-\infty}^{\infty} \tilde{P}(k_x,y_0,k_z,\omega) \te^{-\ti k_y ( y - y_0 ) }  \te^{- \ti \left( k_x x  + k_z z \right) }
\td k_x\td k_z.
\label{Eq:Theory:Pressure_propagated}
\end{equation}
Expressing $\tilde{P}(k_x,y_0,k_z,\omega)$ in terms of the normal velocity $\tilde{V}_{\mathrm{n}}(k_x,y_0,k_z,\omega)$ using the Euler's equation  \eqref{Eq:Theory:Freq_Eulers_equation}, with the normal ---i.e. $y$---derivative at $y = y_0$ calculated by applying the differentiation theorem to \eqref{Eq:Theory:Wave_field_extrapolation} one obtains
\begin{equation}
P(\vx,\omega) = \frac{1}{4\pi^2}\iint_{-\infty}^{\infty} 
\underbrace{ \rho_0 c k \tilde{V}_{\mathrm{n}}(k_x,y_0,k_z,\omega)}_{-\frac{1}{\ti} \frac{\partial}{\partial y} \tilde{P}(k_x,y_0,k_z,\omega) }
\frac{\te^{-\ti k_y ( y - y_0 ) } }{k_y} \te^{- \ti \left( k_x x + k_z z \right) }
\td k_x\td k_z.
\label{Eq:Theory:Velocity_propagated}
\end{equation}
%\footnote{\color{red}{Note, that all these equation may be applied directly for 2D wave field extrapolation (invariant along the $z$-direction) with substitution $k_z = 0$.}}
These formulations are of central importance in the field of Fourier-acoustics. 
They state that an arbitrary sound field is completely determined by either the pressure, or by the normal velocity component, measured along an infinite plane. 
Wave propagation is calculated by multiplying the measured spectra with an exponential term, referred to as the \emph{pressure propagator} $\tilde{G}_p$ in \eqref{Eq:Theory:Pressure_propagated} and the \emph{velocity propagator} $\tilde{G}_v$ in \eqref{Eq:Theory:Velocity_propagated}:
\begin{equation}
\tilde{G}_p(k_x,y-y_0,k_z,\omega) = \te^{-\ti k_y ( y - y_0 ) } ,\ \hspace{1cm}
\tilde{G}_v(k_x,y-y_0,k_z,\omega) = \rho_0 c k \frac{\te^{-\ti k_y ( y - y_0 ) } }{k_y}
\label{Eq:Theory:propagators}
\end{equation}
\begin{align}
\tilde{P}(k_x,y,k_z,\omega) &= \tilde{P}(k_x,y_0,k_z,\omega) \, \tilde{G}_p(k_x,y-y_0,k_z,\omega) \\
                            &= \tilde{V}_{\mathrm{n}}(k_x,y_0,k_z,\omega) \, \tilde{G}_v(k_x,y-y_0,k_z,\omega).
\end{align}
Wave propagation in source-free volumes therefore can be modeled by 2D linear filtering of the sound field, where the filter transfer characteristics are given by the corresponding propagator.
Formulation of the equations in the spatial domain results in 2D spatial convolutions, termed the Rayleigh I. and II. integrals, as it will be further discussed in the latter sections.

As a further important consequence of the wavefield extrapolating equations the $y$-derivative of the angular spectrum can be expressed by differentiating both sides of \eqref{Eq:Theory:Wave_field_extrapolation_2} with respect to the $y$-coordinate
\begin{equation}
\frac{\partial}{\partial y} \tilde{P}(k_x,y,k_z,\omega) = \frac{\partial}{\partial y} \left( \tilde{P}(k_x,y_0,k_z,\omega) \te^{-\ti k_y ( y - y_0 ) } \right) = -\ti k_y \tilde{P}(k_x,y,k_z,\omega),
\label{eq:Theory:Fourier_diff}
\end{equation}
which is the Fourier transform differentiation theorem for wavefield extrapolation.

%
\subsection{Solution in other geometries}
Similarly to the presented Cartesian-solution, the general solution of the free-field homogeneous Helmholtz equation can be found for spherical and cylindrical coordinate systems. 
The representations are given in the form of an infinite series of spherical and cylindrical harmonics respectively, relating the radiated sound at an arbitrary point to the sound field measured on a spherical or an infinite cylindrical surface.
These solutions are of great importance when spherical or circular secondary source distributions are applied for sound field reconstruction. 
Since the present thesis does not include the spectral solution of the reconstruction problem for these geometries, the presentation of the spherical and cylindrical solutions are omitted. 
For a detailed investigation refer to \cite{Williams1999, Zotter2009phd, Ahrens2012}.

%
%
%
%
%
%
%
%
%
%
%
%
\section{Solution of the inhomogeneous wave equation}

\subsection{The Green's function}
%First the solution for \eqref{Eq:Theory:Inhomogene_wave_eq_time_domain} is introduced.
A common way to obtain the solution for the inhomogeneous wave equation is using the \emph{Green's function}. 
We define the $n$-dimensional \emph{Green's function} as the solution for the following equation \cite{Gumerov2004, Williams1999}
\begin{equation}
\nabla^2 g(\vx|\vxo,t) -\frac{1}{c^2}\frac{\partial^2}{\partial t^2} g(\vx|\vxo,t) = -\delta\left( \vx - \vxo \right)\delta\left( t - t_0 \right),
\label{Eq:Theory:Green_function_def}
\end{equation}
with $\vx, \vxo \in \mathbb{R}^{n}$ and $\delta()$ being the Dirac-delta distribution. 
The Green's function describes the sound field at $\vx$ due to an impulsive disturbance located at $\vxo$ at the time instant $t_0$.
The Green's function is often referred to as the \emph{spatio-temporal impulse response} of the domain of interest and its temporal Fourier-transform $G(\vx|\vxo,\omega)$ as the \emph{spatio-temporal transfer function} of a point source at $\vxo$. 
In the followings we assume free-field conditions by implying the Sommerfeld-radiation condition. 
Under these assumptions the \emph{free field Green's function} is translation invariant, denoted by $g(\vx-\vxo,t)$.

The motivation behind the use of the Green's function is that assuming an arbitrary linear differential operator $\mathcal{L}_{\vx}\left\{ \right\}$ acting on a distribution $p(\vx)$ with an arbitrary excitation $-s(\vx)$, the solution of the inhomogeneous differential equation $\mathcal{L}_{\vx}\left(p(\vx)\right) = -s	(\vx)$ may be expressed by the convolution of the Green's function and the load term:
\footnote{Multiplying both sides of the left equation of \eqref{Eq:Theory:Basic_Green_function_eq} by $-s(\vxo)$ and integrating along all dimensions according to $\vxo$ results in
$-\int_{\Omega(\vxo)} s(\vxo)\mathcal{L}_{\vx}\left\{ g(\vx-\vxo) \right\} \td \vxo= s(\vx)$.
Since $\mathcal{L}_{\vx}$ acts only on $\vx$, the operator may be taken outside of the integration.
Expressing the load term by $-\mathcal{L}_{\vx}\left\{ p(\vx) \right\}$ leads to $\mathcal{L}_{\vx}\left\{ \int_{\Omega(\vxo)} s(\vxo) g(\vx-\vxo) \td \vxo \right\} = \mathcal{L}_{\vx}\left\{ p(\vx) \right\}$.
}
\begin{equation}
\mathcal{L}_{\vx}\left\{ g(\vx-\vxo) \right\} = -\delta( \vx-\vxo ) \hspace{3mm} \rightarrow \hspace{3mm}
p(\vx) = \int_{\Omega(\vxo)}  g(\vx-\vxo) s(\vxo) \td \vxo.
\label{Eq:Theory:Basic_Green_function_eq}
\end{equation}

The Green's function is usually obtained by eigenfunction expansion of the operator in a given geometry with specified boundary conditions. 
Under free-space assumptions, where harmonic functions give a full orthogonal basis a straightforward method is to perform a Fourier transform to equation \eqref{Eq:Theory:Green_function_def} at $\vxo = 0$ with respect to space and time, yielding in $\vx \in \mathbb{R}^{3}$
\begin{equation}
\left(-(k_x^2 + k_y^2 + k_z^2) + \left(\frac{\omega}{c} \right)^2\right)\tilde{G}(\mathbf{k},\omega) = -1,
\end{equation}
with $\mathbf{k} = [k_x,\ k_y,\ k_z]^{\mathrm{T}}$.
The Green's function in the wavenumber domain reads \cite{Devaney2012, Watanabe2015}
\begin{equation}
\tilde{G}(\mathbf{k},\omega) = -\frac{1}{\left( \frac{\omega}{c}\right)^2 -  k_x^2 - k_y^2 - k_z^2 }.
\label{Eq:Theory:3D_kxkykzw_Green}
\end{equation}
Applying the Fourier convolution theorem to \eqref{Eq:Theory:Basic_Green_function_eq} the solution of \eqref{Eq:Theory:Inhomogene_wave_eq_time_domain} in the wavenumber domain reads
\begin{equation}
\tilde{P}(\mathbf{k},\omega)  = \tilde{S}(\mathbf{k},\omega) \tilde{G}(\mathbf{k},\omega) = -\frac{\tilde{S}(\mathbf{k},\omega)}{\left( \frac{\omega}{c}\right)^2 -  k_x^2 - k_y^2 - k_z^2 },
\end{equation}
and the solution in the spatio-temporal domain is yielded by the inverse Fourier-transform:
\begin{equation}
p(\vx,t) =\frac{1}{(2\pi)^4} \iiiint^{\infty}_{-\infty} - \frac{\tilde{S}(\mathbf{k},\omega)}{\left( \frac{\omega}{c}\right)^2 -  k_x^2 - k_y^2 - k_z^2 } \te^{-\ti \left( \left< \vk \cdot \vx \right> - \omega t \right) } \td k_x \td k_y \td k_z \td \omega.
\end{equation}
The different representations of the free-field Green's function may be obtained by the corresponding inverse Fourier-transform of \eqref{Eq:Theory:3D_kxkykzw_Green}.
The resulting formulas are collected in Table\ \ref{tab:theory:Greens_fun_representations} by taking only the causal solutions into consideration.

\begin{table}[h!]
\caption{Free field acoustic Green's function representations ($\vxo = 0$) \cite{Devaney2012, Duffy2001:Greens, Ahrens2010a, Ahrens2012, Gibson2008, DeSanto1992}.
$\theta\left( \right)$ denotes the Heaviside step function, $H_0^{(2)}\left( \right)$ is the zeroth order Hankel function of the second kind and $K_0
\left( \right)$ is the modified Bessel function of the second kind.
The conditional expressions ensure that evanescent waves are attenuated with increasing distance from the source. For the sake of brevity in the followings Greens function is expressed only in the propagating region, however it should be kept in mind, that evanescent wavenumber components are defined as given in \eqref{eq:theory:k_y_definition}, resulting in the presented conditional expressions.
}
\FloatBarrier
\renewcommand*{\arraystretch}{2.25}
\label{tab:theory:Greens_fun_representations} 
    \begin{tabular}[h!]{  c | | l |	 l }%\toprule
      & 3-dimensional & 2-dimensional \\ \hline
    $\tilde{G}(k_x,k_y,k_z,\omega)$ & $-\frac{1}{ \left(\frac{\omega}{c}\right)^2 - k_x^2-k_y^2-k_z^2} $ &  $-\frac{1}{\left(\frac{\omega}{c}\right)^2 - k_x^2-k_y^2}\delta(k_z)$ \\ 
    $\tilde{G}(k_x,k_y,z,\omega)$   &  
    \scriptsize	$\begin{aligned}[t]
	-\frac{\ti}{2}\frac{\te^{-\ti\sqrt{(\frac{\omega}{c})^2 - k_x^2 - k_y^2}|z|}}{\sqrt{(\frac{\omega}{c})^2 - k_x^2 - k_y^2}},\hspace{3mm} \text{for} \hspace{1mm}
	\sqrt{k_x^2+k_y^2}	\leq\left| \frac{\omega}{c} \right| \\
	\frac{1}{2}\frac{\te^{-\sqrt{k_x^2 + k_y^2-(\frac{\omega}{c})^2}|y|}}{\sqrt{k_x^2 + k_y^2-(\frac{\omega}{c})^2}},\hspace{3mm}  \text{for} \hspace{1mm}				\sqrt{k_x^2+k_y^2}>\left| 	\frac{\omega}{c} \right| 
	\end{aligned}$ \normalsize    
    &
	$-\frac{1}{\left(\frac{\omega}{c}\right)^2 - k_x^2-k_y^2}$
	\\
    $\tilde{G}(k_x,y,z,\omega)$
    % \footnote{$\tilde{G}(k_x,k_y,z,\omega)$ and $\tilde{G}(k_x,y,z,\omega)$ may be written in a less expressive but briefer form as 
    % $\tilde{G}(k_x,y,k_z,\omega) = -\frac{\ti}{2}\frac{\te^{-\sqrt{(\frac{\omega}{c})^2 - k_x^2 - k_z^2}|y|}}{\sqrt{(\frac{\omega}{c})^2 - k_x^2 - k_z^2}}$ and
    % $\tilde{G}(k_x,y,z,\omega) = -\frac{\ti}{4}H_0^{(2)}\left( -\ti\sqrt{k_x^2-(\frac{\omega}{c})^2  } \sqrt{y^2+z^2} \right)$ valid for arbitrary $k_x$ and $k_z$  }     
    &      
    \scriptsize
    $\begin{aligned}[t] % placement: default is "center", options are "top" and "bottom"
	-\frac{\ti}{4} H_0^{(2)}\left( \sqrt{(\frac{\omega}{c})^2 - k_x^2 } \sqrt{y^2+z^2} \right),\hspace{3mm} \text{for} \hspace{1mm}|k_x|<\left| \frac{\omega}{c} \right| \\ \frac{1}{2\pi} K_0\left( \sqrt{k_x^2 - (\frac{\omega}{c})^2 } \sqrt{y^2+z^2} \right),\hspace{3mm}  \text{for} \hspace{1mm}|k_x|>\left| \frac{\omega}{c} \right| 
	\end{aligned}$ \normalsize
     &     
     \scriptsize	$\begin{aligned}[t]
	-\frac{\ti}{2}\frac{\te^{-\ti\sqrt{(\frac{\omega}{c})^2 - k_x^2 }|y|}}{\sqrt{(\frac{\omega}{c})^2 - k_x^2 }},\hspace{3mm} \text{for} \hspace{1mm}|k_x|		\leq			\left| \frac{\omega}{c} \right| \\
	\frac{1}{2}\frac{\te^{- \sqrt{k_x^2 -(\frac{\omega}{c})^2}|y|}}{\sqrt{k_x^2 -(\frac{\omega}{c})^2}},\hspace{3mm}  \text{for} \hspace{1mm}|k_x|>\left| 					\frac{\omega}{c} \right| 
	\end{aligned}$ \normalsize      \\ 
    $G(x,y,z,\omega)$ 				 &  $\frac{1}{4\pi}\frac{\te^{-\ti\frac{\omega}{c}\sqrt{x^2+y^2+z^2}}}{\sqrt{x^2+y^2+z^2}}$ & \scriptsize$-\frac{\ti}{4} H_0^{(2)}\left( \frac{\omega}{c} \sqrt{x^2+y^2} \right) $\normalsize  \\ 
    $g(x,y,z,t)$ 					 &  $\frac{1}{4\pi}\frac{\delta\left( t - \sqrt{x^2+y^2+z^2}/c \right)}{\sqrt{x^2+y^2+z^2}}$  & $\frac{1}{2\pi}\frac{\theta(t - \sqrt{x^2+y^2}/c)}{\sqrt{t^2 - \left(\frac{\sqrt{x^2+y^2}}{c}\right)^2}}$
    \end{tabular}
\end{table}
\FloatBarrier

\begin{figure}
	\centering
	\begin{overpic}[width = .95\columnwidth]{Figures/Basic_acoustics/greens_function.png}
	\small
%	\put(27,37){(a)}
%	\put(40,47){(b)}
	\end{overpic}
	\caption{Different representations of the 3D free-field Green's function in the angular frequency domain $G(x,y,z,\omega)$, with $\lambda = \frac{c}{f} = \frac{2\pi c}{\omega}$ (a) and in the semi-wavenumber domain $\tilde{G}(k_x,k_y,z,\omega)$ (b) shown at $z=0$.}
	\label{Fig:Theory:Greens_function}
\end{figure}
%
%\vspace{3mm}
In 3-dimensions the 2-dimensional Green's function represents the field of an infinite line source along the $z$-axis, that can be described as a continuous linear distribution of 3D point sources---explaining the infinite tail of the 2D impulse response%: it is given by the sum of 3D impulse responses, delayed and attenuated depending on how far the actual impulse arrives from, and multiplied by 2, since along the vertical line source two point elements contribute to the total field at each time instant
---, thus the relation between the 3D and 2D Green's functions is given as
\begin{equation}
G_{2\mathrm{D}}(x,y,\omega) = \int_{-\infty}^{\infty} G_{3D}(x,y,z,\omega) \td z = \left. \mathcal{F}_{z}\left\{ G_{3D}(x,y,z,\omega) \right\}\right|_{k_z = 0},
\label{Eq:Wave_Theory:2D_Green}
\end{equation} 
obviously holding for any other representation, as it is reflected by the table above.

\subsection{Solution of the general inhomogeneous wave equation} 
\label{Section:Theory:Inhom_wave_eq_solution}
Now the solution of the general inhomogeneous wave equation, given by \eqref{Eq:Theory:Inhomogene_wave_eq_freq_domain} is presented.
From \eqref{Eq:Theory:Basic_Green_function_eq} the solution in the spatial domain is obtained by the convolution of the source term with the Green's function 
\footnote{The second term can be obtained by integration by parts: $
\int \nabla \cdot \mathbf{F}(\vxo)G(\vx-\vxo) \td \vxo = 
\int \nabla \cdot \left( \mathbf{F}(\vxo)G(\vx-\vxo) \right) \td \vxo 
- \int \left< \nabla G(\vx-\vxo)  \cdot \mathbf{F}(\vxo) \right> \td \vxo $.
Applying the Gauss theorem and invoking the Sommerfeld radiation condition reveals, that the first term of the right handside vanishes.}
:
\begin{equation}
P(\vx,t) = -\int_{\Omega(\vxo)} \ti \omega \rho_0 Q(\vxo,\omega)G(\vx-\vxo,\omega) +  \left< \mathbf{F}(\vxo,\omega)\cdot \nabla G(\vx-\vxo,\omega) \right> \td \vxo,
\end{equation}	
where $\Omega(\vx)$ is the domain of interest, containing the source distribution.
From the general solution the case of point-like disturbances is of special interest, with the distribution function described by a Dirac distribution:
%

\begin{figure}
	\centering
	\begin{overpic}[width = .95\columnwidth ]{Figures/Basic_acoustics/monopole_dipole.png}
	\footnotesize
	\put(0,0){(a)}
	\put(35,0){(b)}
	\put(68,0){(c)}
	\end{overpic}
	\caption{ Directivity characteristics of a monopole (a), a dipole with the dipole axis being the $y$ axis (b) and a horizontal quadrupole constructed from two opposing dipoles in the horizontal plane (c). All multipoles are solutions to the wave equation due to its linearity.}
	\label{Fig:Theory:multipoles}
\end{figure}
\begin{itemize}
%
\item supposing, that $Q(\vxo,\omega) = \hat{Q}(\omega)\delta(\vx)$, one obtains the field response to a point-like volume injection
\begin{equation}
P_{\mathrm{m}}(\vx,\omega) = - \frac{\ti \omega \rho_0 \hat{Q}(\omega)}{4\pi} \frac{\te^{-\ti \frac{\omega}{c}|\vx-\vxo|}}{|\vx-\vxo|}.
\end{equation}
This is the field of an \emph{acoustic monopole}, which is defined as a pulsating sphere, with its radius decreased to infinitesimal, with the total volume velocity held constant \cite{Howe2007}.
$\hat{Q}(\omega)$ is often referred as \emph{monopole strength}. Monopoles constitute a good far-field approximation of sources in the velocity field, e.g. a dynamical loudspeaker.
\item assuming a point-like force excitation, described by $\mathbf{F}(\vx,\omega) =  \mathbf{f} \hat{F}(\omega) \delta(\vx)$, where the unit vector $\mathbf{f}$ denotes the direction of the force the solution for the inhomogeneous wave equation is given by
\begin{equation}
P_{\mathrm{d}}(\vx,\omega) = -\hat{F}(\omega) \left< \mathbf{f} \cdot \nabla G(\vx,\omega) \right>.
\end{equation}
The expression describes the field generated by an \emph{acoustic dipole}, with vector $\hat{F}(\omega) \mathbf{f}$ denoting the \emph{dipole moment}.
The terminology reflects that an acoustic dipole can be constructed by two antiphase point sources positioned infinitesimally close to each other 
\footnote{Such a distribution can be described by the directional gradient of a Dirac distribution $s(\vx,\omega) = \left< \mathbf{f} \cdot \nabla \delta(\vx) \right>$.}.
By expressing the gradient of the Green's function the full form of a dipole field is given as
\begin{equation}
P_{\mathrm{d}}(\vx,\omega) =  \hat{F}(\omega)
\cos \theta \left(  \frac{1}{|\vx-\vxo|} + \ti \frac{\omega}{c} \right)
\frac{1}{4\pi}
\frac{\te^{-\ti \frac{\omega}{c}|\vx-\vxo|}}{|\vx-\vxo|}.
\end{equation}
with $\cos \theta = \frac{\left<\mathbf{f} \cdot (\vx-\vxo) \right> } {|\vx-\vxo|}$.
Unlike monopoles, dipoles are directive sources, with the directivity characteristics described by $\cos \theta$.
Dipoles give a good model for e.g. unbaffled loudspeakers, moving freely in the fluid, radiating maximally into the direction of motion $\mathbf{f}$ often termed the \emph{dipole axis}, and without any lateral radiation.
%
\end{itemize}
The importance of monopoles and dipoles along with higher order \emph{multipoles} lies in the far-field approximation of the field of extended sources,
where a complex radiation pattern may be expanded into series of weighted multipole fields, termed \emph{multipole expansion}.
Figure \ref{Fig:Theory:multipoles} presents the directivity pattern of multipoles up to the third order.

\vspace{3mm}
It is important to notice, that comparison of $\tilde{G}(k_x,y,k_z,\omega)$ in Table \ref{tab:theory:Greens_fun_representations}  with the pressure and velocity propagators \eqref{Eq:Theory:propagators} with applying the Fourier differentiation theorem \eqref{eq:Theory:Fourier_diff} reveals that
\begin{align}
\tilde{G}_p(k_x,y,k_z,\omega) &=  2 \ti k_y \tilde{G}(k_x,y,k_z,\omega) 
\hspace{2mm} \rightarrow \hspace{2mm} 
G_p(\vx,\omega) = -2 \frac{\partial}{\partial y} G(\vx,\omega) = 2 P_{\mathrm{d}}(\vx,\omega),
\\
\tilde{G}_v(k_x,y,k_z,\omega) &=  2 \ti \omega \rho_0 \tilde{G}(k_x,y,k_z,\omega) 
\hspace{1mm} \rightarrow \hspace{1mm} 
G_v(\vx,\omega) = 2 \ti \omega \rho_0 G(\vx,\omega) = -2P_{\mathrm{m}}(\vx,\omega),
\end{align}
i.e. the pressure and velocity propagators are given by dipoles and monopoles respectively.
This finding will be further discussed in the section, dealing with the Rayleigh integral formulation.

\section{Boundary integral representation of sound fields}

\subsection{The Kirchhoff-Helmholtz integral equation}
Any sound field obeying the homogeneous Helmholtz-equation may be written in the form of a surface integral above an enclosing surface, termed the \emph{Kirchhoff-Helmholtz integral equation}. 
This integral formulation, solving the homogeneous wave equation with inhomogeneous boundary conditions is of central importance in the field of acoustics, e.g. forms the backbone of the Boundary Element Method, SVD-based Conformal Nearfield Acoustic Holography, and Sound Field Synthesis.

In this section the integral formulation of interior problems in source-free volumes is introduced.
The effect of direct sources inside the enclosure may be straightforwardly included in the following results by the proper addition of the solution of the inhomogeneous Helmholtz equation \cite{Spors2005}.
\begin{figure}[!h]
	\centering
	\begin{overpic}[width = .65\columnwidth ]{Figures/Basic_acoustics/Kirchhoff-Helmholtz.png}
	\scriptsize
	\put(0,23){primary source}
		\put(48,1){$\mathbf{O}$}
		\put(70.5,31){$\vx$}
		\put(37,14){$\vxo$}
		\put(39,22){$\vni(\vxo)$}
		\put(50,40){$\Oi$}
		\put(80,5){$\Oe$}
		\put(80,40.5){$\dO$}
	\end{overpic}
\caption{Geometry for the interior Kirchhoff-Helmholtz integral, representing the sound field inside an enclosure $\Oi$ generated by an exterior sound source in the form of a surface integral along $\dO$. For 2D problems the boundary degenerates to a closed contour, enclosing the area $\Oi$.}
	\label{Fig:Theory:HIE_geometry}
\end{figure}

Let $\Oi$ be an $n$-dimensional enclosure, bounded by the surface $\dO$ with arbitrary position vectors $\vxo, \vx \in \mathbb{R}^{\mathrm{n}}$. 
Refer to figure \ref{Fig:Theory:HIE_geometry} for the geometry. 
For two continuous, differentiable scalar valued functions $\Phi(\vxo)$, $\Psi(\vxo)$ the Green's theorem reads (see \ref{App:Green_theorem} for the derivation)
\begin{equation}
\small
\int_{\Omega}
\left(  \Phi(\vxo) \nabla^2 \Psi(\vxo) - \Psi(\vxo) \nabla^2 \Phi(\vxo)   \right)   \td \Oi(\vxo)= 
\oint_{\dO}  \left(  \Psi(\vxo) \frac{\partial \Phi(\vxo)}{\partial \vni}  - \Phi(\vxo) \frac{\partial \Psi(\vxo)}{\partial \vni}  \right)   \td \dO(\vxo),
\label{Eq:Theory:Greens-theorem}
\end{equation}
with $\frac{\partial}{\partial \vni}$ denoting the inward normal derivative $\langle \vni(\vxo) \cdot \left. \nabla \Phi(\vx)\right|_{\vx = \vxo} \rangle$.
Let $\Phi$ be the pressure field inside the enclosure satisfying the homogeneous Helmholtz equation and let $\Psi$ be the Green's function
\footnote{Although the continuity requirement is not fulfilled for the Green's function, with proper mathematical workaround the singularity at $\vx$ may be excluded from the enclosure \cite{Williams1999}.}
---i.e. a point source located at $\vx$:
\begin{equation}
(\nabla^2 + k^2)P(\vxo,\omega) = 0, \hspace{10mm}
(\nabla^2 + k^2)G(\vxo|\vx,\omega) = -\delta(\vxo - \vx).
\end{equation}

Substituting into the Green's theorem leads to
\begin{equation}
\small
\int_{\Oi} - P(\vxo,\omega) \delta(\vxo - \vx)
  \td \Oi(\vxo) = 
\oint_{\dO}  \left(  G(\vxo|\vx,\omega) \frac{\partial P(\vxo,\omega)}{\partial \vni}  - P(\vxo,\omega)  \frac{\partial G(\vxo|\vx,\omega)}{\partial \vni}  \right)   \td \dO(\vxo).
\end{equation}

The sifting property of the Dirac-delta may be exploited, by taking into account that the singularity is located in the enclosure:
if $\vx$ lies outside the volume the integral is identically zero, while if it is on the surface it is assumed, that "only half of the Dirac-impulse is in the volume". For a rigorous derivation refer to \cite{Williams1999}.
Finally, by exploiting the symmetry and translation invariancy of the free-field Green's function
\begin{equation}
G(\vxo|\vx) = G(\vx|\vxo) = G(\vx-\vxo), \qquad
\frac{\partial}{\partial \vxo} G(\vxo|\vx)= \frac{\partial}{\partial \vxo} G(\vx|\vxo)
\end{equation}
the \emph{Kirchhoff-Helmholtz integral equation} (KHIE) is obtained:
\begin{equation}
\alpha P(\vx,\omega) = 
\oint_{\dO} - \left( 
\frac{\partial P(\vxo,\omega)}{\partial \vni} G(\vx-\vxo,\omega)
-
P(\vxo,\omega)  \frac{\partial G(\vx-\vxo,\omega)}{\partial\vni} 
\right)   \td \dO( \vxo),
\label{Eq:Theory:Kirchhoff-Helmholtz}
\end{equation}
with
\begin{equation*}
\alpha = \begin{cases} 
1           & \hspace{1mm} \forall \hspace{5mm}  \vx \in \Oi  	   \\
\frac{1}{2} & \hspace{1mm} \forall \hspace{5mm}  \vx \in \dO  \\
0 			& \hspace{1mm} \forall \hspace{5mm}  \vx \in \Oe.
\end{cases}
\end{equation*}
Point $\vx$ is termed \emph{evaluation point}, while $\vxo$ is termed the \emph{field point}. 
A frequently used form of the KHIE---utilizing the Euler's equation \eqref{Eq:Theory:Eulers_equation} to express the normal derivative of the pressure in terms of the normal velocity on the surface ($\frac{\partial P(\vxo,\omega)}{\partial \vni} = -\ti \rho_0 \omega V_{\mathrm{n}}(\vxo,\omega)$)---is
\begin{equation}
\alpha P(\vx,\omega) = 
\oint_{\dO}  \left(  
V_{\mathrm{n}}(\vxo,\omega) \underbrace{\ti \rho_0 \omega  G(\vx-\vxo,\omega) }_{-P_{\mathrm{m}}(\vx-\vxo,\omega)}
+
P(\vxo,\omega)  \underbrace{\frac{\partial G(\vx-\vxo,\omega)}{\partial \vni}}_{-P_{\mathrm{d}}(\vx-\vxo,\omega)}
\right)   \td \dO(\vxo),
\label{Eq:Theory:Kirchhoff}
\end{equation}
where the weighting factors containing the Green's function can be recognized as the fields of monopoles and dipoles, functioning as velocity and pressure propagators analogously to the angular spectrum approach.
The actual form of the Green's function depends on the problem dimensionality, given in the previous section.

The equation states that the pressure field inside an enclosure is completely determined by the boundary conditions for the pressure and normal velocity on the boundary surface.
The interior KHIE describes the pressure field only inside the volume of investigation, outside the volume the left hand side is identically zero. For exterior radiation problems the exterior KHIE can be derived in a similar manner, describing the pressure field outside the volume and ensuring zero left hand side inside \cite{Williams1999}.
In both cases the approach is capable of dealing only with \emph{forward propagation problems} i.e. capable to describe the effects of a source distribution based on the radiated field measured on a surface, but unable to describe the source properties from these data. 
This latter scenario is called an acoustic \emph{inverse problem}, forming the basis of Acoustic Holography and Sound Field Synthesis 
\footnote{Actually, the KHIE may be made able to solve inverse problems, by replacing the forward propagating Green's function by the backward propagating Green's function \cite{Wapenaar1989}}.
\vspace{3mm}

KHIE consists of two integral components, termed the \emph{single layer potential} and the \emph{double layer potential}: single layer potential describes the field as the weighted sum of a single layer of point sources, characterized by $ G(\vx|\vxo) $, while the double layer potential describes the field of an ensemble of dipole point sources, described by $\frac{\partial G(\vx|\vxo,\omega)}{\partial \vni}$, realized by two anti-phase point sources: by a double layer.

One drawback of interior HIE is that it overspecifies the problem in order to ensure zero pressure and velocity outside the domain of interest. 
In the aspect of Sound Field Synthesis the presence of both single and double layer potentials is infeasible.
By letting the sound field non-zero outside the enclosure it is possible to completely describe the sound field in the region of interest in terms of only single or double layer potentials by either modifying the Green's function in order to satisfy Dirichlet or Neumann boundary conditions, or to impose these boundary conditions on the sound field $ P(\vxo,\omega)$ itself in an equivalent scattering problem.
In the following sections these approaches are applied for the simplification of the KHIE.



\subsection{The Simple Source Formulation}
The simple source formulation is derived from the KHIE by the construction of a separate exterior and interior radiation problem with prescribing the same inhomogeneous Dirichlet boundary condition for both fields on the boundary surface $\dO$ \cite{Ahrens2012}.
%
\begin{figure}[h!]
	\centering
	\begin{overpic}[width = 1\columnwidth ]{Figures/Basic_acoustics/simple_source_formulation.png}
	\footnotesize
	\put(2, 38){(a)}
	\put(52,38){(b)}
	\put(27, 0){(c)}
	\put(36, 68){$P(\vx,\omega)$}
	\put(77, 68){$P_\mathrm{e}(\vx,\omega) = -P_\mathrm{s}(\vx,\omega)$}
	\put(73, 53){$P_\mathrm{i}(\vx,\omega)$}
	\put(85,57){$\Oi$}
	\put(92,52.5){$\dO$}
	\put(95,47){$\Oe$}
	\put(45,32){$P_\mathrm{t}(\vx,\omega) = P(\vx,\omega) + P_\mathrm{s}(\vx,\omega)$}
	\put(70,8){$\Oe$}
	\put(58,18){$\Oi$}
	\put(65,13.5){$\dO$}
	\end{overpic}
\caption{Illustration of simple source formulation in a 2D problem ($\Omega \subset \mathbb{R}^2$). Figures show the incident/target sound field (a), the field given by the simple source formulation (b) and the scattering of the incident field from a sound soft boundary (c). The incident field is the field of a 2D point source (i.e. a line source) at $\vxs = [-0.4,\ 2.5]^{\mathrm{T}}$. Equation \eqref{Eq:Theory:Simple_source_HIE} was evaluated numerically using an open source C++ Boundary Element software \cite{Fiala2014:BEM}. The figures demonstrate, how simple source formulation expresses the incident field inside $\Oi$, and the $(-1)$ times the scattered field at $\Oe$ in an equivalent sound soft scattering problem. Figure (c) showing the difference between the incident field and the simple source field ((a)-(b)) therefore illustrates the total scattering in the exterior.}
	\label{Fig:Theory:simple_source_formulation}
\end{figure}

Let's assume an exterior sound field $P_{\mathrm{e}}(\vx,\omega)$, satisfying the homogeneous Helmholtz equation at $\vx \in \Oe$, i.e. that all sources are located within the enclosure. 
The exterior wave field is the combination of radiating, or diverging waves. 
On the other hand assume an interior sound field $P_{\mathrm{i}}(\vx,\omega)$ inside the enclosure $\vx \in \Omega$, induced by a sound source located outside the volume of investigation, thus the interior field also satisfies the homogeneous Helmholtz equation constructed by a set of incoming or converging waves.
The two spatially disjunct problems are connected through the following boundary condition written onto the boundary surface
\begin{equation}
P_{\mathrm{e}}(\vxo,\omega) = P_{\mathrm{i}}(\vxo,\omega), \hspace{15mm} \vxo \in \dO.
\end{equation}
Both fields may be expressed in terms of an exterior and an interior KHIE respectively, for the exterior KHIE with inward normals refer to \cite[eq. 8.30]{Williams1999}.
By adding the exterior and interior KHIEs, due to the coupled boundary condition terms, weighted by the pressure on the boundary vanish and the following integral expression is obtained \cite[p.~268.]{Williams1999}
\begin{equation}
\oint_{\dO} 
G(\vx|\vxo,\omega) 
\left(
\frac{\partial P_{\mathrm{e}}(\vxo,\omega)}{\partial \vni} - \frac{\partial P_{\mathrm{i}}(\vxo,\omega)}{\partial \vni} 
\right)
\td \dO ( \vxo)
= 
\begin{cases} 
P_{\mathrm{e}}(\vx,\omega)           & \hspace{1mm} \forall \hspace{5mm}  \vx \in \Omega_e  	   \\
P_{\mathrm{e}}=P_{\mathrm{i}} & \hspace{1mm} \forall \hspace{5mm}         \vx \in \dO  \\
P_{\mathrm{i}}(\vx,\omega) 			& \hspace{1mm} \forall \hspace{5mm}   \vx \in \Oi.
\end{cases}
\label{Eq:Theory:Simple_source_HIE}
\end{equation}
The equation states that either the interior or the exterior sound field, satisfying the homogeneous Helmholtz equation may be represented as a single layer potential.
The \emph{single layer strength function} is given in the integral \eqref{Eq:Theory:Simple_source_HIE} implicitly.
The discontinuity in the pressure gradient is termed the \emph{jump relation}, expressing the fact that the sound field generated by the single layer potential is continuous in pressure on the boundary $\dO$, while the gradient changes sign i.e. \emph{jumps}.

%In terms of sound field synthesis the interior sound field is the desired sound field itself. The simple source formulation therefore states that for an arbitrary geometry the SSD driving function is given by
%\begin{equation}
%D(\vxo,\omega) = 
%\frac{\partial P_{\mathrm{e}}(\vxo,\omega)}{\partial \vni} - \frac{\partial P(\vxo,\omega)}{\partial \vni},
%\label{Eq:Theory:Source_strength}
%\end{equation}
%where $P_{\mathrm{e}}(\vxo,\omega)$ is the corresponding exterior sound field, needed to be calculated in order to solve the SFS problem.

\vspace{3mm}
As pointed out in \cite{Fazi2013:Equivalent_scattering, Fazi2010, Schultz2014:Comparing_approaches, Zotter2013:uniqueness} the following physical interpretation can be assigned to the simple source formulation: 
we assume that the surface $\dO$ represents the boundary of a sound soft scattering object. 
In acoustic scattering problems we consider an a-priori known \emph{incident sound field} $P(\vx,\omega)$ that is reflected by the scattering object, generating the \emph{scattered field} $P_{\mathrm{s}}(\vx,\omega)$.
The field measured in the presence of the obstacle is termed the \emph{total field} $P_{\mathrm{t}}(\vx,\omega)$, given by the sum of the incident and scattered fields.
The scattered field is the solution of the exterior radiation problem, so that the total field obeys homogeneous boundary conditions on the sound soft scatterer surface, i.e. $P_{\mathrm{s}}(\vxo,\omega) = -P(\vxo,\omega) = - P_{\mathrm{e}}(\vxo,\omega), \hspace{.2cm} \vxo \in \dO$.
%In the aspect of SFS the incident field inside the theoretical scatterer is the target sound field itself.
Comparing this result with the simple source formulation it is clear, that the single layer driving function is the derivative of $(-1)$ times the total field on the boundary.
See Figure \ref{Fig:Theory:simple_source_formulation} for an illustration of the simple source formulation and for its interpretation as an equivalent scattering problem.
Obviously, in practical applications usually only the incident field is known analytically.
The general application therefore would require numerical computation method, e.g. BEM. For such a general scenario see Figure \ref{Fig:Theory:simple_source_formulation}, where the single layer weighting factors are calculated numerically for an arbitrary enclosing reflecting surface in a 2D scenario.

%Another demonstration of this principle is shown in Figure \ref{Fig:Theory:monopole_synthesis_by_planar_SDM} (b) where the difference between the synthesized field and the target sound field is the total field in the equivalent scattering problem $P_{\mathrm{t}}(\vx,\omega) = P(\vx,\omega) - P_{\mathrm{e}}(\vx,\omega)$.
%\begin{equation}
%D(\vxo,\omega) = \frac{\partial P_T(\vxo,\omega)}{\partial n}
%=
%\frac{\partial P(\vxo,\omega)}{\partial n} + \frac{\partial P_s(\vxo,\omega)}{\partial n}.
%\label{Eq:Theory:Equivalent_scattering_driv_fun}
%\end{equation}
%Simple source approach---and the equivalent scattering interpretation---gives the analytical driving function for an arbitrary SSD geometry implicitly. 
%Unfortunately the exterior scattering solution is scarcely available analytically except for simple geometries. The general application therefore would require numerical computation method, e.g. BEM. For such a general scenario see Figure \ref{Fig:Theory:simple_source_formulation}, where the driving functions are calculated numerically for an arbitrary enclosing SSD in a 2D scenario.

\subsection{The Rayleigh integrals}
\label{Section:Theory:Rayleigh}

The Rayleigh integrals formulate the sound field with merely the pressure field or the normal velocity measured on an infinite plane. 
The derivation utilizes the Neumann and Dirichlet Green's functions for the geometry, that can be seen in figure \ref{Fig:Theory:Rayleigh_geometry}.

In order to derive the Rayleigh integrals an interior problem is considered by writing the KHIE on a boundary consisting of a simply joint disc ($\dO_P$) and a hemisphere ($\dO_S$, $\dO = \dO_P \cup \dO_S$), shown in Figure\ \ref{Fig:Theory:Rayleigh_geometry}. 
As the radius of the hemisphere is increased to infinity ($r \rightarrow \infty$) the Sommerfeld-radiation condition is invoked and the contribution of the hemisphere vanishes: the radiated field is described by a surface integral written on the infinite plane. 
For the sake of simplicity the plane is located at $y=0$ with its normal given by $\vni = [0,\ 1,\ 0]^{\mathrm{T}}$, i.e. the field at $y>0$ reads
\begin{multline}
P(\vx,\omega) = \lim_{r\rightarrow \infty} \left( \int_{\dO_P} + \int_{\dO_S} \td \dO \right) = \\
\int_{\dO_P}  \left( 
P(\vxo,\omega)  
\left. \frac{\partial G(\vxo|\vx,\omega)}{\partial y_0} \right|_{y_0 = 0} 
-
\left. \frac{\partial P(\vxo|\vx,\omega)}{\partial y_0} \right|_{y_0 = 0} 
G(\vxo|\vx,\omega) 
\right)   \td \dO_P ( \vxo).
\end{multline}

Exploiting that any homogeneous solution of the Helmholtz equation---satisfying free field boundary conditions---may be added to the Green's function, the inhomogeneous wave equation, and the Kirchhoff-Helmholtz integral still holds: either the single or the double layer potential may be eliminated in the integral by setting the homogeneous solution in the Green's function to non-zero:
\begin{itemize}
\item The \emph{Neumann Green's function} eliminates the double layer potential, by describing Neumann boundary conditions for the Green's function on the bounding infinite plane:
\begin{equation}
G_N(\vxo,\omega) = G(\vxo|\vx,\omega) + G'_N(\vxo,\omega),
\end{equation}
\begin{equation}
\frac{\partial G_N(\vx,\omega) }{\partial \vni} = \left. \frac{\partial G_N(\vx,\omega) }{\partial y_0} \right|_{y_0 = 0} = 0.
\label{Eq:Theory:Neumann_Greenfun_def}
\end{equation}
\item The \emph{Dirichlet Green's function} eliminates the double layer potential in the same manner by prescribing
\begin{equation}
G_D(\vxo,\omega) = G(\vxo|\vx,\omega) + G'_D(\vxo,\omega) = 0, \hspace{3mm} \forall \hspace{3mm} \vxo \in \dO_P.
\label{Eq:Theory:Dirichlet_Greenfun_def}
\end{equation}
\end{itemize}

\begin{figure}
\small
  \begin{minipage}[c]{0.45\textwidth}
  \hspace{1cm}
	\begin{overpic}[width = 1\columnwidth ]{Figures/Basic_acoustics/Rayleigh_integral.png}
	\scriptsize
	 	\put(100,16){$x$}		
		\put(52.5,63.5){$y$}
		\put(19.5,38){$\vx$}		
		\put(18,0){$\vx'$}
		\put(44,16){$\vxo$}
		\put(38.5,34){$\vni$}
		\put(62,37){$r$}
		\put(40,50){$\Oi$}
		\put(75,16){$\dO_P$}
		\put(60,59){$\dO_S$}
	\end{overpic} \end{minipage}\hfill
	\begin{minipage}[c]{0.4\textwidth}
    \caption{
Geometry for deriving the Rayleigh integrals. The radius of the hemisphere $\dO_S$ is increased to infinity, therefore only $\dO_P$ contributes to the radiated field at $y>0$.
    } \label{Fig:Theory:Rayleigh_geometry}
  \end{minipage}
\end{figure}

From simple geometrical considerations it can be deduced that the sound fields satisfying the homogeneous Helmholtz equation at $y>0$ and equations \eqref{Eq:Theory:Neumann_Greenfun_def}, \eqref{Eq:Theory:Dirichlet_Greenfun_def} on the infinite plane $y=0$ are given as
\begin{equation}
 G'_N(\vxo,\omega) = G(\vxo|\vx',\omega), \hspace{10mm} G'_D(\vxo,\omega) = -G(\vxo|\vx',\omega),
\end{equation}
with $\vx' = [x,\ -y,\ z]^{\mathrm{T}}$ i.e. by simple mirroring the Green's function to the boundary surface $\dO_P$, thus
\begin{equation}
G_N(\vxo,\omega) = 2G(\vxo,\vx,\omega), \hspace{10mm}  \left. \frac{\partial G_D(\vxo,\omega) }{\partial y_0} \right|_{y_0 = 0} = 2\left. \frac{\partial G(\vxo|\vx,\omega) }{\partial y_0} \right|_{y_0 = 0}
\end{equation}
holds.

%The Neumann and Dirichlet Green's functions therefore satisfy $(\nabla^2+k^2)G_N(\vxo,\omega) = -(\delta(\vxo-\vx) + \delta(\vx'))$ and $(\nabla^2+k^2)G_N(\vxo,\omega) = -(\delta(\vxo-\vx)  - \delta(\vxo-\vx'))$ respectively. 
Substituting them into the Green's theorem \eqref{Eq:Theory:Greens-theorem} yields the Rayleigh I and II integrals respectively \cite{Berkhout1984}:
\begin{eqnarray}
P_N(\vx,\omega) =
\int_{\dO_P}
-2
\left. \frac{\partial P(\vx,\omega)}{\partial y} \right|_{y = 0} 
G(\vx-\vxo,\omega) \td \dO_P ( \vxo).
\label{Eq:Theory:RayleighI}
\\
P_D(\vx,\omega) =
\int_{\dO_P}
2 P(\vxo,\omega)  
\left. \frac{\partial G(\vx-\vxo,\omega)}{\partial y} \right|_{y = 0} 
\td \dO_P ( \vxo)
\label{Eq:Theory:RayleighII}
\end{eqnarray}
with
\begin{align*}
P_N(\vx,\omega) = \begin{cases} 
P(\vx,\omega)           & \hspace{1mm} \forall \hspace{5mm}  y>0  	     \\
P(\vx = \vx',\omega) 			& \hspace{1mm} \forall \hspace{5mm}  y=0  \\
P(\vx',\omega) 			& \hspace{1mm} \forall \hspace{5mm}  y<0
\end{cases}
&&
P_D(\vx,\omega) = \begin{cases} 
P(\vx,\omega)           & \hspace{1mm} \forall \hspace{5mm}  y>0  	     \\
0 			& \hspace{1mm} \forall \hspace{5mm}  y=0  \\
-P(\vx',\omega) 			& \hspace{1mm} \forall \hspace{5mm}  y<0.
\end{cases}
\end{align*}

%Again, the form of Green's function depends on the problem dimensionality and in 2D the Rayleigh integrals express an arbitrary sound field in the form of a line integral.
The Rayleigh integrals provide the radiated field in the form of a spatial convolution over the boundary surface with the convolution kernel being either the Neumann or 
Dirichlet Green's function.
The Rayleigh integrals are of major importance in the theory of diffraction from finite aperture. 
The Rayleigh I integral is also extensively used in the calculation of radiated fields from finite radiators, mounted in infinite walls, e.g. the field of loudspeakers. It states that the radiated field from a rigid vibrating plane can be calculated by summing the field of point sources, driven by the normal velocity distribution, or mathematically speaking: by convolving the Green's function with the velocity distribution over the infinite surface.

\vspace{3mm}
It is important to note that the Rayleigh integrals---besides the presented methodology involving the Neumann and Dirichlet Greeen's function---can be deduced directly using the equivalent scattering interpretation of the simple source approach: for a planar reflecting surface the reflected field can be obtained by mirroring the incident field to the scatterer, resulting in the same formulation.

Another straightforward way to obtain the Rayleigh integrals stems from the direct inverse Fourier-transform of the angular spectrum representations
\eqref{Eq:Theory:Pressure_propagated} and \eqref{Eq:Theory:Velocity_propagated} by applying the Fourier transform convolution theorem and applying that the pressure and velocity propagators are given by the field of dipoles and monopoles, as it was stated in section \ref{Section:Theory:Inhom_wave_eq_solution}.

Obviously, the fact, that all the presented three approaches lead to the same solution stems from the uniqueness of the solution for the simple source formulation problem for a planar boundary.
%
%
%
\chapter{High frequency approximation of wave fields and radiation problems}
\label{sec:high_freq_approx}
The boundary integral representations introduced in the previous section give the possibility for controlling the sound field inside enclosures. 
However, their direct application for sound field synthesis is of great computational complexity for arbitrary geometries.
In order to derive integral representations more suitable for general sound field reproduction, the application of approximate solutions are inevitable.
This chapter presents high frequency asymptotic approximations of sound fields and their integral representations.
These approximations will be of crucial interest in finding the driving function for general loudspeaker contours in the latter sections.

The presented local/asymptotic description of wavefields is not unknown in the fields of acoustics: With minor modifications it is a massively used concept in ray tracing and geometrical optics/acoustics.
However, its application for sound field synthesis problems has been unprecedented so far.

\section{Local attributes of sound fields}
\subsection{The local wavenumber vector}

Assume an arbitrary steady-state harmonic sound field in $\vx \in \mathbb{R}^3$, written in a general polar form with $A^P(\vx,\omega)$, $\phi^P(\vx,\omega) \in \mathbb{R}$
\begin{equation}
P(\vx,\omega) = A^P(\vx,\omega) \, \te^{\ti \phi^P(\vx,\omega)}.
\label{eq:HF_appr:general_sf}
\end{equation}
%
The dynamics of wave propagation is described by the phase of the sound field.
From ray tracing/geometrical optics theory the following quantity is introduced \cite{Carozzi2004, Romer2005}:
%
\importanteq{Local waveno. vector}{
\label{eq:local_wn_vec_def}
\vk^P(\vx) = -\Dx \phi^P(\vx,\omega),
}%
termed the \emph{local wavenumber vector} of sound field $P$, being obviously the generalization of the plane wave wavenumber vector introduced by equation \eqref{Eq:Theory:PW_wavenumber_vec}.\footnote{The negative sign ensures that the phase of the sound field decreases into the propagation direction, similarly to the case of a plane wave}
Note that the local wavenumber vector is frequency dependent, which dependency is suppressed for the sake of brevity.
In Cartesian coordinates the components of the local wavenumber vector are denoted as $\vk^P(\vx)=[k_x^P(\vx),\ k_y^P(\vx),\ k_z^P(\vx)]^{\mathrm{T}}$.
In the following, the existence of the superscript distinguishes local properties from the global ones: the local wavenumber components of a sound field from the wavenumber components of its plane wave decomposition.
The wavenumber vector, defined as the negative gradient of the phase function, points into the direction of maximal phase advance, being perpendicular to isophase surfaces, i.e. it is perpendicular to the wavefront in any position.
For an isotropic medium, where the propagation speed is constant, the wavenumber vector points into the direction of the wave's energy flow, thus towards the local wave propagation direction.\footnote{This statement holds exclusively for isotropic media.
Although the wavenumber vector is always perpendicular to the wavefront, in anisotropic media the energy of a wave not necessarily travels along the path, defined by the wavefront normals \cite{Pollard1977}.}
The illustration of the local wavenumber vector of a point source and a plane wave is depicted in \ref{Fig:HF_appr:local_wavenumber_vector}.
%
\begin{figure}
	\small
	\centering
	\begin{overpic}[width = .9\columnwidth]{Figures/High_freq_approximations/wavenumber_vector.png}
	\put(0,30){a)}
	\put(50,30){b)}
	\put(0,0){c)}
	\put(50,0){d)}
	\end{overpic}
	\caption{Illustration of the local wavenumber vector for a 2D acoustic point source (a,c) and a 2D plane wave (b,d).
(a-b) show an arbitrarily chosen contour of constant phase (isochronous contour), along with the wavenumber vector on this contour.
(c-d) show the $x$-component of the normalized local wavenumber vector ($\hat{k}^P_x(x,y_0)$), measured along the horizontal dashed black lines.
}
	\label{Fig:HF_appr:local_wavenumber_vector}
\end{figure}

Now it is investigated how wave dynamics can be described in terms of the local wavenumber vector.
Substituting the general, polar formulation of an arbitrary sound field \eqref{eq:HF_appr:general_sf} into the Helmholtz equation \eqref{Eq:Theory:Homog_Helmholtz}, expressing the Laplace operator by its definition ($\Lx = \Dx \cdot \Dx$), and expressing the derivative of the polar form yields
\begin{equation}
\left( 
\frac{\Lx A^P}{A^P} 
- 
| \Dx \phi^P |^2
+ 
\ti \left(  
\Lx \phi^P
+ 2\frac{ \left< \Dx \phi^P \cdot \Dx A^P \right> }{A^P} 
\right)
+ \left(\frac{\omega}{c}\right)^2 
\right) 
P = 0,
\label{eq:HF_appr:ray_tracing_helmholtz}
\end{equation}
with the function arguments being suppressed for the sake of transparency.

In order to have the equality satisfied, both the real and imaginary parts of the bracketed term have to vanish, resulting in the following coupled equations:
\begin{eqnarray} \label{eq:HF_appr:eikonal_eq}
\frac{\Lx A^P}{A^P}  - | \Dx \phi^P |^2 + \left(\frac{\omega}{c}\right)^2 = 0, \\ 
\label{eq:HF_appr:transport_eq}
\Lx \phi^P + 2\frac{ \left< \Dx \phi^P \cdot \Dx A^P \right> }{A^P} = 0.
\end{eqnarray}

Under high frequency conditions the phase changes rapidly compared to the amplitude, i.e. $\frac{\nabla^2_{\vx} A^P}{A^P} \ll | \nabla_{\vx} \phi^P |^2$ holds, 
and by utilizing the definition of the local wavenumber vector, equation \eqref{eq:HF_appr:eikonal_eq} leads to the \emph{local dispersion relation}
\importanteq{Local dispersion relation}{
|\vk^P(\vx)|^2 = k^P_x(\vx)^2 + k^P_y(\vx)^2 + k^P_z(\vx)^2 = \left( \frac{\omega}{c} \right)^2 = k^2,
\label{eq:HF_appr:local_dispersion}
}
being a generalization of the dispersion relation, given for plane waves by \eqref{Eq:Theory:dispersion_relation}.
%
The equation holds trivially for simple sound fields: for plane waves, for point sources and for line sources excluding the source position,\footnote{As a well-known fact from the field of electrostatics, the amplitude factor of a point source serves as the Green's function for the Laplace equation, satisfying $\Lx A^P = \Lx \frac{1}{4\pi} \frac{1}{|\vx-\vxo|} = -\delta(\vx-\vxo)$. Hence, for the field of a point source $\frac{\Lx A^P}{A^P}=0$ trivially holds and the local dispersion relation is satisfied, except for the singular point.} however fails in the presence of strong interference phenomena, due to which the amplitude distribution varies rapidly over the space.
Furthermore, the present form of the local dispersion relation is valid only for stationary sound fields in isotropic media.
In Chapter \ref{sec:moving_source_synthesis} the theory will be extended to include non-stationary fields through the example of the sound field, generated by a moving harmonic source.

By applying the local dispersion relation, the \emph{normalized local wavenumber vector} can be defined for a stationary sound field as
\begin{equation}
\vhk^P(\vx) = \frac{\vk^P(\vx)}{|\vk^P(\vx)|} = \frac{\vk^P(\vx)}{\omega/c},
\end{equation}
being a vector of unit length (independent from frequency), pointing in the local propagation direction of the sound field.
%In the field of high frequency geometrical optics the representation of wave fields in $\vx, \vk(\vx)$ is termed the phase space representation \cite{Arnold1995}.
%Over the last decades also the phase space representation of acoustic fields has gained an increasing interest\cite{Steinberg1993, Teyssandier2005}.}.
%
The normalized wavenumber vector, i.e. the normalized phase change of wavefields is a basic concept in ray tracing, massively used for solving wave propagation problems in anisotropic media.
In the field of ray tracing expression $\Gamma(\vx) = -\frac{\phi^P(\vx,\omega)}{k}$ is termed as the \emph{eikonal}, that's gradient defines the local propagation direction of the wavefield: $\nabla \Gamma(\vx) = \vhk(\vx)$.
In that context the local dispersion relation in the form of \eqref{eq:HF_appr:eikonal_eq} is termed the \emph{eikonal equation} \cite{Pierce1991, Kinsler2000}, having to be solved for the eikonal---often at space-variant sound speeds---yielding the phase change of the sound rays over the ray path.
The second basic ray tracing equation, termed the \emph{transport equation}, is given by \eqref{eq:HF_appr:transport_eq}, with its solution providing the intensity change of sound rays.
In the following section the physical interpretation of this equation is investigated in terms of the local wavenumber vector and the local wavefront curvature.

\subsection{The local wavefront curvature}
%
Applying the local wavenumber vector concept, the \emph{local wavefront curvature} of arbitrary sound fields can be introduced.
The wavefront curvature and the radius of curvature give an expressive physical interpretation and coordinate system independent description for the results of the asymptotic approximations developed in the following sections, and serves as a mathematical basis in order to distinguish \emph{divergent} and \emph{convergent} wavefronts.
A wavefield is termed \emph{divergent} with a convex wavefront propagating away from a source distribution and \emph{convergent} or \emph{focused}, if a concave wavefront propagates towards a focal point.
Mathematically, the local vergence of the wavefield may be described by the \emph{principal curvatures} of the wavefront or in a looser sense by the \emph{mean curvature} of the wavefront.

The \emph{principal curvature} components $\kappa_1^P(\vx),\kappa_2^P(\vx)$ are defined geometrically as the reciprocal of the principal radii $\rho_1^P(\vx), \rho_2^P(\vx)$, being the maximal and minimal radii of osculating circles at a point on the wavefront, as illustrated in Figure \ref{Fig:HF_appr:local_wave_curvature}.
Mathematically, the curvatures can be defined via the \emph{Hessian matrix} of the phase function $\mH^P(\vx)$, with the elements in 3D given by
\begin{equation}
H_{ij}^P(\vx) =
\frac{\partial^2}{\partial x_i \partial x_j} \phi^P(\vx,\omega) \hspace{1cm} i,j = 1,2,3.
\end{equation} 
As long as the local dispersion relation holds, the principal curvatures are given by the two non-zero eigenvalues of the Hessian, normalized by $-\frac{\omega}{c}$, as discussed in details in the Appendix \ref{App:Hessian} \cite{Hartmann1999, Hartmann2001}.
A wavefield is then divergent when both principal curvatures are positive \cite{Bleistein1984, Arnold1986, HF_and_Pulse_Scattering1992}.

\begin{figure} 
	\small
  \begin{minipage}[c]{0.55\textwidth}
  \hspace{0cm}
	\begin{overpic}[width = 1\columnwidth ]{Figures/High_freq_approximations/wavefront_curvature.png}
	\small
	\put(45,50.5){$\vxo$}
	\put(46,65){$\hat{\vk}(\vxo)$}
	\put(36,40){$\rho_1$}
	\put(60,28.5){$\rho_2$}
	\put(60.5,60){$\mathbf{v}_1$}
	\put(28.5,61){$\mathbf{v}_2$}
	\put(71,40){$-\phi^P(\vx) = \text{const}$}
	\end{overpic}
	\end{minipage}
	\hspace{10mm}
	\begin{minipage}[c]{0.4\textwidth}
    \caption{
	 Illustration of the principal radii and principal curvatures of an arbitrary smooth wavefront, satisfying the local dispersion relation.
	 The principal radii are denoted by $\rho_1$ and $\rho_2$, with the corresponding tangent vectors $\mathbf{v}_1$ and $\mathbf{v}_2$ respectively, pointing into the direction of the largest and smallest curvature, being two eigenvectors of the phase function's Hessian.
	 The principal curvatures are given by the reciprocal of the principal radii.
	 In the present treatise non-converging wavefields are discussed with both principal curvatures being non-negative.}
	\label{Fig:HF_appr:local_wave_curvature}
	  \end{minipage}
\end{figure}
The \emph{mean curvature} of the wavefront---generally defined as the divergence of the surface normal \cite{Goldman2005}---is given by the divergence of the normalized local wavenumber vector, or the negative trace of the Hessian:
%
\importanteq{Mean wavefront curvature}{
\overline{\kappa}^P(\vx) =
\frac{\Dx \cdot \hat{\vk}^P(\vx)}{2} = - \frac{\Lx \phi^P(\vx,\omega)}{2 k} 
= - \frac{ 1 }{2  k} \left( \frac{\partial^2}{\partial x^2} + \frac{\partial^2}{\partial y^2} + \frac{\partial^2}{\partial z^2} \right) \phi^{P}(\vx,\omega),
\label{eq:HF_appr:curvature}}
with $k = \frac{\omega}{c}$ being the acoustic wavenumber.
Substitution into the transport equation \eqref{eq:HF_appr:transport_eq}, the divergence of the local wavenumber vector can be expressed as
\begin{equation}
\overline{\kappa}^P(\vx)
= -\left< \hat{\vk}^P(\vx)\cdot \frac{ \Dx A^P(\vx,\omega) }{A^P(\vx,\omega)}\right>
= -\frac{ 1 }{ A^P(\vx,\omega)} \frac{\partial A^P(\vx,\omega)}{\partial \hat{\vk}^P}.
\end{equation}
Hence, the transport equation states that at an arbitrary point the relative amplitude change towards the wavefront's propagation direction is given by the mean curvature.
This fact allows a formal definition for the vergence of the sound field in a mean sense: a field is divergent, if its amplitude decreases into the local propagation direction and convergent, if the intensity is focused towards the propagation direction.

As a strict definition---since the mean and the principal curvatures are related as $\overline{\kappa}^P(\vx)  = \frac{1}{2} \left( \kappa_1^P(\vx)+\kappa_2^P(\vx) \right)$---wavefields may be classified as
\begin{equation}
\label{eq:HF_appr:curvature_cases}
\kappa_1^P(\vx),\kappa_2^P(\vx),\overline{\kappa}^P(\vx) 
\begin{cases*}
> 0  \hspace{5mm} \text{for a locally diverging/non-focused wavefield} \\
= 0  \hspace{5mm} \text{for a plane-wave}  \\
< 0  \hspace{5mm} \text{for a locally converging/focused wavefield.} 
\end{cases*}
\end{equation}
Within this thesis only non-converging wavefields will be discussed.
Finally, one can define the \emph{Gaussian curvature} of the wavefront, given by $\kappa^P_1 \cdot \kappa^P_2$, being only negative, if the wavefront has a saddle point in the point of investigation.

As a summary of the foregoing, Table \ref{tab:HF_appr:local_prop} gives the local wavenumber vector, the local wavefront curvatures and principal radii for frequently used sound field models. 
The Hessian of the Green's function's phase, required for intermediate calculations is given by \eqref{Eq:App:Greens_f_hessian}.
\begin{table}[h!] \begin{center}
\caption{Local wavenumber vector $\vk^P$, local wavefront curvatures $\kappa$ and radii $\rho$ of a 3D point source, an infinite vertical line source (i.e. a 2D point source) and a 3D plane wave.
The phase of a line source is obtained from its high frequency approximation, given by \eqref{eq:HF_approx:2D_vs_3D_GF}}
\renewcommand*{\arraystretch}{2}
\label{tab:HF_appr:local_prop}
\scalebox{1}{ 
\begin{tabular*}{\textwidth}{@{\extracolsep{\fill}} c || c | c | c }
               &\parbox{3cm}{3D point source $\vx = \posvec{3}{x}{y}{z}$}&\parbox{3cm}{2D point source $\vx = \posvec{2}{x}{y}$}&              plane wave                \\ \hline
 $P(\vx,\omega)$           &$\frac{1}{4\pi}\frac{\te^{-\ti k |\vx|}}{\vx}$  &$-\frac{\ti}{4}H_0^{(2)}(k |\vx|)$& $\te^{-\left<\vk \cdot \vx\right>}$    \\
 $\vk^P(\vx)$              &$\frac{k}{|\vx|} \cdot \posvec{3}{x}{y}{z}$     &$\frac{k}{|\vx|} \cdot \posvec{3}{x}{y}{0}$&              $\vk$                     \\
 $\rho_{1,2}^P(\vx)$       &       $|\vx|$                                  &  $|\vx| \hspace{2mm}, \hspace{2mm} \infty$&  			   $\infty$				         \\
 $\kappa_{1,2}^P(\vx)$     &       $\frac{1}{|\vx|}$   						&$\frac{1}{|\vx|}\hspace{2mm},\hspace{2mm} 0$&  			 0				         \\
 $\overline{\kappa}^P(\vx)$& $\frac{1}{|\vx|}$         					    &  $\frac{1}{2 |\vx|}$   					&  			     0         
\end{tabular*}}
\end{center} \end{table}

\subsection{High frequency gradient approximation}
As a further approximation in the high frequency domain, the gradient of an arbitrary sound field may be expressed in a simplified form in terms of the local wavenumber vector.
By applying the product rule of differentiation, the gradient of an arbitrary polar form sound field, described by \eqref{eq:HF_appr:general_sf}, reads as
\begin{multline}
\Dx P(\vx,\omega) = \\ \left(  \frac{\Dx A^P(\vx,\omega)}{A^P(\vx,\omega)} + \ti \Dx \phi^P(\vx,\omega) \right) P(\vx,\omega) =  \left(  \frac{\Dx A^P(\vx,\omega)}{A^P(\vx,\omega)} - \ti \vk^P(\vx) \right) P(\vx,\omega).
\end{multline}
%In the frequency domain of interest the sound field's phase function varies rapidly compared to the envelope of the oscillation, which must hold both to apply the Kichhoff approximation and the stationary phase approximation in the following.
In the high frequency region $|\vk^P(\vx)| \approx \left( \frac{\omega}{c} \right) \gg \left| \frac{ \Dx A^P(\vx,\omega)}{A^P(\vx,\omega)} \right|$ holds, and the gradient can be approximated as
\importanteq{High freq. gradient appr.}{
\Dx P(\vx,\omega) \approx - \ti \vk^P(\vx) P(\vx,\omega).
\label{eq:HF_approx:gradient_appr}
}

\vspace{3mm}
For an interpretation of the local wavenumber concept and the high frequency gradient approximation, the first order Taylor expansion of the phase function may be expressed around an arbitrary point $\vxo$, reading
\begin{equation}
\phi^P(\vx,\omega) \approx \phi^P(\vxo,\omega) + \left< (\vx-\vxo) \cdot \Dx \phi^P(\vxo,\omega) \right>.
\end{equation}
By substitution into \eqref{eq:HF_appr:general_sf} with a slowly varying amplitude function---i.e. $A^P(\vx)$ is approximated by the first order Taylor expansion coefficient---in the proximity of $\vxo$ the sound field is approximated by
\begin{equation}
\label{Eq:HF_approx:plane_wave_approximation}
P(\vx,\omega) \approx P(\vxo,\omega) \te^{-\ti  \left< \vk^P(\vxo) \cdot \left( \vx - \vxo \right) \right>}.
\end{equation}
Therefore, each point of a sound field is approximated as a local elementary plane wave with the wavenumber and angular frequency given by $\vk^P(\vx)$ and $\omega$, respectively.
Furthermore, expressing the gradient of the local plane wave representation \eqref{Eq:HF_approx:plane_wave_approximation} leads to the high frequency gradient approximation \eqref{eq:HF_approx:gradient_appr}, which is obviously the gradient of locally plane wavefields.

\subsection*{Application example: Stereophony}
\label{Sec:stereophony}

As an application example for the local wavenumber vector concept, the resultant sound field of two 3D point sources is investigated, modeling a stereo loudspeaker setup.

\begin{figure}
  \begin{minipage}[c]{0.45\textwidth}
  \hspace{1cm}
	\begin{overpic}[width = \textwidth ]{Figures/High_freq_approximations/stereo_geometry.png}
	\small
	\put(97,7){$x$}
	\put(49,100){$y$}
	\put(93,73){$\vx_1$}
	\put(-3.25,73){$\vx_2$}
	\put(87,7){$x_1$}
	\put(40.5,75){$y_1$}
	\put(49.5,27){$\phi_0$}
	\put(41,40){$\phi_p$}
	\put(52,2){$\vk^P(\vx)$}
	\put(18,94){\parbox{.5in}{phantom source}}
	\end{overpic}  \end{minipage}\hfill
	\begin{minipage}[c]{0.4\textwidth}
    \caption{
       General two-channel stereophonic geometry consisting of two point sources, positioned symmetrically to the $y$-axis, termed the \emph{stereo axis}.
       The \emph{aperture angle} is usually set to $2\phi_0 = 60^{\circ}$ and the listener's position is at the origin \cite{Rumsey2001}.
       Simple amplitude panning techniques apply intensity difference between the loudspeaker pair, so that the listener perceives the illusion of a single sound source, termed the \emph{phantom source}, positioned along the \emph{active arc} between the two loudspeakers.
    } \label{Fig:HF_appr:stereophony_geometry}
  \end{minipage}
\end{figure}
%
The point sources are positioned at $\vx_1 = \posvec{3}{x_1}{y_1}{z_1 = 0}$ and $\vx_2 = \posvec{3}{-x_1}{y_1}{z_1 = 0}$ in a standard stereo ensemble, with the stereo axis being the $y$-axis. 
The geometry is illustrated in Figure \ref{Fig:HF_appr:stereophony_geometry} \cite{SpringerHandbook2008}.
In the case of \emph{amplitude panning}, the sources are driven in-phase, with only their frequency independent amplitude factor $A_1$, $A_2$ differing.
The resultant sound field reads as
\begin{equation}
P(\vx,\omega) = 
\frac{A_1}{4\pi}\frac{\te^{-\ti \frac{\omega}{c}|\vx - \vx_1|} }{|\vx - \vx_1|} + 
\frac{A_2}{4\pi}\frac{\te^{-\ti \frac{\omega}{c}|\vx - \vx_2|} }{|\vx - \vx_2|}.
\end{equation}

\begin{figure}[]
	\small
	\centering
	\begin{overpic}[width = 1\columnwidth ]{Figures/High_freq_approximations/stereophony.png}
	\put(2,2){(a)}
	\put(62,2){(b)}
	\end{overpic}
	\caption{
Sound field generated in a typical stereo setup. 
The point sources are positioned with a base angle of $\phi_0 = 30^\circ$, with their distances from the origin being $R_0 = 2.5~\mathrm{m}$.
The gain factors $A_1, A_2$ were selected, so that the angle of the local wavenumber vector at the origin would equal to $\phi_p = 10^\circ$.
In Figure (a) contour lines indicate isochronous surfaces with the normalized local wavenumber vector displayed along the stereo axis.
Figure (b) shows the normalized wavenumber components along $x=0$.
Note that due to interference phenomena the amplitude distribution changes rapidly, and as a consequence the local dispersion relation \eqref{eq:HF_appr:local_dispersion} does not hold in particular positions:
The length of the wavenumber vector decreases between the sources where standing waves occur, and increases to infinity on the parts where the amplitude vanishes and the phase changes rapidly due to destructive interference.
}
\label{Fig:HF_appr:stereophony_wave_number}
\end{figure}

Generally, for an arbitrary receiver position $\vx$, the phase of the resultant field and the local wavenumber vector can be only described by a complex formula, as it is derived in \ref{App:stereophony}.
From the aspect of stereophonic applications, the investigation of the local propagation direction on the stereo axis is sufficient, since the listener's position is assumed to be the origin.
On the stereo axis, i.e. along the $y$-axis, the local wavenumber vector can be simplified to
\begin{equation}
\vk^P(0,y,0) = - \left. \Dx \phi^P(\vx,\omega) \right|_{x=0,z=0} =
k \begin{bmatrix} \frac{A_1 - A_2  }{ A_1 + A_2  } \frac{x-x_1}{|\vx-\vx_1|}  \\[.7em] \frac{y-y_1}{|\vx-\vx_1|} \\[.7em] \frac{z-z_1}{|\vx-\vx_1|}= 0 \\[0.5em]    \end{bmatrix}. 
\label{Eq:HF_approx:stereo_local_wavenumber}
\end{equation}
Hence, the local wavenumber vector can be steered along the stereo axis by applying appropriate frequency independent gains to the point source pair, in order to control the $k_x^P$ component.
The local wavenumber vector for a general stereophonic scenario is illustrated in Figure \ref{Fig:HF_appr:stereophony_wave_number}.
Assuming that the local wavenumber vector determines the apparent position of the phantom source, with the appropriate choice of the source gains the desired phantom source direction can be set.

%
Assuming that the position of the phantom source or the target propagation direction angle measured from the stereo axis is prescribed---denoted by $\phi_p$ in Figure \eqref{Fig:HF_appr:stereophony_geometry}---the required gain factors may be expressed from \eqref{Eq:HF_approx:stereo_local_wavenumber}.
The local propagation angle of the resultant field at the origin $\mathbf{0}$ can be written in terms of the local wavenumber components as
\begin{equation}
\tan \phi_p = \frac{k_x^P(\mathbf{0})}{k_y^P(\mathbf{0})} = \frac{A_1-A_2}{A_1+A_2}\frac{x_1}{y_1}.
\end{equation}
Exploiting that $\tan \phi_0 = \frac{x_1}{y_1}$ leads to the the formula
\begin{equation}
\frac{A_1 - A_2}{A_1 + A_2} = \frac{\tan \phi_p}{\tan \phi_0},
\end{equation}
which is identical to the well-known \emph{tangent law} of stereophony \cite{Pulkki1997, Pulkki2001a, Pulkki2001:phd, SpringerHandbookSpeech2008}, originally derived from a different consideration \cite{Bennett1985}.
The tangent law therefore ensures the matching of the local propagation directions of the target field and the reproduced wavefronts in the proximity of the listener's position, i.e. over the sweet spot.

Obviously, the tangent law expresses merely the relationship between $A_1$ and $A_2$, the exact value of the gain factors can be calculated by applying some type of normalizing strategy \cite{Moore1990}.
A frequently used strategy is to keep the intensity of the reproduced field at a constant value, by requiring $A_1^2 + A_2^2 = \text{constant}$.
Alternatively, it may be exploited that the amplitude of the resultant field on the stereo axis equals to $\frac{1}{4\pi}\frac{A_1+A_2}{|\vx-\vx_1|}$ (as given by \eqref{Eq:AppB:stereo_amplitude}), in order to match the amplitude of the reproduced field to that of the phantom point source.

\section{The Kirchhoff approximation}
%
The Kirchhoff approximation is an important high frequency asymptotic approximation of the Kirchhoff-Helmholtz integral.
Based on the equivalent scattering interpretation the simple source formulation may be simplified in the high frequency region using the \emph{Kirchhoff/Physical optics approximation}, applied frequently to estimate scattering from random surfaces \cite{Voronich1999, Tsang2000}.
In order to estimate the scattered field---and its normal derivative on the scatterer surface---two approximations are applied:
\begin{figure} 
	\centering
	\begin{overpic}[width = .95\columnwidth ]{Figures/High_freq_approximations/Kirchhoff_approximation.png}
	\small
	\put(0, 0){(a)}
	\put(53,0){(b)}
	\put(-2.5,23){$\vk^P(\vxo)$}
	\put(-3,3.5){$\vxs$}
	\put(8,13){illuminated region}
	\put(27,29){shadow region}
	%	
	\put(58.25,3){$\vxs$}
	\put(71.5,12){$\vno$}
	\put(77,17.5){$\vni$}
	\put(84.5,15){$\vk^P(\vxo)$}
	\put(77.5,5){$\vk^{P_\mathrm{s}}(\vxo)$}
	\put(92.5,2.5){\parbox{.5in}{tangent plane}}
	\end{overpic}
\caption{Illustration of the geometrical optics approximation (a) and the tangent plane approximation (b)}
	\label{Fig:Theory:KH_approximation_a}
\end{figure}

\begin{itemize}
%
\item According to \emph{geometrical optics} or \emph{ray acoustics}, the scatterer surface can be divided into an \emph{illuminated} and a \emph{shadow region}: only those parts of the scatterer surface contribute to the scattered field that are directly illuminated by the primary source, i.e. where the local propagation directions of the incident and the reflected field---determined locally by the scatterer surface's normal---coincide \cite{doi:10.1121/1.1916538}.
In the field of high frequency boundary element method this is termed as \emph{determining the visible elements} on the boundary \cite{Herrin2003}.
Mathematically, this requirement is formulated as weighting the integral, describing the scattered field, by the windowing function
\begin{equation}
w(\vxo) = \begin{cases}
                        1, \hspace{3mm} \forall \hspace{3mm} \langle \vk^P(\vxo) \cdot \vni(\vxo) \rangle > 0 \\
                        0  \hspace{3mm} \text{elsewhere},
                    \end{cases}
\label{eq:theory:windowing_function}
\end{equation}
where $\mathbf{k}^P(\vxo)$ denotes the local wavenumber vector of the incident sound field at $\vxo$ and $ \vni(\vxo)$ is the inward normal of the surface element. 
For an illustration see Figure \ref{Fig:Theory:KH_approximation_a} (a).
%
This windowing means the neglection of both diffracting waves around the scattering object (as well as the so-called \emph{creeeping rays} \cite{Bleistein1984}) and reflections from one part of the scatterer to an other \cite{Pignier2015}. 
Due to this latter restriction, the Kirchhoff approximation may be applied only to convex surfaces, free of secondary reflections.
%
\item As a second simplification, the \emph{tangent plane approximation} is applied on the illuminated region.
It is supposed that there exists a local relation between the incident and the scattered field at each point on the surface.
By assuming that the incident wave is reflected locally obeying the Snell's law \cite{Voronich2007}---its amplitude changes proportionally to the local \emph{reflection index}, with the angle of incidence equaling the angle of reflection measured from the local normal---the following relations are yielded for a sound soft scatterer: \cite{Bleistein1984, Bleistein2000, Pike2002}
\begin{equation}
P_{\mathrm{s}}(\vxo,\omega) = -P(\vxo,\omega), \hspace{5mm} \frac{\partial}{\partial \vni} P_{\mathrm{s}}(\vxo,\omega) = -\frac{\partial}{\partial \vno} P(\vxo,\omega), \hspace{5mm} \vxo \in \dO,
\label{Eq:SFS_theory:tangent_plane}
\end{equation}
where $P(\vx,\omega)$ is the incident field and $P_{\mathrm{s}}(\vx,\omega)$ is the scattered field.
The approximation therefore calculates the reflected wavefield by modeling each point of the scatterer as an infinite tangential plane. 
For low-frequencies and non-smooth boundaries the surface can not be considered locally planar, introducing further artifacts.\footnote{In order to overcome these limitations several curvature correctional and iterative approaches exist \cite{Elfouhaily2004}.}

%
\end{itemize}

\begin{figure}
	\centering
	\begin{overpic}[width = 1\columnwidth]{Figures/High_freq_approximations/KH_approx.png}
	\small
	\put(2, 38){(a)}
	\put(52,38){(b)}
	\put(20, 0){(c)}
	\end{overpic}
\caption{Illustration of the Kirchhoff approximation in a 2D problem ($\Omega \subset \mathbb{R}^2$), applied for the calculation of the scattering of a 2D point source, positioned at $\vxs = \posvec{2}{0.4}{2.5}$, oscillating at $f_0 = 1~\mathrm{kHz}$.
Figure (a) depicts the numerical evaluation of the Kirchhoff approximation \eqref{Eq:SFS_theory:Kirchhoff_appr}.
Figure (b) describes the total field of the point source in the presence of a sound soft scattering object.
Figure (c) shows the absolute value of the total field on a logarithmic scale.
Inside the enclosure the sound field should be identically zero if no approximations were applied, hence in these region the non-zero field indicates the error of the Kirchhoff approximation.
In Figure (a) the illuminated/active part of the scatterer contour is denoted by solid black line, whilst the shadow region is denoted by dotted line.}
	\label{Fig:Theory:KH_approximation}
\end{figure}
%
These approximations can be utilized in order to approximate the single source formulation. 
According to the equivalent scattering interpretation, the external field is given by the scattered sound field as $P_\mathrm{e}(\vx) = -P_{\mathrm{s}}(\vx)$.
Reformulating \eqref{Eq:SFS_theory:tangent_plane} merely in terms of the inward normal vector and applying the geometrical optics windowing function, one obtains the Kirchhoff approximation of the simple source formulation 
\importanteq{Kirchhoff approximation}{
\oint_{\dO} 
- 2w(\vxo)\,
\frac{\partial P(\vxo,\omega)}{\partial \vni} \,
G(\vx-\vxo,\omega) \,
\td \dO ( \vxo) 
\approx
\begin{cases} 
P(\vx,\omega)     & \hspace{1mm} \forall \hspace{5mm}   \vx \in \Oi \\
P=-P_{\mathrm{s}}  & \hspace{1mm} \forall \hspace{5mm}         \vx \in \dO  \\
-P_{\mathrm{s}}(\vx,\omega)    & \hspace{1mm} \forall \hspace{5mm}  \vx \in \Oe.
\end{cases}
\label{Eq:SFS_theory:Kirchhoff_appr}
}
The integral gives a fair approximation for smooth, convex surfaces in the high frequency and farfield region, where the wavelength and the wavefront curvature are significantly smaller, than the dimensions of the scattering object.\footnote{According to \cite[Eq.(2.7.12)]{Blenstein1975} the approximation holds, when $k \cdot \rho_{1,2} \gg 1$, where $\rho_{1,2}$ are the local principal radii of the curved scatterer and $k$ is the wavenumber.}
The Kirchhoff approximation of the 2D example presented in Section \ref{Sec:SimpleSourceFormulation} is illustrated in Figure \ref{Fig:Theory:KH_approximation}. 
The lack of diffracted waves around the enclosure gives rise to artifacts on parts of the space where the local propagation direction of the incident field is nearly parallel with the scatterer contour.

%\newpage
\section{The stationary phase approximation}

This section introduces a basic tool of asymptotic analysis, the \emph{stationary phase approximation (SPA)}, being of central importance in the present thesis.
It allows the evaluation of integrals of complex functions by assuming that the greatest contribution stems from critical points in the integral path.
In the following chapters the SPA allows the extraction of asymptotic, local solutions from the global ones for radiation and reproduction problems, written in terms of either boundary or spectral integrals.

For the sake of brevity, the following notation convention is used hereinafter, as given also in the nomenclature:
Given an $n$-dimensional function $f(\vx)$ with $\vx = \posvec{4}{x_1}{x_2}{...}{x_i}$, $i = 1,2,...,n$, the first and second partial derivatives with respect to the $i$-th and $j$-th coordinates $x_i$, $x_j$ evaluated at the position $\vx^*$ are denoted as
\begin{equation}
\left. \frac{\partial}{\partial x_i} f(\vx) \right|_{\vx = \vx^*} = f'_{x_i}(\vx^*), \hspace{1cm}
\left. \frac{\partial^2}{\partial x_i \partial x_j} f(\vx) \right|_{\vx = \vx^*} = f''_{x_i x_j}(\vx^*).
\end{equation}

\subsection{The integral approximation}
%
Generally speaking, the SPA yields the approximate value for the integrals of complex valued functions, written in the general polar form as
\begin{equation}
\label{Eq:SPAintegral_1d_nd}
I_{1\mathrm{D}} = \int\limits_{-\infty}^{\infty} A(x) \, \te^{\ti \phi(x)} \, \td x,
\hspace{20mm} 
I_{n\mathrm{D}} = \int\limits_{-\infty}^{\infty} A(\vx) \, \te^{\ti \phi(\vx)} \, \td \vx
\end{equation}
in one and $n$ dimensions respectively, when $\te^{\ti \phi(\vx)}$ is highly oscillating and $A(\vx)$ is comparably slowly varying.

%\paragraph{1D SPA:} 
For the SPA of the 1D integral, a rigorous derivation, based on integration by parts, is given in \cite{Blenstein1975, Bleistein1984, Williams1999}.
Less formally, the method relies on the second order truncated Taylor series of the exponent around the \emph{stationary point} $x^*$, defined as the point in the integration path, satisfying
\begin{equation}
\phi'_x(x^*) = 0, \hspace{1cm} \text{and} \hspace{1cm} \phi''_{xx}(x^*) \neq 0.
\end{equation}
The Taylor series around the stationary point reads as
\begin{equation}
\phi(x) \approx \phi(x^*) + \frac{1}{2}\phi''_{xx}(x^*)(x-x^*)^2.
\end{equation}
%
Supposing that the amplitude $A(x)$ is a slowly varying smooth function compared to $\phi(x)$, it is assumed that where the phase varies (i.e.\ $\phi'_x(x) \neq 0$), the integral of rapid oscillation cancels out, thus the greatest contribution to the total integral comes from the immediate surroundings of the stationary point.
Moreover, in the proximity of the stationary point $A(x)$ can be regarded to be constant, with the value $A(x^*)$.
%
With these considerations the integral becomes
\begin{align}
I_{1D} \approx A(x^*)\,\te^{+\ti\phi(x^*)} 
\int\limits_{-\infty}^{\infty} \te^{+\ti \frac{1}{2}\phi''_{xx}(x^*)(x-x^*)^2} \, \td x.
\end{align}
The remaining integral can be evaluated analytically, resulting in the stationary phase approximation of \eqref{Eq:SPAintegral_1d_nd} \cite[Ch.\ 2.8]{Blenstein1975}, reading as
\importanteq{1D stationary phase approximation}{
\label{Eq:SPAResult}
I_{1D} \approx \sqrt{\frac{2\pi}{| \phi''_{xx}(x^*) |}} F(x^*) \, \te^{+\ti \phi(x^*) + \ti \frac{\pi}{4}\,\mathrm{sgn}\left(  \phi''_{xx}(x^*) \right)}.
}

%\paragraph{Multidimensional SPA:} 
Similarly, in higher dimensions a \emph{simple stationary point} is defined as
\begin{equation}
\label{Eq:ndim_stat_point}
\left.
\Dx \phi(\vx)\right|_{\vx = \vx^*} = 0,
\end{equation}
where 
\begin{equation}
\det \mH(\vx^*) \neq 0,
\hspace{5mm} 
H_{ij}(\vx^*) = \left. \left[
\frac{\partial^2 \phi(\vx)}{\partial x_i \partial x_j} 
\right] \right|_{\vx = \vx^*},
\hspace{5mm}
i,j = 1,2,...,n
\end{equation}
holds, with $\mH$ being the Hessian matrix of the phase function.
The multidimensional formula for the integral value reads as
\importanteq{Multi-dimensional SPA}{
\label{Eq:SPAResult_nd}
I_{nD} \approx \sqrt{\frac{(2\pi)^n}{|\det \mH(\vx^*)|}} F(\vx^*) \te^{\ti \phi(\vx^*) + \ti \frac{\pi}{4}\,\mathrm{sgn}\left( \mH(\vx^*) \right)},
}
where $\mathrm{sgn}\left( \mH(\vx^*) \right)$ is the signature of the Hessian: the number of positive eigenvalues minus the number of negative eigenvalues \cite{Bleistein2000}.

In the following, the physical interpretation of the SPA is discussed when applied to boundary and spectral integrals of sound fields and simple examples are given for its application.
The conclusions of the presented examples will be further utilized in the following chapters.

\subsection{Asymptotic approximation of boundary integrals}
\label{Sec:HS_approx:SPA_for_Rayleigh}
First, the physical interpretation of the stationary position is discussed for the case when the SPA is applied to boundary integrals, for the sake of simplicity, through the example of the Rayleigh I integral.
%

Assume that the Rayleigh integral describes an arbitrary sound field at $y>y_0$ in terms of a boundary integral along the plane $\vxo = \posvec{3}{x_0}{y = y_0}{z_0}$, according to \eqref{Eq:Theory:RayleighI}.
It is supposed that all sources of sound are located behind the Rayleigh plane in the half space $y<y_0$, generating a non-converging wavefront.
For the application of the SPA high frequency conditions are standard prerequisites in order to ensure a highly oscillating exponential.
Therefore, the gradient can be expressed by its high frequency approximation \eqref{eq:HF_approx:gradient_appr}, resulting in the high frequency Rayleigh integral
\begin{equation}
\label{eq:HF_approx:HighF_Rayleigh}
P(\vx,\omega) = 2 \iint_{-\infty}^{\infty} \ti k_y^P(\vxo) P(\vxo,\omega) \, G(\vx-\vxo,\omega) \, \td x_0 \, \td z_0.
\end{equation}
The goal is to evaluate the Rayleigh integral for a given receiver position $\vx$, by applying the SPA.
With the involved functions written in polar form, the integral reads as
\begin{equation}
P(\vx,\omega) = 2 \iint_{-\infty}^{\infty} k_y^P(\vxo) A^P(\vxo,\omega ) A^G(\vx-\vxo,\omega) \te^{\ti \left( \phi^P(\vxo,\omega) + \phi^G(\vx-\vxo,\omega) + \frac{\pi}{2} \right)} \, \td x_0 \, \td z_0.
\end{equation}
According to \eqref{Eq:ndim_stat_point} the stationary position for the integral is found where the phase gradient vanishes.
Exploiting that the constant phase shift $+\frac{\pi}{2}$ vanishes due to differentiation, the stationary position $\vxo^*(\vx)$ for a given receiver position $\vx$ is found where
\begin{equation}
\left.
\begin{bmatrix} \frac{\partial}{\partial x_0} \\[.7em] \frac{\partial}{\partial z_0} \\[0.5em]  \end{bmatrix}
\phi^P(\vxo,\omega)
\right|_{\vxo=\vxo^*(\vx)}
= 
\left.
-\begin{bmatrix} \frac{\partial}{\partial x_0} \\[.7em] \frac{\partial}{\partial z_0} \\[0.5em]  \end{bmatrix}
\phi^G(\vx-\vxo,\omega) 
\right|_{\vxo=\vxo^*(\vx)}
\end{equation}
is satisfied.
By the definition \eqref{eq:local_wn_vec_def}, the derivatives describe the corresponding components of the local wavenumber vector.
Since two components completely determine the local wavenumber vector, therefore in the stationary position
\importantalign{Stationary position of boundary integrals}{
k^P_x(\vxo^*(\vx)) 
&= 
k^G_x(\vx-\vxo^*(\vx))
\\ \nonumber
k^P_z(\vxo^*(\vx))
&=
k^G_z(\vx-\vxo^*(\vx))
\\ \nonumber
\vk^P(\vxo^*(\vx)) &= \vk^G(\vx-\vxo^*(\vx))= - \vk^G(\vxo^*(\vx)-\vx)
}%\end{align}
holds. 
In the right-hand side the chain rule\footnote{Since the derivative is taken w.r.t. $\vxo$, according to the chain rule, the sign of the Green's function's derivative is reversed, resulting in $\frac{\partial}{\partial x_0} \phi^G(\vx-\vxo,\omega) = -\phi^{G'}_{x_0} (\vx-\vxo,\omega)$ and $\frac{\partial}{\partial z_0} \phi^G(\vx-\vxo,\omega) = -\phi^{G'}_{z_0} (\vx-\vxo,\omega)$.} and the reciprocity of the Green's function was exploited.
%
\begin{figure}
\small
  \begin{minipage}[c]{0.58\textwidth}
	\small
	\begin{overpic}[width = \textwidth ]{Figures/High_freq_approximations/rayleigh_stat_point.png}
	\put(96,30){$x$}
	\put(15,80){$y$}
	\put(78.5,60){$\vx$}
	\put(62,29.5){$\vxo^*(\vx)$}
	\put(70,42){$\vk^P(\vxo^*(\vx))$}
	\put(58,20){$\vk^G(\vxo^*(\vx) - \vx)$}
	\end{overpic}  \end{minipage}\hfill
	\begin{minipage}[c]{0.4\textwidth} \hspace{2mm}
    \caption{
       2D Geometry for the physical interpretation of the stationary position for the Rayleigh integral.
       The stationary position is found along the integral surface/contour where the local propagation direction---and the local wavenumber vector---of the described wavefield and the spherical field of a point source, positioned at $\vx$ coincide.
       Equivalently, it means that the local propagation direction of the described field at $\vxo^*(\vx)$ equals with that of the Green's function ,positioned at $\vxo^*(\vx)$, measured at the receiver position $\vx$.
       } 
       \label{Fig:HF_appr:rayleigh_stat_point}
  \end{minipage}
\end{figure}
%

Hence, the SPA 'compares' the propagation direction/wavefronts of the described field and the Green's function along the integral path.
The stationary position for a given receiver position is given by that point $\vxo^*(\vx)$ where the local propagation direction of the described wavefield is opposite to that of a monopole field centered at the receiver position $\vx$.
Obviously, by translating back the 3D Green's function into $\vxo^*(\vx)$, its wavenumber vector at $\vx$ will coincide with the described field's wavenumber vector. 
In other words, since the Rayleigh integral describes a sound field as the resultant field of a planar distribution of point sources, for a given receiver point that point source will have the greatest contribution, that's sound field/wavefront propagates into the same direction as the primary sound field/wavefront.

This interpretation is illustrated in Figure \ref{Fig:HF_appr:rayleigh_stat_point}, with the example of a 2D point source, described by the 2D Rayleigh integral.
For the case of a point source at $\vxs$ the stationary position is found at the intersection of vector $\vx-\vxs$ and the integration path.
In a 3D example, if the primary field is a spherical one, the stationary point is found at the intersection of the Rayleigh plane and the vector pointing from the source into the evaluation position.

\subsection*{Application example \#1: Asymptotic evaluation of the Rayleigh integral}
\label{Sec:HF:RayleighSPA}
As an application example for the SPA, the evaluation of the Rayleigh integral around the stationary point is investigated in further details.
As a result it is described, how the local properties (its amplitude and phase) of wavefronts change over the propagation path/ray path.

The stationary point was found on the Rayleigh plane where the local propagation direction of the primary sound field coincides (with a negative sign) with the spherical wavefront of the Green's function, positioned at the receiver point. 
In order to evaluate integral \eqref{eq:HF_approx:HighF_Rayleigh} around its stationary point according to \eqref{Eq:SPAResult_nd} (with $n=2$), the signature and the determinant of the Hessian in the stationary position is required.
In the present geometry, the Hessian is given by the sum of the individual Hessians:
\begin{equation}
\mH^{P \cdot G}(\vxo) = \mH^{P}(\vxo) + \mH^{G}(\vx - \vxo)  =
\begin{bmatrix} 
\frac{\partial^2}{\partial x_0^2} & \frac{\partial^2}{\partial x_0 \partial z_0} \\[.7em]
\frac{\partial^2}{\partial x_0 \partial z_0} & \frac{\partial^2}{\partial z_0^2}\\[0.5em] \end{bmatrix} 
\left( \phi^P(\vxo,\omega) + \phi^G(\vx-\vxo,\omega)  \right).
\label{eq:HF_appr:Hessian}
\end{equation}
In the stationary position the primary wavefront and the Green's function wavefront are tangential (as it can be seen in Figure \ref{Fig:HF_appr:rayleigh_stat_point}).
Due to the spherical nature of the latter one, in the stationary point the 3 eigenvectors of the above individual Hessians can be chosen to coincide, therefore their principal curvatures (eigenvalues) are additive.
Thus, the resultant Hessian \eqref{eq:HF_appr:Hessian} can be expressed in terms of the principal curvatures of the primary sound field $\kappa_1^P, \kappa_2^P$ and the Green's function $\kappa_1^G, \kappa_2^G$ as
\begin{equation}
\scriptstyle
\mH^{P \cdot G}(\vxo^*(\vx)) = -k
\mathbf{V}
\begin{bmatrix} 
\kappa_1^P(\vxo^*(\vx)) + \kappa_1^G(\vx-\vxo^*(\vx)) & 0 \\[.3em]
0 & \kappa_2^P(\vxo^*(\vx)) + \kappa_2^G(\vx-\vxo^*(\vx)) \\[.5em] \end{bmatrix}
\mathbf{V}^{\mathrm{T}},
\end{equation}
with $\scriptstyle \mathbf{V} = \begin{bmatrix} 
v_{1 x} & v_{2 x} \\[.1em]
v_{1 z} & v_{2 z}\\[.3em] \end{bmatrix}$ being a matrix, constructed from the $x,z$-components of the eigenvectors/principal directions, corresponding to $\kappa_1^P$ and $\kappa_2^P$, as shown in Figure \ref{Fig:HF_appr:local_wave_curvature}.
For a more detailed description refer to Appendix \ref{App:Hessian}.
The determinant of the Hessian reads as
\begin{multline}
\label{Eq:HF_approx:H_det_Rayleigh}
\mathrm{det} \, \mH^{P \cdot G}(\vxo^*(\vx)) = 
\left(\kappa_1^P(\vxo^*(\vx)) + \kappa_1^G(\vx-\vxo^*(\vx))\right)
\left(\kappa_2^P(\vxo^*(\vx)) + \kappa_2^G(\vx-\vxo^*(\vx))\right)
\cdot \\ \cdot
k^2 \underbrace{\left( v_{1 x}v_{2 z}-v_{2 x}v_{1 z} \right)^2}_{\hat{k}_y^P(\vxo^*(\vx))^2},
\end{multline}
where the underbraced part is the $y$-coordinate of a unit vector, being perpendicular to $\mathbf{v}_1$ and $\mathbf{v}_2$, i.e. of the normalized local wavenumber vector.
By taking into consideration that for a divergent field both curvatures of the wavefront are positive and the signature of the Hessian equals (-2), substitution into \eqref{Eq:SPAResult_nd} yields the asymptotic Rayleigh integral, reading
\begin{align}
\label{eq:HF_approx:asymptotic_rayleigh}
P(\vx,\omega) &= 
4\pi \frac{P(\vxo^*(\vx),\omega) G(\vx-\vxo^*(\vx),\omega)}
{\sqrt{\left(\kappa_1^P(\vxo^*(\vx)) + \kappa_1^G(\vx-\vxo^*(\vx))\right)}
\sqrt{\left(\kappa_2^P(\vxo^*(\vx)) + \kappa_2^G(\vx-\vxo^*(\vx))\right)}}\\
&=
\textstyle \sqrt{\frac{\rho_1^P(\vxo^*(\vx)) \cdot \rho_1^G(\vx-\vxo^*(\vx))}{\rho_1^P(\vxo^*(\vx)) + \rho_1^G(\vx-\vxo^*(\vx))}}
\sqrt{\frac{\rho_2^P(\vxo^*(\vx)) \cdot \rho_2^G(\vx-\vxo^*(\vx))}{\rho_2^P(\vxo^*(\vx)) + \rho_2^G(\vx-\vxo^*(\vx))}}
P(\vxo^*(\vx),\omega) G(\vx-\vxo^*(\vx),\omega), \nonumber
\end{align}
written in terms both of the principal curvatures and radii.

Substituting the exact formulation of the 3D Green's function and by exploiting that according to \eqref{eq:app:propagated_radii}, the principal radii increase proportional with the Euclidean distance along the local propagation direction, i.e. $\rho_i^P(\vx) = \rho_i^P(\vxo^*(\vx)) + \rho_i^G(\vx-\vxo^*(\vx))$ holds, the asymptotic formula takes the form
\importanteq{Asymptotic Rayleigh integral}{
\label{eq:HF_approx:ray_propagation}
P(\vx,\omega) =
\underbrace{\sqrt{ \frac{ \rho_1^P(\vxo^*(\vx)) \cdot \rho_2^P(\vxo^*(\vx)) }{ \rho^P_1(\vx) \cdot \rho^P_2(\vx) } }}_
{\substack{\text{amplitude change} \\ \text{over propagation}} }
\underbrace{\te^{-\ti \omega \frac{|\vx-\vxo^*(\vx)| }{c} }}_ 
{\substack{\text{phase change} \\ \text{over propagation}} }
P(\vxo^*(\vx),\omega),
}
where $\rho^P_1 \cdot \rho^P_2$ is the reciprocal of the Gaussian curvature of the wavefront.
Thus, in a ray tracing manner the wavefield is approximated locally, based on its value at the stationary position: 
the numerator of the amplitude factor approximates the pressure field's amplitude in the source position, attenuated by the denominator--describing the attenuation factor for the source-to-receiver distance---while the simple phase shift term corresponds to the propagation time delay.
The equation reflects the fact that \emph{the intensity of a 3D wavefield is proportional to the Gaussian curvature of the wavefront}, being a well-known fact in the field of optics \cite[Sec. 3.1]{Born1970}, \cite[Sec. 1.3]{Bouche1997}.
Similarly, a 2D wavefront's amplitude attenuates proportionally to the only non-zero curvature, given by $\sim \sqrt{\rho^P}$, which fact can be deduced from the SPA of the 2D Rayleigh integral.

Note that since points $\vx$ and $\vxo^*(\vx)$ are related by the local wavenumber vector, therefore equation \eqref{eq:HF_approx:ray_propagation} generally describes how the field's amplitude and phase change along the direction of the local wavenumber vector, i.e. along the path of propagation.
Departing from the Rayleigh plane concept, as a more general statement, any propagating pressure field may be approximated along the propagation path as
\begin{equation}
P(\vx+\td x \cdot \hat{\vk}^P(\vx),\omega) =
\sqrt{ \frac{ \rho_1^P(\vx) \rho_2^P(\vx) }{ \left(\rho_1^P(\vx) + \td x \right) \left(\rho_2^P(\vx) + \td x\right) } }
\, \te^{-\ti \omega \frac{\td x}{c} } \, P(\vx,\omega)
\end{equation}
as long as high frequency/farfield assumptions hold.\footnote{From the above equation, the relative amplitude change can be expressed by applying the L'Hospital's rule, reading as
%\begin{equation}
$
\frac{\left< \hat{\vk}^P(\vx) \cdot \Dx A^P(\vx,\omega) \right>}{A^P(\vx,\omega)} = \lim_{\td x \rightarrow 0} 
\frac{\sqrt{ \frac{ \rho_1^P(\vx) \rho_2^P(\vx) }{ \left(\rho_1^P(\vx) + \td x \right) \left(\rho_2^P(\vx) + \td x\right) } }-1}{\td x}
= 
-\frac{1}{2}\frac{\rho_1^P(\vx) + \rho_2^P(\vx)}{\rho_1^P(\vx) \rho_2^P(\vx)} = -\overline{\kappa}^P(\vx),
$%\end{equation}
which result is in agreement with the definition of the mean curvature, originally obtained from the transport equation \eqref{eq:HF_appr:curvature}.}

\subsection*{Application example \#2: The Kirchhoff approximation}
%
As a second application example for the SPA of boundary integrals, an alternative derivation of the Kirchhoff approximation is presented, obtained directly from the Kirchhoff-Helmholtz integral.
Suppose that an interior radiation problem is described by the KHIE inside an enclosure $\Omega$, bounded by $\dO$. 
The field is given by
\begin{equation}
P(\vx,\omega) = 
\oint_{\dO} - \left( 
\frac{\partial P(\vxo,\omega)}{\partial \vni} G(\vx-\vxo,\omega)
-
P(\vxo,\omega)  \frac{\partial G(\vx-\vxo,\omega)}{\partial\vni} 
\right)  \td \dO( \vxo).
\end{equation}
Assuming high frequency conditions, both the sound field and the Green's function normal derivatives may be approximated using the high frequency gradient approximation, resulting in
\begin{equation}
P(\vx,\omega) = 
\oint_{\dO} 
\left( \ti k_{\mathrm{n}}^P(\vxo) + \ti k_{\mathrm{n}}^G(\vx-\vxo) \right) %FS: correct sign now, ok with PhD ch. 2.2.2
\, P(\vxo,\omega) \, G(\vx-\vxo,\omega) \, \td \dO( \vxo).
\end{equation}
%
\begin{figure}
  \begin{minipage}[c]{0.775\textwidth}
	\begin{overpic}[width = 1\columnwidth]{Figures/High_freq_approximations/KHIE_stat_point.png}
	\small
%	\put(13.5,36.5){$\vxs$}
	\put(30,31.5){$\vxo^*(\vx)$}
	\put(56.5,23){$\vx$}	
	\put(11.2,27){$\vk^G(\vxo^*(\vx)-\vx)$}
	\put(64,20){$\vk^G(\vx-\vxo^*(\vx))$}	
	\put(41.5,27){$\vk^P(\vxo^*(\vx))$}
	\end{overpic}
	\end{minipage}
  \begin{minipage}[c]{0.18\textwidth}
\caption{2D geometry for the illustration of the stationary position for the Kirchhoff-Helmholtz integral.
}
	\label{Fig:HF_appr:KH_approximation_HF}
	\end{minipage}
\end{figure}
%
Again, it can be assumed that for a given receiver position $\vx$ most part of the integral cancels out, and the field is dominated by one particular stationary point on the surface.
Obviously, the stationary point is found on $\dO$ where the phase gradient vanishes, i.e. where the local wavenumber vector/local propagation direction of the described sound field and the Green's function positioned at $\vx$ coincide, satisfying $\vk^P(\vxo^*(\vx))= \vk^G(\vx-\vxo^*(\vx)) = -\vk^G(\vxo^*(\vx)-\vx)$.
This interpretation is illustrated in Figure \ref{Fig:HF_appr:KH_approximation_HF} in case of a primary point source.

As an approximation, the amplitude factor of the integral can be substituted by its stationary value, i.e. with $k_{\mathrm{n}}^G(\vx-\vxo) = k_{\mathrm{n}}^P(\vxo)$.
Furthermore, only that part of the integral path contributes to the total sound field that serves as a stationary point for any receiver position inside the enclosure,
resulting in the windowing function \eqref{eq:theory:windowing_function} and the KHIE may be further simplified towards
\importanteq{Kirchhoff approximation}{
P(\vx,\omega) = 
\oint_{\dO} 
2 w(\vxo) \ti k_{\mathrm{n}}^P(\vxo)  \,
P(\vxo,\omega) \, G(\vx-\vxo,\omega) \, \td \dO( \vxo).
\label{Eq:HF_appr:Kirchhoff_approximation}
}
%\fscom{nice: this links perfectly to my PhD (2.33) when exchanging (1+cos) to 2*win and replacing velocity with pressure along SSD}
This is obviously equivalent to the Kirchhoff approximation \eqref{Eq:SFS_theory:Kirchhoff_appr}, derived by physically motivated considerations from the equivalent scattering interpretation of the simple source formulation.

\subsection{Asymptotic approximation of spectral integrals}
\label{Sec:SPA_for_Fourier}
Now the physical interpretation of the stationary position is discussed when the SPA is applied for spectral integrals.
The forward and inverse Fourier transforms of an arbitrary sound field $P(\vx,\omega)$, written in a general polar form, are given by
\begin{equation}
\tilde{P}(k_x,y,k_z,\omega) = \iint_{-\infty}^{\infty} A^P(\vx,\omega)\te^{\ti \phi^P(\vx,\omega)} \te^{\ti k_x x} \te^{\ti k_z z} \td x \td z,
\label{eq:forward_transform}
\end{equation}
\begin{equation}
P(\vx,\omega) = \frac{1}{(2\pi)^2} \iint_{-\infty}^{\infty} A^{\tilde{P}}(k_x,y,k_z,\omega)\te^{\ti \Phi^{\tilde{P}}(k_x,y,k_z,\omega)}  \te^{-\ti k_x x} \te^{-\ti k_z z} \td k_x \td k_z,
\label{eq:inverse_transform}
\end{equation}
with $\tilde{P}(k_x,y,k_z,\omega) = A^{\tilde{P}}(k_x,y,k_z,\omega)\te^{\ti \Phi^{\tilde{P}}(k_x,y,k_z,\omega)}$, where $A^{\tilde{P}}, \Phi^{\tilde{P}}\in\mathbb{R}$.
The forward and inverse transforms describe projection and composition of the sound field $P$ to and from \emph{spectral plane waves} respectively (see Section \ref{Sec:thoery:angular_Spectrum}).
The propagation direction of these spectral waves (i.e. their wavenumber vector) is completely determined by $k_x$ and $k_z$ along with the acoustic wavenumber $k$, via the dispersion relation.

Supposing that the sound field fulfills the SPA requirements---i.e. under high frequency assumptions---the forward transform \eqref{eq:forward_transform}
may be evaluated asymptotically, by applying the stationary phase method \cite{Arnold1995, Tinkelman2005}.
The stationary point $\vx^*(k_x,k_z)$ is found for given spectral $k_x$ and $k_z$ values where the gradient of the exponent is zero.
Assuming that the local dispersion relation holds, two local wavenumber components completely define the local wavenumber vector and the stationary position for the spectral integral is found where
\importantalign{Stationary position of spectral integrals}{
\left.
\frac{\partial}{\partial x} \phi^P(\vx,\omega)
\right|_{\vx=\vx^*(k_x,k_z)} + k_x &= 0 \hspace{3mm} \rightarrow \hspace{3mm} k_x^P(\vx^*(k_x,k_z)) = k_x, \\
\left.\frac{\partial}{\partial z} \phi^P(\vx,\omega)
\right|_{\vx=\vx^*(k_x,k_z)} + k_z &= 0 \hspace{3mm} \rightarrow \hspace{3mm} k_z^P(\vx^*(k_x,k_z)) = k_z, \\
\left.\Dx \phi^P(\vx,\omega)
\right|_{\vx=\vx^*(\vk)} + \vk  &= 0 \hspace{3mm} \rightarrow \hspace{3mm} \vk^P(\vx^*(\vk)) = \vk
}
is satisfied, with $\vk$ being the wavenumber vector of the spectral plane wave.

This finding states that each point in the angular spectrum of a sound field is dominated by the parts of the space where the local propagation direction coincides with the corresponding spectral plane wave's global propagation direction.
The local wavenumber components therefore may be also defined alternatively as the stationary points of the spatial Fourier transform \eqref{eq:forward_transform}, as a function of space.\footnote{This definition if often termed \emph{Lagrange submanifolds}, playing a central role in phase space representation of sound fields \cite{Steinberg1993, Arnold1995, Tinkelman2005}.}
The interpretation of the stationary position for the Fourier transform SPA is illustrated in Figure \ref{Fig:Theory:stat_pos_in_kx} through the exemplary transformation of a point source.

\begin{figure}
	\small
	\centering
	\begin{overpic}[width = 1\columnwidth]{Figures/High_freq_approximations/fourier_stat_point.png}
	\small
	\put(0,0){(c)}
	\put(60,29){(b)}
	\put(60,0){(d)}
	\put(0,37){(a)}
	\put(54,40){$x$}
	\put(6.5,53){$y$}
	\put(54,12.5){$x$}
	\put(6.5,35){$y$}
	\put(37,50.25){$\vk$}
	\put(31,12){$x^*(\vk)$}
	\put(86,35.25){$x^*(k_x = 0.5k)$}
	\end{overpic}
	\caption{Illustration of the stationary position for the SPA of the Fourier transform in case of a 3D point source, with its one-dimensional Fourier transform evaluated along the $x$-axis. 
Figure (a) presents a spectral basis function (i.e. a horizontal plane wave), with the exemplary wavenumber vector defined by $k_x = 0.5 k$. 
For this spectral component the stationary phase point (indicated by white arrow) is found in the field of the point source (shown in Figure (c)) where the local propagation direction of the point source coincides with that of the plane wave.
The coincidence of the local propagation directions is ensured by the assumption that in the plane of investigation $k_z^G(x,y,0) \equiv 0$, and the spectral plane wave is assumed to propagate with $k_z = 0$.
The spectrum, shown in Figure (d) (as given analytically in Table \eqref{tab:theory:Greens_fun_representations}), is dominated around $k_x = 0.5k$ by this stationary position, denoted by $x^*(k_x)$ in Figure (b).}
	\label{Fig:Theory:stat_pos_in_kx}
\end{figure}

The counterpart of this statement is that the greatest contribution to the inverse transform \eqref{eq:inverse_transform} is associated to those plane waves---the stationary phase of the inverse integral for given a $\vx$---whose wave number vector coincide with the local wavenumber components of the sound field at $\vx$.

So far, it has been assumed that in the region of investigation (along an infinite plane or line, depending on the transform dimensionality) the stationary phase position and thus each propagation direction is unique along the integral path/surface.
This trivially does not hold for the case of e.g. a plane wave or for complex acoustic fields produced by multiple sources of sound.
The SPA, however, can be extended for multiple stationary positions and the result of the approximation is obtained by summing the SPA contributions over the stationary positions \cite[p. 129]{Bleistein2000}.
In the present treatise this limitation is not investigated further, since the results involving the SPA hold without any modification for a virtual plane wave as well, as a limiting case.

\subsection*{Application example \#1: 1D spectrum of the Green's function}
\label{sec:greens_function_spectrum}

As an example, the 1D spatial Fourier transform of the 3D Green's function is investigated with the transform taken along the $x$-dimension.
As a result, a frequently used high frequency asymptotic approximation of the Hankel function is obtained.
For the sake of simplicity the point source is located in the origin.

The exact solution for the problem is available analytically in Table \eqref{tab:theory:Greens_fun_representations}, given by the second order Hankel function in the propagation region:
\begin{equation}
\tilde{G}(k_x,y,z,\omega) = \frac{1}{4\pi} \int_{-\infty}^{\infty} \frac{\te^{-\ti k \sqrt{x^2 + y^2 + z^2}}}{\sqrt{x^2 + y^2 + z^2}} \, \te^{\ti k_x x} \, \td x = 
-\frac{\ti}{4} H_0^{(2)}\left( \sqrt{k^2- k_x^2} \sqrt{y^2 + z^2} \right)
\label{Eq:HF_approx:Greens_spectrum_defintion}
\end{equation}
In this simple case, the stationary positions can be found explicitly for a given wavenumber and the SPA of the Fourier transform can be evaluated analytically. 
By definition the stationary position for an arbitrary spectral wavenumber $k_x$ is found where the $x$-derivative of the phase function vanishes and $x^*(k_x)$ satisfies
\begin{equation}
k^G_x(x^*(k_x)) = 
k \frac{x^*(k_x)}{\sqrt{x^*(k_x)^2 + y^2 + z^2}} = k_x 
\hspace{1cm} \rightarrow \hspace{1cm} 
x^*(k_x) = \rho \frac{k_x}{k_{\rho}},
\label{eq:HF_approx:greens_spectrum_stat_point}
\end{equation}
with $\rho = \sqrt{y^2+z^2}$ being the radial distance from the $x$-axis and $k_{\rho} = \sqrt	{k^2-k_x^2}$ being the corresponding radial wavenumber.
For the geometric interpretation of the stationary point refer to Figure \ref{Fig:Theory:stat_pos_in_kx}.
At the stationary point the phase of the integrand and its second derivative reads
\begin{equation}
\phi^{G}(x^*(k_x)) = - \rho k_{\rho}, \hspace{1cm}
\phi^{''G}_{xx}(x^*(k_x)) =  -k \frac{y^2+z^2}{\sqrt{ x^*(k_x)^2 +y^2+z^2 }^3} = - \frac{k_{\rho}^3}{k^2 \rho}.
\end{equation}
Substitution into the SPA \eqref{Eq:SPAResult} with exploiting that $-k \sqrt{x^*(k_x)^2 + y^2 + z^2} = -\rho \frac{k^2}{k_{\rho}}$ and accounting for the negative sign of the second derivative yields the asymptotic form of the 3D point source spectrum
\importanteq{Field of a linear radiator}{
\tilde{G}(k_x,y,z,\omega) = -\frac{\ti}{4} H_0^{(2)}\left( k_{\rho} \rho \right) \approx \frac{1}{\sqrt{8\pi \ti}} \frac{\te ^{-\ti \rho k_{\rho}}}{\sqrt{ \rho k_{\rho} }}.
\label{Eq:25D_WFS:3D_Greens_asymp_spectrum}
}
This result is the well-known asymptotic expansion of the Hankel function for large arguments \cite[10.17.6]{Olver:2010:NHMF}, given generally as
\begin{equation}
H_0^{(2)}(z)\approx \sqrt{\frac{2 \ti}{\pi z}} \te^{-\ti z}.
\label{Eq:HF_approx:Hankel_asymptotic_form}
\end{equation}

\begin{figure}[]
	\small
	\centering
	\begin{overpic}[width = 0.9\columnwidth ]{Figures/High_freq_approximations/greens_stat_pos_2.png}
	\small
	\put(-2,0){(a)}
	\put(45,0){(b)}
	\put(53,37){$z=0$}
	\put(-0.5,11.75){$x$}
	\put(34.5,10.5){$y$}
	\put(14.75,40.5){$z$}
	%
	\put(99,9){$x$}
	\put(72,37){$y$}
	\put(66,8.5){$x^*(k_x)$}
	\put(80,20.5){$k_x$}
	\put(71.5,31.5){$k_{\rho}$}
	\put(80.5,30){$\vk$}
	\put(75,17.5){$\rho$}
	\end{overpic}
	\caption{Interpretation of the Green's function's spectrum as the field of a line source, with harmonic spatial distribution, described by wavenumber $k_x$, evaluated at $x = 0$.
	Such a source (shown in Figure (a)) radiates a cylindrical sound field, with the radial wavenumber $k_{\rho}$ and the longitudinal wavenumber $k_x$, so that $k = \sqrt{k_x^2+k_{\rho}^2}$ is satisfied.
	In case of $k_x=0$, this corresponds to the field of the 2D Green's function.
	From geometrical considerations, and applying the interpretation of the SPA for boundary integrals, the stationary position for integral \eqref{Eq:HF_approx:Greens_spectrum_defintion} is found at $x^*(k_x) = \rho \frac{k_x}{k_{\rho}}$, as shown in Figure (b).}
%	Based on this interpretation the stationary position for integral \eqref{Eq:HF_approx:Greens_spectrum_defintion} can be found by :
%	for a given wavenumber $k_x$ and for a given radial distance $\rho$ that part of the $x$-axis will be the stationary point from which the emerging wavefront at $\vx$ coincides with that of a plane wave propagation into the direction $\vk = \posvec{2}{k_x}{k_{\rho}}$.
%	From simple geometric considerations it is found at $x^*(k_x) = r_0 \frac{k_x}{k_{\rho}}$.}
	\label{Fig:Theory:greens_stat_pos}
\end{figure}
\vspace{3mm}
In the particular case under consideration, when the function to be Fourier transformed is the Green's function, the Fourier integral \eqref{Eq:HF_approx:Greens_spectrum_defintion} can be interpreted as the sound field of an infinite line source, with a harmonic spatial distribution described by $k_x$, evaluated at $x = 0$.
Such a line source radiates attenuating conical wavefronts, propagating radially away from the $x$-axis with the local wavenumber vector given by $\vk^P(\vx) = \posvec{2}{k_x}{k_\rho}$, as illustrated in Figure \ref{Fig:Theory:greens_stat_pos} (a).
The attenuation of the waves depends on the propagation direction: lateral waves (with small $k_{\rho}$) are extremely enhanced.
%For this special case the stationary position defined by \eqref{eq:HF_approx:greens_spectrum_stat_point} gains a simple geometrical interpretation, shown in Figure \ref{Fig:Theory:greens_stat_pos} (b).
 
The DC ($k_x = 0$) component of the spectrum \eqref{Eq:25D_WFS:3D_Greens_asymp_spectrum} describes the sound field generated by an infinite line source along the $x$-axis, i.e. a 2D point source. 
The high frequency approximation of the 2D Green's function---which therefore stems from the asymptotic approximation of \eqref{Eq:Wave_Theory:2D_Green} \cite[p. 118]{Williams1999}---is thus given by
\importanteq{Approximate 2D Green's function}{
G_{2\text{D}}(\vx,\omega) \approx \frac{1}{\sqrt{8\pi \ti}}\frac{\te^{-\ti k |\vx|}}{\sqrt{k |\vx|}} =  \sqrt{\frac{2 \pi |\vx|}{\ti k }}G_{3\text{D}}(\vx,\omega),
\label{eq:HF_approx:2D_vs_3D_GF}
}
with $\vx = \posvec{2}{y}{z}$. 
This result indicates that a 2D point source generates cylindrical wavefronts, with its phase function---and its local wavenumber vector---coinciding with that of a 3D point source, measured at $z=0$.
The sound field attenuates proportionally to $\frac{1}{\sqrt{|\vx|}} = \frac{1}{\sqrt{\rho^{G_{2\mathrm{D}}}}} = \sqrt{\kappa^{G_{2\mathrm{D}}}}$, where $\kappa^{G_{2\mathrm{D}}}$ is the horizontal principal curvature of the wavefront (with the vertical one being zero).
Opposed to a 3D source's flat frequency response, a 2D one exhibits a frequency response of $\sim 1/\sqrt{\ti \omega}$, corresponding with the infinite tail of a 2D field's impulse response.

\subsection*{Application example \#2: 2D spectrum of the Green's function}
\label{Sec:HF_approx:1D_Greens}
As a second example, the 2D spatial Fourier transform of the Green's function is discussed.
The Fourier transform reads as
\begin{equation}
\label{eq:HF_approx:2D_FFT}
\tilde{G}(k_y,y,k_z,\omega) = \iint^{\infty}_{-\infty} G(x,y,z,\omega) \, \te^{\ti k_x x} \, \te^{\ti k_z z} \, \td x \, \td z.
\end{equation}
On a fixed $y = \text{const}$ plane the stationary point for the integral is found where 
\begin{align}
\label{eq:HF_approx:2D_FFT_stat_pos}
k_x^G(x^*(k_x),y,z^*(k_z)) &= k_x, \hspace{3mm} \rightarrow \hspace{3mm} k\frac{x^*(k_x)}{|\vx^*(k_x,k_z)|} = k_x \\
k_z^G(x^*(k_x),y,z^*(k_z)) &= k_z, \hspace{3mm} \rightarrow \hspace{3mm} k\frac{z^*(k_z)}{|\vx^*(k_x,k_z)|} = k_z \\
|\vk^G(\vx^*(k_x,k_z))| &= |\vk|,  \hspace{3mm} \rightarrow \hspace{3mm} k\frac{y}{|\vx^*(k_x,k_z)|} = k_y
\end{align} 
holds, i.e where the local propagation direction of the spherical wavefront coincides with that of the spectral plane wave, described by $k_x, k_z$.

The determinant of the phase function's Hessian can be given in terms of the principal curvatures (known analytically for the Green's function), by using the same considerations as used in Section \ref{Sec:HF:RayleighSPA}.
By definition, around the stationary position $k_y^G(\vx^*(k_x,k_z)) = k_y$ holds, and the determinant reads as
\begin{equation}
\mathrm{det} \, \mH^{G}(\vx^*(k_x,k_z)) = k^2 \kappa^G_1(\vx^*(k_x,k_z))\kappa^G_2(\vx^*(k_x,k_z)) \hat{k}^G_y(\vx^*(k_x,k_z))^2 = \frac{k_y^2}{|\vx^*(k_x,k_z)|^2}.
\end{equation}
Accounting for the positive curvatures the signature of the Hessian equals (-2), and the 2D Fourier transform can be approximated by the 2D SPA of \eqref{eq:HF_approx:2D_FFT} as
\begin{equation}
\tilde{G}(k_y,y,k_z,\omega) = \frac{2\pi}{\sqrt{|\mathrm{det} \, \mH^{G}(\vx^*(k_x,k_z))|}} \, \frac{1}{4\pi} \frac{\te^{-\ti k |\vx^*(k_x,k_z)|}}{|\vx^*(k_x,k_z)|}
%\te^{\ti k_x x^*(k_x)} \te^{\ti k_z z^*(k_z)} %typo propably from spatial FT?!
\, \te^{-\ti \frac{\pi}{2}}.
\end{equation}
Substituting the determinant and expressing the stationary positions by \eqref{eq:HF_approx:2D_FFT_stat_pos} leads finally to
\importanteq{Field of a planar radiator}{
\tilde{G}(k_y,y,k_z, \omega) =\frac{1}{2} \frac{\te^{-\ti k_y y } }{\ti k_y} =
\frac{1}{2} \frac{\te^{-\ti \sqrt{\left(\frac{\omega}{c}\right)^2-k_x^2-k_z^2} y } }{ \ti \sqrt{\left(\frac{\omega}{c}\right)^2-k_x^2-k_z^2} }.
\label{eq:HF_approx:Greens_2D_Spectrum}
}
Comparison with Table \eqref{tab:theory:Greens_fun_representations} reveals that in this special case, the 2D SPA yields the exact spectrum of the Green's function in the propagation region.

Similarly to the previous example, the above equation describes the field of an infinite planar set of point sources with a harmonic spatial distribution, generating plane waves into the direction $\vk = \posvec{3}{k_x}{k_y}{k_z}$, measured at $\vx = \posvec{3}{0}{y}{0}$.
Furthermore, at $k_x = k_z = 0$ the spectrum yields the 1D Green's function
\importanteq{1D Green's function}{
G_{1\text{D}}(y,\omega) = \frac{1}{2} \frac{\te^{-\ti k y } }{\ti k},
\label{eq:HF_approx:1D_Green}
}
describing the field of a vibrating infinite planar surface, with the frequency response given by $\sim \frac{1}{\ti \omega}$ and the impulse response being a Heaviside step function.
The 1D Green's function therefore realizes the full integration of the source time history, while the 2D Green's function's impulse response can be interpreted as the half-integration of the source signal \cite{Deregowski1983, Wang2009, Schultz2013:IIR_prefilters, Wang2016}.
%
%
%
\chapter{Theory of Sound Field Synthesis}
\label{sec:sound_field_synthesis}
\begin{figure}[b!]
	\centering
	\begin{overpic}[width = .8\columnwidth ]{Figures/SFS_theory/general_sfs.png}
	\small
	\put(0,26){virtual source}
	\put(44.5,0.5){$\mathbf{0}$}
	\put(71,31){$\vx$}
	\put(39,21){$\vni$}
	\put(43,15){$\vxo$}
	\begin{turn}{27}
	\put(57,-3){$|\vx - \vxo|$}
	\end{turn}
	\put(50,35){$\Omega$}
	\put(80,20.5){$\dO$}
	\end{overpic}
	\caption{Geometry for the general sound field synthesis problem}
	\label{Fig:Theory:general_sfs_geometry}
\end{figure}

In the following, the general sound field synthesis (SFS) problem is formulated. 
Consider a source-free volume $\Omega \subset \mathbb{R}^3$, bounded by a continuous set of acoustic sources, forming the boundary surface $\dO$.
The enclosing source ensemble is termed the \emph{secondary source distribution (SSD)}.
%Obviously, in the aspect of practical applications only 3D problems are of importance ($n=3$).%, however for the sake of computational simplicity several simulations 
The general geometry is depicted in Figure \ref{Fig:Theory:general_sfs_geometry}.
For the sake of simplicity it is assumed that the boundary is acoustically transparent and the secondary sources are acoustic point sources, described by the free field Green's function.
Since closed-box, dynamic loudspeakers can be modeled as 3D monopoles in the low-frequency region, this choice of secondary sources is reasonable. 

With these assumptions, the synthesized pressure at any receiver position $\vx \in \Omega$ is given by the superposition of the fields of individual secondary sources, written as a single layer potential \cite{Ahrens2010phd,Ahrens2012,Wierstorf2014,Schultz2014:Comparing_approaches}
\begin{equation}
P(\vx,\omega) = \oint_{\dO} D(\vxo,\omega) \, G(\vx - \vxo , \omega ) \td \dO ( \vxo ),
\label{Eq:Theory:3D_SFS}
\end{equation}
with $G(\vx,\omega)$ denoting the 3D free field Green's function.
The weighting factor $D(\vxo,\omega)$, i.e. the loudspeaker driving signal is termed the \emph{driving function} for the given SSD. 
The sound field synthesis problem can be formulated as follows:
Given a \emph{target sound field} or the sound field of a \emph{virtual source} $P(\vx,\omega)$, the goal is to solve the above integral equation for $D(\vxo,\omega)$, so that the superposition of the SSD's sound field---the \emph{synthesized field}---equals to the target sound field.
The problem is therefore an inverse problem, with unique solution for general enclosures \cite{Fazi2010}.

Comparing the general SFS formulation \eqref{Eq:Theory:3D_SFS} with the Kirchhoff-Helmholtz integral \eqref{Eq:Theory:Kirchhoff-Helmholtz}, it becomes obvious that SFS with a single layer SSD is not able to ensure identically zero sound field outside the enclosure.
Practically, the dipole sources that would cancel the field of the monopoles outside the volume, are excluded from the superposition.
In the present thesis, free field conditions are assumed: the exterior sound field satisfies the Sommerfeld radiation condition, thus the effects of the listening environment, present in practical applications (i.e. wall reflections), are not considered. 
For the inclusion of room effects to the SFS problem refer to \cite{Spors2005, app8010016}.

\vspace{3mm}
The general 3D SFS setup, as discussed above, requires an enclosing surface of 3D point sources, making practical implementations hardly realizable.
In practice it is often sufficient to restrict the reproduction to the $z=0$ plane, containing a 2D contour of secondary sources.
This reproduction scenario is termed \emph{2.5D synthesis}, referring to the fact that although the problem dimensionality is reduced to $n=2$, still, the 2D SSD contour consists of 3D point source elements.
In this geometry, the general 2.5D synthesis problem is formulated as
\begin{equation}
P(\vx,\omega) = \oint_{C} D(x_0,y_0,0,\omega) \, G(x - x_0, y-y_0, 0 , \omega ) \, \td s( x_0, y_0 ),
\label{Eq:Theory:25D_SFS}
\end{equation}
where $C(x_0,y_0)$ is the SSD contour and $\td s$ is the infinitesimal arc length.
Obviously, neither 2D nor 3D sound fields can be perfectly synthesized in this geometry, due to dimensionality inconsistency between the target field and the SSD.
Overcoming the artifacts of this \emph{dimensionality mismatch} is the central question of practical sound field synthesis and is the main topic of the present chapter.

In the followings, this chapter presents approaches, in order to solve the 3D and 2.5D SFS problems including physically based implicit, and particularly mathematical explicit solutions. 

\section{Implicit solution: Wave Field Synthesis}

The implicit solution for the SFS problem aims at the derivation of a single layer potential representation of the target sound field, containing the required SSD driving function implicitly.
In this section it is discussed, how these surface, or---more practically---contour integral representations of an arbitrary 3D sound field may be expressed in the form of \eqref{Eq:Theory:3D_SFS} and \eqref{Eq:Theory:25D_SFS}, from which the driving function can be extracted.

\subsection{3D Wave Field Synthesis}

In case of a 3D SFS problem, obtaining the implicit solution is straightforward, based on the boundary integral representations discussed in the previous chapters.
Assume a general enclosing 3D SSD surface, consisting of 3D point sources.
Comparing the Kirchhoff approximation of the Kirchhoff-Helmholtz integral \eqref{Eq:SFS_theory:Kirchhoff_appr} or \eqref{Eq:HF_appr:Kirchhoff_approximation} with the general SFS equation \eqref{Eq:Theory:3D_SFS} reveals that the Kirchhoff approximation implicitly contains the driving function $D(\vxo,\omega)$ for a general enclosing SSD surface.
The driving function is given by
\begin{equation}
D(\vxo,\omega) = - 2w(\vxo)\frac{\partial P(\vxo,\omega)}{\partial \vni}, 
\label{Eq:Theory:2D_3D_WFS_driv_fun}
\end{equation}
or making use of the high frequency gradient approximation, as
\importanteq{3D WFS driving function}{
D(\vxo,\omega) = 2 w(\vxo) \ti k_{\mathrm{n}}^P(\vxo) P(\vxo,\omega)
\label{Eq:Theory:2D_3D_WFS_driv_fun_2}
}
with $k_{\mathrm{n}}^P(\vxo)$ being the normal component of the target field's local wavenumber vector taken on the SSD and $w(\vxo)$ being the window function, as introduced in the previous chapter:
\begin{equation}
\label{Eq:Theory:SSD_selection}
w(\vxo) = \begin{cases}
                        1, \hspace{3mm} \forall \hspace{3mm} \langle \mathbf{k}^P(\vxo) \cdot \mathbf{n}_{\text{in}}(\vxo) \rangle > 0 \\
                        0  \hspace{3mm} \text{elsewhere}.
                    \end{cases}
\end{equation}
In the context of WFS, the windowing is termed as \emph{secondary source selection} \cite{nicol19993d, Spors2007:DAGA:SS_selection_criterion, Spors2007}, selecting the \emph{active secondary sources}, contributing to the synthesized field.

The driving function \eqref{Eq:Theory:2D_3D_WFS_driv_fun_2} is a common generalization of the 3D WFS driving function, given by \cite[(20)]{Zotter2013:uniqueness} specifically for a virtual point source.
Both formulations are valid in the high frequency region within the validity of the Kirchhoff approximation: in the farfield of the virtual source distribution, i.e. where the local plane wave approximation of the virtual field holds.

In the special case of an infinite planar boundary surface, located along the plane $y = y_0$, the Kirchhoff-Helmholtz integral degenerates to the Rayleigh I (Neumann) integral, representing the field of any source distribution, located at $y<y_0$, in terms of a single layer potential, as discussed in Section \ref{Section:Theory:Rayleigh}.
Therefore, in this geometry with a planar SSD, the driving function \eqref{Eq:Theory:2D_3D_WFS_driv_fun} is capable of the perfect synthesis of an arbitrary virtual sound field in the listening half-space $y>y_0$, without any approximations involved.
In this case, no windowing is required, i.e. $w(\vxo) \equiv 1$, and the normal derivative is simply given by the $y$-derivative of the target/virtual sound field.

\vspace{3mm}
The following physical interpretation can be assigned to the 3D WFS driving function:
As it was discussed in Section \ref{Sec:HF_approx:1D_Greens}, the frequency response of the 1D Green's function, given by \eqref{eq:HF_approx:1D_Green}---representing an infinite planar distribution of 3D point sources---is proportional to $\sim \frac{1}{\ti k} = \frac{c}{\ti \omega}$ expressing an infinite impulse response of a step function, performing the integration of the source time history. 
In case of a non-homogeneous distribution, a directivity factor is also present, given by $\frac{1}{\ti k_y}$ for a harmonic distribution as given by \eqref{eq:HF_approx:Greens_2D_Spectrum} and approximated locally by \eqref{Eq:HF_approx:H_det_Rayleigh}.
This directivity factor of $\frac{1}{\ti k^P_y(\vxo)}$ indicates that a surface of point sources generates a larger pressure field into lateral directions than into the normal direction.
Within the validity of the Kirchhoff approximation, this statement can be extended towards arbitrary SSD surfaces, radiating with an enhanced intensity into locally lateral directions.

The factor $\ti k_{\mathrm{n}}^P(\vxo)$ in \eqref{Eq:Theory:2D_3D_WFS_driv_fun_2} (and in the Rayleigh/Kirchhoff-Helmholtz integral) may be therefore interpreted as a correction term, ensuring flat frequency response for the SSD surface by inverse filtering (taking the time derivative) of the excitation signal, and compensating for the SSD's directive radiation characteristics, locally.

\subsection*{Application example: 3D synthesis of a virtual point source}

\begin{figure}  
\small
  \begin{minipage}[c]{0.64\textwidth}
	\begin{overpic}[width = 1\columnwidth ]{Figures/SFS_theory/3D_WFS_general.png}
	\small
	\put(2,53){(a)}
	\put(2,1){(b)}
	\end{overpic}   \end{minipage}\hfill
	\begin{minipage}[c]{0.35\textwidth}
    \caption{3D synthesis of a 3D point source ,located at $\vxs = \posvec{3}{0.4}{2.5}{0}$, radiating at $f_0 = 1.5~\mathrm{kHz}$.
    The SSD surface is chosen to be independent of the $z$-coordinate, as illustrated in Figure \ref{fig:SFS_theory:WFS_geometry}.
	For the numerical calculation, the SSD was truncated along the vertical dimension by choosing parameters, so that diffraction effects due to the truncation are negligible in the simulation results.
    Figure (a) depicts the real part of the synthesized field and Figure (b) presents the absolute error of synthesis (the discrepancy between the synthesized and the target sound field) in a logarithmic scale, measured in the horizontal plane, containing the virtual point source.
	The active arc of the SSD is denoted by solid black line and the inactive part with dotted by black line.
    }
\label{fig:SFS_theory:3D_WFS_general}  \end{minipage}
\end{figure}

As a simple example, the 3D WFS of a virtual point source is discussed.
Assume a 3D point source, located behind the SSD at $\vxs = \posvec{3}{x_s}{y_s}{z_s}$ with $y_s < y_0$.
Substituting the Green's function into \eqref{Eq:Theory:2D_3D_WFS_driv_fun_2} yields the point source specific high frequency 3D WFS driving function
\begin{equation}
\label{Eq:SFS_theory:3D_WFS_ps_driv_fun}
D(\vxo,\omega) = w(\vxo)  \frac{\ti k }{2\pi} \frac{\left< \vxo-\vxs \cdot \vni(\vxo) \right> }{|\vxo-\vxs|} \, \frac{\te^{-\ti k |\vxo-\vxs|}}{|\vxo-\vxs|},
\end{equation}
being equivalent with \cite[Eq. 20.]{Zotter2013:uniqueness} and \cite[Eq. 19.]{Spors2008:WFSrevisited}.

The result of synthesis is depicted in Figure \ref{fig:SFS_theory:3D_WFS_general} for the special case of an SSD surface, being invariant to translation along the $z$-axis, as illustrated by Figure \ref{fig:SFS_theory:WFS_geometry} in the following section.
The driving function ensures amplitude correct synthesis within the validity of the Kirchhoff approximation:
amplitude errors arise 
\begin{itemize}
\item in the proximity of secondary sources where local curvature of the SSD surface is large, due to the local failure of the tangent plane approximation
\item in the proximity of secondary sources where the normal component of the local wavenumber vector is small---i.e. at parts of the SSD that are nearly parallel to the virtual field's local propagation direction---since at these positions the high frequency gradient approximation fails, with also the lack of diffracted waves leading to amplitude errors.
\item at space regimes for which the above described secondary sources serve as stationary positions, as discussed in Section \ref{Sec:HS_approx:SPA_for_Rayleigh}.
\end{itemize}

The driving function \eqref{Eq:SFS_theory:3D_WFS_ps_driv_fun} expresses the loudspeaker driving signal for a virtual point source, radiating at a single frequency component.
Assuming a wideband source excitation with the time history given by $s(t)$ and its frequency content being $S(\omega)$, the time domain WFS driving function is given by the inverse temporal Fourier transform of \eqref{Eq:SFS_theory:3D_WFS_ps_driv_fun}, weighted by the excitation spectrum:
\begin{equation}
d(\vxo,t) = \frac{w(\vxo)}{2\pi} \int_{-\infty}^{\infty} \frac{\left< \vxo-\vxs \cdot \vni(\vxo) \right> }{|\vxo-\vxs|} \frac{\ti \omega }{2\pi c} S(\omega) \frac{\te^{ \ti \omega ( t - \frac{|\vxo-\vxs|}{c} ) }}{|\vxo-\vxs|} 
\, \td \omega.
\end{equation}
By realizing that $\ti \omega \, S(\omega)$ describes the temporal derivative of the source time history (being equivalent with filtering the source signal with a 6 dB/octave highpass filter), and exploiting the Fourier transform shift theorem, the time domain 3D WFS driving function for a virtual point source is obtained as
\begin{equation}
\label{Eq:SFS_theory:3D_WFS_ps_driv_fun_td}
d(\vxo,t) = \frac{w(\vxo)}{2\pi c} \frac{\left< \vxo-\vxs \cdot \vni(\vxo) \right> }{|\vxo-\vxs|} \frac{s'_t(t - \frac{|\vxo-\vxs|}{c} )}{|\vxo-\vxs|},
\end{equation}
where differentiation with respect to time compensates for the SSD frequency response, as discussed above.

\subsection{The 2.5D Kirchhoff approximation}

Before getting involved with the questions of 2.5D Wave Field Synthesis a further simplification of the Kirchhoff approximation is introduced.
This simplification reduces the 3D Kirchhoff integral into a 2D contour integral representing a 3D sound field as the superposition of 3D Green's functions.
The approximation is therefore referred to as the \emph{2.5D Kirchhoff integral}, frequently occurring in the field of seismic migration and inversion problems.
The dimensionality reduction is performed by applying the stationary phase approximation to the Kirchhoff integral along the vertical dimension.

Assume a 3D interior radiation problem, with the sound field under consideration described by the Kirchhoff integral \eqref{Eq:HF_appr:Kirchhoff_approximation} written on a surface, being translation invariant along the $z$-axis.
The problem geometry is depicted in Figure \ref{fig:SFS_theory:WFS_geometry}.
In the following the receiver position is assumed to be at $z=0$ inside the enclosure at $\vx = \posvec{3}{x}{y}{0} \in \Omega$.
%
\begin{figure}  
\begin{minipage}[c]{0.6\textwidth}
  \hspace{0cm}
	\begin{overpic}[width = 1\columnwidth ]{Figures/SFS_theory/WFS_geometry.png}
	\small
	\put(82,51.5){$x$}
	\put(91.5,33){$y$}
	\put(95,65.5){$z$}
	\put(48,35.5){$\vx$}
	\put(63.5,42.5){$\vxo$}
	\put(4,22.5){plane of interest}
	\put(30,8){$\dO$: 3D surface}
	\put(48,24.5){$C$: 2.5D contour}
	\end{overpic}  \end{minipage}\hfill
	\begin{minipage}[c]{0.37\textwidth}
    \caption{
    Geometry for the derivation of 2.5D Kirchhoff integral.
The enclosing surface $\dO(x_0,y_0)$ is chosen to be independent of the $z$-coordinate in order to be able to evaluate the Kirchhoff integral with respect to $z_0$ using the SPA. 
If the sound field to be described is a 2D one propagating in the direction parallel to the listening plane, then the surface can be interpreted as a continuous set of infinite vertical line sources along $C$, capable of the perfect description of a 2D field inside the enclosure by a 2D countour integral.}
\label{fig:SFS_theory:WFS_geometry}  
\end{minipage}
\end{figure}
%
In this special geometry the integral variables are separable and the Kirchhoff integral can be written as
\begin{equation}
P(\vx,\omega) = 
\oint_{C} \int_{-\infty}^{\infty} 
2 w(\vxo) \ti k_{\mathrm{n}}^P(\vxo) 	
P(\vxo,\omega) G(\vx-\vxo,\omega) \, \td z_0 \, \td s(x_0, y_0),
\label{Eq:SFS_theory:Kirchhoff_spec_geom}
\end{equation}
with the integral variable $\td s$ being the infinitesimal arc length along the contour $C(x_0,y_0) = \dO(x_0,y_0,0)$.

The integral along $z_0$ is approximated applying the stationary phase approximation.
Application of the 1D SPA formulation \eqref{Eq:SPAResult} requires the definition of the stationary position and the sign of the integrand's phase function's second vertical derivative (i.e. the signature of the 1D Hessian) at the stationary position.

\paragraph{Definition of the vertical stationary position:}
The vertical stationary position in the geometry under investigation is straightforward:
Since the contour of integration is chosen to lie at the $z=0$ plane, therefore the vertical stationary position has to be found at $z_0^* = 0$.
Based on the foregoing this requirement can be formulated as
\begin{equation}
k_z^P(x_0,y_0,0) = k_z^G(x-x_0,y-y_0,0) = 0,
\end{equation}
stating the trivial fact that a sound field can be described by a 2.5 dimensional contour integral only in the plane where all the sound sources are located, and which plane the emerging waves propagate parallel with.
In the plane of investigation $k_z^P(x,y,0) \equiv 0$ holds for the field of 3D sources located at the plane of investigation and of 2D sources being invariant along the vertical dimension.
Throughout the present thesis when dealing with 2.5D synthesis problems these types of virtual fields are considered exclusively.

\paragraph{Definition of the Hessian's signature:}
An important property of wavefields under consideration is that their vertical curvature---given by $\phi^{P''}_{zz}(x_0,y_0,0,\omega)$---is one of their principal curvatures itself, denoted by $\Kv^P$.
Hence, as it is discussed in Appendix \ref{App:Hessian} in the stationary position ($z_0 = 0$) the principal curvatures are additive, and the second phase derivative is the negative sum of the principal curvatures of the target sound field and the Green's function:
\begin{multline}
\phi^{P + G''}_{zz}(x_0,y_0,0,\omega) = \phi^{P''}_{zz}(x_0,y_0,0,\omega) +\phi^{G''}_{zz}(x-x_0,y-y_0,0,\omega) = \\ = -k\left( \Kv^P(x_0,y_0,0) + \Kv^G(x-x_0,y-y_0,0) \right).
\label{Eq:SFS_theory:Curvatures_Addition}
\end{multline}
Note that here it was exploited that $\phi^{P''}_{zz} = \phi^{P''}_{z_0 z_0}$ and $\phi^{G''}_{zz} = \phi^{G''}_{z_0 z_0}$ holds.
By definition, for an arbitrary diverging sound field (including the Green's function) the principal curvatures are positive and $\mathrm{sgn} \left( \phi^{P + G''}_{zz}(x_0,y_0,0,\omega) \right) = -1$ trivially holds.
For a converging wavefront the signature of the resultant curvature depends on the receiver position: in regions of the receiver plane where sound field $P$ locally converges the resultant curvature is positive, while in regions where the sound field diverges, e.g. after passing a focal point the resultant curvature is negative.
In the present thesis only locally diverging wavefields are discussed.

With these considerations application of the SPA to \eqref{Eq:SFS_theory:Kirchhoff_spec_geom} results in the \emph{2.5D Kirchhoff integral} for diverging sound fields, reading as
\importantmline{2.5D Kirchhoff integral}{
P(\vx,\omega) =
\oint_{C}
2 w(\vxo)
\sqrt{\frac{2 \pi}{\ti |\phi^{P''}_{zz}(\vxo,\omega) +\phi^{G''}_{zz}(\vx-\vxo,\omega)|}} \cdot \\ \cdot
\underbrace{\ti k_{\mathrm{n}}^P(\vxo) 	P(\vxo,\omega) }_{ \approx  -\frac{\partial P(\vxo,\omega)}{\partial \vn_{\mathrm{in}}}}
G(\vx-\vxo,\omega) \td s(x_0,y_0), 
\label{Eq:SFS_thery:25_KI}
}
with both $\vx = \posvec{3}{x}{y}{0}$ and $\vxo = \posvec{3}{x_0}{y_0}{0}$ denoting in-plane positions in the following when 2.5D scenarios are considered.
Again, in this formulation the coincidence of the second derivatives with respect to $z$ and $z_0$ of both $P$ and $G$ is exploited.
% and \fscom{HF directional gradient}.
%\fscom{it might be worth to discuss $\phi^{P''}_{zz}$ vs. $\phi^{P''}_{z_0z_0}$ to make calculus concise}

\subsection{2.5D Wave Field Synthesis}

The 2.5D Kirchhoff integral implicitly contains the 2.5D WFS driving function for a continuous contour of 3D point sources at the $z = 0$ plane.
%Comparing the expression for the synthesized field \eqref{Eq:Theory:25D_SFS} with \eqref{Eq:SFS_thery:25_KI} implicates that the required driving functions can be extracted from the 2.5D Kirchhoff integral.
The resulting driving function is, however, still dependent on the listener position through the argument of $\phi^{G''}_{zz}(\vx-\vxo,\omega)$, which dependency may be avoided by fixing the listener position.
This strategy would only allow the synthesis of the virtual field, optimized to a single, fixed receiver position termed the \emph{reference point}, while in other points in the listening plane amplitude errors would be present.
In the following it is presented how the driving function can be further manipulated in order to ensure correct synthesis along an arbitrary receiver curve termed the \emph{reference curve} within the validity of the stationary phase approximation resulting in what was termed \emph{generalized 2.5D WFS theory} \cite{Firtha2016}.


As it was stated in Section \ref{Sec:HS_approx:SPA_for_Rayleigh} for any receiver position $\vx$ the Kirchhoff integral is dominated by that stationary contour element $\vxo^*(\vx)$, from which the emerging spherical wavefronts locally coincide with the target field wavefront, i.e. where $\vk^P(\vxo^*(\vx)) = \vk^G(\vx - \vxo^*(\vx))$ is satisfied.
As a consequence, the 2.5D Kirchhoff integral may be further approximated by expressing the amplitude factor with its value at the stationary position as
\begin{multline}
P(\vx,\omega) = 
\oint_{C}
\! 2 w(\vxo) 
\sqrt{\frac{2 \pi}{\ti |\phi^{P''}_{zz}(\vxo^*(\vx)) +\phi^{G''}_{zz}(\vx-\vxo^*(\vx))|}}
\cdot \\ \cdot
\ti k_{\mathrm{n}}^P(\vxo) 	P(\vxo,\omega)
G(\vx-\vxo,\omega) \td s(x_0,y_0),
\label{Eq:SFS_theory:25D_KI_appr}
\end{multline}
where $\vxo^*(\vx)$ is defined by the implicit relation above.

The statement can be expressed by reversing causality, forming the main idea of 2.5D WFS theory: 
every point $\vxo$ on the secondary distribution contributes to the total sound field at the set of positions $\vx(\vxo)$ where the local propagation direction of a point source positioned at $\vxo$ coincides with that of the target field, i.e. where their local wavenumber vectors coincide.
Hence $\vx$ and $\vxo$ are \emph{stationary point pairs}, mutually determining each other.
By reversing the causality choosing $\vxo$ as an independent parameter the driving function can be extracted from \eqref{Eq:SFS_theory:25D_KI_appr} resulting in the \emph{generalized 2.5D WFS driving function}
\importanteq{2.5D WFS driving function}{
D(\vxo, \omega) = w(\vxo) 
\sqrt{\frac{8\pi}{\ti k}}\sqrt{\dref(\vxo)}
\ti k_{\mathrm{n}}^P(\vxo) 	P(\vxo,\omega),
\label{Eq:SFS_theory:25D_WFS_driv_fun}
}
with the term $\dref(\vxo)$ denoting the \emph{referencing function}, defined as
\begin{align}
\label{Eq:SFS_theory:Referencing function}
\dref(\vxo) &= \frac{k}{|\phi^{P''}_{zz}(\vxo) +\phi^{G''}_{zz}(\vxref(\vxo)-\vxo)|} =\\
		 	&= \frac{1}{\Kv^P(\vxo)+ \Kv^G(\vxref(\vx)-\vxo)}  = \\
			&= \frac{\Rv^P(\vxo) \cdot \Rv^G(\vxref(\vx)-\vxo)}{\Rv^P(\vxo)+ \Rv^G(\vxref(\vx)-\vxo)} 
,
\end{align}
where $\Kv^P$, $\Kv^G$ and $\Rv^P$,$\Rv^G$ are the vertical principal curvatures and radii of the involved sound fields, respectively.
Position $\vxref(\vxo)$ is the \emph{reference position} for the SSD element at $\vxo$, for which receiver position $\vxo$ serves as a stationary phase point on the SSD, defined by
\importanteq{2.5D WFS reference position}{
\vk^P(\vxo) = \vk^G(\vxref(\vxo) - \vxo).
\label{Eq:SFS_theory:WFS_General_Stat_pos}
}
By substituting the explicit expression for the Green's function's wavenumber ($\vk^G(\vx) = k\frac{\vx}{|\vx|}$) into \eqref{Eq:SFS_theory:WFS_General_Stat_pos} the set of positions for which a given $\vxo$ serves as stationary point reads as 
\begin{align} 
\vxref(\vxo) = \vxo + \hat{\vk}^P(\vxo) |\vxref(\vxo)-\vxo|,
\label{Eq:SFS_theory:SSD_ref_positions}
\end{align} 
with $\hat{\vk}^P(\vxo)$ denoting the normalized local wavenumber vector, being the unit vector into the target field's local propagation direction.
The equation describes straight lines passing through $\vxo$ into the direction of the local wavenumber vector of the target sound field $\vk^P(\vxo)$.
Along this straight line inside the listening region the virtual field wavefront matches the actual SSD element's wavefront.
Each SSD element therefore dominates the synthesized field towards the direction of the virtual field's local propagation direction measured at the SSD position, and a unique reference position can be defined for each SSD element along this path where the target field's and the secondary source's wavefronts match in amplitude as well.

In practical applications the set of the reference positions for all secondary sources form a prescribed continuous \emph{reference curve} $C_{\mathrm{ref}}$, so that the reference position for an SSD element is found at the intersection of the straight line, described by \eqref{Eq:SFS_theory:SSD_ref_positions} and the reference curve.
The reference curve must be a smooth convex curve inside the listening region, ensuring that each reference point has a unique stationary point pair.
Once the reference position $\vxref(\vxo) \in C_{\mathrm{ref}}$ is known for each SSD element the WFS driving function \eqref{Eq:SFS_theory:25D_WFS_driv_fun} can be evaluated.
Referencing the WFS driving function is therefore done by prescribing a unique reference point for each SSD element with the aid of the SPA, so that the set of these reference points form the continuous reference curve.
The resulting driving function will result in amplitude correct synthesis over the reference curve within the validity of the integral formulation \eqref{Eq:SFS_theory:25D_KI_appr}.

The location of the reference position for a given SSD element is illustrated in Figure \ref{fig:SFS_theory:WFS_ref_point} for the case of a virtual point source. 
Once the reference position is expressed for each SSD element, the driving function can be evaluated.

%
\begin{figure}
	\centering
	\begin{overpic}[width = .75\columnwidth]{Figures/SFS_theory/WFS_ref_point.png}
	\small
	\put(31,32){$\vxo$}
	\put(48,25){$\vxref(\vxo)$}
	\begin{turn}{-12.5}
	\put(28,33){$\vk^P(\vxo)$}
	\end{turn}
	\begin{turn}{18}
	\put(52,1){reference curve}
	\end{turn}
	\end{overpic}
    \caption{
    Location of the reference position for an SSD element positioned at $\vxo$.
    Due to the phase characteristics of the Green's function the reference position $\vxref(\vxo)$ for an arbitrary SSD element can be found at the intersection of the reference curve and the line emerging from $\vxo$ pointing into the local wavenumber vector of the virtual field $\vk^P(\vxo)$.
	The location of the arbitrarily chosen reference curve is denoted by dashed black line with solid line indicating the positions, for which a stationary SSD position can be found.
	Amplitude correct synthesis may be only achieved along this part of the reference curve.
   }
\label{fig:SFS_theory:WFS_ref_point}  
\end{figure}
%
In order to gain a physical interpretation of the structure of the resulting driving function, the referencing function can be expressed in terms of the vertical principal radii of the virtual field and the Green's function, yielding
\importanteq{Detailed 2.5D WFS driving function}{
\label{Eq:SFS_theory:25D_WFS_driv_fun_ver_2}
\scriptstyle
D(\vxo, \omega) = 
\underbrace{\sqrt{\frac{2\pi \Rv^G(\vxref(\vxo)-\vxo) }{\ti k}}}_{{\substack{\text{SSD}\\\text{compensation}}}}
\underbrace{\sqrt{ \frac{\Rv^P(\vxo) }{\Rv^P(\vxo) +  \Rv^G(\vxref(\vxo)-\vxo) } }}_{{\substack{\text{virtual source}\\\text{compensation}}} 
 \hspace{1mm} = \hspace{1mm}\sqrt{\frac{\Rv^P(\vxo)}{\Rv^P(\vxref(\vxo))}}
}
\underbrace{2 w(\vxo)  \ti k_{\mathrm{n}}^P(\vxo) 	P(\vxo,\omega)}_{\substack{\text{2D high freq.}\\\text{driving function}}}.
}
Again, $\rho_v^P$ and $\rho_v^G$ denote the principal radii of the virtual field and the Green's function along the vertical direction, with the absolute value operation in \eqref{Eq:SFS_theory:Referencing function} omitted due to their positive sign for diverging virtual fields.
The Green's function's principal radius is given simply as $\Rv^G(\vxref(\vxo)-\vxo) = |\vxref(\vxo)-\vxo|$ and the virtual field's principal radius $\Rv^P(\vxref(\vxo))$ is expressed by applying \eqref{eq:app:propagated_radii}.

The terms in the driving function can be identified as compensation factors for the \emph{dimensionality mismatch}, emerging in the 2.5D Kirchhoff integral.
Expressing the Kirchhoff approximation \eqref{Eq:SFS_theory:Kirchhoff_appr} for an entirely 2D problem, an arbitrary 2D sound field may be described in the area of investigation by a contour integral.
The enclosing boundary can be interpreted as the continuous distribution of two dimensional secondary point sources described by the 2D Green's function, representing infinite vertical line sources in three dimensions.
The 2D Green's function is weighted by the normal derivative of the sound field, taken on the SSD contour.
%
Application of the 2.5D WFS driving function aims to describe a 3D sound field in terms of a 2D contour integral with the kernel being the 3D Green's function,
weighted by the normal derivative of the 3D sound field.
This results in a dimensionality mismatch for both the virtual field and the secondary source elements.
The interpretation of the compensation factors in the driving function is then the following:
\begin{itemize}
\item Term $\sqrt{\frac{ 2\pi |\vxref(\vxo)-\vxo| }{ \ti k }}$ is the compensation factor for the \emph{secondary source dimensionality mismatch}.
	Comparison with \eqref{eq:HF_approx:2D_vs_3D_GF} indicates that the compensation factor approximates the frequency response and attenuation factor of the 2D Green's function in terms of the 3D Green's function.
	Obviously, the attenuation factors can be matched only at a particular distance from a given SSD element $\vxo$, chosen to be at the reference position $\vxref(\vxo)$.
	On the other hand, the frequency response compensation term ensures the flat frequency response of the SSD: a 2D contour of point sources exhibits the frequency response of $\sim \frac{1}{\sqrt{\ti k}}$, which along with the normal derivative term would result in a transfer function of $\sim \sqrt{\ti k}$, that has to be compensated for.
%
\item The virtual source compensation factor resolves the \emph{virtual source dimensionality mismatch}, correcting the virtual source attenuation factor from a 2D to a 3D one.
It is assumed that the general relationship between a 2D and a 3D sound field, generated by the same planar source distribution at $z = 0$ reads as
\begin{equation}
P_{3\text{D}}(\vx,\omega) = \sqrt{\frac{\ti k}{2\pi}}
\frac{P_{2\text{D}}(\vx,\omega)}{\sqrt{\Rv^P(\vx)}},
\label{Eq:SFS_thory:2D_3D_relation}
\end{equation}
at $x = \posvec{3}{x}{y}{0}$.
This is a straightforward generalization of \eqref{eq:HF_approx:2D_vs_3D_GF} towards general sound fields.
Expressing a 2D sound field at $\vxref(\vxo)$ in terms of the 2D Kirchhoff integral and rewriting in terms of the corresponding 3D sound fields by applying \eqref{Eq:SFS_thory:2D_3D_relation} leads to the virtual source correction factor under discussion.
A detailed explanation for the virtual source dimensionality compensation is given for the special case of a virtual point source in \cite{Voelk2012}.
\end{itemize}
Referencing the synthesis therefore can be interpreted physically as adjusting both attenuation correction factors for each SSD element to be amplitude correct on the reference curve, performed by prescribing a frequency independent curvature correction factor.

% TODO: TD WFS
%In practical applications the driving functions are implemented in the temporal domain.
%Taking the inverse Fourier transform of \eqref{Eq:SFS_theory:25D_WFS_driv_fun} yields the 2.5D WFS driving functions in the time domain, reading
%\begin{equation}
%d(\vxo, t) = w(\vxo) 
%\sqrt{ 8\pi \dref(\vxo)}
%\sqrt{\frac{\ti \omega}{c} }\hat{k}_{\mathrm{n}}^P(\vxo) 	P(\vxo,\omega),
%\end{equation}

%
\begin{figure}  
\small
  \begin{minipage}[c]{0.64\textwidth}
	\begin{overpic}[width = 1\columnwidth ]{Figures/SFS_theory/25D_WFS_general.png}
	\small
	\put(2,53){(a)}
	\put(2,1){(b)}
	\end{overpic}   \end{minipage}\hfill
	\begin{minipage}[c]{0.35\textwidth}
    \caption{2.5D synthesis of a 3D point source located at $\vxs = \posvec{3}{0.4}{2.5}{0}$, radiating at $f_0 = 1.5~\mathrm{kHz}$.
    Figure (a) depicts the real part of the synthesized field, (b) presents the absolute error of synthesis in a logarithmic scale.
	The reference curve was defined by simple rescaling of the SSD contour, however, an arbitrary convex reference contour could be chosen.
	The active arc of the SSD is denoted by solid black line, and the inactive part with dotted by black line.
	The reference position on the reference curve for each active SSD element is evaluated numerically.
	Obviously in the present geometry there exist secondary sources, for which no unique reference position can be found.
	In order to ensure smooth driving function and avoid truncation artifacts for these SSD positions the referencing function is extrapolated.
    }
\label{fig:SFS_theory:25D_WFS_generals}   \end{minipage}
\end{figure}
\vspace{3mm}
The introduced driving function is capable of the synthesis of arbitrary sound fields applying arbitrary shaped convex SSDs, referencing the synthesis to an arbitrary reference curve. 
The result of such a general 2.5D WFS scenario is presented in Figure \ref{fig:SFS_theory:25D_WFS_generals}.
As the image depicting the synthesis error indicates: on those part of the reference curve for which a stationary point pair can be found on the SSD amplitude correct synthesis is ensured, as the error exhibits a minimum.
%

If a parametrization of the SSD contour and the reference curve along with an analytical virtual source model is known, the referencing function can be expressed analytically, resulting in closed form driving function specific to the SSD and the referencing contour. 
The following two examples are presented in order to demonstrate the analytical application of the presented driving function.

% TODO: OUT TO INTRODUCTION (to the entire thesis or for this chapter)
%The presented referencing approach not only allows the derivation of referencing functions for arbitrary SSD-reference curve geometries, but also the analysis of former WFS approaches in the aspect of the positions of amplitude correct synthesis.
%These former approaches include traditional WFS, which optimizes the synthesis of a point source to a reference line as discussed via the following example \cite{Berkhout1993:Acoustic_control_by_WFS,  Start1997:phd, Verheijen1997:phd}, and revisited WFS formulation, which applies a target field independent constant referencing function without taking the virtual source dimensionality mismatch into consideration \cite{Spors2008:WFSrevisited}.
%The further analysis of the latter approach is not included in the present thesis, a thorough discussion of the topic can be found in \cite{Firtha2016}.

\subsection*{Application example \#1: Synthesis of a 3D point source applying a linear SSD}

As a first example assume an infinite linear SSD located at $\vxo = \posvec{3}{x_0}{0}{0}$.
The reference contour is set to be an infinite line parallel to the SSD, located at $\vxref = \posvec{3}{x_0}{y_{\mathrm{ref}}}{0}$.
This geometry has a distinctive role in the field of sound field synthesis, being the arrangement for which traditional WFS was first formulated \cite{Berkhout1988, Berkhout1993:Acoustic_control_by_WFS, Start1997:phd, Verheijen1997:phd}.
In this arrangement the driving function may be derived from the vertical SPA of the Neumann Rayleigh integral, which describes a sound field precisely in terms of a planar single layer potential.
Therefore application of a linear SSD involves the least approximations, avoiding errors present in the Kirchhoff approximation.
Furthermore, choosing a reference line parallel to the SSD ensures the existence of a unique reference position for each SSD element, therefore amplitude correct synthesis may be ensured over the entire reference line.
Finally, explicit solution can be found directly for this special geometry as described in the following section.

Generally speaking, the evaluation of the referencing function \eqref{Eq:SFS_theory:Referencing function} requires the definition of the Green's function's principal radius, centered at $\vxo$ and measured at the reference position.
For the special case of the Green's function this is given by the simple Euclidean distance of the two points.
Hence, the determination of the distance between the reference position and the corresponding secondary sources is required by solving equation \eqref{Eq:SFS_theory:WFS_General_Stat_pos} for $|\vxref(\vxo)-\vxo|$, termed the \emph{reference distance}.
The terminology indicates that it denotes the distance measured from the individual secondary sources at which the synthesis is optimized.

In the arrangement under discussion both $\vxo$ and $\vxref$ are lying along infinite parallel lines, with the $y$-coordinates of both curves fixed to constant, thus for the second coordinates of equation \eqref{Eq:SFS_theory:WFS_General_Stat_pos}
\begin{equation}
y_{\mathrm{ref}} = y_0 + \hat{k}_y^P(\vxo) |\vxref(\vxo)-\vxo|
\end{equation}
must hold.
With $y_0=0$ the above equation yields the reference distance for the present geometry
\begin{equation}
|\vxref(\vxo)-\vxo| = \frac{y_{\mathrm{ref}}}{\hat{k}_y^P(\vxo)} = \Rv^G(\vxref(\vx)-\vxo). 
\end{equation}
Substitution into \eqref{Eq:SFS_theory:25D_WFS_driv_fun_ver_2} yields the linear 2.5D WFS driving function, ensuring amplitude correct synthesis of an arbitrary sound field on a reference line, reading as
\begin{equation}
D(\vxo, \omega) = 
\sqrt{\frac{8\pi}{\ti k}}\sqrt{\Rv^P(\vxo)}\sqrt{ \frac{\yref}{\yref + \Rv^P(\vxo) \hat{k}_y^P(\vxo)}}
\ti k_y^P(\vxo) P(\vxo,\omega).
\end{equation}

\begin{figure}
\centering
	\begin{overpic}[width = 1\columnwidth ]{Figures/SFS_theory/25D_WFS_linear_SSD.png}
	\small
	\put(0, 0){(a)}
	\put(47,0){(b)}	
	\end{overpic}   
    \caption{2.5D synthesis of a 3D point source located at $\vxs = \posvec{3}{0}{-2}{0}$, radiating at $f_0 = \mathrm{kHz}$ with the reference line set at $y_{\mathrm{ref}} = 1.5~\mathrm{m}$.
    Figure (a) depicts the real part of the synthesized field, (b) shows the error of synthesis.
    Based on the equivalent scattering interpretation of the synthesis the discrepancy between the synthesized field and the virtual field at $y<0$ can be interpreted as the field of a point source reflected from a planar scatterer surface. 
    Due to the problem symmetry the scattered field is given amplitude correctly along $y = - y_{\mathrm{ref}}$.
    }
\label{fig:SFS_theory:25D_WFS_linear_ssd}  
\end{figure}
Finally, expressing the pressure field and the normalized local wavenumber vector of the virtual point source results in the virtual source-SSD shape-receiver shape specific driving function
\importanteq{Linear WFS point source driv. fun.}{
D(\vxo, \omega) =  -\frac{1}{4\pi}
\sqrt{ \frac{8\pi}{\ti k} }
\sqrt{ \frac{y_{\mathrm{ref}}}{y_{\mathrm{ref}} -y_s } }
\ti k y_s \frac{\te^{-\ti k |\vxo-\vxs|}}{|\vxo-\vxs|^{\frac{3}{2}}}.
\label{eq:SFS_theory:WFS_point_source}
}
This result is equivalent with the traditional WFS driving function of a point source \cite[(2.27)]{Verheijen1997:phd}, \cite[(3.16)\&(3.17)]{Start1997:phd}, and furthermore identical to the farfield/high frequency approximated explicit solution presented in the next section \cite[(25)]{Spors2010:analysis_and_improvement}, \cite[Ch. 2.3]{Schultz2016}. 
The result of synthesis is depicted in Figure \ref{fig:SFS_theory:25D_WFS_linear_ssd} confirming that by applying the derived driving function amplitude correct synthesis is ensured along the reference line.

Taking the temporal inverse Fourier transform of the driving function weighted by $S(\omega)$ yields the temporal driving function for a virtual point source with the source time history of $s(t)$.
%\begin{equation}
%d(\vxo, t) =  -\frac{1}{2\pi} \int_{-\infty}^{\infty}
%\sqrt{\frac{\ti k}{2\pi}}
%\sqrt{\frac{y_{\mathrm{ref}}}{y_{\mathrm{ref}} -y_s } } y_s \frac{\te^{\ti \omega ( t - \frac{|\vxo-\vxs|}{c} )}}{|\vxo-\vxs|^{\frac{3}{2}}} \td \omega.
%\end{equation}
Exploiting the shift theorem and the associativity of convolution yields the temporal driving function
\importanteq{Linear time domain point source driv. fun.}{
d(\vxo, t) =  -
\sqrt{\frac{1}{2\pi c}}
\sqrt{\frac{y_{\mathrm{ref}}}{y_{\mathrm{ref}} -y_s } } y_s \frac{ s^{\nshortmid}_t( t - \frac{|\vxo-\vxs|}{c} )}{|\vxo-\vxs|^{\frac{3}{2}}},
}
where $s^{\nshortmid}_t( t ) = h(t) \ast_t s( t )$, with $\ast_t$ denoting convolution in the time domain.
The source time history is pre-equalized with a filter, exhibiting the frequency response of $H(\omega) = \sqrt{\ti \omega}$ being a half-differentiator. 
The filter compensates the frequency response of a 2D SSD contour as the part of the secondary source compensation factor, as discussed above, and is therefore always the part of the 2.5D diving function for an arbitrary target sound field and for an arbitrary SSD shape.
The impulse response of the SSD compensation filter can be expressed by differentiating the half-integrator's impulse response as given in \cite{Deregowski1983}
\begin{equation}
h(t) = \frac{\delta(t)}{\sqrt{\pi t}} - \frac{1}{2} \frac{\theta(t)}{t^{3/2}},
\end{equation}
where $\theta(t)$ is the Heaviside step function.
Practical implementation of this prefilter is discussed in details in \cite{Schultz2013:IIR_prefilters} applying IIR filters, while in \cite[Sec. 2.5]{Schultz2016} the ideal FIR filter coefficients are given analytically.

\subsection*{Application example \#2: Synthesis of a plane wave applying a circular SSD}

As a second example the synthesis of a plane wave applying a circular SSD, centered at the origin with the radius of $R_{\mathrm{SSD}}$ is presented.
The synthesis is referenced to a concentric circle inside the SSD with the radius of $R_{\mathrm{ref}}$.
For this geometry the explicit driving function is also available \cite{Ahrens2008:Analytical_Circ_Spherical_SFS, Ahrens2009:circularSSD_mismatch, Ahrens2009:circular25D_SFR}, which are however not discussed in details in the present thesis.

Again, the system of equations describing the reference distance for each SSD element is given by
\begin{align}
\vxref(\vxo) &= \vxo + \hat{\vk}^P(\vxo) |\vxref(\vxo)-\vxo|,
\\
|\vxref(\vxo)| &= R_{\mathrm{ref}}.
\end{align}
Expressing the reference distance leads to a second order equation.
By exploiting that $|\vxo| = R_{\mathrm{SSD}}$, $|\hat{\vk}^P(\vxo)| = 1$ and taking only the smaller root into consideration---corresponding to the closer arc of the reference circle to the actual SSD position---yields the reference distance
\begin{equation}
|\vxref(\vxo)-\vxo| = R_{\mathrm{SSD}} \left( \hat{k}^P_r(\vxo) + \sqrt{ \hat{k}^P_r(\vxo)^2 + \left( \frac{R_{\mathrm{ref}}}{R_{\mathrm{SSD}}} \right)^2 - 1 } \right),
\label{eq:SFS_theory:pw_circ_ref}
\end{equation}
with $\hat{k}^P_r(\vxo)$ denoting the radial component of the normalized wavenumber vector.
Applying this reference distance in the general 2.5D WFS driving function \eqref{Eq:SFS_theory:25D_WFS_driv_fun} allows the synthesis of an arbitrary sound field referenced on a reference circle inside the SSD.

Assume the special case of a virtual 2D plane wave, propagating parallel to the synthesis plane described by the wavenumber vector $\vk^{\mathrm{PW}} = \posvec{3}{k_x^{\mathrm{PW}}}{k_y^{\mathrm{PW}}}{0}$.
For a 2D sound field invariant along the vertical dimension the vertical wavefront curvature is zero ($\phi^{''P}_{zz}(\vxo) = 0$) and the referencing function is the reference distance itself.
The driving function synthesizing a 2D plane wave is then given as
\begin{equation}
\label{eq:SFS_theory:WFS_plane_wave}
D(\vxo, \omega) = -w(\vxo) 
\sqrt{\frac{8\pi}{\ti k}}\sqrt{|\vxref(\vxo)-\vxo|}
\ti k_r^{\mathrm{PW}}(\vxo) 	\te^{-\ti \left< \vk^{\mathrm{PW}} \cdot \vxo \right> },
\end{equation}
with the reference distance given by \eqref{eq:SFS_theory:pw_circ_ref}.
As an application of this driving function a simple example is depicted in Figure \ref{fig:SFS_theory:25D_WFS_circular_ssd} for the synthesis of a harmonic plane wave.

\begin{figure}
\centering
	\begin{overpic}[width = 1\columnwidth ]{Figures/SFS_theory/25D_WFS_circular_SSD.png}
	\put(0,1){(a)}
	\put(48,1){(b)}
	\end{overpic}   
    \caption{2.5D synthesis of a 2D plane wave with the angular frequency $f_0 = 1~\mathrm{kHz}$ propagating into the direction $\vk^{\mathrm{PW}} = \posvec{3}{k_x^{\mathrm{PW}} }{0}{0}$.
    The SSD is a circular one, with the radius of $R_{\mathrm{SSD}} = 2~\mathrm{m}$.
    The reference curve is a circle with the radius of $R_{\mathrm{ref}} = 1.5~\mathrm{m}$.
    Figure (a) depicts the real part of the synthesized field, (b) shows the error of synthesis.
    }
\label{fig:SFS_theory:25D_WFS_circular_ssd}  
\end{figure}

In order to find the driving signal for the synthesis of a plane wave carrying a broadband excitation time history $s(t)$, \eqref{eq:SFS_theory:WFS_plane_wave} is inverse Fourier transformed weighted by the excitation spectrum, resulting in
\begin{equation}
d(\vxo, t) = -w(\vxo) 
\sqrt{\frac{8 \pi }{c	}|\vxref(\vxo)-\vxo|}  \,
\hat{k}_r^{\mathrm{PW}}(\vxo)  h(t) \ast_t s( t - \frac{1}{c}\left< \hat{\vk}^{\mathrm{PW}} \cdot \vxo \right>),
\end{equation}
where $h(t) = \mathcal{F}^{-1}_{\omega}\left\{ \sqrt{\ti \omega} \right\}$ is the SSD compensation filter, performing half-derivation on the time history and $\hat{\vk}^{\mathrm{PW}} = \posvec{3}{\hat{k}^{\mathrm{PW}}_x}{\hat{k}^{\mathrm{PW}}_y}{0}$ is a unit vector pointing into the plane wave propagation direction.

\section{Explicit solution: Spectral Division Method}

The explicit solution for the general sound field synthesis problem aims at the direct solution of the inverse problems, described by integral equations \eqref{Eq:Theory:3D_SFS} and \eqref{Eq:Theory:25D_SFS}.

Generally speaking the explicit methodology utilizes compact operator theory by exploiting that integral \eqref{Eq:Theory:3D_SFS} constitutes a compact Fredholm operator with the kernel being the Green's function \cite{MorseFeshbach1953, Ahrens2012}.
Such an operator and the involved acoustic fields can by expanded into the series of orthogonal eigenfunctions of the wave equation on a control surface, that form a complete basis of the required solution.
The inverse problem can be straightforwardly solved for the driving function expansion coefficients by a comparison of the corresponding eigenvalues, as long as none of the expansion coefficients of the operator kernel is zero (otherwise the problem is termed \emph{ill-conditioned}).
Finally the explicit analytical solution is found for the driving function as an infinite sum of the weighted basis functions.
The method is often referred to as \emph{mode-matching} solutions, since the eigenfunctions of a given geometry are termed the \emph{modes}.
This solution is unique for general enclosures
%\footnote{In contrary sound field control utilizing the Kirchhoff-Helmholtz formulation would be non-unique on the eigenfrequencies of the enclosure due to resonance phenomena. \fscom{maybe its worth to add: full KHI is unique, single layer can be made unique with CHIEF points like BEM}}
and also for the (strictly speaking) non-enclosing planar case as shown in \cite{Zotter2013:uniqueness} and \cite{Fazi2010}, respectively.
In the latter case the compact operator degenerates to the continuous eigenvalue domain instead of countable eigenvalues as presented in the followings.

The determination of the appropriate eigenfunctions for a general geometry is a tough challenge.
For spherical and circular geometries spherical and circular harmonics form the demanded basis functions. 
For a rigorous treatment for mode-matching SFS using spherical and circular SSDs see \cite{Ahrens2008:Analytical_Circ_Spherical_SFS, Ahrens2009:circularSSD_mismatch, Ahrens2009:circular25D_SFR, Zotter2009phd, Ahrens2010phd, Ahrens2012, Schultz2014:Comparing_approaches} and \cite{Koyama2014, Koyama2014:phd} for the cylindrical solution.
In the present thesis only the planar and linear geometries are investigated in details. %\fscom{arbitrary SSD are rather treated with Kirchhoff, SPA and HF approximations}
%However, under the validity of the tangent plane approximation---i.e. under high frequency assumptions with small SSD curvatures---the SSD surface/contour can be considered locally planar/linear and the spatial form of the explicit driving functions are valid for arbitrary shaped SSDs.
This 

%\begin{figure}
%	\centering
%	\begin{overpic}[width = .85\columnwidth]{Figures/SFS_theory/planar_linear_geometry.png}
%	\footnotesize
%	\put(0, 0){(a)}
%	\put(45,0){(b)}
%	\end{overpic}
%\caption{Secondary source distribution geometry .}
%	\label{Fig:Theory:planar_linear_geometry}
%\end{figure}

\subsection{3D Spectral Division Method}

Assume an infinite planar SSD located at $\vxo = \posvec{3}{x_0}{0}{z_0}$, degenerated from the geometry introduced for the Rayleigh integrals in the previous chapter, shown in Figure \ref{Fig:Theory:Rayleigh_geometry}.
The half-space of synthesis is chosen to be at $y>0$, therefore all the virtual sources are assumed to be located at $y<0$.
The synthesized field in this geometry is given by a Fredholm integral of the first kind 
\begin{equation}
P(\vx,\omega) = \iint_{-\infty}^{\infty} D(x_0,z_0,\omega) G(x-x_0,y,z-z_0, \omega) \td x_0 \td z_0 = D(x,z,\omega)\ast_{x,z} G(x,y,z,\omega),
\end{equation}
describing a continuous convolution along the SSD plane.
Here $G(x,y,z,\omega)$ denotes the sound field of a secondary source element placed at the origin and $\ast_{x,z}$ denotes convolution along the $x$ and $z$ dimensions.

For this geometry the orthogonal basis is given by the continuous set of exponentials, and the decomposition of the involved quantities is given by a double Fourier transform \cite{Arfken2005, Ahrens2012, Schultz2014:Comparing_approaches}, with the physical interpretation of a plane wave decomposition as discussed in Section \ref{Sec:thoery:angular_Spectrum}.
Applying the convolution theorem to the angular spectrum representation, the convolution transforms into a multiplication \cite{Girod2001}:
\begin{equation}
\tilde{P}(k_x,y,k_z, \omega) = \tilde{D}(k_x,k_z, \omega) \cdot \tilde{G}(k_x,y,k_z, \omega).
\end{equation}
%
The expansion coefficients are therefore obtained by a comparison of the spectral coefficients and the driving function in the wavenumber and the spatial domain takes the form:
\begin{equation}
\tilde{D}(k_x,k_z,\omega) = \frac{\tilde{P}(k_x,y,k_z, \omega)}{ \tilde{G}(k_x,y,k_z, \omega)} = 
\frac{\mathcal{F}\left\{ P(\vx,\omega) \right\} }
{  \mathcal{F}\left\{ G(\vx,\omega) \right\} },
\label{Eq:Theory:Dkxkz}
\end{equation}
\importanteq{3D SDM driving fun.}{
D(x_0,z_0,\omega) = \frac{1}{4\pi^2} \iint_{-\infty}^{\infty} \frac{\tilde{P}(k_x,y,k_z, \omega)}{ \tilde{G}(k_x,y,k_z, \omega)} \te^{-\ti (k_x x_0 + k_z z_0)} \td k_x \td k_z,
\label{Eq:Theory:Dkx_inverse_Fourier}
}
respectively.
Since the driving function spectrum is yielded by a division in the spectral domain the approach is termed the \emph{Spectral Division Method} \cite{Ahrens2010a, Ahrens2011:icassp, Ahrens2010:Ambisonics_w_planar_linear, Ahrens2012:Ambisonics_for_planar_linear}.

Substituting the $k_x-k_z$ representation of the 3D Green's function given by \eqref{eq:HF_approx:Greens_2D_Spectrum} the driving function \eqref{Eq:Theory:Dkx_inverse_Fourier} reads as
\begin{equation}
D(x_0,z_0,\omega) = \frac{1}{4\pi^2} \iint_{-\infty}^{\infty} 2\ti k_y \frac{\tilde{P}(k_x,y,k_z, \omega)}{ \te^{ -\ti k_y  y  } } \te^{-\ti (k_x x_0 + k_z z_0)} \td k_x \td k_z.
\end{equation}
with $k_y$ defined by \eqref{eq:theory:k_y_definition}.	
Expressing the target field spectrum by extrapolating from the plane $y = 0$ according to \eqref{Eq:Theory:Wave_field_extrapolation}---i.e. as 
$
\tilde{P}(k_x,y,k_z, \omega) = \tilde{P}(k_x,0,k_z, \omega)  \te^{ -\ti k_y  y }
$---
the exponential pressure propagators cancel out and the driving function becomes independent from the $y$-coordinate. 
The driving function in the wavenumber domain therefore reads as
\begin{equation}
\tilde{D}(k_x,k_z,\omega) = 2\ti k_y \tilde{P}(k_x,0,k_z,\omega) = -2 \left. \frac{\partial}{\partial y} \tilde{P}(k_x,y,k_z,\omega) \right|_{y = 0},
\label{Eq:Theory:Planar_explicit_driv_fun}
\end{equation}
for which it was exploited that according to \eqref{eq:Theory:Fourier_diff} multiplication by $\ti k_y$ represents differentiation along the $y$-dimension.
Straightforwardly, the explicit expression of the driving function in the spatial domain is obtained by the corresponding inverse Fourier transform according to \eqref{Eq:Theory:Dkx_inverse_Fourier}:
\begin{equation}
D(x_0,z_0,\omega) = -2 \left. \frac{\partial}{\partial y} P(\vx,\omega) \right|_{y = 0}.
\label{Eq:Theory:Planar_explicit_driv_fun_spatial}
\end{equation}
The planar explicit driving function is thus equivalent to the implicit solution, in a planar geometry provided by the Rayleigh integral.
The coincidence of the explicit and implicit driving functions is a consequence of the uniqueness of the problem in the present geometry.
It is also indirectly proven that the wavefield extrapolation equations are the spectral domain representations of the Rayleigh integrals.

However, an important difference between the implicit and explicit solution exists: until \eqref{Eq:Theory:Dkx_inverse_Fourier} the present method does not pose any constraints on the actual form of the Green's function. 
Theoretically an arbitrary transfer function may be assigned for the secondary sources. 
As long the spectrum of the transfer function does not exhibit zeros unique driving function may be derived applying the explicit methodology.

If the secondary sources are 3D point sources the following physical interpretation can be assigned to the explicit solution: 
as it was stated in Section \ref{Sec:HF_approx:1D_Greens} a planar distribution of point sources with a harmonic spatial distribution described by $k_x, k_z$ radiate plane waves with the same wavenumber components and a wavenumber/direction dependent amplitude factor $\frac{1}{2 \ti k_y}$ (c.f. \eqref{eq:HF_approx:Greens_2D_Spectrum}, degenerating at $k_x = k_z = 0$ to the 1D Green's function).
The driving function \eqref{Eq:Theory:Planar_explicit_driv_fun} thus compensates the planar SSD's response for the synthesis of a single plane wave component.
Finally, the explicit driving function for an arbitrary virtual field is found as the sum of the individual plane wave driving function weighted by the virtual field' plane wave expansion coefficients.


\subsection*{Application example: Synthesis of a 3D point source using a planar SSD}
%\fscom{well done, I've tried it for the Acta 2014 article, but could not go any further, have to check your solution in detail. However, it might be worth that this issue from the acta in the appendix is solved here!}
\begin{figure}
	\centering
	\begin{overpic}[width = 1\columnwidth]{Figures/SFS_theory/Planar_SDM.png}
	\small
	\put(0, 0){(a)}
	\put(47,0){(b)}
	\end{overpic}
\caption{
Synthesis of a virtual point source using a planar SSD applying the SDM driving function.
The SSD is located at $\vxo = [x_0,\ 0,\ z_0]^{\mathrm{T}}$ denoted by solid black line. 
The virtual source is located at $\vxs = [0,\ -2,\ 0]^{\mathrm{T}}$ oscillating at $f_0 = 1 ~\mathrm{kHz}$.
The figures depict the real part of the synthesized field (a) and the deviation from the target sound field (b) measured at $z=0$.}
	\label{Fig:Theory:monopole_synthesis_by_planar_SDM}
\end{figure}

The application of the planar explicit solution is presented via the synthesis of a 3D virtual point source positioned at $\vxs = \posvec{3}{x_s}{y_s}{z_s}$ behind the SSD plane, located at $\vxo = \posvec{3}{x_0}{0}{z_0}$.
The wavenumber domain representation of the driving function is obtained by substituting the angular spectrum of a 3D point source into \eqref{Eq:Theory:Dkxkz} with applying the Fourier transform shift theorem\footnote{This is the corrected version of \cite[eq. (A11)]{Schultz2014:Comparing_approaches}.}
\begin{equation}
\tilde{D}(k_x,k_z,\omega) =  \frac{-\frac{\ti}{2} \frac{ \te^{-\ti k_y ( y - y_s )} }{ k_y} \te^{\ti (k_x x_s +k_z z_s)} }{-\frac{\ti}{2} \te^{-\ti k_y  y } / k_y   } = \te^{-\ti k_y y_s}\te^{\ti (k_x x_s +k_z z_s)}.
\label{Eq:Theory:Monopole_SDM_planar_driv_fun}
\end{equation}
The double inverse Fourier transform can be carried out analytically by taking the $y$-derivative of the Weyl's integral representation of the Green's function (See \cite{Lalor1969} or \cite[(2.65)]{Williams1999}):
\begin{equation}
\frac{\partial}{\partial y} G(\vxo - \vxs,\omega ) = 
\frac{1}{4\pi^2} \iint_{-\infty}^{\infty} -\frac{1}{2} \te^{ -\ti k_y  ( y - y_s ) }
\te^{\ti (k_x x_s + k_z z_s)} \te^{-\ti (k_x x_0 + k_z z_0)} \td k_x \td k_z
.
\label{Eq:Theory:Weyls_derivative}
\end{equation}
Comparing \eqref{Eq:Theory:Monopole_SDM_planar_driv_fun} and \eqref{Eq:Theory:Weyls_derivative} it is revealed that the driving function in the spatial domain is given by
\begin{equation}
D(x_0,z_0,\omega) = -2 \frac{\partial}{\partial y} \left. G(\vxo - \vxs,\omega )\right|_{y = 0} = -\frac{y_s}{2\pi} \left( \frac{1}{|\vxo-\vxs|} + \ti k\right) \frac{\te^{-\ti k |\vxo-\vxs|}}{|\vxo-\vxs|^2},
\end{equation}
which is in agreement with equation \eqref{Eq:Theory:Planar_explicit_driv_fun_spatial}.

The result of synthesizing the steady-state field of a point source is illustrated in Figure \ref{Fig:Theory:monopole_synthesis_by_planar_SDM}. 
In the target half space $y>0$ perfect synthesis is achieved, as it is indicated in Figure \ref{Fig:Theory:monopole_synthesis_by_planar_SDM} (b), depicting the discrepancy between the synthesized and the target sound field. 
Obviously, the figure also presents the result of 3D planar WFS of a spherical wave without applying the high frequency gradient approximation.


\subsection{2.5D Spectral Division Method}
\label{Sec:25D_SDM}

As the geometry for the derivation of the 2.5D explicit driving function assume an infinite linear distribution of secondary point sources, located at $\vxo = \posvec{3}{x_0}{0}{0}$.
The synthesized field in this arrangement reads as
\begin{equation}
P(x,y,z,\omega) = \int_{-\infty}^{\infty} D(x_0,\omega) G(x-x_0,y,z,\omega) \td x_0.
\label{Eq:SFS_Theory:linear_synth_field_spatial}
\end{equation}
Similarly to the planar case the basis functions for a linear SSD are given by exponentials:
by realizing that the above equation is a convolution along the $x$-axis, the convolution can be transformed into a multiplication by means of a spatial forward Fourier transform
\begin{equation}
\tilde{P}(k_x,y,z, \omega) = \tilde{D}(k_x,\omega) \cdot \tilde{G}(k_x,y,z, \omega).
\label{Eq:SFS_Theory:linear_synth_field_spectral}
\end{equation}
The driving function spectrum is then obtained as a spectral ratio
\begin{equation}
\tilde{D}(k_x,\omega) = \frac{\tilde{P}(k_x,y,z, \omega)}{\tilde{G}(k_x,y,z, \omega)} = \frac{\mathcal{F}_x \left\{ P(\vx,\omega) \right\}}{\mathcal{F}_x \left\{ G(\vx,\omega) \right\}},
\label{Eq:SFS_Theory:LinearSDM_spectral}
\end{equation}
and the frequency domain driving function is obtained as the spatial inverse Fourier transform with respect to $k_x$
\begin{equation}
D(x_0,\omega) = \frac{1}{2\pi} \int_{-\infty}^{\infty} \frac{\tilde{P}(k_x,y,z, \omega) }{\tilde{G}(k_x,y,z, \omega)} \te^{-\ti k_x x_0} \td k_x.
\label{Eq:Theory:LinearSDM1}
\end{equation}
Again, theoretically the transfer function may describe the field of an arbitrary sound source, as long as it does not exhibit zeros in order to keep the problem well-conditioned.

\vspace{3mm}
Unlike the planar case the present driving function depends on the listener position: Equation \eqref{Eq:Theory:LinearSDM1} may be solved only for positions on the surface of a cylinder with fixed radius $d = \sqrt{y^2 + z^2}$ \cite[p.~60.]{Ahrens2010phd}.
This is a direct consequence of the fact that the pressure of an arbitrary 3D sound field measured on the SSD does not determine completely the pressure on the reference line---and vice versa---.
Furthermore, an infinite line source---i.e. the SSD---can only radiate wavefronts with cylindrical symmetry as it was discussed in details in \ref{sec:greens_function_spectrum}.
Phase correct synthesis therefore can be achieved only in a plane containing the SSD in which the radial wavenumber of the synthesized field and the target field coincide. 
Amplitude correct synthesis is ensured in this plane at a distance $\dref = \sqrt{y^2 + z^2}$, for which the driving function is calculated.

For practical applications the plane of synthesis is chosen to be the horizontal plane $z=0$, requiring that for the virtual field $k_z(x,y,0) = 0$ holds.
The driving function thus reads as
\importanteq{2.5D SDM driving function}{
D(x_0,\omega) = \frac{1}{2\pi} \int_{-\infty}^{\infty} \frac{\tilde{P}(k_x,\yref,0, \omega) }{\tilde{G}(k_x,\yref,0, \omega)} \te^{-\ti k_x x_0} \td k_x.
\label{Eq:Theory:Linear_SDM}
}
In this geometry amplitude correct synthesis is restricted to the \emph{reference line} by setting $y = \yref$.

Similarly to the 3D case the following physical interpretation can be assigned to the 2.5D explicit solution:
Given an infinite distribution of point sources along the $x$-axis with a harmonic spatial distribution described by $k_x$, the radiated sound field is given by
\begin{multline}
\int_{-\infty}^{\infty} G(x - x_0,y,z) \te^{\ti k_x x_0} \td x_0 = \hat{G}(k_x,y,z,\omega) \te^{-\ti k_x x} = \\
=  -\frac{\ti}{4} H_0^{(2)}\left( \sqrt{\left( \frac{\omega}{c} \right)^2-k_x^2} \sqrt{y^2+z^2} \right)  \te^{-\ti k_x x},
\end{multline}
at $x=0$ resulting in the 2D Green's function as discussed in Section \ref{sec:greens_function_spectrum}.
Such a source radiates cylindrical symmetric sound fields with conical wavefronts as depicted in Figure \eqref{Fig:Theory:greens_stat_pos} (a). 
Along a fixed reference line at $z=0$ the SSD reproduces a harmonic spatial distribution $\te^{-\ti k_x x}$ attenuated approximately by $\frac{1}{\sqrt{k_y}|y_{\mathrm{ref}}|}$, corresponding to attenuating plane waves with $k_z=0$.
Therefore, the wavenumber domain driving function ensures the compensation of the linear SSD response for the synthesis of a single plane wave component propagating in the plane of synthesis.
Obviously, for sound fields that can be expanded into the series of plane waves with $k_z=0$ the driving function is the weighted sum of the plane wave driving function, resulting in \eqref{Eq:Theory:Linear_SDM}.

It is worth noting that the analytic Fourier transform coefficients of the target sound field are available only for limited simple virtual source models. 
Even in these cases the inverse transform of the driving function can rarely be evaluated analytically, therefore numerical transforms are needed.
For a practical and optimized implementation of the SDM for an arbitrary target sound field refer to \cite{ahrens2013a:efficientSDM}.

\subsection*{Application example: Synthesis of a 3D point source using a linear SSD}

As an example for the 2.5D SDM the reproduction of a 3D point source is presented.
The virtual source is located at $\vxs = \posvec{3}{x_s}{y_s}{0}$, with $y_s<0$. 
The SSD is a linear set of 3D point sources, located along $\vxo = \posvec{3}{x_0}{0}{0}$.
The explicit driving function for a linear SSD is given by \eqref{Eq:Theory:Linear_SDM}. 
Substituting the 1D spectra of the virtual and the secondary point sources along with applying the Fourier shift theorem the driving function is given in the propagation region as
\begin{equation}
\hat{D}(k_x,\omega) = 
\frac{ -\frac{\ti}{4} H_0^{(2)} \left( \sqrt{ \left(\frac{\omega}{c}\right)^2 - k_x^2} |\yref - y_s| \right)  \te^{\ti k_x x_s} }
     { -\frac{\ti}{4} H_0^{(2)} \left( \sqrt{ \left(\frac{\omega}{c}\right)^2 - k_x^2} |\yref| \right)  }
,
\end{equation}
and the spatial inverse Fourier transform yields the spatial domain driving function, reading as
\importanteq{Linear SDM point source driv. fun.}{
\label{Eq:Theory:SDM_point_source}
D(x_0,\omega) = \frac{1}{2\pi} \int_{-\infty}^{\infty} 
\frac{  H_0^{(2)} \left( \sqrt{ \left(\frac{\omega}{c}\right)^2 - k_x^2} |\yref - y_s| \right)  }
     {  H_0^{(2)} \left( \sqrt{ \left(\frac{\omega}{c}\right)^2 - k_x^2} |\yref|       \right)  }
\te^{- \ti k_x (x_0 - x_s)}
\td k_x.}
The synthesized field using this driving function is depicted in \ref{Fig:Theory:monopole_synthesis_by_linear_SDM} (a). 
As it can be seen from Figure (b) displaying the discrepancy between the synthesized field and the target field, application of the explicit driving function ensures perfect synthesis on the reference line. 
In other points amplitude errors are present.

\begin{figure}
	\centering
	\begin{overpic}[width = 1\columnwidth]{Figures/SFS_theory/Linear_SDM.png}
	\footnotesize
	\put(0, 0){(a)}
	\put(45,0){(b)}
	\end{overpic}
\caption{Synthesis of a virtual point source employing a linear SSD applying the 2.5D SDM driving function.
The SSD is located at $\vxo = [x_0,\ 0,\ 0]^{\mathrm{T}}$, denoted by a solid black line. 
The virtual source is located at $\vxs = [0,\ -2,\ 0]^{\mathrm{T}}$ oscillating at $f_0 = 1 ~\mathrm{kHz}$. 
The reference line is at $\yref = 1.5~\mathrm{m}$.
The figure depicts the synthesized field at the synthesis plane ($z = 0$) with (a) depicting the real part of the synthesized field, (b) depicting the error of synthesis.}
	\label{Fig:Theory:monopole_synthesis_by_linear_SDM}
\end{figure}
%
As discussed in \cite{Spors10ahrens:analysis} the derived driving function spectrum can be simplified by applying the large-argument/asymptotic approximation of the Hankel function, given by \eqref{Eq:HF_approx:Hankel_asymptotic_form}.
The asymptotic form gives a fair approximation for \eqref{Eq:Theory:SDM_point_source} if $k_y |\yref| \gg 1$ holds, valid in the farfield of the SSD in front of the virtual source, where $k_y \gg k_x$ dominates the inverse transform.
Applying the Hankel function's approximation the inverse transform can be carried out analytically, resulting in
\begin{equation}
D(x_0,\omega) \approx \frac{1}{2} \sqrt{\frac{\yref}{\yref-y_s}} \ti \frac{\omega}{c} \frac{y_s}{|\vxo-\vxs|} H_1^{(2)}\left( \frac{\omega}{c} |\vxo-\vxs| \right).
\end{equation}
A further large-argument approximation of the first order Hankel function returns the 2.5D WFS driving function for a 3D point source referencing the synthesis on a reference line, given by \eqref{eq:SFS_theory:WFS_point_source}. 
This indicates that the implicit solution constitutes a high frequency approximation for the explicit solution in case of a virtual point source.
The equivalence of the SDM and 2.5D WFS referencing the synthesis of a virtual plane wave on a reference line was further discussed in \cite{Firtha2016, Schultz2016:DAGA,Schultz2016}.
In the following the general relation of the explicit solution and 2.5D WFS is investigated.

\subsection{Explicit solution in the spatial domain}

The determination of a single spectral coefficient for the explicit solution requires the knowledge of the entire target field over the boundary surface in order to perform the spectral decomposition. 
The explicit solution is therefore often termed a \emph{global solution}.
In contrary, the implicit solution requires the value of the local field variables only at the actual SSD position at which the driving function is to be expressed.
The implicit solution is thus referred to as a \emph{local solution}.
In the followings it is presented how the global solution can be approximated asymptotically by the application of the stationary phase method, resulting in an alternative local solution.

As it was discussed in Section \ref{Sec:SPA_for_Fourier} the stationary phase approximation allows the evaluation of forward and inverse Fourier integrals around stationary positions in the spatial and spectral domain.
The SPA therefore may be employed in order to give an approximate formulation for the 2.5D explicit driving function solely in the spatial domain.

The complete derivation is presented in details in Appendix \ref{App:SDM_SPA}, here only the result of the approximation is discussed.
The derivation consists of two main steps:
\begin{enumerate}
	%
	\item First the spectral driving function is expressed in an asymptotic form, resulting in \eqref{eq:hfapproxspectra}. 
	The calculus can be done by assuming that the involved spectra are obtained via the SPA of the corresponding forward Fourier transforms. 
	This step links the spectral coefficients to stationary positions on the reference line (see \eqref{eq:xP_xG_in_spatial_domain}).
	%
	\item It is followed by the inverse Fourier transform of the asymptotic spectral driving function.
	The evaluation of the inverse transform with the SPA relates the forward transform stationary positions to positions along the SSD.
\end{enumerate}
As the result of the derivation for a linear SSD located along $\vxo = \posvec{3}{x_0}{0}{0}$ the asymptotic SDM driving function, expressed entirely in the spatial domain reads as
\importanteq{2.5D explicit driv. fun.}{
\label{eq:SFS_theory:spatial_sdm}
D(x_0,\omega) \approx 
\sqrt{\frac{ \left| \phiGxx(\vxref(\vxo)-\vxo,\omega )\right|^2}{\left| \phiPxx(\vxref(\vxo),\omega) - \phiGxx(\vxref(\vxo)-\vxo,\omega)\right|}}
\sqrt{\frac{\ti}{2\pi}} 
\frac{P(\vxref(\vxo),\omega)}{G(\vxref(\vxo)-\vxo,\omega)}.
}
In the driving function $\vxref(\vxo) = \posvec{3}{x_{\mathrm{ref}}(x_0)}{y}{0}$ is the reference position for the SSD element at $x_0$, measured along a reference line, satisfying the relation
\importanteq{2.5D SDM reference position}{
\label{Eq:stationary_evaluation_points}
\vk^P(\vxref(\vxo)) = \vk^G(\vxref(\vxo) - \vxo).
}
Hence, in the explicit driving function for a given SSD coordinate $x_0$ the reference point $\vxref$ is found on the reference line, where the local propagation direction of the target field $P$ coincides with that of a point source positioned at $\posvec{3}{x_0}{0}{0}$. 
For an illustration refer to Figure \ref{fig:SFS_theroy:explicit_sol_stationary_points_2}.

\begin{figure}[t!]
\small
  \begin{minipage}[c]{0.6\textwidth}
%  \hspace{1cm}
	\small
%	\centering
%	\hspace{-30mm}
	\begin{overpic}[width = \textwidth ]{Figures/SFS_theory/explicit_sol_stationary_point.png}
	\put(96,30){$x$}
	\put(15,80){$y$}
	\put(60,29.5){$x_0$}
	\put(59,73){$\vk^P(\vxref(x_0))$}
	\put(75,59){$\vxref(x_0)$}
	\end{overpic}  \end{minipage}\hfill
	\begin{minipage}[c]{0.35\textwidth}
    \caption{
       Illustration of the stationary position $\vxref(x_0)$ for the evaluation of the spatial explicit driving function. 
	   For a given SSD position $x_0$ the reference (stationary) position is found on a given reference line, where the virtual field propagation direction coincides with that of the SSD element under discussion.
	   Under the validity of the Kirchhoff approximation this principle may be extended towards arbitrary SSD and reference contours. 
       } 
       \label{fig:SFS_theroy:explicit_sol_stationary_points_2}
  \end{minipage}
\end{figure}

The driving function \eqref{eq:SFS_theory:spatial_sdm} states that an arbitrary sound field may be synthesized by finding the positions along the reference line, where the propagation direction/wavefront of the target field matches the field of the actual secondary sources.
In this stationary position the driving function is obtained by the ratio of the target field and the actual SSD element, corrected by the factor, containing the wavefront radii/curvatures at the same position.
Therefore the explicit, global solution can be approximated by local wavefront matching.

One important fact is pointed out here: although having derived the above driving function in terms of a forward and an inverse spatial Fourier transform along a straight line, there is no restriction on the $y$-coordinate of the stationary point in \eqref{eq:SFS_theory:spatial_sdm} due to the local approximations involved: the $y$-coordinate might be $x_0$-dependent.
This means that an arbitrary referencing curve may be defined as $\vxref(x_0)$, and the driving function can be calculated by finding the stationary positions 
satisfying $k_x^P(\vxref(x_0)) = k_x^G(\vxref(x_0) - \vx_0)$ along this curve.
Evaluating the driving function in the stationary positions will result in amplitude correct synthesis along the reference curve. 
This means that the presented driving function is equivalent to the 2.5D WFS driving function with the important difference that here the target field needs to be evaluated on the reference curve.
Furthermore, within the validity of the Kirchhoff approximation the SSD does not necessarily need to be linear: the spatial explicit driving function can be applied using an arbitrary shaped SSD contour.
In that case \eqref{eq:SFS_theory:spatial_sdm_2} has to be evaluated with $\hat{k}_y^{P}(\vxref(x_0)) \rightarrow \hat{k}_{\mathrm{n}}^{P}(\vxref(\vxo))$, i.e. with the wavenumber component along the normal direction of the stationary SSD element.

With the considerations above and by expressing the second derivatives in terms of the principal radii according to \eqref{Eq:App:Hessian_inplane} the explicit driving function can be cast into the final form
\footnotesize
\importantmline{Detailed 2.5D explicit driv. fun.}{
\label{eq:SFS_theory:spatial_sdm_2}
D(\vxo,\omega) =
\underbrace{ \sqrt{\frac{\ti k}{2\pi \Rh^G(\vxref(\vxo)-\vxo)}} }_{{\substack{\text{SSD}\\\text{compensation}}}}
\underbrace{\sqrt{\frac{\Rh^P(\vxref(\vxo))}{\Rh^P(\vxref(\vxo))-\Rh^G(\vxref(\vxo)-\vxo) }}}_{{\substack{\text{virtual source}\\\text{compensation}}}
 \hspace{1mm} = \hspace{1mm} \sqrt{ \frac{\Rh^P(\vxref(\vxo))}{\Rh^P(\vxo)}} } 
\frac{\hat{k}_{\mathrm{n}}^{P}(\vxref(\vxo)) P(\vxref(\vxo),\omega)}{G(\vxref(\vxo)-\vxo,\omega)},
}
\normalsize
with $\vxo = \posvec{3}{x_0}{y_0}{0}$ and $\vxref(\vxo) = \posvec{3}{x_{\mathrm{ref}}(\vxo)}{y_{\mathrm{ref}}(\vxo)}{0}$ now denoting arbitrary SSD and reference contours.
The formulation implies the fact that similarly to the implicit solution, the explicit driving function also requires the derivative of the target field, measured on the reference position.
The driving function explicitly contains a virtual source compensation factor, compensating for the relative amplitude change of the virtual sound field between the SSD and the reference curve in terms of its horizontal principal radius.
Furthermore, the transfer function of the SSD contour---or more specifically the stationary SSD element--- is compensated regarding both its frequency response being a half-integrator and its attenuation factor.
%%\begin{equation}
%%\small
%%\sqrt{\frac{\ti k}{2\pi \Rh^G(\vxref(\vxo)-\vxo)}} \frac{\hat{k}_y^{P}(\vxref(\vxo))}{G(\vxref(\vxo)-\vxo,\omega)} = 
%%2 \sqrt{\frac{2 \pi |\vxref(\vxo)-\vxo|}{\ti k}} \frac{\ti k_y^{G}(\vxref(\vxo)-\vxo)  }{\te^{-\ti k |\vxref(\vxo)-\vxo|}} 
%%\end{equation} %FS:ok
%

%It should be noted that as an important difference from the 2.5D WFS driving functions, up to this point the above driving functions are independent from the problem dimensionality.
%The same driving function holds for the exemplary case of a 2D SFS scenario, in which case the SSD compensation factor becomes independent from the reference position.

\subsection*{Application example: Synthesis of a 3D point source using a linear SSD}

In the following a simple example is presented in order to demonstrate the validity of the spatial SDM driving function for the synthesis of a virtual 3D point source.
For the synthesis a linear secondary source distribution is applied, located at $\vxo = \posvec{3}{x_0}{0}{0}$.
The virtual source is positioned at $\vxs = [x_s,\ y_s,\ 0]^{\mathrm{T}}$ with $y_s < 0$, with the reference curve chosen to be a circle around the virtual point source with the radius of $R_{\mathrm{ref}}$.
Along with the equation describing the reference curve $\vxref(x_0) = \posvec{3}{x_{\mathrm{ref}}(x_0)}{y_{\mathrm{ref}}(x_0)}{0}$ the stationary points satisfy the following equations
\begin{align}
\vk^G(\vxref(\vxo)-\vxs) &= \vk^G(\vxref(\vxo)-\vxo), \\
|\vxref - \vxs|    &= R_{\mathrm{ref}}.
\end{align}
The~solution for the equations is given by
\begin{align}
\label{Eq:SFS_theory:spatial_SDM_circle_ref_points}
\vxref(\vxo) = \vxs + R_{\mathrm{ref}}\frac{\vxo-\vxs}{|\vxo-\vxs|}.
\end{align}
Substituting the spherical virtual field into \eqref{eq:SFS_theory:spatial_sdm_2}---with the principal radii given by simple distances from the point sources---yields the explicit driving function in the spatial domain for a virtual point source
\begin{equation}
D(x_0) =
\sqrt{\frac{|\vxref-\vxs|}{|\vxref-\vxs|-|\vxref-\vxo| }}
\sqrt{\frac{\ti k}{2\pi |\vxref-\vxo|}} 
\hat{k}_y^G(\vxref(\vxo)-\vxs)
\frac{G(\vxref(\vxo)-\vxs)}{G(\vxref(\vxo)-\vxo)}
\end{equation}
Finally, substituting the reference position coordinates along the reference circle \eqref{Eq:SFS_theory:spatial_SDM_circle_ref_points} specifies the driving function, optimizing the synthesis on the reference circle
\begin{equation}
\label{Eq:SFS_theory:linear_SSD_ref_circle}
D(x_0) =-y_s
\sqrt{\frac{R_{\mathrm{ref}}-|\vxo-\vxs|}{R_{\mathrm{ref}}}}
\sqrt{\frac{\ti k }{2\pi}} 
\frac{\te^{-\ti k |\vxo-\vxs|}}
{ |\vxo-\vxs|^{\frac{3}{2}} }.
\end{equation}
%
\begin{figure}
\centering
	\begin{overpic}[width = 1\columnwidth ]{Figures/SFS_theory/25D_spatial_SDM_linear_SSD.png}
	\put(0, 0){(a)}
	\put(45,0){(b)}
	\end{overpic}   
    \caption{2.5D synthesis of a 3D point source located at $\vxs = \posvec{3}{0}{-2}{0}$, radiating at $f_0 =~1~\mathrm{kHz}$.
	The synthesis is referenced on a circle around the virtual source, with a radius of $R_{\mathrm{ref}} = 4~\mathrm{m}$.
    Figure (a) depicts the real part of the synthesized field, (b) shows the error of synthesis.
    }
\label{fig:SFS_theory:25D_spatial_SDM_linear_ssd}  
\end{figure}
Investigating Figure \ref{fig:SFS_theory:25D_spatial_SDM_linear_ssd} verifies that the synthesis is optimized on the prescribed reference curve.

Although having derived the above driving function from the pressure measured along the reference curve, \eqref{Eq:SFS_theory:linear_SSD_ref_circle} is already written merely in terms of the target field measured on the actual SSD element, equivalently to the WFS solution.
In the followings this relation is generalized by expressing the explicit driving function for an arbitrary target sound field that is written in terms of pressure measured along the SSD, revealing the general connection between the implicit and explicit solutions.

\section{Relation of implicit and explicit solutions}

The explicit SFS driving function, given by \eqref{eq:SFS_theory:spatial_sdm_2} requires the knowledge of the virtual sound field, measured along the reference curve.
By applying the findings presented in Section \ref{Sec:HF:RayleighSPA}, for the special case of a linear SSD the required quantities can be expressed in terms of the the virtual field properties, measured on the SSD. 
The obtained formulation allows the comparison of the implicit and explicit solutions.

In order to express the driving function merely in terms of the involved quantities measured along the SSD, all the local wavenumber vector, the principal radii and the pressure of the target field has to be defined along the $y=0$ line.
For the local wavenumber vector this relationship can be simply established: Assuming isotropic medium, the propagation direction---and the local wavenumber vector---does not change along the propagation path, thus $\vk^P(\vxref(\vxo)) = \vk^P(\vxo)$ holds.

The change of the principal radii over the propagation path is given by \eqref{eq:app:propagated_radii}, stating that the principal radii of an arbitrary field increase linearly along the path of propagation.
The relation between the target field, measured at an arbitrary reference position and measured along the SSD can be established by the asymptotic evaluation of the Rayleigh integral, given for a general wavefield by \eqref{eq:HF_approx:asymptotic_rayleigh}.
However, both relations were only formulated for an independent receiver position $\vx$ in terms of a dependent SSD position $\vxo^*(\vx)$, related through the implicit relation
\begin{equation}
\label{Eq:SFS_theory:Rayleigh25D_horizontal_stat_point}
\vk^P(\vxo^*(\vx)) = \vk^G( \vx - \vxo^*(\vx)).
\end{equation}
Comparing this definition of the stationary points for the Rayleigh integral \eqref{Eq:SFS_theory:Rayleigh25D_horizontal_stat_point} and the stationary SDM evaluation points \eqref{Eq:stationary_evaluation_points}, it is revealed that they describe stationary point pairs.
Hence, without the loss of generality both the principal radii and the pressure field, measured on the reference curve can be formulated with choosing the SSD position $\vxo$ as an independent variable, resulting in
\begin{equation}
\label{Eq:principal_radii_addition}
\Rh^P(\vxref(\vxo)) = \Rh^P(\vxo) + \Rh^G(\vxref(\vxo)-\vxo),
\end{equation}
and
\small
\begin{multline}
\label{Eq:Asymptotic_Rayleigh_integral}
P(\vxref(\vxo),\omega) = 
4\pi
\sqrt{\frac{\Rh^P(\vxo) \cdot \Rh^G(\vxref(\vxo)-\vxo)}{\Rh^P(\vxo) + \Rh^G(\vxref(\vxo)-\vxo}} \cdot \\ \cdot
\sqrt{\frac{\Rv^P(\vxo) \cdot \Rv^G(\vxref(\vxo)-\vxo)}{\Rv^P(\vxo) + \Rv^G(\vxref(\vxo)-\vxo)}}
P(\vxo,\omega) G(\vxref(\vxo)-\vxo,\omega)
,
\end{multline} %FS: ok
\normalsize
where positions $\vxref$ and $\vxo$ are related through \eqref{Eq:stationary_evaluation_points} and where it was exploited that the principal radii are given by the vertical and horizontal radii components in the present geometry.

With substituting all these expressions into the spatial explicit driving function \eqref{eq:SFS_theory:spatial_sdm_2} and after simplification the driving function takes the form
\begin{equation}
D(x_0,\omega) =
2\sqrt{\frac{2\pi  \Rv^G(\vxref(\vxo)-\vxo)}{\ti k}} 
\sqrt{\frac{\Rv^P(\vxo) }{\Rv^P(\vxo) + \Rv^G(\vxref(\vxo)-\vxo)}}
\ti k_y^{P}(\vxo)
P(\vxo,\omega),
\end{equation}
or expressed in terms of the second phase derivatives 
\begin{equation}
D(x_0,\omega) =
2\sqrt{\frac{2\pi }{\ti }} 
\frac{ 1 }{\sqrt{|\phi^{P''}_{zz}(\vxo) + \phi^{G''}_{zz}(\vxref(\vxo)-\vxo)|}}
\ti k_y^{P}(\vxo)
P(\vxo,\omega).
\end{equation}
Comparison with \eqref{Eq:SFS_theory:25D_WFS_driv_fun_ver_2} and \eqref{Eq:SFS_theory:25D_WFS_driv_fun} reveals that the asymptotic SDM driving function exactly coincides the 2.5D WFS driving function when applied for a linear---and within the validity of the Kirchhoff approximation for an arbitrary shaped---SSD.
It is therefore verified that under high frequency assumptions WFS is the asymptotic and local approximation of the global explicit solution.

An important difference is however that the WFS driving function was obtained from the 2.5D Neumann Rayleigh integral in an intuitive manner, by introducing the reference curve concept with interchanging the role of the receiver position and its stationary SSD position. 
On the other hand the explicit driving function \eqref{eq:SFS_theory:spatial_sdm} inherently contains the horizontal SPA and the reference curve concept.

\section{Synthesis applying discrete secondary source distribution}
\label{Sec:Aliasing}
%\fscom{the findings in this section are very good, text deserves critical review, shorter sentences, more clear statements, I corrected some major issues}

Throughout this thesis synthesis applying the continuous distribution of secondary sources has been discussed so far.
In practical applications the SSD is realized by a densely spaced loudspeaker ensemble with the source elements positioned at discrete locations.
The violation of the continuous SSD assumption leads to severe artifacts in the synthesized field, commonly referred to as \emph{spatial aliasing phenomena}.

An advantage of the explicit solution is that it allows the analytical description of aliasing artifacts, which can be directly applied to the results of WFS as well, due to the presented asymptotic equivalence of the two methods.
As a complex application example for the equivalence of the explicit and implicit solutions, this section discusses the analysis and mitigation of spatial aliasing in a unified manner.

\subsection{Description of spatial aliasing}

Physically, spatial aliasing can be interpreted as follows: in case of steady-state analysis above a certain temporal frequency---termed as the \emph{aliasing frequency}--- the field of the individual secondary sources no longer form a continuous virtual wavefront but rather the fields of the individual secondary sources create a complex interference pattern.
The actual value of the aliasing frequency highly depends on the local propagation direction of the virtual wavefront: waves propagating laterally to the SSD are more likely to cause aliasing due to the rapid change of phase between adjacent secondary sources.

In the foregoing only steady-state analysis of wavefields was discussed.
However, spatial aliasing gains a simple physical interpretation when time domain analysis is considered.
As discussed already, the 2.5D WFS driving function applies a half-differentiation to the input signal in order to compensate for the infinite tail (i.e. the half-integrator characteristics) of the secondary source response, emerging due to the geometry of the SSD.
Compensation is therefore performed by each SSD element canceling out the fields of the adjacent secondary sources after the virtual wavefront is emitted.
In case of a discrete SSD this cancellation can not be performed above the aliasing frequency, hence in the synthesized field the spherical wavefields of the individual secondary sources will be present.
Spatial aliasing therefore manifests in a series of echoes---each produced by one individual secondary source element---following the intended virtual wavefront carrying the driving function of the individual secondary sources, high-pass filtered above the aliasing frequency \cite{spors2009spatial}. 

Note that in a given receiver position due to the short time interval between the arrival of the high-pass filtered echo wavefronts, aliasing is perceived rather as the coloration of the virtual sound field, than actual echoes/reverberation, while due to the precedence effect the localization of the virtual field is not degraded \footnote{This is true only for the synthesis of non-focused fields.
In case of reproducing e.g. a focused point source spatial aliasing appears as pre-echoes, arriving before the intended virtual wavefront \cite{Spors2009:FocusedSourceAliasing}.}.
A detailed investigation of the perception of spatial aliasing in WFS can be found in \cite{8371275}.
In the followings only an objective description of the above phenomena is presented.

\begin{figure}
\centering
	\begin{overpic}[width = 1\columnwidth]{Figures/SFS_theory/Spatial_alising.png}
	\put(0, 0){(a)}
	\put(50,0){(b)}
	\end{overpic}   
    \caption{2.5D synthesis of a 3D point source located at $\vxs = \posvec{3}{0}{-2}{0}$, radiating at $f_0 =~2~\mathrm{kHz}$ (a), or emitting an impulse, bandlimited to $15~\mathrm{kHz}$ measured at $t_0 = 10~\mathrm{ms}$ (b).
    The synthesis is referenced on a reference line with $\yref = 1.5~\mathrm{m}$.
	Synthesis is performed using a linear SSD with the secondary source spacing set to $\Delta x = 10~\mathrm{cm}$.}
\label{fig:SFS_theory:Spatial_alising}  
\end{figure}

In a recent publication it was shown that by applying the generalized WFS framework introduced in the present thesis, spatial aliasing artifacts can be directly described in the spatial domain \cite{Winter2018:GeometricModel}.
As an alternative, the following derivation approaches the problem by utilizing the spectral description of the synthesized field applying a discretized SSD, as introduced in \cite{Ahrens2012}.

Assume the synthesis of an arbitrary sound field by applying a linear SSD, located at $\vxo = \posvec{3}{x_0}{0}{0}$.
The synthesized field in the spatial and wavenumber domains is given by \eqref{Eq:SFS_Theory:linear_synth_field_spatial} and \eqref{Eq:SFS_Theory:linear_synth_field_spectral} respectively, in case of a continuous SSD.
Discretization of the SSD can be mathematically modeled by the sampling of the driving function with a sampling distance being the actual loudspeaker spacing $\Delta x$:
\begin{equation}
D^{\mathrm{S}}(x,\omega) = \sum_{\eta = -\infty}^{\infty} D(x,\omega) \cdot \delta \left(x - \eta \Delta x \right).
\end{equation}
Since each secondary source is represented by a Dirac delta, hence in this model the SSD consists of a discrete set of 3D point sources.
Exploiting that the spectrum of a series of Dirac deltas is a pulse train itself \cite{Girod2001}, and applying the sifting property of the Dirac delta, the wavenumber content of the sampled driving function reads as
\begin{equation}
\tilde{D}^{\mathrm{S}}(k_x,\omega) = \frac{1}{\Delta x} \sum_{\eta = -\infty}^{\infty} \tilde{D}\left(k_x - \eta \frac{2\pi}{\Delta x},\omega \right).
\end{equation}
% Add: extended secondary sources
\begin{figure}
\centering
	\begin{overpic}[width = 1\columnwidth ]{Figures/SFS_theory/Aliased_spectrum.png}
	\put(0,0){(a)}	
	\put(48,0){(b)}
	\put(18,30){$\begin{matrix}
				\text{propagation} \\ \text{region} \end{matrix}$	}
	\put(7.2,10){$\begin{matrix}
				\text{evanescent} \\ \text{region} \end{matrix}$	}
				\begin{turn}{60}
	\put(28,-18){$|k_x| = \omega/c$}
	\end{turn}
	\end{overpic}   
    \caption{Illustration of the discretization process of the linear WFS driving function. 
	Figure (a) shows the spectrum of the theoretically continuous driving function $\tilde{D}(k_x,\omega)$, Figure (b) shows the spectrum of the discretized driving function $\tilde{D}^{\mathrm{S}}(k_x,\omega)$ with the baseband spectrum repeating on the multiples of the sampling wavenumber.
    The secondary source distance is set to $\Delta x = 10~\mathrm{cm}$, corresponding to the sampling wavenumber $k_{x,s} \approx 63~\mathrm{rad/m}$ and resulting in an aliasing frequency $\omega_a \approx 2\pi \cdot 1.7~\mathrm{krad/s}$.}
\label{fig:SFS_theory:Aliased_spectrum}  
\end{figure}
Thus, as the result of discretization the driving function spectrum is obtained as the infinite sum of the original spectrum, repeating on the multiples of the sampling wavenumber $k_{x,s} = \frac{2\pi}{\Delta x}$.
Mathematically, spatial aliasing originates from the overlapping of the repetitive spectra.

The discretization process of the driving function is illustrated in the $k_x-\omega$ domain in Figure \ref{fig:SFS_theory:Aliased_spectrum} via the example of discretizing the driving function for a virtual point source, given by \eqref{eq:SFS_theory:WFS_point_source}.
From simple geometrical considerations it can be deduced that below a given angular frequency---the spatial aliasing frequency, obtained from $\frac{\omega_a }{c} = \frac{k_{x,s}}{2}= \frac{\pi}{\Delta x}$---no spectral overlapping occurs between the propagation regions.
Since WFS theory assumes high frequency/farfield conditions, neglecting the evanescent components is feasible.\footnote{A more detailed analysis on the aliasing occurring in the reproduction of virtual point sources is found in \cite{spors2009spatial}, where aliasing components are classified based on whether propagating/evanescent component overlaps into the propagating or evanescent region of the baseband.} 
Above the aliasing frequency high wavenumber components ($|k_x|>k_{x,s}/2$) of translated spectra overlap into the propagation region of the baseband driving function spectrum.

The wavenumber content of the synthesized field, applying a discrete SSD can be written as
\begin{equation}
\tilde{P}(k_x,y,z, \omega) = \tilde{D}^{\mathrm{S}}(k_x,\omega) \cdot \tilde{G}(k_x,y,z, \omega) = \frac{1}{\Delta x}
\sum_{\eta = -\infty}^{\infty} \tilde{D}\left(k_x - \eta \frac{2\pi}{\Delta x},\omega \right)  \cdot \tilde{G}(k_x,y,z, \omega).
\label{Eq:SFS_theory:Aliased_field_sp}
\end{equation}
The reproduction process is illustrated by Figure \ref{fig:SFS_theory:Aliased_repr_field} with the involved quantities measured along an arbitrary reference line, $y \gg 0$.
As it is demonstrated in \ref{fig:SFS_theory:Aliased_repr_field} (b) the transfer function from the SSD to the reference line, i.e. the Green's function acts as a spatial low-pass filter (with analogy to signal processing acting as a reconstruction filter), restricting the reproduced field on a given angular frequency to $|k_x| < \omega/c$, i.e. to the propagation region.
Note that again, nearfield investigation, where the spectrum of the Green's function would exhibit a high evanescent contribution is out of the scope of the present thesis.

\begin{figure}
\centering
	\begin{overpic}[width = 1\columnwidth]{Figures/SFS_theory/Aliased_repr_field.png}	
	\put(1,38){(a)}	
	\put(51.5,38){(b)}
	\put(27.5,0){(c)}
	\end{overpic}   
    \caption{Illustration of aliased synthesis of a virtual point source in the setup, used for \ref{fig:SFS_theory:Aliased_spectrum}.
    Figure (a) shows the spectrum of the discretized driving function with overlapping spectral repetition.
    Figure (b) shows the spectrum of the 3D Green's function and Figure (c) shows the spectrum of the synthesized field, both measured along the reference line.}
\label{fig:SFS_theory:Aliased_repr_field}  
\end{figure}
Figure \ref{fig:SFS_theory:Aliased_repr_field} (c) reflects the fact that aliasing components present in the synthesized field are described mathematically as additive, overlapping spectral components remaining in the propagation region of the baseband spectrum, described by \eqref{Eq:SFS_theory:Aliased_field_sp} with $|\eta|>0$.
This additive error manifests in additive wavefronts superimposed on to the non-aliased ideal wavefront (described by the $\eta = 0$ component in \eqref{Eq:SFS_theory:Aliased_field_sp}) in the time-domain.

As discussed earlier, components in the spectrum of the synthesized field correspond to synthesized attenuating plane waves oscillating at the angular frequency $\omega$ and propagating into the direction described by $k_x$ in the plane of synthesis.
Hence Figure \ref{fig:SFS_theory:Aliased_repr_field} (c) suggests that arbitrary directed plane waves with the angular frequency below the aliasing frequency can be synthesized with a discretized SSD.
Above the aliasing frequency lateral waves cannot be synthesized without the presence of aliasing plane wave components, propagating into the opposite direction (with reversed $k_x$ value).
Finally, above twice the aliasing frequency even plane waves propagating perpendicular to the SSD can not be synthesized without aliasing.

\subsection{Avoiding spectral overlapping}

A straightforward way to avoid spectral overlapping is to spatially bandlimit the driving function to the Nyquist wavenumber $k_{x,\mathrm{Nyq}} = k_{x,s}/2$ before discretization, i.e. by eliminating plane wave components that would cause spatial aliasing.
This requires the application of a spatial low-pass filter that's implementation is straightforward only for the present, linear SSD case.
A detailed investigation on the direct spatial pre-filtering of the linear SFS driving function can be found in \cite{Firtha2012:isma, Ahrens2012}.\footnote{There it is discussed that pre-filtering the SFS driving function is equivalent with extending the virtual source distribution according the impulse response of the spatial filter impulse response.}
However, applying the local wavenumber vector concept and exploiting the general asymptotic equivalence of the SDM and WFS, a more simple, general anti-aliasing strategy may be introduced, as given in the following.

According to \eqref{eq:xP_xG_in_spatial_domain} a given wavenumber component $k_x$ in the SDM driving function is dominated by that particular position of the SSD, where the local propagation direction of the virtual wavefield coincides with that of the actual spectral plane wave, i.e. where $k_x^P(x_0) = k_x$ holds.
Hence, on a given angular frequency only those parts of the SSD will cause spectral overlapping, where the $x$-component of the virtual field's local wavenumber vector is above the Nyquist wavenumber, i.e. where
\begin{equation}
|k_x^P(x_0)| = \frac{\omega}{c} |\hat{k}_x^P(x_0)| > k_{x,\mathrm{Nyq}} = \frac{\pi}{\Delta x}
\end{equation}
is satisfied.
Therefore, within the validity of the SPA spatial aliasing components can be assigned to particular positions along the SSD.
This formulation allows the elimination of the overlapping spectral components by simple temporal low-pass filtering the driving function, with the angular cut-off frequency given as
\begin{equation}
D(x_0,\omega) = 0, \hspace{5mm} \text{where} \hspace{5mm} \omega \geq \frac{\pi}{\Delta x} \frac{c}{|\hat{k}_x^P(x_0)|}.
\label{Eq:SFS_theory:cutoff_freq}
\end{equation}
\begin{figure}
\centering
	\begin{overpic}[width = 1\columnwidth ]{Figures/SFS_theory/AntiAliased_spectrum.png}
	\put(1,38){(a)}	
	\put(51.5,38){(b)}
	\put(27.5,0){(c)}
	\end{overpic}   
    \caption{Illustration of the ideal anti-aliasing filtering of the WFS driving function for a virtual point source in the setup, used for Figure \ref{fig:SFS_theory:Aliased_spectrum}.
    Figure (a) shows the spatially band-limited driving function spectrum, low-pass filtered with the low-pass filtering performed in spatial domain used the introduced approach.
    Figure (b) shows the spectrum of the discretized driving function containing non-overlapping spectral repetition.
    Figure (c) shows spectrum of the synthesized field, measured along the reference line.}
\label{fig:SFS_theory:anti-aliased_spectrum}  
\end{figure}

This finding may be generalized towards the application of an arbitrary shaped SSD, within the validity of the Kirchhoff approximation.
In this general scenario the SSD is assumed to be locally linear, and the role of $k_x^P$ is taken over by the tangential component of the local wavenumber vector.
Hence, the ideal anti-aliasing condition for a general SFS problem is formulated as
\importanteq{Symmetric anti-aliasing condition}{
D(\vxo,\omega) = 0, \hspace{5mm} \text{where} \hspace{5mm} \omega \geq \frac{\pi}{\Delta x} \frac{c}{|\hat{k}_t^P(\vxo)|},
\label{Eq:SFS_theory:general_cutoff_freq_v0}
}
where $\hat{k}_t^P$ is the normalized local wavenumber vector component, tangential with the SSD.

The above formulation can be interpreted as a frequency dependent secondary source selection criterion: basically the window function in the general WFS driving function \eqref{Eq:Theory:SSD_selection} became frequency dependent.
The application of an ideal low-pass filter, given directly by \eqref{Eq:SFS_theory:general_cutoff_freq_v0} would result in truncation effects emerging from the muted and unmuted transition of the SSD, due to the discontinuity of the driving function \cite{Start1997:phd}.
%Mathematically this is modeled as the multiplication of the driving function with a window function, resulting in a ringing effect in the wavenumber domain.
Hence, in practical applications anti-aliasing should be performed by an appropriately smooth low-pass filter design with the cut-off frequency given by \eqref{Eq:SFS_theory:general_cutoff_freq_v0}.

\begin{figure}  
\small
  \begin{minipage}[c]{0.64\textwidth}
	\begin{overpic}[width = 1\columnwidth ]{Figures/SFS_theory/Antialiased_synth.png}
	\small
	\put(2,53){(a)}
	\put(2,1){(b)}
	\end{overpic}   \end{minipage}\hfill
	\begin{minipage}[c]{0.35\textwidth}
    \caption{2.5D synthesis of a 3D point source, located in $\vxs = \posvec{3}{1.5}{3}{0}~\mathrm{m}$, emitting a bandlimited impulse, applying an arbitrary shaped discrete SSD with the secondary source spacing being $\Delta x = 10~\mathrm{cm}$, and with the snapshot taken at $t_0 \approx 6~\mathrm{ms}$.
    Figure (a) shows the effect of the discrete SSD, resulting in aliasing echoes following the intended wavefront.
    Figure (b) shows the result of spatial anti-aliasing filtering.
    As a result anti-aliased synthesis may be achieved behind the virtual wavefront into the particular direction, denoted by dashed arrow.
    The arrow originates at the SSD element with no angular bandwidth limitation, i.e. performing full-band synthesis.
    This full-band SSD element is found, where the local propagation direction of the virtual field coincides with the SSD normal (i.e. the tangential local wavenumber component is zero).}
\label{fig:SFS_theory:anti-aliased_synthesis}   \end{minipage}
\end{figure}

In the following the locations of anti-aliased synthesis is investigated.
Although spectral overlapping can be avoided by the presented strategy, due to the non-directive nature of the 3D secondary point sources, lateral aliasing waves are present in the synthesized wavefield, emerging from the partial reproduction of the mirror spectra.
Investigating Figure \ref{fig:SFS_theory:anti-aliased_spectrum} (c) suggests that at a given angular frequency $\omega$ with a given SSD element at $\vxo$ anti-aliased synthesis can be performed into directions for which $|k^P_t(\vxo)| < k_{t,\mathrm{Nyq}}$ holds.
Above the Nyquist wavenumber, components of the mirror spectra are reproduced as well, present as lateral waves in the synthesized field. 
It is restated here that one particular SSD element at $\vxo$ dominates the synthesized field along the local propagation direction of the synthesized field, hence along a straight line through the SSD element into the direction of the virtual field's local wavenumber vector.
This means that anti-aliased synthesis can be achieved along a beam, centered around the normal vector of the SSD element, where $|k^P_t(\vxo)| = 0$ holds locally, and the opening angle of the beam decreases with frequency.
At high frequencies therefore anti-aliased, full band synthesis can be only achieved with that SSD element, where $|k^P_t(\vxo)| = 0$ holds locally, and the positions of full band, anti-aliased synthesis in the reproduced field will be that, dominated by the full-band SSD element.

Figure \ref{fig:SFS_theory:anti-aliased_synthesis} illustrates the effect of the presented anti-aliasing strategy.
It is verified that into the direction dominated by the SSD element at $\vxo$, where $k^P_t(\vxo) = 0$ holds, aliasing components can be almost entirely eliminated behind the virtual wavefront, and still into this particular direction full-band synthesis can be achieved.
Into other directions lateral aliasing components are present and the synthesized virtual wavefront is bandlimited.

\begin{figure}
\small
  \begin{minipage}[c]{0.64\textwidth}
	\begin{overpic}[width = 1\columnwidth ]{Figures/SFS_theory/AntiAliased_spectrum_asymm.png}
	\small
	\put(2,1){(a)}
	\put(54.5,13.5){$k^P_{x,0}$}
	\end{overpic}   	
	\begin{overpic}[width = 1\columnwidth ]{Figures/SFS_theory/Steered_antialias_synth.png}
	\small
	\put(2,1){(b)}
	\end{overpic}   \end{minipage}\hfill	
	\begin{minipage}[c]{0.35\textwidth}
    \caption{2.5D synthesis of a 3D point source, located in $\vxs = \posvec{3}{1.5}{3}{0}~\mathrm{m}$, emitting a bandlimited impulse, applying an arbitrary shaped discrete SSD with the secondary source spacing being $\Delta x = 10~\mathrm{cm}$, and with the snapshot taken at $t_0 \approx 6~\mathrm{ms}$.
	Figure (a) shows the sampled driving function spectrum with asymmetric bandwidth limitation.
	Figure (b) presents the synthesized field.
	The synthesis is performed by optimizing spatial anti-aliasing into the direction, indicated by dashed arrow in Figure (b) by simple bandlimitation of the driving function in the angular frequency domain, according to \eqref{Eq:SFS_theory:general_cutoff_freq}.
	The corresponding center wavenumber is indicated by dotted line in Figure (a).	
    }
\label{fig:SFS_theory:anti-aliased_synthesis_asymm}   \end{minipage}
\end{figure}
\vspace{3mm}
So far only symmetrical anti-aliasing filtering was discussed.
However, overlapping of the mirror spectra may be eliminated by bandlimiting the driving function wavenumber content around an arbitrary chosen center wavenumber (below the Nyquist wavenumber) to the bandwidth of $\frac{2\pi}{\Delta x}$, as illustrated in Figure \ref{fig:SFS_theory:anti-aliased_synthesis_asymm} (a).
The center wavenumber $k^P_{t,0}$ defines the direction into which full-band, anti-aliased synthesis can be achieved: the SSD element, with $k^P_t(\vxo) = k^P_{t,0}$ performs full-band synthesis and dominates the synthesized field into the direction $\vk^P(\vxo)$.
Asymmetric anti-aliasing filtering can be performed by the angular low-pass filtering of the driving function according to
\importanteq{Asymmetric anti-aliasing condition}{
D(\vxo,\omega) = 0, \hspace{5mm} \text{where} \hspace{5mm} \omega \geq \frac{\pi}{\Delta x} \frac{c}{|\hat{k}_t^P(\vxo) - \hat{k}^P_{t,0}|}.
\label{Eq:SFS_theory:general_cutoff_freq} 
}
The result of asymmetric anti-aliasing filtering is presented in Figure \ref{fig:SFS_theory:anti-aliased_synthesis_asymm} (b), illustrating how anti-aliased synthesis may be optimized into an arbitrary direction.
This local increase of synthesis accuracy is referred to as \emph{Local Wave Field Synthesis (LWFS)} in the related literature, being the subject of extensive study in the recent years. \cite{ahrens2010local, 5946329, spors2011local, Winter15:EURONOISE, Winter15:AES, Hahn17:EUSIPCO, Hahn16:AES, Winter2016-TASL}
Local increase of accuracy is usually achieved by some spatial bandwidth limitation strategy applied to the driving function, performed in simple SSD geometries (e.g. linear, circular SSDs).
The above derivation hence gives an asymptotic approximation of these spatial bandwidth limitation techniques, valid for an arbitrary SSD contour.

\subsection{Avoiding the reproduction of mirror spectra}

As presented in the foregoing, even besides ideal anti-aliasing filtering (spatial bandlimition) of the driving function, lateral aliasing components following the virtual wavefront will still be present.
Mathematically, these aliasing artifacts originate from the reproduction of the adjacent mirror spectra components above the Nyquist wavenumber, as illustrated in Figure \ref{fig:SFS_theory:anti-aliased_spectrum} (c).
Obviously, the reproduction of mirror spectra can be avoided only by applying a secondary source distribution with spatially low-pass filtered transfer function---instead of the simple 3D Green's function---with high attenuation factor above the Nyquist wavenumber.\footnote{In fact, mirror spectrum reproduction may be avoided by applying more strict low-pass anti-aliasing, by also filtering out those components, which would be present as lateral mirror spectra after discretization. 
Hence the spectrum of the filtered driving function would be a romboid shape in Figure \ref{fig:SFS_theory:anti-aliased_spectrum} (a), entirely bandlimited to $\omega = \frac{k_{x,s}}{c}$.
This strategy is equivalent with the solution, given in \cite{Winter2018:GeometricModel}.}
This requirement is fulfilled by directive secondary sources, radiating with low intensity into lateral directions, i.e. to large tangential wavenumbers.
In a signal processing analogy, directive secondary sources therefore act as reproduction filters.

Physically, source directivity stems from the physical extension of radiating surfaces due to the constructive and destructive interference of waves, originating from different positions on the surface.
At high frequencies and in the farfield (compared to the physical extension) baffled vibrating surfaces can be modeled as directive point sources.
In fact, this can be proven by using the local wavenumber concept as presented in Appendix \ref{App:Planar_radiators}.
The optimal shape of extended secondary sources in the aspect of suppressing the mirror spectra has been studied in the related literature \cite{Verheijen1997:phd}.
Here the optimal SSD directivity is discussed within the context of the generalized WFS framework.

The farfield approximation of an extended radiator as a directive point source is given by \eqref{Eq:App:directive_monopole}.
The field of a directive point source at the horizontal plane, containing the source (i.e. at $\theta = 0$) is given by
\begin{equation}
G_{\Theta}(\vx-\vxo,\omega) = \Theta(\phi(\vx,\vxo),\omega) \cdot
G(\vx-\vxo,\omega),
\end{equation}
with the directivity function given by $\Theta(\cdot)$. 
Here $\phi$ denotes the polar angle, measured from the main direction of the source, in case of modeling baffled sources given by the source-surface normal: $\cos \phi = \frac{\left< \vx-\vxo \cdot \vn(\vxo) \right>}{|\vx-\vxo|}$.

First the wavenumber domain representation of a theoretical directive point source is investigated.
The transfer function of a directive point source at the origin, measured along a fixed $y$ reads as 
\begin{equation}
\tilde{G}_{\Theta}(k_x,y,\omega) = \int_{-\infty}^{\infty} \Theta(\phi(\vx),\omega) \cdot
G(\vx,\omega) \, \te^{\ti k_x x} \td x.
\end{equation}
As an approximation, the amplitude factor of the integral is approximated around its stationary point.
For the sake of  simplicity it is assumed that the directivity function is real-valued.
Since in case of baffled radiators the directivity function is the 2D Fourier transform of the surface normal velocity distribution, the directivity is real-valued, if the radiator is symmetrical both to the $x$- and $z$-axes (i.e. its is even function).
This holds for simple geometries, e.g. for a circular piston, which model is applied frequently for modeling a dynamic loudspeaker.

Following \eqref{eq:xP_xG_in_spatial_domain}, by assuming a real-valued directivity the stationary point of the integral is simply found where $k_x^{G}(\vx^*(k_x),y) = k_x$ is satisfied.
Around the stationary point the directivity function is approximated by its stationary value and the integral simplifies to
\begin{equation}
\tilde{G}_{\Theta}(k_x,y,\omega) = \Theta(\phi(\vx^*(k_x)),\omega) \cdot \int_{-\infty}^{\infty} 
G(\vx,\omega) \, \te^{\ti k_x x} \td x =
\Theta(\phi(\vx^*(k_x)),\omega) \cdot 
\tilde{G}(k_x,y,\omega),
\end{equation}
thus in the wavenumber domain the directivity function acts as a simple spatial filter transfer function.
Since for the Green's function's local wavenumber vector $k_x^{G}(\vx) = k \sin \phi(\vx) = k_x$ holds (c.f. \eqref{Eq:App:ps_k_vec}), therefore as a final result the transfer function of a directive point source is given by
\begin{equation}
\tilde{G}_{\Theta}(k_x,y,\omega) = 
\Theta(\arcsin \frac{k_x}{k},\omega) \cdot 
\tilde{G}(k_x,y,\omega),
\end{equation}
while the spectrum of the synthesized field, applying a discrete secondary distribution of directive point sources becomes
\begin{equation}
\tilde{P}(k_x,y,\omega) = \tilde{D}^S(k_x,\omega) \cdot \Theta(\arcsin \frac{k_x}{k},\omega) \cdot \tilde{G}(k_x,y,\omega).
\end{equation}
This allows one to derive the directivity characteristics of an ideal spatial low-pass secondary source, which suppresses lateral waves above the Nyquist wavenumber, hence avoiding the reproduction of mirror spectra.
The transfer function of the ideal anti-aliasing SSD is given as
\begin{equation}
\Theta(\arcsin \frac{k_x}{k},\omega) = 0, \hspace{5mm} \text{if} \hspace{5mm} k_x \geq \frac{\pi}{\Delta x}
\end{equation}
which can be formulated in the angular domain as
\importanteq{Ideal anti-aliasing SSD directivity}{
\Theta(\phi,\omega) = 0, \hspace{5mm} \text{if} \hspace{5mm} \sin \phi \geq \frac{\pi}{\Delta x} \frac{c}{\omega}.
}
\begin{figure}  
\small
  \begin{minipage}[c]{0.64\textwidth}
%	\begin{overpic}[width = 1\columnwidth]{Figures/SFS_theory/anti-aliased_spectrum_DirSSD.png}
%	\small
%	\put(2,1){(a)}
%	\end{overpic}   	
	\begin{overpic}[width = 1\columnwidth ]{Figures/SFS_theory/AntiAliased_synth_dir_SSD.png}
%	\small
%	\put(2,1){(b)}
	\end{overpic}   \end{minipage}\hfill	
	\begin{minipage}[c]{0.35\textwidth}
    \caption{2.5D synthesis of a 3D point source, located in $\vxs = \posvec{3}{1.5}{3}{0}~\mathrm{m}$, emitting a bandlimited impulse, applying an arbitrary shaped discrete SSD with the secondary source spacing being $\Delta x = 10~\mathrm{cm}$.
	The synthesis is performed by ideal anti-aliasing filtering of the driving function and applying the distribution of ideally directive secondary sources.
    }
\label{fig:SFS_theory:anti-aliased_synthesis_ideal_synth}   \end{minipage}
\end{figure}
The theoretical, ideal secondary sources therefore would only radiate within a beam with a unit amplitude, so that the width of the beam decreases at increasing frequency.
By applying such ideal secondary sources, aliasing could be theoretically completely avoided, still ensuring full band, anti-aliased synthesis into one particular direction, as described above.
Into other directions the synthesized wavefront is bandlimited (low-pass filtered).
Obviously, unlike for the anti-aliasing filtering, the direction of perfect synthesis is fixed towards the main lobe of the applied secondary sources.

The result of synthesis applying secondary sources with ideal directivity can be seen in Figure \ref{fig:SFS_theory:anti-aliased_synthesis_ideal_synth}, depicting the theoretical best case scenario when applying a discrete SSD.

\vspace{3mm}
Obviously, the above ,,ideally directed secondary sources" are not realizable, however the presented framework is useful for predicting the suppression
factor in the wavenumber region, once the directivity of the applied secondary sources is known.
As a simple example, the directivity of a circular piston is given by \eqref{Eq:App_Circ_piston}, reading
\begin{equation}
\Theta(\arcsin \frac{k_x}{k},\omega) = 2\frac{J_1\left( r_0 k_x\right)}{r_0 k_x},
\end{equation}
with $J_1(\cdot)$ being the first order Bessel function.
The circular piston therefore acts as a spatial low-pass filter, with the $-3~\mathrm{dB}$ cut-off wavenumber being $k_{x,c} \approx \frac{2.22}{r_0}$, calculated numerically.
Real-life dynamic loudspeakers with circular membranes are often modeled as circular pistons at frequencies, at which the modal behaviour of the diaphragm is not considerable.
In a linear array the largest possible loudspeaker radius is the half of the loudspeaker spacing ($r_0 = \Delta x/2$), hence, the lowest achievable cut-off wavenumber is 
$k_{x,c} = \frac{4.44}{\Delta x}$.
This is still higher, than the Nyquist wavenumber ($k_{x,\mathrm{Nyq}} = \frac{\pi}{\Delta x}$), meaning that even with the closest physically possible loudspeaker spacing in case of an in-line loudspeaker array, spatial aliasing components will be slightly present in the reproduced field.
%\fscom{good points here, in essence this is what I treated little bit differently, but with same message and comparing to a line piston that exhibit sinc spectrum. Depending on the gaps between piston aliasing is differently strong.}
%
%

\begin{appendices}
\chapter{Appendix A}
\section{Definition and properties of Fourier transform and the Dirac delta}
\label{App:Fourier_def}

\paragraph{Temporal Fourier transform:}\mbox{} \\
Given a four dimensional function $f(\vx,t)$, depending on both time and the spatial position.
The forward and inverse temporal Fourier transform is defined as 
\begin{equation}
\label{eq:temporal_fourier_transform_def}
F(\vx,\omega) = \mathcal{F}_t \left\{ f(\vx,t) \right\} = \int\limits_ {-\infty}^{\infty} f(\vx,t) \te^{-\ti \omega t} \td t,
\end{equation}
\begin{equation}
\label{eq:temporal_inverse_fourier_transform_def}
f(\vx,t) = \mathcal{F}_{\omega}^{-1} \left\{ F(\vx,\omega) \right\} = \frac{1}{2\pi} \int_ {-\infty}^{\infty} F(\vx,\omega) \te^{ \ti \omega t} \td \omega.
\end{equation}
Note that capital letter indicates that the function is taken in the angular frequency domain.
%
\paragraph{Spatial Fourier transforms:}\mbox{} \\
Following the convention, given in e.g. \cite{Ahrens2012} the spatial Fourier transform is defined as follows:
\begin{itemize}
\item in one dimension:
\begin{equation}
\label{eq:spatial_fourier_transform_def}
\tilde{F}(k_x,y,z,\omega) = \mathcal{F}_x \left\{ F(\vx,\omega) \right\} = \int_ {-\infty}^{\infty} F(\vx,\omega) \, \te^{\ti k_x x} \, \td x,
\end{equation}
\begin{equation}
\label{eq:spatial_inverse_fourier_transform_def}
F(\vx,\omega) = \mathcal{F}_{k_x}^{-1} \left\{ \hat{F}(k_x,y,z,\omega) \right\} = \frac{1}{2\pi} \int_ {-\infty}^{\infty} \tilde{F}(k_x,y,z,\omega) \, \te^{ -\ti k_x x} \td k_x,
\end{equation}
\item in two dimensions:
\begin{equation}
\label{eq:spatial_fourier_transform_def_2d}
\tilde{F}(k_x,y,k_z,\omega) = \iint_ {-\infty}^{\infty} F(\vx,\omega) \te^{\ti \left( k_x x + k_z z \right) } \td x \, \td z,
\end{equation}
\begin{equation}
\label{eq:spatial_inverse_fourier_transform_def_2d}
F(\vx,\omega) = \frac{1}{\left( 2\pi \right)^2} \iint_ {-\infty}^{\infty} \tilde{F}(k_x,y,k_z,\omega) \, \te^{ -\ti \left( k_x x + k_z z \right)} \td k_x \, \td k_z,
\end{equation}
\item in three dimensions:
\begin{equation}
\label{eq:spatial_fourier_transform_def_3d}
\tilde{F}(\vk,\omega)= \iiint_ {-\infty}^{\infty} F(\vx,\omega) \te^{ \ti \left< \vk \cdot \vx \right>} \td x \,\td y\,\td z,
\end{equation}
\begin{equation}
\label{eq:spatial_inverse_fourier_transform_def_3d}
F(\vx,\omega) = \frac{1}{\left( 2\pi \right)^3} \iiint_{-\infty}^{\infty} \tilde{F}(\vk,\omega) \te^{ -\ti \left< \vk \cdot \vx \right>} \td k_x \, \td k_y \, \td k_z.
\end{equation}
\end{itemize}
Hence, hat over the function symbol indicates that the function is taken in the wavenumber domain.
Note that the exponent of the spatial Fourier transform is taken with a reversed sign, compared to the temporal transform.
Writing an arbitrary function in the form of a spatio-temporal inverse Fourier transform
\begin{equation}
f(\vx,t) = \frac{1}{\left( 2\pi \right)^4} \iiiint_{-\infty}^{\infty} \hat{F}(\vk,\omega) \te^{ \ti \left( \omega t - \left< \vk \cdot \vx \right> \right)} \td k_x \, \td k_y \, \td k_z \td \omega
\end{equation}
basically describes the expansion of an arbitrary function into the linear combination of plane waves, propagating to direction $\vk$.
The reversed sign therefore ensures that this simple physical interpretation can be assigned to the Fourier transform.

\paragraph{Fourier transform properties:}\mbox{} \\
Several important properties of the Fourier transform, applied frequently in the present thesis are the following.
\begin{itemize}
\item Shift theorem:
\begin{equation}
\int_{-\infty}^{\infty} f(x-x_0) \te^{\ti k_x x} \td x = \mathcal{F}_x \left\{ f(x-x_0) \right\} = \hat{F}(k_x) \te^{\ti k_x x_0}.
\end{equation}
In case of temporal Fourier transform the right side is with reversed exponent.
\item Convolution theorem:
\begin{equation}
\int_{-\infty}^{\infty} f(x-x_0) g(x_0) \td x_0 \, \te^{\ti k_x x} \td x = \mathcal{F}_x \left\{ f(x) \ast_x g(x) \right\} = \hat{F}(k_x) \cdot \hat{G}(k_x).
\end{equation}
\item Differentiation property:
\begin{equation}
\int_{-\infty}^{\infty} \frac{\partial}{\partial x} f(x) \te^{\ti k_x x} = \mathcal{F}_x \left\{ \frac{\partial}{\partial x} f(x) \right\} = 
-\ti k_x \hat{F}(k_x).
\end{equation}
In case of temporal Fourier transform the right side is with reversed sign.
\item Scaling property:
\begin{equation}
\int_{-\infty}^{\infty} f(a x) \te^{\ti k_x x} = \mathcal{F}_x \left\{ f( a x) \right\} = 
\frac{1}{|a|}\hat{F}(\frac{k_x}{a}).
\end{equation}
In case of temporal Fourier transform the right side is with reversed sign.
\end{itemize}

\paragraph{Properties of Dirac delta:}\mbox{} \\
The Dirac delta is a generalized function (or distribution) applied frequently in modeling acoustic phenomena, defined as
\begin{equation}
\delta(x) = 
\begin{cases}
\infty, & \hspace{1mm} x = 0\\
0, & \hspace{1mm}  x \neq 0
\end{cases},
\hspace{1cm}
\text{with}
\hspace{2cm}
\int_{-\infty}^{\infty} \delta(x) \td x = 1.
\end{equation}
Several important properties of the Dirac delta, applied frequently in the present thesis are the following.
\begin{itemize}
\item Forward Fourier transform:
\begin{equation}
\mathcal{F}_x \left\{\delta(x-x_0)\right\} = \int_{-\infty}^{\infty} \delta(x-x_0) \te^{\ti k_x x} \td x =   \te^{\ti k_x x_0}.
\end{equation}
\item Inverse Fourier transform:
\begin{equation}
\delta(x-x_0) = \frac{1}{2\pi} \int_{-\infty}^{\infty} \te^{-\ti k_x (x-x_0)} \td k_x =  \mathcal{F}^{-1}_{k_x} \left\{ \te^{\ti k_x x_0} \right\}.
\end{equation}
\item Sifting property:
\begin{equation}
\int_{-\infty}^{\infty} \delta(x-x_0) f(x) \td x = f(x_0).
\end{equation}
\item Generalized sifting property:
\begin{equation}
\int_{-\infty}^{\infty} f(x) \delta(g(x)) \td x = \sum_{i} \frac{f(x_i)}{\left| \frac{\partial}{\partial x} g(x) \right|_{x = x_i}}, \hspace{5mm} \text{where} \hspace{5mm} g(x_i) = 0.
\end{equation} 
\end{itemize}
\chapter{Appendix B}
\section{Notes on the Hessian of the phase function}
\label{App:Hessian}

\subsection{Definition of the principal curvatures and principal directions}
Assume a wavefield, described by the general polar form $P(\vx,\omega) = A^P(\vx,\omega)\te^{\ti \phi^P (\vx,\omega)}$.
Supposing that the amplitude changes slowly compared to the phase function, the local dispersion relation $| \Dx \phi(\vx,\omega) |= \frac{\omega}{c} = k$ holds and the equation, describing an arbitrary wavefront, i.e. $\phi^P(\vx,\omega) - C = 0$ is by definition the \emph{normalform} of the given surface \cite{Hartmann1999, Hartmann2001}.
The Hessian matrix of the function is given by the symmetric matrix
\begin{equation}
\label{Eq:App:Hessian}
\mH^P(\vx) =
\frac{\partial^2}{\partial x_i \partial x_j} \phi^P(\vx,\omega) 
=
 \begin{bmatrix} 
\phi^{P''}_{xx}(\vx,\omega) & \phi^{P''}_{xy}(\vx,\omega) & \phi^{P''}_{xz}(\vx,\omega) \\[.7em]
\phi^{P''}_{xy}(\vx,\omega) & \phi^{P''}_{yy}(\vx,\omega) & \phi^{P''}_{yz}(\vx,\omega) \\[.7em]
\phi^{P''}_{xz}(\vx,\omega) & \phi^{P''}_{yz}(\vx,\omega) & \phi^{P''}_{zz}(\vx,\omega) \\[0.5em]    \end{bmatrix}, \hspace{3mm} i,j = 1,2,3,
\end{equation}
with the eigenvalues $\lambda_1, \lambda_2, \lambda_3$ and the corresponding eigenvectors $\mathbf{v}_1, \mathbf{v}_2, \mathbf{v}_3$.
Since the function under consideration is a normalform, therefore the following properties hold
\begin{itemize}
\item $\lambda_3 = 0$, with the corresponding eigenvector given by $\mathbf{v}_3 = - \frac{1}{k} \Dx \phi^P(\vx,\omega) = \hat{\vk}^P(\vx)$, i.e. being the normal of the wavefront
\item $\lambda_1 = -k \cdot \kappa^P_1(\vx)$ and $\lambda_2 = -k \cdot	\kappa^P_2(\vx)$ are proportional to the \emph{main or principal curvatures} of the wavefront, where $\rho^P_1(\vx) = \frac{1}{\kappa^P_1(\vx)}$ and $\rho^P_2(\vx) = \frac{1}{\kappa^P_2(\vx)}$ are the \emph{principal radii}.
The principal curvatures and radii are defined as the following:
Consider all the planes, containing the normal of the surface at the point of investigation. The planes are defined by the surface normal and vector $\mathbf{v}$, being a tangent vector of the surface.
The curvature is defined as the quadratic form 
\begin{equation}
\kappa = \mathbf{v}^{\mathrm{T}} \mH^P \mathbf{v}.
\label{Eq:App:curvature_def}
\end{equation}
The main curvatures are then defined as the minimum and maximum values of curvature, i.e. the reciprocal of the osculating circles' radii (the principal radii).
The corresponding eigenvectors, $\mathbf{v}_1$ and $\mathbf{v}_2$ are tangential, orthogonal unit vectors, pointing into the direction of the maximal and minimal curvatures.
For an illustration refer to Figure \ref{Fig:HF_appr:local_wave_curvature}.
\end{itemize}
Finally, as the general case the Hessian matrix of an arbitrary wavefront can be written in a spectral form, in terms of the principal curvatures and the corresponding eigenvectors as
\begin{equation}
\mH^P = -k \left( \kappa_1  \mathbf{v}_1 \mathbf{v}_1^\mathrm{T} + \kappa_2 \mathbf{v}_2 \mathbf{v}_2^\mathrm{T} \right) = -k \mathbf{V} \mathbf{K} \mathbf{V}^{\mathrm{T}},
\label{Eq:App:Hessian_w_curvature}
\end{equation}
where $\mathbf{V} = \begin{bmatrix} \mathbf{v}_1 & \mathbf{v}_2 \\\end{bmatrix}$ and $\mathbf{K} = \begin{bmatrix} \kappa_1 & 0 \\[.0em] 0 & \kappa_2 \\[0.0em] \end{bmatrix}$ are the matrices of the eigenvectors and the curvatures.

For the special case of the three-dimensional Green's function positioned at the origin, the Hessian matrix is given as
\begin{equation}
\label{Eq:App:Greens_f_hessian}
\mH^G(\vx) = -\frac{k}{|\vx|}
\begin{bmatrix} 
\left( 1-\frac{x^2}{|\vx|^2} \right) & \frac{x y}{|\vx|^2}                  & \frac{x z}{|\vx|^2}                 \\[.7em]
\frac{y x}{|\vx|^2}                  & \left( 1-\frac{y^2}{|\vx|^2} \right) &\frac{y z}{|\vx|^2}                  \\[.7em]
\frac{z x}{|\vx|^2}                  & \frac{z y}{|\vx|^2}                  &\left( 1-\frac{z^2}{|\vx|^2} \right) \\[0.5em]    \end{bmatrix}
\end{equation} 
%
For this matrix only the eigenvector, corresponding to $\lambda_3$ is well-defined over the spherical/\emph{umbilical} wavefront, being the normal vector of the surface.
Eigenvectors $\mathbf{v}_1$ and $\mathbf{v}_2$ may be arbitrary orthogonal vector-pair in the tangent plane of the surface, in the point of investigation $\vx$. 
The corresponding principal curvatures are $\kappa_1^G(\vx) = \kappa_2^G(\vx) = \frac{1}{|\vx|}$.

\vspace{3mm}
In the present treatise, when dealing with 2.5D problems, it is a standard prerequisition that in the plane of investigation ($z = 0$) all the involved wavefields propagate along the horizontal direction ($k_z(x,y,0) \equiv 0$). 
In this special case, the Hessian of the phase function becomes
\begin{equation}
\mH^P(\vx) =  \begin{bmatrix} 
\phi^{P''}_{xx}(\vx,\omega) & \phi^{P''}_{xy}(\vx,\omega) & 0 \\[.7em]
\phi^{P''}_{xy}(\vx,\omega) & \phi^{P''}_{yy}(\vx,\omega) & 0 \\[.7em]
0 & 0 & \phi^{P''}_{zz}(\vx,\omega) \\[0.5em]    \end{bmatrix},
\end{equation}
with the trivial eigenvector/principal direction $\mathbf{v_2} = \posvec{3}{0}{0}{1}$, and the corresponding principal curvature $\kappa_2^P(\vx) = -\frac{1}{k} \phi^{P''}_{zz}(\vx,\omega)$.
Furthermore, considering that the eigenvector with a zero eigenvalue is given by $\mathbf{v}_3 = \hat{\vk}^P(\vx) = \posvec{3}{\hat{k}_x^P(\vx)}{\hat{k}_y^P(\vx)}{0}$, and $\mathbf{v}_1$ is orthogonal to  $\mathbf{v}_2$ and $\mathbf{v}_3$, therefore $\mathbf{v}_1 = \posvec{3}{\hat{k}_y^P(\vx)}{\hat{k}_x^P(\vx)}{0}$ holds.
Applying \eqref{Eq:App:Hessian_w_curvature}, the elements of the Hessian matrix can be expressed as
\begin{equation}
\label{Eq:App:Hessian_inplane}
\mH^P(\vx) = -k	 \begin{bmatrix} 
\hat{k}_y^{P}(\vx)^2 \kappa_1^P(\vx) & \hat{k}_x^{P}(\vx)\hat{k}_y^{P}(\vx)\kappa_1^P(\vx) & 0 \\[.7em]
\hat{k}_x^{P}(\vx)\hat{k}_y^{P}(\vx) \kappa_1^P(\vx) & \hat{k}_x^{P}(\vx)^2\kappa_1^P(\vx) & 0 \\[.7em]
0 & 0 & \kappa_2^P(\vx) \\[0.5em]    \end{bmatrix}.
\end{equation}

\vspace{3mm}
In the aspect of the present treatise, the signature and the determinant of the Hessian in the stationary position is of importance.
In the followings, these properties will be discussed when the SPA is applied for the Rayleigh integral.

\subsection{Hessian for the SPA applied for the Rayleigh integral}

Assume that the Rayleigh integral is written on the plane $y = 0$ for an arbitrary sound field $P$, with the high frequency gradient approximation applied, reading as
\begin{equation}
P(\vx,\omega) = 2 \int_{-\infty}^{\infty} \ti k_y^P(\vxo) P(\vxo, \omega) \, G(\vx-\vxo,\omega) \, \td x_0 \, \td z_0.
\end{equation}
The elements of the 3x3 Hessian of the integrand's phase function (with suppressing its space dependency) are given as
\begin{equation}
\label{eq:app:Hessian_for_Rayleigh}
H_{ij}^{P \cdot G} = H_{ij}^P + H_{ij}^G = \frac{\partial^2}{\partial x_{0 i} \partial x_{0 j}}\left( \phi^P(\vxo,\omega) + \phi^G(\vx-\vxo,\omega) \right), \hspace{3mm} i,j = 1,2,3.
\end{equation}
The eigenvalues and eigenvectors of $\mH^P$ and $\mH^G$ are the principal curvatures and the corresponding unit vectors of the target field and the Green's function.

By definition, the stationary position for the integral is found where
\begin{equation}
\label{eq:app:Rayleigh_stat_point}
\Dxo \phi^P(\vxo,\omega) = -\Dxo \phi^G(\vx-\vxo,\omega).  %\\
\end{equation}
Geometrically speaking, the stationary position $\vxo^*(\vx)$ is found, where the normals of the involved wavefronts coincide on the Rayleigh plane, i.e. where the wavefront of $P$ is tangential with the spherical wavefront of the Green's function.
Therefore, in the stationary position the tangent planes of the involved wavefronts coincide. 

Since the principal directions for the Green's function's wavefront are arbitrary, orthogonal unit vector-pair in the tangent plane, in the stationary position they can be chosen to coincide with the principal directions of $\mH^P$.
Therefore, at the stationary point the eigenvectors of $\mH^P$ and $\mH^G$ coincide and their eigenvalues are additive.
The eigenvalues of the resultant matrix are therefore simply given as
\begin{align}
\label{eq:app:propagated_curvature}
\lambda^{P \cdot G}_1(\vx) &= \lambda_1^P(\vxo^*(\vx)) + \lambda_1^G(\vx-\vxo^*(\vx)) = -k \left( \kappa_1^P(\vxo^*(\vx)) + \kappa_1^G(\vx-\vxo^*(\vx)) \right), \\
\lambda^{P \cdot G}_2(\vx) &= \lambda_2^P(\vxo^*(\vx)) + \lambda_2^G(\vx-\vxo^*(\vx)) = -k \left( \kappa_2^P(\vxo^*(\vx)) + \kappa_2^G(\vx-\vxo^*(\vx)) \right). \\
\lambda^{P \cdot G}_3(\vx) &= \lambda_3^P(\vxo^*(\vx)) + \lambda_3^G(\vx-\vxo^*(\vx)) = 0.
\end{align}
%\vspace{0.5mm}

In the aspect of the present thesis, it is important to investigate the local principal curvatures and the principal radii of the wavefront of $P$ at the evaluation point $\vx$, i.e. how these quantities change over the propagation from the Rayleigh plane.
According to the SPA, $P(\vx,\omega)$ is obtained from the stationary value of the Rayleigh integral.
Therefore, curvature of $P$, measured at $\vx$ is obtained as the eigenvalues (normalized by $-k$) of the integrand's Hessian, taken at the stationary point, given by
\begin{equation}
\mH^{P}(\vx) = \frac{\partial^2}{\partial x_{i} \partial x_{j}} \phi^P ( \vx,\omega) = \frac{\partial^2}{\partial x_{i} \partial x_{j}}\left( \phi^P(\vxo^*(\vx),\omega) + \phi^G(\vx-\vxo^*(\vx),\omega) \right).
\end{equation}
First, the phase Hessian at the receiver position $\mH^{P}(\vx)$ is expressed.
By applying the chain rule, the elements of the Hessian can be written, as
\begin{multline}
H_{ij}^P(\vx) 
= 
\frac{\partial}{\partial x_{j}} \left( \frac{\partial x_{0 k}}{\partial x_{i} } \frac{\partial \phi^P(x_{0 k},\omega)}{\partial x_{0 k} } + 
\frac{\partial ( x_k - x_{0 k}) }{\partial x_{i} }  \frac{\partial \phi^G(x_k-x_{0 k},\omega)}{\partial( x_k - x_{0 k}) }   \right) = \\
\frac{\partial^2 x_{0 k}}{\partial x_{i} \partial x_{j}} 
\underbrace{
\left( \frac{\partial \phi^P(x_{0 k},\omega)}{\partial x_{0 k} } 
+ \frac{\partial \phi^G(x_{kl}-x_{0 kl},\omega)}{\partial( x_k - x_{0 k}) } \right)}_{ = 0} +
\\ 
 \frac{\partial x_{0 k}}{\partial x_{i} } \frac{\partial x_{0 l}}{\partial x_{j}} 
\underbrace{ \frac{\partial^2 \phi^P(x_{0 k},\omega)}{\partial x_{0 k}\partial x_{0 l} }}_{H^P_{kl}}
+  \frac{\partial ( x_{k} - x_{0 k}) }{\partial x_{i} } 
 \frac{\partial ( x_{l} - x_{0 l}) }{\partial x_{j} }
\underbrace{ \frac{\partial^2 \phi^G(x_{kl}-x_{0 kl},\omega)}{\partial( x_k - x_{0 k}) \partial ( x_{l} - x_{0 l})} }_{H^G_{kl}},
\end{multline}
where the first underbraced part equals zero due to the definition of the stationary position.
By introducing the matrix $\Dx \vxo$ for the rate of change of the stationary position, with respect to the change in any coordinate of the evaluation point, defined as
\begin{equation}
\label{eq:app:stat_point_grad}
[\Dx \vxo]_{lj} =  \frac{\partial x_{0 l}	}{\partial x_j} =  
\begin{bmatrix} \frac{\partial \vxo^*(\vx)}{\partial x} \hspace{1mm} \bigg| & \hspace{-2.5mm}  \frac{\partial \vxo^*(\vx)}{ \partial y} \hspace{1mm} \bigg| & \hspace{-2.5mm} \frac{\partial \vxo^*(\vx)}{ \partial z} 
 \\[.3em] \end{bmatrix},
\end{equation}
the Hessian under consideration can be written in the matrix form
\begin{equation}
\label{eq:app:Hpij}
\mH^P(\vx) = (\Dx \vxo)^{\mathrm{T}}  \mH^P(\vxo) (\Dx \vxo) + \left( \mathbf{I} - \Dx \vxo \right)^{\mathrm{T}} \mH^G(\vx-\vxo) \left( \mathbf{I} - \Dx \vxo \right).
\end{equation}

In order to express the gradient of the stationary position \eqref{eq:app:stat_point_grad}, its definition \eqref{eq:app:Rayleigh_stat_point} is reconsidered:
\begin{equation}
\frac{\partial}{\partial x_{0 k}} \phi^P(\vxo^*(\vx),\omega) - \frac{\partial}{\partial x_{0 k}}\phi^G(\vx-\vxo^*(\vx),\omega)  = 0.
\end{equation}
Taking a further derivative with respect to $x_j$ with applying the chain rule results in
\begin{equation}
\underbrace{\frac{\partial^2}{\partial x_{0 l} \partial x_{0 k}} \phi^P(\vxo^*(\vx),\omega)}_{H^P_{kl}} \frac{\partial x_{0 l}}{\partial x_j}
- \underbrace{\frac{\partial^2}{\partial x_{0 l} \partial x_{0 k}} \phi^G(\vx-\vxo^*(\vx),\omega)}_{H^G_{kl} }  \frac{\partial}{\partial x_j} \left( x_l - x_{0 l} ) \right) = 0,
\end{equation}
or written in a matrix form
\begin{equation}
\mH^P(\vxo) \, \Dx \vxo - \mH^G(\vx-\vxo) \left( \mathbf{I} - \Dx \vxo \right) = 0 \hspace{2mm} \rightarrow \hspace{2mm} \left( \mH^P +  \mH^G \right) \, \Dx \vxo = \mH^G. 
\end{equation}
In order to invert the matrix $\mH^P +  \mH^G$, the involved matrices are expressed in the form, given in \eqref{Eq:App:Hessian_w_curvature}, i.e. by transforming it into its eigenspace:
\begin{equation}
 \mathbf{V} \left( \mathbf{K}^P + \mathbf{K}^G \right) \mathbf{V}^{\mathrm{T}} \cdot \Dx \vxo =   \mathbf{V} \mathbf{K}^G \mathbf{V}^{\mathrm{T}},
\end{equation}
where $\mathbf{K}^P$ and $\mathbf{K}^G$ are 2x2 diagonal matrices of the curvatures of $P$ and $G$ at the stationary point, and $\mathbf{V}$ is a 3x2 matrix, consisting of the two corresponding eigenvectors.
Since the two columns of $\mathbf{V}$ are orthonormal, and the inverse of a 2x2 diagonal matrix can be calculated easily, therefore the gradient of the stationary position reads as
\begin{equation}
\Dx \vxo =  \mathbf{V} \frac{\mathbf{K}^G}{\mathbf{K}^P + \mathbf{K}^G} \mathbf{V}^{\mathrm{T}} = 
\mathbf{V} 
\begin{bmatrix}
\frac{\kappa_1^G}{\kappa_1^P + \kappa_1^G} & 0 \\[.5em]
0 & \frac{\kappa_2^G}{\kappa_2^P + \kappa_2^G}
\\[0.3em]    \end{bmatrix}
\mathbf{V}^{\mathrm{T}},
\end{equation}
and obviously
\begin{equation}
\mathbf{I} - \Dx \vxo =  \mathbf{V} \frac{\mathbf{K}^P}{\mathbf{K}^P + \mathbf{K}^G} \mathbf{V}^{\mathrm{T}}
\end{equation}
holds.
	
Finally, the Hessian at the evaluation point, given by \eqref{eq:app:Hpij} is expressed in the same eigenspace with $\mH^P= -k \mathbf{V} \mathbf{K}^P \mathbf{V}^{\mathrm{T}}$ and $\mH^G = -k \mathbf{V} \mathbf{K}^G \mathbf{V}^{\mathrm{T}}$.
Exploiting that $\mathbf{V}^{\mathrm{T}}\mathbf{V} = \mathbf{I}$ leads to
\begin{align}
\mH^P(\vx) = 
-k \mathbf{V} \frac{ \mathbf{K}^P \mathbf{K}^G  }{\mathbf{K}^P + \mathbf{K}^G }   \mathbf{V}^{\mathrm{T}} 
&= -k \mathbf{V} 
\begin{bmatrix}
\frac{\kappa_1^P \kappa_1^G}{\kappa_1^P + \kappa_1^G} & 0 \\[.5em]
0 & \frac{\kappa_2^P \kappa_2^G}{\kappa_2^P + \kappa_2^G}
\\[0.3em]    \end{bmatrix}
\mathbf{V}^{\mathrm{T}}
= \\
&=
-k \mathbf{V} 
\begin{bmatrix}
\frac{1}{\rho_1^P+\rho_1^G} & 0 \\[.5em]
0 & \frac{1}{\rho_2^P+\rho_2^G}
\\[0.3em]    \end{bmatrix}
\mathbf{V}^{\mathrm{T}}
\end{align}
This result states, that if the Rayleigh integral describes a sound field at $\vx$, then the principal curvatures and radii of the field can be written as
\importantalign{Curvature change over propagation}{
\kappa^P(\vx) &= \frac{\kappa^P(\vxo^*(\vx)) \kappa^G(\vx-\vxo^*(\vx)) }{\kappa^P(\vxo^*(\vx)) + \kappa^G(\vx-\vxo^*(\vx)}, \nonumber \\
\rho^P(\vx) &= \rho^P(\vxo^*(\vx)) + \rho^G(\vx-\vxo^*(\vx)). 
\label{eq:app:propagated_radii}
}
Furthermore, the corresponding eigenvectors, i.e the direction of the largest and smallest curvature on the wavefront does not change along the direction of propagation.

Finally, the signature and the determinant of the Hessian's submatrices is investigated

\vspace{3mm}
\paragraph{Evaluation of the Rayleigh integral along the $xz$-dimensions:}
In case the Rayleigh integral is approximated by the SPA with respect to both $x$- and $z$-directions, the Hessian for the SPA may be expressed from \eqref{eq:app:Hessian_for_Rayleigh}, by removing the rows and columns, that contain the $y$ derivatives, hence by forming its 2x2 principal submatrix.
By removing the same rows and columns from the spectral description, based on \eqref{Eq:App:Hessian_w_curvature}, the Hessian of the integrand's phase can be expressed in the stationary point as
\begin{equation}
\resizebox{.95\hsize}{!}{$
\mH^{P \cdot G}(\vxo^*(\vx)) = -k 
\begin{bmatrix} 
v_{1 x}^2 \left( \kappa_1^P+\kappa_1^G \right)+ v_{2 x}^2 \left( \kappa_2^P+\kappa_2^G \right) & 
v_{1 x}v_{1 z} \left( \kappa_1^P+\kappa_1^G \right)+ v_{2 x}v_{2 z} \left( \kappa_2^P+\kappa_2^G \right) \\[.7em]
v_{1 x}v_{1 z} \left( \kappa_1^P+\kappa_1^G \right)+ v_{2 x}v_{2 z} \left( \kappa_2^P+\kappa_2^G \right) & 
v_{1 z}^2 \left( \kappa_1^P+\kappa_1^G \right)+ v_{2 z}^2 \left( \kappa_2^P+\kappa_2^G \right) \\[0.5em]    \end{bmatrix},
$}\end{equation}
with $\mathbf{v}_1 = \posvec{3}{v_{1 x}}{v_{1 y}}{v_{1 z}}$, $\mathbf{v}_2 = \posvec{3}{v_{2 x}}{v_{2 y}}{v_{2 z}}$, $\kappa^P = \kappa^P(\vxo^*(\vx))$, $\kappa^G = \kappa^G(\vx-\vxo^*(\vx))$.

The eigenvalues of this submatrix cannot be expressed in a general way, however the \emph{interlacing inequalities of principal submatrices} ensure that they have the same sign as $\lambda_2$ and $\lambda_3$.
The signature of the Hessian is therefore 
\begin{itemize}
\item assuming a divergent sound field, the eigenvalues of the Hessian are negative (the curvatures are positive) and the signature is given by (-2).
\item assuming a convergent sound field with both principal curvature being negative \emph{on the Rayleigh plane}, the signature of the Hessian depends on the evaluation position $\vx$.
On the parts of the space, where the curvature of wavefront $P$ is greater than that of the Green's function, the eigenvalues of the Hessian are positive and its signature is 2.
On other parts of the space, the signature is (-2).
\end{itemize}
In practice, it means that if the Rayleigh integral describes a sound field, propagating towards a focus point, then the signature for an evaluation point between the Rayleigh plane and the focus point is given by 2, and in other parts of the space, where the waves already diverge after passing the focus point, the signature is -2.

The determinant of of the Hessian is given by
\begin{equation}
\mathrm{det} \, \mH^P(\vxo^*)(\vx)  = -k \left( \kappa_1^P+\kappa_1^G \right) \left( \kappa_2^P+\kappa_2^G \right) \left( v_{2 x} v_{1_z} - v_{1 x} v_{2 z} \right)^2.
\end{equation}
By the definition of the cross product of vectors, the term $\left( v_{2 x} v_{1_z} - v_{1 x} v_{2 z} \right)$ is the second coordinate of the vector, being perpendicular to $\mathbf{v}_1$ and $\mathbf{v}_2$, i.e. of the normalized local wavenumber vector:
\begin{equation}
\resizebox{1\hsize}{!}{$
\mathrm{det} \, \mH^P(\vxo^*(\vxref))  = -k \left( \kappa_1^P(\vxo^*(\vx))+\kappa_1^G(\vx-\vxo^*(\vx)) \right) \left( \kappa_2^P(\vxo^*(\vx))+\kappa_2^G(\vx-\vxo^*(\vx)) \right) \hat{k}_y^P(\vxo^*(\vx))^2. $}
\end{equation}
This finding is not limited to the Rayleigh integral: if the Kirchhoff-Helmholtz integral is written onto a smooth, convex surface with the surface's curvature being significantly smaller than the wavefront curvature, then the surface can be considered locally plane, and the above given description holds with the substitution $\hat{k}_y^P(\vxo^*(\vx)) \rightarrow \hat{k}_{\mathrm{n}}^P(\vxo^*(\vx))$, being the normal component of the local wavenumber vector.
This statement is a consequence of the invariance of the determinant with respect to a linear transform.
	
The same formulation holds for the evaluation of a 2D Fourier integral.
In this case the determinant reads as
\begin{equation}
\mathrm{det} \, \mH^P(\vxo^*(\vx))  = -\frac{1}{k} \kappa_1^P(\vxo^*(k_x,k_z)) \kappa_2^P(\vxo^*(k_x,k_z	)) k_y^2.
\end{equation}


\paragraph{Evaluation of the Rayleigh integral along the $z$-dimension:}
In the specific case of the derivation of the 2.5D Rayleigh integral, only the integration along the $z$-dimension is approximated and the Hessian is simply given by $\phi''_{zz}(\vxo) =\phi^{P''}_{zz}(\vxo) + \phi^{G''}_{zz}(\vx-\vxo)$.
Requiring $k_z(\vx) \equiv 0$ to be satisfied in the horizontal plane of investigation guarantees that the second derivative is the principal curvature itself, thus around the stationary position
\begin{equation}
\phi''_{zz}(\vxo^*(\vx)) = -k \left( \kappa_2^P(\vxo^*(\vx)) + \kappa_2^G(\vx-\vxo^*(\vx)) \right).
\end{equation}
holds.
\chapter{Appendix C}
\section{Notes on the Hessian of the phase function}
\label{App:Hessian}

\subsection{Definition of principal curvatures and principal directions}
Assume a wave field, described by the general polar form $P(\vx,\omega) = A^P(\vx,\omega)\te^{\ti \phi^P (\vx,\omega)}$!
Supposing, that the amplitude changes slowly, compared to the phase function the local dispersion relation $| \nabla \phi(\vx,\omega) |= k$ holds and the equation describing an arbitrary wavefront i.e. $\phi^P(\vx,\omega) - C = 0$ is the \emph{normalform} of the given surface \cite{Hartmann1999}.
The Hessian matrix of the function is given by the symmetric matrix 
\begin{equation}
H = - \begin{bmatrix} 
\phi^{P''}_{xx}(\vx,\omega) & \phi^{P''}_{xy}(\vx,\omega) & \phi^{P''}_{xz}(\vx,\omega) \\[.7em]
\phi^{P''}_{xy}(\vx,\omega) & \phi^{P''}_{yy}(\vx,\omega) & \phi^{P''}_{yz}(\vx,\omega) \\[.7em]
\phi^{P''}_{xz}(\vx,\omega) & \phi^{P''}_{yz}(\vx,\omega) & \phi^{P''}_{zz}(\vx,\omega) \\[0.5em]    \end{bmatrix},
\end{equation}
with the eigenvalues $\lambda_1, \lambda_2, \lambda_3$ and the corresponding eigenvectors $\mathbf{v}_1, \mathbf{v}_2, \mathbf{v}_3$.
Since the function under consideration is a normalform, therefore the following properties hold
\begin{itemize}
\item $\lambda_3 = 0$ with the eigenvector $\mathbf{v}_3 = - \nabla \phi^P(\vx,\omega) /k = \hat{\vk}^P(\vx)$, i.e. being the normal of the wavefront
\item $\lambda_1 = \kappa^P_1(\vx)$, $\lambda_2 = \kappa^P_2(\vx)$ being the \emph{main or principal curvatures} of the wavefront and $\rho^P_1(\vx) = 1/\kappa^P_1(\vx)$, $\rho^P_2(\vx) = 1/\kappa^P_2(\vx)$ being the \emph{principal radii}.
The principal curvatures and radii are defined as the following:
Consider all the planes containing the normal of the surface at the point of investigation, defined by the surface normal and vector $\mathbf{v}$, being the tangent of the surface.
The curvature is defined as the quadratic form 
\begin{equation}
\kappa = \mathbf{v} H \mathbf{v}^{\mathrm{T}}.
\label{Eq:App:curvature_def}
\end{equation}
The main curvatures are then defined as the minimum and maximum values of curvature, i.e. the reciprocal of the osculating circles radii (the principal radii).
The corresponding eigenvectors $\mathbf{v}_2$ and $\mathbf{v}_3$ are the tangential, orthogonal unit vectors pointing into the direction of the maximal and minimal curvatures.
\end{itemize}

Note, that for the special case of the three-dimensional Green's function positioned at $\vxs$, only eigenvector $\lambda_3$ is well-defined over the spherical /\emph{umbilical} wavefront, being a radial directed normal vector.
$\lambda_2$ and $\lambda_3$ may be arbitrary orthogonal vectors in the plane, tangent to the surface in the point of investigation $\vxo$, with the corresponding principal curvatures being $\kappa_1 = \kappa_2 = 1/|\vxo - \vxs| $.


In the present treatise when dealing with 2.5D problems it is a standard prerequisition, that in the plane of investigation ($z = 0$) all the involved wave fields propagate along the horizontal direction ($k_z(x,y,0) \equiv 0$), thus the Hessian of the phase function becomes
\begin{equation}
H = - \begin{bmatrix} 
\phi^{P''}_{xx}(\vx,\omega) & \phi^{P''}_{xy}(\vx,\omega) & 0 \\[.7em]
\phi^{P''}_{xy}(\vx,\omega) & \phi^{P''}_{yy}(\vx,\omega) & 0 \\[.7em]
0 & 0 & \phi^{P''}_{zz}(\vx,\omega) \\[0.5em]    \end{bmatrix},
\end{equation}
with the trivial eigenvector/principal direction $\mathbf{v_2} = \posvec{3}{0}{0}{1}$ and the corresponding principal curvature $\kappa_2^P(\vx) =  \phi^{P''}_{zz}(\vx,\omega)$.
Furthermore, considering, that the eigenvector with a zero eigenvalue is given by $\mathbf{v}_3 = \hat{\vk}^P(\vx) = \posvec{3}{\hat{k}_x^P(\vx)}{\hat{k}_y^P(\vx)}{0}$ and $\mathbf{v}_1$ is orthogonal to  $\mathbf{v}_2$ and $\mathbf{v}_3$, therefore $\mathbf{v}_1 = \posvec{3}{\hat{k}_y^P(\vx)}{\hat{k}_x^P(\vx)}{0}$ holds.
From simple geometrical considerations and applying \eqref{Eq:App:curvature_def} the elements of the Hessian matrix can be expressed as
\begin{equation}
\label{Eq:App:Hessian_inplane}
H = - \begin{bmatrix} 
\hat{k}_y^{P}(\vx)^2 \kappa_1^P(\vx) & \hat{k}_x^{P}(\vx)\hat{k}_y^{P}(\vx)\kappa_1^P(\vx) & 0 \\[.7em]
\hat{k}_x^{P}(\vx)\hat{k}_y^{P}(\vx) \kappa_1^P(\vx) & \hat{k}_x^{P}(\vx)^2\kappa_1^P(\vx) & 0 \\[.7em]
0 & 0 & \kappa_2^P(\vx) \\[0.5em]    \end{bmatrix}.
\end{equation}

\vspace{3mm}
In the aspect of the present treatise the signature and the determinant of the Hessian in the stationary position is of importance.
In the followings these properties will be discussed when the SPA is applied for the Rayleigh integral.


\subsection{Hessian of the SPA applied for the Rayleigh-integral}

Assume the Rayleigh-integral, written onto the plane $y = 0$, with the high-frequency gradient approximation applied:
\begin{equation}
P(\vx,\omega) = 2 \int_{-\infty}^{\infty} \ti k_y^P(\vxo) P(\vxo, \omega) G(\vx-\vxo,\omega) \td x_0  \td z_0.
\end{equation}
The phase function under consideration and hence the Hessian of the phase is given by the sum
\begin{equation}
\phi(\vx,\omega) = \phi^P(\vxo,\omega) + \phi^G(\vx-\vxo,\omega), \hspace{1cm} H = H^P+ H^G.
\end{equation}
The Hessian is investigated in the stationary position.
The stationary position is found where the normals of the involved wavefronts coincide on the Rayleigh plane, i.e. where the wavefront of $P$ is tangential with the spherical wavefront of the Green's function.
Furthermore, since the principal directions of the Green's function's wavefront are not well-defined, they can be chosen to coincide with the principal directions of $P$.
Therefore at the stationary point the eigenvectors of $H^P$ and $H^G$ coincide and their eigenvalues are additive.
The eigenvalues of the resultant matrix are therefore simply given as
\begin{align}
\lambda_1 &= \lambda_1^P + \lambda_1^G = \kappa_1^P(\vxo) + \kappa_1^G(\vx-\vxo), \\
\lambda_2 &= \lambda_2^P + \lambda_2^G = \kappa_2^P(\vxo) + \kappa_2^G(\vx-\vxo). \\
\lambda_3 &= \lambda_3^P + \lambda_3^G = 0.
\end{align}

\vspace{3mm}
In the case of approximating the integral with respect to both $x$ and $z$ direction the Hessian of the SPA may be formed from the 3x3 Hessian by removing the rows and columns containing the $y$ derivatives, hence by forming its 2x2 principal submatrix.
The eigenvalues of this submatrix cannot be expressed in a general way, however the signature of the Hessian can be determined, by applying the \emph{interlacing inequalities of principal submatrices}:
In case of a hermitian matrix with the real eigenvalues in an increasing order, $\lambda_1, \lambda_2, ..., \lambda_n$, than for the $k-$th eigenvalue $\lambda'_k$ of its principal submatrix
\begin{equation}
\lambda_k \leq \lambda'_k \leq \lambda_{k+1}
\end{equation}
holds.
This inequality ensures that the eigenvalues of the 2x2 Hessian of the SPA has the same sign as $\lambda_2$ and $\lambda_3$, and the signature of the Hessian is 
\begin{itemize}
\item assuming a divergent sound field, the eigenvalues of the Hessian are positive and the signature is given by 2.
\item assuming a convergent sound field with both principal curvature being negative \emph{on the Rayleigh plane}, the signature of the Hessian depends on the evaluation position $\vx$.
On the parts of the space where the curvature of wavefront $P$ is greater in magnitude than that of the Green's function the eigenvalues of the Hessian are negative and it signature is -2.
On other parts of the space the signature is 2.
\end{itemize}
In practice with a simple example it means, that if the Rayleigh integral describes a sound field propagating toward a point, than the signature for an evaluation point between the Rayleigh plane and the focus point is given by -2, and in other parts of the space, where the waves already diverge after passing the focus point , the signature is 2.

In the special case of the derivation of the 2.5D Rayleigh integral only the integration along the $z$-dimension is approximated and the Hessian is simply given by $\phi''_{zz}(\vxo) =\phi^{P''}_{zz}(\vxo) + \phi^{G''}_{zz}(\vx-\vxo)$.
Requiring, that in the horizontal plane of investigation $k_z(\vx) \equiv 0$ guarantees, that the second derivative is the principal curvature itself (see below), thus
\begin{equation}
\phi''_{zz}(\vxo) = \kappa_2^P(\vxo) + \kappa_2^G(\vx-\vxo).
\end{equation}
Obviously for the sign of the second derivative the description given for the 2D SPA case holds.

Note, that the description above is not limited for the Rayleigh integral: if the Kirchhoff-Helmholtz integral is written onto a smooth convex surface with the surface's curvature being significantly smaller than the wavefront curvature, than the surface can be considered locally plane, and the above given description holds.

\chapter{Appendix D}
\section{2.5D Kirchhoff integral for a 3D point source}
\label{App:25D_KI}

Assume a 2D point source, located at $\vxs = \posvec{2}{x_s}{y_s}$.
The generated sound field inside the domain, bounded by the contour $C$ can be described by the 2D Kirchhoff approximation in terms of the 2D Green's function.
At position $\vx \in \mathbb{R}^2$ the field reads
\begin{equation}
P_{2\text{D}}(\vx,\omega) = 2 \oint_C w(\vxo) \ti k^P_{\mathrm{n}}(\vxo) P_{2\text{D}}(\vxo,\omega) G_{2\text{D}}(\vx-\vxo,\omega) \td C,
\end{equation} 
with in the present case $P_{2\text{D}}(\vx,\omega) = G_{2\text{D}}(\vx-\vxs,\omega)$.
The equality holds withing the validity of the stationary phase approximation.

As a first step assume, that in three dimensions, the sound field of a vertical line source is to be described at $z=0$ in terms of the 3D Green's function.
The equation can be rewritten as
\begin{equation}
P_{2\text{D}}(\vx,\omega) = 2 \oint_C w(\vxo) \ti k^P_{\mathrm{n}}(\vxo) P_{2\text{D}}(\vxo,\omega)\int_{-\infty}^{\infty}  G_{3\text{D}}(\vx-\vxo,\omega) \td z_0 \td C.
\end{equation}
The inner integral can be approximated by the SPA with the stationary point lying trivially at $z=0$, resulting in the approximation of the 2D Green's function presented in equation \eqref{eq:HF_approx:2D_vs_3D_GF}:
\begin{equation}
P_{2\text{D}}(\vx,\omega) = 2 \oint_C w(\vxo) \ti k^P_{\mathrm{n}}(\vxo) P_{2\text{D}}(\vxo,\omega)  \sqrt{\frac{2 \pi |\vx-\vxo|}{\ti k }}G_{3\text{D}}(\vx-\vxo,\omega) \td C.
\label{Eq:App:25D_KI_2D_ps}
\end{equation}

Now assume, that instead of a 2D point source, a 3D point source is to be described by \eqref{Eq:App:25D_KI_2D_ps}.
Since the integral holds only for two dimensional sound fields the 3D field has to be formulated in terms of a 2D sound field with appropriate correction factors.
Since $P$ is the sound field of a point source in 2 or 3 dimensions
\begin{equation}
P_{2\text{D}}(\vx,\omega) = \sqrt{\frac{2 \pi |\vx-\vxs|}{\ti k }} P_{3\text{D}}(\vx,\omega), \hspace{1cm}
P_{2\text{D}}(\vxo,\omega) = \sqrt{\frac{2 \pi |\vxo-\vxs|}{\ti k }} P_{3\text{D}}(\vxo,\omega)
\end{equation} 
holds.
Rewriting \eqref{Eq:App:25D_KI_2D_ps} in terms of the 3D field variables
\begin{multline}
\sqrt{\frac{2 \pi |\vx-\vxs|}{\ti k }} P_{3\text{D}}(\vx,\omega) = 2 \oint_C w(\vxo) \ti k_{\mathrm{n}}^P(\vxo) \sqrt{\frac{2 \pi |\vxo-\vxs|}{\ti k }} P_{3\text{D}}(\vxo,\omega) \\
  \sqrt{\frac{2 \pi |\vx-\vxo|}{\ti k }}G_{3\text{D}}(\vx-\vxo,\omega) \td C.
\end{multline}
Finally, simplification and rearrangement leads to the 2.5D formulation of the Kirchhoff integral for the special case of a describing a 3D point source
\begin{equation}
P_{3\text{D}}(\vx,\omega) = 2 \oint_C w(\vxo) \ti k_{\mathrm{n}}^P(\vxo) \sqrt{\frac{2 \pi |\vx-\vxo|}{\ti k }} \sqrt{ \frac{|\vxo-\vxs|}{|\vx-\vxs|}} P_{3\text{D}}(\vxo,\omega) 
 G_{3\text{D}}(\vx-\vxo,\omega) \td C.
\end{equation}
The horizontal stationary position for the integral is then found, where $|\vx-\vxs| = |\vx - \vxo(\vx)| + |\vxo(\vx)-\vxs|$ is satisfied.

\end{appendices}


\bibliographystyle{plain}
\bibliography{dissertation}

\end{document}
