\documentclass[12pt,a4paper]{report}
%
\usepackage{layouts}
\usepackage{amsmath}
\usepackage{a4wide}
\usepackage[T1]{fontenc}
\usepackage[utf8]{inputenc}
\usepackage{xcolor}
\usepackage{listings}
\usepackage{graphicx,overpic,subfigure}
\usepackage{tikz}
\usetikzlibrary{positioning,arrows}
\usepackage{booktabs} 			% Nice tables
\usepackage{csquotes}			% Quotation
\usepackage{multirow} 			% Multirow cells in tables
\usepackage{rotating}
\usepackage{pdflscape}
\usepackage[small,bf]{caption}
\usepackage{ae,aecompl}
\usepackage{url}
\usepackage[american]{babel}
\usepackage{hyperref}
\usepackage{nomencl}
\usepackage[toc,page]{appendix}
\usepackage{amssymb}
\usepackage{steinmetz}
\usepackage{palatino}
\usepackage{array}
\usepackage{booktabs}
\usepackage{footnote}
\usepackage{multicol}
\makesavenoteenv{tabular}
%

\newcount\posveccount
\newcommand*\posvec[1]{
        \global\posveccount#1
        [
        \posvecnext
}
\def\posvecnext#1{
        #1
        \global\advance\posveccount-1
        \ifnum\posveccount>0
                ,\
                \expandafter\posvecnext
        \else
                ]^{\mathrm{T}}
        \fi
}

\newcount\colveccount
\newcommand*\colvec[1]{
        \global\colveccount#1
        \begin{bmatrix}
        \colvecnext
}
\def\colvecnext#1{
        #1
        \global\advance\colveccount-1
        \ifnum\colveccount>0
                \\[5pt]
                \expandafter\colvecnext
        \else
                \end{bmatrix}
        \fi
}

%\setcounter{secnumdepth}{2}

\newcommand{\dint}{\int\!\!\!\!\!\int}
\newcommand{\tint}{\int\!\!\!\!\int\!\!\!\!\int}
\newcommand{\qint}{\int\!\!\!\!\int\!\!\!\!\int\!\!\!\!\int}
\newcommand{\td}{\mathrm{d}}
\newcommand{\te}{\mathrm{e}}
\newcommand{\ti}{\mathrm{j}}
\newcommand{\sinfi}{\sin\varphi}
\newcommand{\cosfi}{\cos\varphi}
\newcommand{\sinteta}{\sin\theta}
\newcommand{\costeta}{\cos\theta}
\newcommand{\yref}{y_{\mathrm{ref}}}
\newcommand{\dref}{d_{\mathrm{ref}}}
\newcommand{\vx}{\mathbf{x}}
\newcommand{\vn}{\mathbf{n}}
\newcommand{\vxo}{\mathbf{x}_0}
\newcommand{\vni}{\mathbf{n}_{\mathrm{in}}}
\newcommand{\vno}{ \mathbf{n}_{\mathrm{out}} }
\newcommand{\vxs}{\mathbf{x}_{\mathrm{s}}}
\newcommand{\vxref}{\mathbf{x}_{\mathrm{ref}}}
\newcommand{\vk}{\mathbf{k}}
\newcommand{\vhk}{\hat{\mathbf{k}}}
\newcommand{\kn}{k_\mathrm{n}}
\newcommand{\Oi}{\Omega_{\mathrm{i}}}
\newcommand{\Oe}{\Omega_{\mathrm{e}}}
\newcommand{\dO}{\partial \Omega}
%
\renewcommand{\arraystretch}{1}

\title{Reproduction of moving sound sources applying a unified Wave Field Synthesis framework}
\date{\today \\
Budapest University of Technology and Economics, \\ Dept. of Networked Systems and Services, \\ Laboratory of Acoustics and Studio Technologies}
\author{Gergely Firtha}
\makenomenclature

\begin{document}
\pagenumbering{roman}

\maketitle
\tableofcontents
\printnomenclature
%
%\chapter{Introduction}
%
%\chapter{Mathematical Preliminaries}
%\label{sec:math}
%\section{Functions and distributions}
\section{Definition and properties of Fourier-transforms}

The temporal Fourier transform (or frequency content) of a function $f(t)$ is defined in the same manner as in the related literature e.g. \cite{Ahrens2012, Ahrens2010a}:
%
\begin{align}
\mathcal{F}_{t} \left\{  f(t) \right\}  = F(\omega) = \int_{-\infty}^{\infty} f(t) \, \te^{-\ti \,\omega \, t} \, \td t, \\
\mathcal{F}^{-1}_{t} \left\{  F(\omega) \right\}  = \frac{1}{2\pi} \int_{-\infty}^{\infty} F(\omega)\, \te^{\ti\, \omega\, t} \, \td \omega.
\label{Eq:Math:Temp_Fourier}
\end{align}
%
Similarly, the spatial Fourier transform (or wavenumber content) is defined as
%
\begin{align}
\mathcal{F}_{x} \left\{  f(x) \right\}  = \tilde{f}(k_x) = \int_{-\infty}^{\infty} f(x)\, \te^{\ti\, k_x\, x}\, \td x, \\
\mathcal{F}^{-1}_{x} \left\{  \tilde{f}(k_x) \right\}  = \frac{1}{2\pi} \int_{-\infty}^{\infty} \tilde{f}(k_x)\, \te^{-\ti\, k_x\, x}\, \td k_x.
\end{align}

Several important properties of Fourier-transform, used frequently trough the thesis are:
\begin{itemize}
\item Shift theorem
%
\item Convolution theorem
%
\item Differentiation theorem
\begin{equation}
\mathcal{F}_{t} \left\{ \frac{\partial}{\partial t} f(t) \right\}  = \ti \omega F(\omega).
\label{Eq:Math:Fourier_tr_diff}
\end{equation}
%
\item Similarity theorem
\begin{equation}
\mathcal{F}_{t} \left\{ f(a t) \right\}  = \frac{1}{|a|} F(\frac{\omega}{a}).
\label{Eq:Math:Fourier_tr_similarity}
\end{equation}
\end{itemize}
  

\section{Stochastic signal theory basics}
%
\chapter{Theory of wave propagation and radiation problems}
\pagenumbering{arabic}
\label{sec:theory}
In the following chapters the theoretical basis of sound radiation and sound field reproduction is introduced. The section starts with discussing the physics of sound propagation and radiation by deriving the formulation and solution of the governing homogeneous and inhomogeneous wave equations. Various integral representation of sound fields are presented including wave-number domain representation and the Kirchhoff-Helmholtz integral. After formulating the basic sound field synthesis problem, the explicit (Spectral Division Method) and implicit (Wave Field Synthesis) solutions are discussed for a linear loudspeaker distribution.
%

Sound is a mechanical disturbance propagating in an elastic fluid, causing an alternation in the pressure (along with density) and in the displacement of the medium's particles. The propagation of the disturbance is described fully by the acoustic wave equation. First the homogeneous wave equation is introduced briefly, which is valid for \emph{source-free} domains. For a detailed treatise on the derivation please refer to \cite{Beranek1993, Morse1968, Williams1999, Blackstock2000}.

\subsection{The homogeneous wave equation}

Consider a homogeneous, elastic fluid, modeled as an ideal gas with no viscosity. In the aspect of the present thesis it is feasible to investigate sound propagation solely in air at room temperature. 

%
The domain of investigation ie. where sound waves propagate is termed as \emph{sound field} hereinafter.
The acoustical quantities of the the sound field is described by \emph{dynamic field variables} in each point $\vx$ at each time instant $t$: the vector variable \emph{particle velocity} $\mathbf{v}(\vx,t)$ and the scalar \emph{instantaneous sound pressure} $p(\vx,t)$ superimposed onto the static pressure $P_0 \approx 10^5~\mathrm{Pa}$.
The medium is quiescent, meaning on average each particle is at rest with zero particle displacement --thus zero particle velocity-- and with the static pressure $P_0$. 
The presence of a sound wave causes incremental change in the instantaneous pressure and the particle velocity.
%

In order to apply a linear model for sound propagation two assumptions are made.
Since the traveling speed of thermal diffusion is small compared to the speed of sound and the acoustical wave length in the audible frequency range, it is feasible to assume, that heat exchange in the wave due to compression and expansion is negligible: the state changes are modeled as adiabatic changes.
Furthermore the alternation of the instantaneous sound pressure must be small compared to the static pressure, so the non-linear adiabatic state-change characteristics can be linearized around $P_0$. This later assumption is also fulfilled for pressure magnitudes below the threshold of pain of the human auditory system, as it was pointed out in \cite{Gumerov2004, Ahrens2012}.
%

The linear wave equation may be derived by utilizing two fundamental physical principles.
\begin{itemize}
\item \emph{The equation of motion:} By applying Newton's second law for an infinitesimal small volume of gas we obtain the connection between the particle velocity vector and the pressure field at each point at each time instant. The resulting \emph{Euler's equation} states that the force acting on the volume, which is proportional to the pressure on the surface causes an acceleration, which is given by the time derivative of the particle velocity:
\begin{equation}
\nabla p(\vx,t) = -\rho_0 \frac{\partial}{\partial t} \mathbf{v}(\vx,t),
\label{Eq:Theory:Eulers_equation}
\end{equation}
\nomenclature[2]{$\nabla$}{Gradient operator. In Descartes-coordinates it is given by $\nabla = \frac{\partial}{\partial x} \mathbf{e}_x + \frac{\partial}{\partial y} \mathbf{e}_y + \frac{\partial}{\partial z} \mathbf{e}_z$}.
%
where $\nabla$ is the gradient operator and $\rho_0$ is the fluid ambient density. In room temperature for the above given static pressure it is given as $\rho_0 = 1.18~\mathrm{kg}/\mathrm{m}^3$.
Equation \eqref{Eq:Theory:Eulers_equation} may be transformed wrt. time into the angular frequency domain, supposing steady-state harmonic wave solutions. Using the differentiation theorem the frequency domain Euler's equation reads
\begin{equation}
\nabla P(\vx,\omega) = \ti \omega \rho_0 \mathbf{V}(\vx,\omega),
\label{Eq:Theory:Freq_Eulers_equation}
\end{equation}

\item \emph{The gas law:} For adiabatic processes the change of state is governed by the following equation
\begin{equation}
P V^{\gamma} = \mathrm{constant},
\label{Eq:Theory:Adiabatic_change}
\end{equation}
where $\gamma = c_P/c_V$ is the ratio of specific heat of the fluid in constant pressure and with constant volume. For air $\gamma = 1.4$. Linerization of \eqref{Eq:Theory:Adiabatic_change} and expressing the change in pressure --equaling the instantaneous pressure-- one obtains
\begin{equation}
\Delta P = p(\vx,t) = -\gamma P_0 \frac{\Delta V}{V_0},
\end{equation}
where $V_0$ is the undisturbed volume. The relative change of volume may be expressed as the sum of particle displacement over the boundary surface. Applying the definition of divergence and expressing the equation in terms of particle velocity yields
\begin{equation}
\frac{\partial}{\partial t} p(\vx,t) = -\gamma P_0 \nabla \cdot \mathbf{v}(\vx,t),
\label{Eq:Theory:Second_eq}
\end{equation}
where $\nabla \cdot$ is the \emph{divergence operator}.
\end{itemize}
%
Taking the derivative wrt. time of equation \ref{Eq:Theory:Second_eq} and the divergence of equation \ref{Eq:Theory:Eulers_equation} the particle velocity may be eliminated. By using the \emph{Laplacian-operator} $\nabla \cdot \nabla = \nabla^2$ the scalar linear homogeneous wave equation is obtained for the sound pressure
\begin{equation}
\nabla^2 p(\vx,t) - \frac{1}{c^2} \frac{\partial^2}{\partial t^2} p(\vx,t) = 0,
\label{Eq:Theory:Scalar_wave_equation}
\end{equation}
\nomenclature[1]{$c$}{Speed of sound}%
\nomenclature[3]{$\nabla^2$}{Laplacian operator. In Descartes-coordinates: $\nabla^2 = \frac{\partial^2}{\partial x^2} + \frac{\partial^2}{\partial y^2} +  \frac{\partial^2}{\partial z^2}$}%
where $c \equiv \sqrt{ \frac{\gamma P_0}{\rho_0} }$ is the speed of the sound wave in the medium. For air in room temperature it is given as $c = 343.1 ~ \mathrm{m}/\mathrm{s}$.

The instantaneous pressure may be also eliminated. In this case the vector wave equation is obtained for the particle velocity, valid for curl-free media
\begin{equation}
\nabla^2 \mathbf{v}(\vx,t) - \frac{1}{c^2} \frac{\partial^2}{\partial t^2} \mathbf{v}(\vx,t) = 0,
\label{Eq:Theory:Vector_wave_equation}
\end{equation}
where $\nabla^2 = \nabla \left( \nabla \cdot \right)$.
%
The wave equations fully describe the properties of acoustic wave propagation as long as the the assumptions are fulfilled.
%

\vspace{3mm}
%
Equation \eqref{Eq:Theory:Scalar_wave_equation} can be transformed into the frequency domain by performing a temporal Fourier-transform according to \eqref{Eq:Math:Temp_Fourier}. By using the differentiation property of Fourier-transform given in \eqref{Eq:Math:Fourier_tr_diff} the \emph{homogeneous Helmholtz-equation is obtained}:
\begin{equation}
\nabla^2 P(\vx,\omega) + k^2 P(\vx,\omega) = 0,
\label{Eq:Theory:Homog_Helmholtz}
\end{equation}
where $k$ is the \emph{acoustic wavenumber}, which is related to the temporal frequency trough the \emph{dispersion relation}:
\begin{equation}
k = \frac{\omega}{c}.
\end{equation}
%
Equation \eqref{Eq:Theory:Homog_Helmholtz} must hold for every physically possible \emph{steady-state} wave form with harmonic time-dependence for a source-free volume (which latter is indicated with the zero load term on the right side). In the aspect of the present thesis the time-domain wave equation is rarely solved, therefore in the followings the general solution for the Helmholtz-equation is presented.

\subsection{Solution of the homogeneous wave equation}

Since understanding plane wave theory is of major importance in the aspect of the present thesis we will focus our interest to the free-field solution of the homogeneous Helmholtz-equation in Cartesian coordinate system.

The Descartes coordinate form of the Laplace-operator is given by
\begin{equation}
\nabla^2 = \frac{\partial^2}{\partial x^2} + \frac{\partial^2}{\partial y^2} +  \frac{\partial^2}{\partial z^2}.
\end{equation}
The general solution for the Helmholtz-equation is obtained by the separation of variables \cite{Devaney2012}: let's try to find the solution in the form
\begin{equation}
P(\vx,\omega) = \hat{P}(\omega) X(x)Y(y)Z(z).
\label{Eq:Theory:Seperated_variables}
\end{equation}
Let's substitute it into \eqref{Eq:Theory:Homog_Helmholtz} and divide both sides by $\hat{P}(\omega) X(x)Y(y)Z(z)$!
\begin{equation}
\underbrace{\frac{\partial^2 X(x)}{\partial x^2}\frac{1}{X(x)}}_{-k_x^2} + 
\underbrace{\frac{\partial^2 Y(y)}{\partial y^2}\frac{1}{Y(y)}}_{-k_y^2} + 
\underbrace{\frac{\partial^2 Z(z)}{\partial z^2}\frac{1}{Z(z)}}_{-k_z^2}
= - k^2.
\label{Eq:Theory:Seperated_variables_expanded}
\end{equation}
Since in the equation each term contains a total derivative --independent from any other variable-- equality may hold only if each term is constant. These constant are denoted by $k_x-k_y-k_z$. Consequently each part of the equation leads to a simple eigenvalue problem, for which the eigenfunction solution is well-known. Given eg. for $x$-variable it reads
\begin{equation}
\frac{\partial^2 X(x)}{\partial x^2} = -k_x^2 X(x) \hspace{5mm} \rightarrow \hspace{5mm} X(x) = A_1 \te^{-\ti k_x x} + A_2 \te^{\ti k_x x}.
\end{equation}
The solutions may substituted back to equation \eqref{Eq:Theory:Seperated_variables}. In order to include every possible solution the general solution for the free-field homogeneous Helmholtz-equation is yielded by summation over all possible values of $k_x-k_y-k_z$ weighted by arbitrary constants. However, the variables are not independent, since for a fixed temporal frequency they are related according the dispersion relation
(resulting from \eqref{Eq:Theory:Seperated_variables_expanded}):
\begin{equation}
k^2 = \left( \frac{\omega}{c} \right)^2 = k_x^2 + k_y^2 + k_z^2.
\end{equation}
As a dependent variable we will use $k_y$ trough this treatise so that
\begin{equation}
k_y = \sqrt{ k^2 - k_x^2 - k_z^2 }.
\end{equation}
Using this and by denoting the arbitrary constant by $\hat{P}(k_x,k_z, \omega)$ the general solution reads
\begin{equation}
P(\vx,\omega) = \frac{1}{4\pi^2}\iint_{-\infty}^{\infty} \hat{P}(k_x,k_z, \omega)  \te^{- \ti \left( k_x x + k_y y + k_z z \right) }
\td k_x\td k_z.
\label{Eq:Theory:Helmholtz_Inverse_Fourier}
\end{equation}
Constant $\frac{1}{4\pi^2}$ is introduced due to proper Fourier-transform normalization as we will see later.

\vspace{3mm}
One separated solution from the integral is in the form of \cite{Williams1999}
\begin{equation}
P(\vx,\omega) = \hat{P}(\omega) \te^{-\ti \left( k_x x + k_y y + k_z z \right) } =  \hat{P}(\omega) \te^{-\ti \mathbf{k}^{\mathrm{T}} \vx },
\end{equation}
where $\mathbf{k}^{\mathrm{T}} = [k_x,\ k_y,\ k_z]$ is the wavenumber vector, with the length equaling the acoustic wavenumber $k = \sqrt{ \mathbf{k}^{\mathrm{T}}  \mathbf{k}}$.
The solution represents a \emph{plane wave} component traveling in the direction $\mathbf{k}$ with the acoustic wavelength of $\lambda = 2\pi/k$. The terminology indicates that the surface of constant phase points are lying along an infinite plane. Refer to figure xy for the illustration of a traveling plane wave.

\begin{figure}[!h]
	\centering
	\begin{overpic}[width = 1\columnwidth]{Figures/Theory/plane_wave.png}
	\end{overpic}
\caption{Plane Wave}
	\label{Fig:Theory:plane_wave}
\end{figure}

As it is indicated in the figure $k_x-k_y-k_z$ variables are the $x-y-z$ directional components of the wavenumber vector. For the sake of simplicity assume that $k_z = 0$, thus the propagation direction of the plane wave is parallel with the $z=0$ plane. In this case the wavenumber components are expressed as
\begin{eqnarray}
k_x = k \sin \theta , \\
k_y = k \cos \theta .
\end{eqnarray}
  
\subsubsection{Evanescent waves}
It is important to note, that there is no constraint on the values of $k_x^2$ and $k_z^2$ as long as they are real, their value may span from $-\infty$ to $\infty$. The plane wave equation is satisfied also when $k_x>k$ or $k_z>k$. Resulting from the dispersion relation in these cases $k_y$ becomes complex, reading
\begin{equation}
k_y = -\ti \sqrt{ k_x^2 + k_z^2 - k^2 }=-\ti k_y',
\end{equation}
by ignoring the non-physical positive sign solution.
The expressed waves describe plane waves, propagating perpendicular to the $y$-axis, exhibiting an exponential decaying amplitude along $y$-direction (see Figure \ref{Fig:Theory:plane_wave} (b)):
\begin{equation}
P(\vx,\omega) = \hat{P}(\omega) \te^{-k_y' y} \te^{-\ti \left( k_x x + k_z z \right) }.
\end{equation}

In these cases component of the wave-length is shorter, then the acoustic wave-length. As a consequence the wave can not propagate from the $y = 0$ surface, but an exponentially decaying radiation phenomena occurs. These type of waves are termed \emph{evanescent waves} opposed to \emph{propagating waves}, when all wavenumber components are real valued.

Evanescent waves are often the results of the difference between the speed of sound in different materials: in solids the speed of sound is significantly higher, than in the air. As a consequence in case of e.g. a vibrating solid surface important higher-order modes will not be radiated into the free-space, since the wave length on the surface of the object is shorter, than the acoustic wave length would be in air. In these cases air above the surface acts as a hydrodynamic short-circuit.

Due to the foregoing the presence of evanescent waves is of central importance in the aspect of \emph{Nearfield Acoustic Holography}, when one needs a high-resolution image from the velocity distribution on the vibrating object's surface.
However plane wave contribution is often neglected in the aspect of Sound Field Synthesis, when the listener is relatively far from the secondary loudspeaker array.


\subsubsection{The Angular Spectrum and Wave Field Extrapolation}

Loosely speaking an arbitrary 3-dimensional function could be expanded into exponential function via a spatial Fourier-transformation, for which a Fourier-transform exist. The dispersion relation sifts the solutions of the Helmoltz-equation. This process could be treated mathematically by multiplying the 3-dimensional inverse Fourier-transform by $\delta(k_y - \sqrt{k^2-k_x^2-k_z^2})$ and integrating wrt. $k_y$.
As stated in the foregoing an arbitrary free-field sound field can be therefore written in the form of equation \eqref{Eq:Theory:Helmholtz_Inverse_Fourier}.
This formulation is termed as the \emph{angular spectrum representation} \cite{Ahrens2010phd, Ahrens2012, Williams1999} or the \emph{plane wave expansion} \cite{Spors2005} of the sound field.
By expressing the pressure at the infinite plane $y=0$ the interpretation of $\hat{P}(k_x,k_y, \omega)$ is revealed
\begin{equation}
P(x,0,z,\omega) = \frac{1}{4\pi^2}\iint_{-\infty}^{\infty} \hat{P}(k_x,k_z, \omega)  \te^{- \ti k_x x} \te^{- \ti k_z z  }
\td k_x\td k_z.
\label{Eq:Theory:P_x0z}
\end{equation}
It is clear, that \eqref{Eq:Theory:P_x0z} expresses a double inverse Fourier-transform wrt. $k_x-k_z$-variables. 
The \emph{angular spectrum}, or \emph{plane wave expansion coefficients} $\hat{P}(k_x,k_z, \omega)$ can be therefore expressed as the forward Fourier-transform of the pressure distribution at $y=0$
\begin{equation}
\hat{P}(k_x,k_z, \omega) = \iint_{-\infty}^{\infty} P(x,0,z,\omega)  \te^{ \ti k_x x} \te^{ \ti k_z z  }\td x \td z.
\end{equation}
From now on the domain, denoted by $k_x$,$k_z$ is termed as the \emph{wavenumber domain}.
Equation \eqref{Eq:Theory:Helmholtz_Inverse_Fourier} therefore constitutes a connection between the pressure distribution of an arbitrary sound field measured an arbitrary point and at the plane $y=0$. In the wave-number domain the equation reads:
\begin{equation}
\mathcal{F}_x\mathcal{F}_z \left\{ P(\vx,\omega) \right\} = \hat{P}(k_x,y,k_z,\omega) = \hat{P}(k_x,0,k_z,\omega) \te^{-\ti k_y y}.
\label{Eq:Theory:Wave_field_extrapolation}
\end{equation}
Note, that wave propagation is determined by the phase change of the plane wave expansion's $y$-component, therefore generally speaking the following equation holds:
\begin{equation}
\hat{P}(k_x,y,k_z,\omega) = \hat{P}(k_x,y',k_z,\omega) \te^{-\ti k_y ( y - y' ) }.
\end{equation}

By taking the inverse Fourier transform of both sides:
\begin{equation}
P(\vx,\omega) = \frac{1}{4\pi^2}\iint_{-\infty}^{\infty} \hat{P}(k_x,y',k_z,\omega) \te^{-\ti k_y ( y - y' ) }  \te^{- \ti \left( k_x x + k_y y + k_z z \right) }
\td k_x\td k_z.
\label{Eq:Theory:Pressure_propagated}
\end{equation}
Taking the normal derivative of $\hat{P}(k_x,y',k_z,\omega)$ using the Fourier-transformation differentiation theorem \eqref{Eq:Math:Fourier_tr_diff} and using the Euler's equation \eqref{Eq:Theory:Eulers_equation} to relate the pressure and the normal velocity the following expression is obtained
\begin{equation}
P(\vx,\omega) = \frac{1}{4\pi^2}\iint_{-\infty}^{\infty} \rho_0 c k\hat{V}_{\mathrm{n}}(k_x,y',k_z,\omega) \frac{\te^{-\ti k_y ( y - y' ) } }{k_y} \te^{- \ti \left( k_x x + k_y y + k_z z \right) }
\td k_x\td k_z.
\label{Eq:Theory:Velocity_propagated}
\end{equation}

These formulations are extremely important, and plays a leading role in the field of Fourier-acoustics. They state that an arbitrary sound field is completely determined by either the pressure, or by the normal velocity component, measured along an infinite plane. Wave propagation is calculated by multiplying the measured spectra with an exponential term, referred as the propagators: from \eqref{Eq:Theory:Pressure_propagated} term $\te^{-\ti k_y ( y - y' ) }$ is referred as the \emph{pressure propagator} and from \eqref{Eq:Theory:Velocity_propagated} term $\frac{\te^{-\ti k_y ( y - y' ) } }{k_y}$ is called the \emph{velocity propagator}.

These statements are completely equivalent with the \emph{Ralyeigh-integral} formulations of sound fields, and these integral formulations could be derived directly from equations \eqref{Eq:Theory:Pressure_propagated}-\eqref{Eq:Theory:Vector_wave_equation}. In the present thesis however the Rayleigh-integrals will be derived from the Kirchhoff-Helmholtz integral formulation.
The importance of these statements will be further investigated in the latter sections, dealing with Sound Field Synthesis using a planar secondary source distribution.

\vspace{3mm}
Similarly to the Cartesian-solution the general solution for the free-field homogeneous Helmholtz equation can be found for spherical and cylindrical coordinate system. The results are given also in the form of an infinite series of spherical and cylindrical harmonics.These functions connect the radiated sound at an arbitrary point, and the sound field, measured on a spherical or infinite cylindrical surface. These solutions are of great importance when spherical, or circular secondary source distributions are applied for sound field reconstruction. Since the present thesis deal exclusively with planar and linear loudspeaker arrays, there the presentation of the spherical and cylindrical solutions are omitted. For a detailed investigation
please refer to \cite{Williams1999, Zotter2009phd, Ahrens2012}

\subsubsection{Boundary conditions}
\label{Section:Theory:Boundary_conditions}

\begin{figure}
	\centering
	\begin{overpic}[width = .5\columnwidth]{Figures/Theory/boundary_conditions.png}
	\put(20,14){$\mathbf{n}$}
	\put(45,30){$\Omega_i$}
	\put(80,55){$\Omega_e$}
	\put(10,44){$\partial \Omega$}
	\end{overpic}
	\caption{Boundary conditions}
	\label{Fig:Theory:bounday_condition}
\end{figure}

So far we considered wave propagation in free-field, ie. no boundaries are present. In order to be able to solve the wave equation both inital conditions and boundary must be known. As initial conditions trough the present thesis we suppose  \emph{Cauchy initial conditions}, by setting $p(\vx,0) = 0$, $\frac{\partial}{\partial t}p(\vx,t)|_{t=0} = 0$.

In the presence of boundaries the wave field must satisfy prescribed boundary conditions.
If the domain of interest is the exterior of the enclosing boundary, while the sources of radiation is inside the volume --or it is the vibrating boundary surface itself-- the problem to be solved is termed as an \emph{exterior radiation problem}. On the other hand, if we want to determine the sound field inside a source-free volume we are facing an \emph{interior problem}.

The boundary conditions are typically continuous pressure or particle velocity. By supposing zero pressure or velocity on the boundary surface \emph{homogeneous boundary conditions} are considered. Non-zero field variables on the other hand represent a vibrating surface and are termed \emph{inhomogeneous bondary conditions}.

In the aspect of this thesis two important types of boundary conditions are of interest:
\begin{itemize}
\item \emph{Dirichlet boundary condition}:
Dirichlet boundary condition prescribes the pressure, measured on the boundary surface. The homogeneous Dirichlet boundary conditions are thus
\begin{equation}
P(\vx,\omega) = 0, \hspace{3mm} \forall \hspace{3mm} \vx \in \partial \Omega.
\end{equation}
These types of boundaries are called \emph{sound-soft}, or \emph{pressure release} boundaries. These types of boundary conditions are used to model eg. the surface of the ocean wrt. a wave propagating in the water \cite{Blackstock2000, Ziomek1995}.

The inhomogeneous Dirichlet boundary condition assumes a prescribed pressure value on the boundary surface:
\begin{equation}
P(\vx,\omega) = f_D(\vx,\omega), \hspace{3mm} \forall \hspace{3mm} \vx \in \partial \Omega.
\end{equation}

\item \emph{Neumann boundary condition}:
Neumann boundary condition gives the normal derivative of the pressure on the boundary surface, ie. prescribes the normal velocity of the surface. For the sake of simplicity the normal derivative taken on the surface uses the following notation and definition
\begin{equation}
\frac{\partial}{\partial n} f(\vx)\equiv \left. \frac{\partial}{\partial \mathbf{n}(\vx)} f(\vx) \right|_{\partial \Omega} \equiv \left. \langle \nabla f(\vx), \mathbf{n}(\vx) \rangle \right|_{\partial \Omega},
\end{equation}
where $ \mathbf{n}(\vx) $ is the normal vector of the boundary surface. For interior problems the inward pointing normal is used.

Homogeneous Neumann boundary condition are
\begin{equation}
\frac{\partial}{\partial  n }P(\vx,\omega)= 0.
\end{equation}
These type of boundaries are termed as \emph{sound hard}, or \emph{rigid} boundaries representing the fact, that it is ensured, that any incident wave can not move the boundary surface.

Inhomogeneous Neumann boundary conditions are given by
\begin{equation}
\frac{\partial}{\partial n }P(\vx,\omega) = f_N(\vx,\omega), \hspace{3mm} \forall \hspace{3mm} \vx \in \partial \Omega.
\end{equation}
Vibrating surfaces --eg. a mounted loudspeaker, or a baffled piston-- are most often modeled with these type of boundary conditions.

\end{itemize}

\newpage
\subsection{The inhomogeneous wave equation and the Green's function}

So far we investigated wave propagation in source-free volumes. Sources may be included into the wave-equation resulting in the inhomogeneous wave equation
\begin{equation}
\nabla^2 p(\vx,t) -\frac{1}{c^2}\frac{\partial^2}{\partial t^2}p(\vx,t) = -q(\vx,t),
\label{Eq:Theory:Inhomogene_wave_eq_time_domain}
\end{equation}
and by transforming wrt. time in the inhomogeneous Helmholtz-equation:
\begin{equation}
(\nabla^2 + k^2 ) P(\vx,\omega ) = -Q(\mathrm{x},\omega).
\end{equation}
Term $q(\vx,t)$ is referred as the \emph{load term}, and it describes the spatial extension and time history of the excitation.

It should be noted, that the solution of the inhomogeneous wave equation is not unique, since any solution for the homogeneous wave equation may be added to the solution, the inhomogeneous wave equation is still satisfied. In order to obtain a unique solution again, we impose \emph{Cauchy initial conditions}.

Note, that since the free-field solution is considered no boundary conditions are assumed except for the \emph{Sommerfeld-radiation condition}, which ensures that no reflected wave are present: the sound field consist of only outgoing waves. Mathematically it is ensured by implying boundary conditions at infinity:
\begin{equation}
\lim_{r \rightarrow \infty} r \left( \frac{\partial}{\partial r}P(\vx,\omega) +\ti \frac{\omega}{c}P(\vx,\omega) \right) = 0.
\label{Eq:Theory:Sommerfeld_radiation_condition}
\end{equation}

\vspace{3mm}
A common way to obtain the solution for the inhomogeneous wave equation is using the \emph{Green's function}. We define the \emph{3D free-field Green's function} as the solution for the following equation \cite{Gumerov2004, Williams1999}
\begin{equation}
\nabla^2 g(\vx|\vxo,t) -\frac{1}{c^2}\frac{\partial^2}{\partial t^2} g(\vx|\vxo,t) = -\delta\left( \vx - \vxo \right)\delta\left( t - t_0 \right),
\label{Eq:Theory:Green_function_def}
\end{equation}
where $\delta()$ is the Dirac-delta distribution. The Green's function therefore describes the propagation of waves measured at $\vx$, deriving from a point source located at $\vxo$ with an impulse excitation at $t_0$. The Green's function is often referred as the \emph{spatio-temporal impulse response} of the domain of interest.
Similarly the temporal Fourier-transform $G(\vx|\vxo,\omega)$ is referred as the \emph{spatio-temporal transfer function} of a point source, or an \emph{acoustic monopole}, located at $\vxo$. 

The motivation behind the use of the Green's function is the following:
Let's assume an arbitrary linear differential operator $\mathcal{L}\left\{ \right\}$ acting on a distribution $p(\vx)$. The Green's function of the operator is then defined as
\begin{equation}
\mathcal{L}\left\{ g(\vx-\vxo) \right\} = -\delta( \vx-\vxo ).
\label{Eq:Theory:Basic_Green_function_eq}
\end{equation}
The Green's function may be used to solve the following equation!
\begin{equation}
\mathcal{L}\left\{ p(\vx) \right\} = -f(\vx)
\label{Eq:Theory:General_Green}
\end{equation}
Let's multiply both sides of \eqref{Eq:Theory:Basic_Green_function_eq} by the source term $f(\mathbf{x_0})$ and integrate over the domain of investigation. Using the Dirac-delta sifting property:
\begin{equation}
\int_{\vxo} \mathcal{L}\left\{ g(\vx-\vxo) \right\} f(\vxo) \td \vxo  = -\int_{\vxo } \delta( \vx-\vxo ) f(\vx) \td \vxo  = -f(\vx).
\end{equation}
Due to linearity integration and the differential operator may be interchanged:
\begin{equation}
\mathcal{L}\left\{ \int_{\vxo }  g(\vx-\vxo) f(\vxo) \td \vxo  \right\} 
= -f(\vx).
\end{equation}
Comparing with \eqref{Eq:Theory:General_Green} it is revealed, that the solution takes the form:
\begin{equation}
p(\vx) = \int_{\vxo }  g(\vx-\vxo) f(\vxo) \td \vxo.
\end{equation}
Thus once the Green's function is found for a differential equation with given boundary and initial conditions, the solution for arbitrary load terms may be found by convolving the Green's function with the load term in each spatial and temporal dimensions.

The Green's function is usually obtained by eigenfunction expansion. For the present case the Green's function may be obtained by taking the spatio-temporal Fourier-transform of \eqref{Eq:Theory:Green_function_def}. By using the Fourier-transform differentiation theorem and the Fourier-transform of the Dirac-delta function, by denoting $\mathbf{k} = [k_x,\ k_y,\ k_z]^{\mathrm{T}}$:
\begin{equation}
(-(k_x^2 + k_y^2 + k_z^2) + \left(\frac{\omega}{c} \right)^2)G(\mathbf{k},\omega) = 1.
\end{equation}
Thus the Green's function in the wavenumber space reads \cite{Devaney2012, Watanabe2015}
\begin{equation}
G(\mathbf{k},\omega) = \frac{1}{\left( \frac{\omega}{c}\right)^2 - \mathbf{k}^{\mathrm{T}} \mathbf{k}}.
\label{Eq:Theory:3D_kxkykzw_Green}
\end{equation}
Using the convolution theorem the wavenumber representation of the solution using the Green's function reads
\begin{equation}
P(\mathbf{k},\omega)  = F(\mathbf{k},\omega) G(\mathbf{k},\omega) =\frac{F(\mathbf{k},\omega)}{\left( \frac{\omega}{c}\right)^2 -  \mathbf{k}^{\mathrm{T}} \mathbf{k} },
\end{equation}
and the solution in the spatio-temporal domain is yielded by the inverse Fourier-transform:
\begin{equation}
p(\vx,t) =\frac{1}{(2\pi)^4} \iiiint^{\infty}_{-\infty} \frac{F(\mathbf{k},\omega)}{\left( \frac{\omega}{c}\right)^2 -  \mathbf{k}^{\mathrm{T}} \mathbf{k} } \te^{-\ti \left( \mathbf{k}^{\mathrm{T}}\vx - \omega t \right) } \td k_x \td k_y \td k_z \td \omega.
\end{equation}

\begin{figure}
	\centering
	\begin{overpic}[width = 1\columnwidth]{Figures/Theory/point_source.png}
	\end{overpic}
	\caption{3D point source}
	\label{Fig:Theory:point_source}
\end{figure}

The different representations of the 3D free-field Green's function may be obtained by the corresponding inverse Fourier-transform of \eqref{Eq:Theory:3D_kxkykzw_Green} (setting $\mathbf{x_0} = 0$, ie. the point source is placed at the origin) \cite{Devaney2012, Ahrens2010a, Ahrens2012}
\begin{equation}
G(k_x,y,k_z,\omega ) = -\frac{\ti}{2}\frac{\te^{-\ti \sqrt{ \left( \frac{\omega}{c} \right)^2 - k_x^2 - k_z^2  }}|y|}{\sqrt{ \left( \frac{\omega}{c} \right)^2 - k_x^2 - k_z^2  }} ,
\label{Eq:Theory:3D_kxykzw_Green}
\end{equation}
\begin{equation}
G(k_x,y,z,\omega) = -\frac{\ti}{4} H_0^{(2)}\left( \sqrt{ \left( \frac{\omega}{c} \right)^2 - k_x^2 } \sqrt{y^2 + z^2} \right)
\label{Eq:Theory:3D_kxyzw_Green}
\end{equation}
\begin{equation}
G(\vx,\omega) = \frac{1}{4\pi} \frac{\te^{ -\ti \frac{\omega}{c}|\vx| } }{ |\vx| },
\label{Eq:Theory:3D_xyzw_Green}
\end{equation}
\begin{equation}
g(\vx,t) = \frac{1}{4\pi} \frac{\delta \left( t - \frac{|\vx|}{c}  \right)}{|\vx|}.
\label{Eq:Theory:3D_xyzt_Green}
\end{equation}
The solution for the inhomogeneous wave equation is then yielded by convolving \eqref{Eq:Theory:3D_xyzt_Green} over the extension and time history of the source term.

In the field of Sound Field Synthesis the application of 2-dimensional Green's function is frequent. The 2D Green's function is the particular solution of the 2D inhomogeneous Helmhotz-equation with a spatial impulse load term, and it reads \cite{Gibson2008}
\begin{equation}
G_{2\mathrm{D}}(\vx,\omega) = \frac{\ti}{4}H_0^{(2)}\left( \frac{\omega}{c} \sqrt{x^2 + y^2} \right),
\end{equation}
which is the transfer function of a 2-dimensional point source.
In 3-dimensions the 2D Green's function can be interpreted as the field of an infinite vertical line source.

\begin{figure}
	\centering
	\begin{overpic}[width = .5\columnwidth]{Figures/Theory/dipole_source.png}
	\end{overpic}
	\caption{3D dipole}
	\label{Fig:Theory:dipole_source}
\end{figure}

\vspace{3mm}
For the solution of the wave equation in enclosures the Green's function must satisfy the imposed boundary conditions. In these cases the Green's function --except for special cases-- varies with $\vxo$, therefore it is translation variant.
For an more general treatment please refer to \cite{Spors2005}. 
When the Green's function satisfies Neumann-boundary conditions (ie. $G(\vx_{S},\omega) = 0$) it is called \emph{Neumann Green's function}, while if it satisfies Dirichlet boundary conditions it is termed as \emph{Dirichlet Green's function}.
For special geometries the Neumann and Dirichlet Green's functions may be expressed analytically, as we will see in the following section.

\vspace{3mm}
Besides simple monopoles the superposition of several point sources is of importance in this thesis. The superposition of two anti-phase monopoles placed infinitesimally close to each other forms an acoustic dipole. Mathematically its field is derivad as the directional gradient of the field of a monopole, eg. in the $y$-direction:
\begin{equation}
G_d(\vx|\vxo,\omega) =
\frac{\partial}{\partial y} G(\vx|\vxo,\omega)|= 
- \frac{y}{4\pi}\left( \frac{1}{| \vx - \vxo |} + \ti \frac{\omega}{c} \right) 
\frac{\te^{-\ti \frac{\omega}{c}  | \vx - \vxo | }}{| \vx - \vxo |^2}.
\end{equation}
The sound field of this dipole is depicted in figure \ref{Fig:Theory:dipole_source}.

\newpage
\subsection{The Kirchhoff-Helmholtz integral equation}

In the foregoing of this chapter we were dealing with the free-space solutions of the wave equation. In the following the presence of enclosures will be investigated.

Any sound field obeying the homogeneous Helmholtz-equation may be written in the form of a surface integral above an enclosing surface, termed as the \emph{Kirchhoff-Helmholtz integral equation}. This integral formulation, which solves the homogeneous wave equation with inhomogeneous boundary conditions is of central importance in the field of acoustics, eg. forms the backbone of the Boundary Element Method, and SVD-based Conformal Nearfield Acoustic Holography.

In this chapter we investigate the integral formulation of interior problems in source-free volumes.
Note, that the effect of direct sources may be straightforwardly included in the following results by the proper addition of the solution of the inhomogeneous Helmholtz equation. See \cite{Spors2005} for the examination of this case.
\begin{figure}[!h]
	\centering
	\begin{overpic}[width = .65\columnwidth]{Figures/Theory/Kirchhoff-Helmholtz.png}
	\end{overpic}
\caption{Kirchhoff-Helmholtz integral geometry}
	\label{Fig:Theory:HIE_geometry}
\end{figure}

Let $\Omega$ be a 3D volume, bounded by the surface $\partial \Omega$ with arbitrary position vectors $\vxo$ and $\vx$. Refer to figure xy for the geometry. For two continuous, differentiable scalar valued functions $\Phi(\vxo)$, $\Psi(\vxo)$ the Green's theorem reads (see \ref{App:Green_theorem} for the derivation)
\begin{equation}
\iiint_{\Omega}          \left(  \Phi(\vxo) \nabla^2 \Psi(\vxo) - \Psi(\vxo) \nabla^2 \Phi(\vxo)   \right)   \td \Omega = 
\iint_{\partial \Omega}  \left(  \Psi(\vxo) \frac{\partial \Phi(\vxo)}{\partial n}  - \Phi(\vxo) \frac{\partial \Psi(\vxo)}{\partial n}  \right)   \td \partial \Omega,
\label{Eq:Theory:Greens-theorem}
\end{equation}
with $\frac{\partial}{\partial n}$ denoting the inward normal derivative. Let these functions satisfy the homogeneous and inhomogeneous Helmholtz-equations:
\begin{equation}
(\nabla^2 + k^2)\Phi(\mathbf{x_0}) = 0,
\label{Eq:Theory:Phi}
\end{equation}
\begin{equation}
(\nabla^2 + k^2)\Psi(\vx|\vxo) = -\delta(\vx - \vxo).
\label{Eq:Theory:Psi}
\end{equation}
Although the continuity requirement is not fulfilled for $\Psi$, with proper mathematical workaround the singularity may be excluded from the volume integral\cite{Williams1999}. By setting the homogeneous solution of \eqref{Eq:Theory:Psi} to 0 and assuming acoustically transparent boundary the particular solution is given by the free-field Green's function
\begin{equation}
\Psi(\vx | \vxo) = G(\vx| \vxo,\omega) = \frac{1}{4\pi} \frac{\te^{-\ti \frac{\omega}{c} |\vx-\vxo|}}{|\vx-\vxo|},
\end{equation}
describing the field of a point source located at $\vx$ measured at $\vxo$. For acoustic problems the scalar valued function satisfying the Helmholtz equation is generally the pressure field, thus $\Phi(\vxo) = P(\vxo,\omega) $
 
Let's combine equations \eqref{Eq:Theory:Greens-theorem}-\eqref{Eq:Theory:Phi}-\eqref{Eq:Theory:Psi}!
\begin{equation}
\iiint_{\Omega} - P(\vxo) \delta(\vx - \vxo)
  \td \Omega(\vxo) = 
\iint_{\partial \Omega}  \left(  G(\vx|\vxo) \frac{\partial P(\vxo)}{\partial n}  - P(\vxo)  \frac{\partial G(\vx|\vxo)}{\partial n}  \right)   \td \partial \Omega ( \vxo),
\end{equation}

The sifting property of the Dirac-delta may be exploited, by taking into account, that the singularity is located in the enclosure: if $\vxo$ lies outside the volume the integral is identically zero, while if it is on the surface it is assumed, that "only half of the Dirac-impulse is in the volume". As a result the \emph{Kirchhoff-Helmholtz integral}, or \emph{Helmholtz integral equation (HIE)} is obtained:
\begin{equation}
\alpha P(\vx,\omega) = 
\iint_{\partial \Omega}  \left( 
P(\vxo,\omega)  \frac{\partial G(\vx|\vxo,\omega)}{\partial n}  -  
G(\vx|\vxo,\omega) \frac{\partial P(\vxo,\omega)}{\partial n} 
\right)   \td \partial \Omega ( \vxo),
\label{Eq:Theory:Kirchhoff-Helmholtz}
\end{equation}
with
\begin{equation*}
\alpha = \begin{cases} 
1           & \hspace{1mm} \forall \hspace{5mm}  \vxo \in \Omega_i  	   \\
\frac{1}{2} & \hspace{1mm} \forall \hspace{5mm}  \vxo \in \partial \Omega  \\
0 			& \hspace{1mm} \forall \hspace{5mm}  \vxo \in \Omega_e.
\end{cases}
\end{equation*}
Point $\vx$ is termed as \emph{evaluation point}, while $\vxo$ is termed the \emph{field point}. Utilizing the Euler's equation \eqref{Eq:Theory:Eulers_equation}
\begin{equation}
\frac{\partial P(\vxo,\omega)}{\partial n} = \ti \rho_0 c k V_{\mathrm{n}}(\vxo,\omega),
\end{equation}
ie. the normal component of the velocity on the surface:
\begin{equation}
\alpha P(\vx,\omega) = 
\iint_{\partial \Omega}  \left(  
P(\vxo,\omega)  \frac{\partial G(\vx|\vxo,\omega)}{\partial n}  -
\ti \rho_0 c k V_{\mathrm{n}}(\vxo,\omega) G(\vx|\vxo,\omega) 
\right)   \td \partial \Omega ( \vxo),
\label{Eq:Theory:Kirchhoff}
\end{equation}

The equation states that the pressure field inside an enclosure is completely determined by the boundary conditions for the pressure and normal velocity on the boundary surface.
The interior HIE describes the pressure field only inside the volume of investigation, outside the volume the pressure field is identically zero. For exterior radiation problems the exterior HIE can be derived in a very similar manner --By calculating with the outward normal velocity on the surface and positioning the Dirac-delta outside the enclosure--, and the formulation will result in correct pressure field outside the volume and zero pressure inside \cite{Williams1999}.
In both cases the method is capable of dealing only with forward propagation problems.
\vspace{3mm}

It should be noted, that HIE is consist of two integral components:
\begin{equation}
S_{\mathrm{monopole}}(\vx,\omega) = 
\iint_{\partial \Omega}  \frac{\partial P(\vxo)}{\partial n} G(\vxo|\vx) \td \partial \Omega (\vxo),
\label{Eq:Theory:Single_Layer_Potential}
\end{equation}
\begin{equation}
S_{\mathrm{dipole}}(\vx,\omega) = 
\iint_{\partial \Omega}  P(\vxo,\omega)  \frac{\partial G(\vx|\vxo,\omega)}{\partial n}   \td \partial \Omega (\vxo).
\label{Eq:Theory:Double_Layer_Potential}
\end{equation}
The components are termed \emph{single layer potential} and \emph{double layer potential} respectively in the field of potential theory. The terminology represents the fact, that single layer potential describes the field as the weighted sum of a single layer of monopoles, characterized by $ G(\vxo|\vx) $. On the other hand double layer potential describes the field of an ensemble of dipoles whose field is described by 
$\frac{\partial G(\vx|\vxo,\omega)}{\partial n}$ and which can be physically realized by two anti-phase monopoles, thus by a double layer.

\vspace{3mm}
One drawback of interior HIE is that it overspecifies the problem in order to ensure zero pressure and velocity outside the domain of interest. In the aspect of Sound Field Synthesis the presence of both single and double layer potentials is infeasible. By letting the sound field non-zero outside the enclosure it is possible to completely describe the sound field in the region of interest in terms of only single or double layer potentials.
This can be done by modifying the Green's function in order to satisfy Dirichlet or Neumann boundary conditions, or to impose these boundary conditions to the sound field $ P(\vxo,\omega) $ itself in an equivalent scattering problem (leading to the simple source formulation).
In the following section the former approach is applied for planar boundaries resulting in the Rayleigh integral theorem.

\subsection{The Rayleigh-integrals}
\label{Section:Theory:Rayleigh}

The Rayleigh integrals formulate the sound field with merely the pressure field or the normal velocity measured on an infinite plane. The derivation utilizes the Neumann and Dirichlet Green's functions for the geometry, that can be seen in figure \ref{Fig:Theory:Rayleigh_geometry}.

In this scenario we investigate an interior problem with the volume bounded by a plane and a hemisphere: the HIE is written into this two surfaces. As we increase the radius of the hemisphere to infinity ($r_s \rightarrow \infty$) the Sommerfeld-radiation condition is invoked and the integral on the sphere vanishes: the radiated field is described by a surface integral written on the infinite plane. Also by noticing that $\frac{\partial}{\partial n} = \frac{\partial}{\partial y_0}$:
\begin{multline}
P(\vx,\omega) = \lim_{r\rightarrow \infty} \left( \int_{\partial \Omega_P} + \int_{\partial \Omega_S} \td \partial \Omega \right) = \\
\iint_{\partial \Omega_S}  \left( 
P(\vxo,\omega)  \frac{\partial G(\vx|\vxo,\omega)}{\partial y_0}  -
\ti \rho_0 c k V_{\mathrm{n}}(\vxo,\omega)  G(\vx|\vxo,\omega) 
\right)   \td \partial \Omega_S ( \vxo).
\end{multline}

It can be easily proven, that any homogeneous solution of the Helmholtz equation -- satisfying free field boundary conditions -- may be added to the Green's function, the Kirchhoff-Helmholtz integral still holds. In order to eliminate either the single or the double layer potential in the HIE we set the homogeneous solution in the Green's function to non-zero.\begin{itemize}
\item \emph{Neumann Green's function} eliminate the double layer potential, by describing Neumann boundary conditions for the Green's function on the bounding infinite plane:
\begin{equation}
G_N = G + g_N,
\end{equation}
\begin{equation}
\frac{\partial G_N}{\partial n}|_{\partial \Omega_P} = 0.
\label{Eq:Theory:Neumann_Greenfun_def}
\end{equation}
\item \emph{Dirichlet Green's function} eliminate the double layer potential in the same manner by prescribing
\begin{equation}
G_D(\vxo) = G + g_D = 0, \hspace{3mm} \forall \hspace{3mm} \vxo \in \partial \Omega_S.
\end{equation}
\end{itemize}

\begin{figure}
	\centering
	\begin{overpic}[width = .5\columnwidth]{Figures/Theory/Rayleigh_integral.png}
	\end{overpic}
	\caption{Rayleigh geometry}
	\label{Fig:Theory:Rayleigh_geometry}
\end{figure}
\vspace{3mm}
First we are looking for the Neumann's Green function for the planar geometry under discussion:
For the free field Green's function the partial derivative on the plane $y_0 = 0$ is given by
\begin{equation}
\frac{\partial}{\partial y} G(\vx|\vxo,\omega)|_{y_0 = 0} = 
- \frac{y}{4\pi}\left( \frac{1}{| \vx - \vxo |} + \ti \frac{\omega}{c} \right) 
\frac{\te^{-\ti \frac{\omega}{c}  | \vx - \vxo | }}{| \vx - \vxo |^2}
\label{Eq:Theory:monopole_y_derivative}
\end{equation}

The construction of Neumann's Green function is now straightforward. In order ensure that \eqref{Eq:Theory:Neumann_Greenfun_def} is fulfilled the solution must have the form
\begin{equation}
\frac{\partial}{\partial y} g_N(\vx|\vxo,\omega)|_{y_0 = 0} = \frac{\partial}{\partial y} G(\vx|\vxo,\omega)|_{y_0 = 0},
\end{equation}
which is the normal derivative of a point source, positioned at $-y$, thus the image of $G(\vx|\vxo,\omega)$ mirrored on the infinite plane. Since this mirror singularity lies outside the domain of investigation, therefore in the $y > 0$ volume it satisfies the homogeneous Helmholtz equation. The Neumann Green's function on the plane then takes the form
\begin{equation}
G_N(\vx|\vxo, \omega)|_{y = 0} = 
2G(\vx|\vxo, \omega)|_{y = 0} = 
\frac{1}{2\pi} \frac{\te^{-\ti \frac{\omega}{c} |\vx-\vxo|}}{ |\vx-\vxo| }
\end{equation}
and the HIE is simplfied into the \emph{Rayleigh's first integral} \cite{Berkhout1984}
\begin{multline}
P(\vx,\omega) =
- 2 \iint_{\partial \Omega_S} \frac{\partial P(\vxo)}{\partial n} G(\vxo|\vx) \td \vxo 
=
 - \frac{\ti \rho_0 c k}{2\pi} \iint_{\partial \Omega_S} V_{\mathrm{n}}(\vxo,\omega)   \frac{\te^{-\ti \frac{\omega}{c} |\vx-\vxo|}}{ |\vx-\vxo| } \td \vxo.
\label{Eq:Theory:RayleighI}
\end{multline}

\vspace{3mm}
The construction of Dirichlet Green's function is simple: by driving the mirror source anti-phase $G_D = G + g_D = 0$ is obtained on the surface, while for the derivative
\begin{equation}
\frac{\partial}{\partial y} G_N(\vx|\vxo,\omega)|_{y_0 = 0} = 2 \frac{\partial}{\partial y} G(\vx|\vxo,\omega)|_{y_0 = 0}
\end{equation}
holds. Substituting that into the HIE the double-layer potential is eliminated and \emph{Rayleigh's second integral} is yielded:
\begin{equation}
P(\vx,\omega) = 
2 \iint_{\partial \Omega_S}  P(\vxo,\omega)  \frac{\partial G(\vx|\vxo,\omega)}{\partial y_0}     \td \vxo.
\label{Eq:Theory:RayleighII}
\end{equation}

The Rayleigh I integral is of major importance in the aspect of Sound Field Synthesis, and in the theory of diffraction from finite aperture. It is also extensively used in the calculation of radiated fields from finite radiators, mounted in infinite walls, eg. field of loudspeakers. It states that the radiated field from a rigid vibrating plane can be calculated by summing the field of monopoles, driven by the normal velocity distribution, or mathematically speaking: by convolving the Green's function with the velocity distribution over the infinite surface.

\vspace{3mm}
Finally inverse Fourier-transform with respect to time leads us to the time domain version of the Rayleigh I integral\cite{Pierce1991}:
\begin{equation}
p(\vx,t) = \frac{\rho_0}{2\pi} \iint_{\partial \Omega_S} \frac{\partial}{\partial t} \frac{v_{\mathrm{n}}(\vxo,t-\frac{ | \vx-\vxo | }{c})}{| \vx-\vxo |} \td \vx_0,
\end{equation}
Note, that since the Rayleigh integral describes the field of an ensemble of point source on the plane $y = 0$, therefore the pressure field is the solution for the following equation \cite{Pierce1991}
\begin{equation}
\nabla^2 p(\mathbf{x},t) - \frac{1}{c^2}\frac{\partial^2 p(\mathbf{x},t)}{\partial t^2} = -2\rho_0 \frac{\partial}{\partial t} v_{\mathrm{n}}(x,z,t)\delta(y).
\end{equation}

\newpage


\chapter{Theory of Sound Field Synthesis}
\label{sec:SFS_theory}
\section{The problem formulation}
\begin{figure}[b!]
	\centering
	\begin{overpic}[width = .8\columnwidth]{Figures/SFS_theory/general_sfs.png}
	\scriptsize
	\put(0,26){virtual source}
	\put(45,0.5){$\mathbf{0}$}
	\put(71,31){$\vx$}
	\put(43,15){$\vxo$}
	\begin{turn}{27}
	\put(57,-3){$|\vx - \vxo|$}
	\end{turn}
	\put(50,35){$\Omega$}
	\put(80,20.5){$\partial \Omega$}
	\end{overpic}
	\caption{Geometry for the general Sound Field Synthesis problem}
	\label{Fig:Theory:general_sfs_geometry}
\end{figure}


Now we are able the formulate the general Sound Field Synthesis problem. Consider a source-free volume $\Omega \subset \mathbb{R}^n$, bounded by a continuous set of acoustic sources forming the boundary surface $\partial \Omega$.
The enclosing source ensemble is termed as the \emph{secondary source distribution (SSD)}.
The general geometry is depicted in Figure \ref{Fig:Theory:general_sfs_geometry}.
For the sake of simplicity we assume, that the boundary is acoustically transparent and the secondary sources are acoustic point sources, i.e. described by the $n$-dimensional free field Green's function. Unless it is denoted otherwise, $G(\vx,\omega)$ refers to the 3D Green's function in the followings.
%Since dynamic loudspeakers can be modeled as 3D monopoles in the low-frequency region, this assumption is feasible for $n = 3$. The incorporation of non-ideal secondary source elements in the generalized SFS theory will be discussed in the next chapter.

With these assumptions the pressure at any $\vx \in \Omega$ is given by the sum of the individual SSD elements, written as a single layer potential \cite{Ahrens2012,Ahrens2010phd,Wierstorf2014,Schultz2014:Comparing_approaches}:
\begin{equation}
P(\vx,\omega) = \oint_{\partial \Omega} D(\vxo,\omega) G(\vx - \vxo , \omega ) \td \partial \Omega ( \vxo ).
\label{Eq:Theory:3D_SFS}
\end{equation}
The weighting factor $D(\vxo,\omega)$ is termed as the \emph{driving function} for the given SSD. 
The Sound Field Synthesis problem can be formulated as the following:
Given a \emph{target sound field}, or the sound field of a \emph{virtual source} $P(\vx,\omega)$, our aim is to solve the integral equation for $D(\vxo,\omega)$, so that the weighted sum of the SSD's sound field---i.e. the \emph{synthesized field}---equals to the target sound field. 
The problem is therefore an inverse problem and has a unique solution for general enclosures.

Comparing with the Kirchhoff-Helmholtz integral formulation \eqref{Eq:Theory:Kirchhoff-Helmholtz} it becomes clear, that SFS with a single layer SSD is not able to ensure identically zero sound field outside the enclosure. Practically, the dipole sources that would cancel the field of the monopoles outside the volume are removed from the surface.
In the present thesis free-field conditions are assumed: the exterior sound field satisfies the Sommerfeld radiation condition, thus the effect of the listening environment in practical applications is not considered. For the inclusion of room effects to the SFS problem refer to \cite{Spors2005}.

In the followings mainly planar and linear SSD geometries are considered employing 3-dimensional secondary sources.
Since dynamic loudspeakers can be modeled as 3D monopoles in the low-frequency region, this choice of SSD elements is feasible. %\footnote{Dynamic loudspeakers actually can be modeled as point sources with respect to the velocity potential, forming a pulsating point source. Due to the virtual source and secondary source interchangeability this means, that when the target sound field is that of a point source, the virtual source model will be a point source with respect to the velocity potential in practical applications.}.

% Note: Morse-Ingard formulates static and moving sources via the velocity potential. Ahrens also gives the basic SFS equations wrt. the velocity potential. However KHIE is valid for pressure, not the velocity potential: so here we should note that we model our sources as eg. point sources of pressure which has less phsyical meaning than a velocity potential point source, since the latter would model a more phsyical pulsating singularity.

\paragraph{Planar SSD geometry:}
The geometry for the planar case may be derived in the same manner as the geometry for the Rayleigh-integral: consider an enclosure of $\Omega \subset \mathbb{R}^3$, bounded by the surface consisting of a simply connected disc and a hemisphere. Refer to \cite[p.~84,p.~275]{Ahrens2012, Williams1999} for the geometry. For the sake of convenience the disc is located in the plane $\vxo = [x_0,\ 0,\ z_0]^{\mathrm{T}}$. By increasing the radius of the hemisphere to infinity and by invoking the Sommerfeld radiation condition the reproduced field is written as an integral over an infinite plane $\vxo$:
\begin{equation}
P(\vx,\omega) = \iint_{-\infty}^{\infty} D(\vxo,\omega) G(\vx - \vxo , \omega ) \td x_0 \td z_0,
\label{Eq:Theory:3D_planar_SFS}
\end{equation}
and $\Omega$ becomes the half-space $y>0$, often termed as \emph{target half-space}.
$p(\vx,t)$ therefore satisfies the inhomogeneous wave equation with homogeneous Neumann boundary condition
\begin{equation}
\nabla^2 p(\vx,t) - \frac{1}{c^2}\frac{\partial^2}{\partial t^2} p(\vx,t) = - d(x,z,t)\delta(y).
\label{Eq:Theory:3D_planar_SFS_time}
\end{equation}
The planar SSD geometry is depicted in Figure \ref{Fig:Theory:planar_linear_geometry} (a).

\paragraph{Linear SSD geometry:}
From the practical point of view the application of a planar loudspeaker geometry is unfeasible.
Instead, in practical arrangements a linear ensemble of 3D point sources is utilized.
For a linear SSD positioned at $\vxo = [x_0,\ 0,\ 0]^{\mathrm{T}}$ the synthesized field reads
\begin{equation}
P(\vx,\omega) = \int_{-\infty}^{\infty} D(\vxo,\omega) G(\vx - \vxo , \omega ) \td x_0.
\label{Eq:Theory:Linear_SFS}
\end{equation}
%
\begin{figure} 
	\centering
	\begin{overpic}[width = .8\columnwidth]{Figures/SFS_theory/planar_linear_geometry.png}
	\put(-10,5){(a)}
	\put(45,5){(b)}
	\footnotesize
	\put(34.5,41){$x$}
	\put(40,26){$y$}
	\put(16,54){$z$}
	%
	\put(91,41){$x$}
	\put(96,26){$y$}
	\put(84,26){$\yref$}
	\put(62,26){$-\yref$}
	\put(72,54){$z$}
	\end{overpic}
	\caption{Geometry for the SFS problem applying a planar (a) and linear (b) set of secondary sources. In both cases from practical reasons 3D point sources are considered as SSD elements.}
	\label{Fig:Theory:planar_linear_geometry}
\end{figure}
%
Equation \eqref{Eq:Theory:Linear_SFS} describes a cylindrically symmetric sound field with the symmetry axis being the SSD. In practice we restrict the investigation of the synthesized field to the horizontal half-plane containing the SSD, ie. $z = 0, y>0$, termed as the \emph{synthesis-plane}.
Furthermore, even the explicit solution for the linear problem allows us to ensure theoretically perfect synthesis only along a line parallel to the SSD, termed as the \emph{reference line}. 
Refer to Figure \ref{Fig:Theory:planar_linear_geometry} for the linear SFS geometry.
Obviously, in this case the SSD is no longer an enclosing surface: the target space is of $\Omega \subset \mathbb{R}^2$, with the application of 3D point sources instead of 2D ones. Due to this dimensional mismatch---resulting in severe restrictions on the nature of the target sound field---this type of synthesis is referred to as \emph{2.5D synthesis}. 

\vspace{3mm}
There are several approaches to solve the SFS problem including physically based implicit and particularly  mathematical explicit solutions. 
Explicit solutions aim to solve the inverse problem directly, while implicit approaches transform the KHIE to the form of \eqref{Eq:Theory:3D_SFS} with taking the SSD geometry into consideration, thus the obtained single layer potential implicitly contains the driving functions.
For special geometries---planar, linear, spherical, circular or cylindrical SSDs---analytical expressions are available. In the following these approaches are outlined focusing on planar and linear SSD arrays.

\paragraph{Virtual Source Models}
Regarding the target sound field two approaches exist: \emph{data-based rendering} and \emph{model-based rendering}. The first is applied for the resynthesis of a wave field captured by a microphone array (citation). In the second case an analytical description of the target sound field is available. In the present treatise exclusively model-based synthesis is considered. The discussed models include virtual plane waves, point sources and infinite line sources.

% no 3D monopole can be synthesized with 2D arrays

\newpage
\section{Explicit solution: The Spectral Division Method}

The explicit solution for the general SFS problem utilizes compact operator theory by exploiting that integral \eqref{Eq:Theory:3D_SFS} constitutes a compact Fredholm operator with the kernel being the Green's function $G(\vx - \vxo , \omega )$ \cite{Ahrens2012,MorseFeshbach1953}.
Such an operator and the involved acoustic fields can by expanded into the series of orthogonal eigenfunctions of the wave equation on the boundary surface $\partial \Omega$, that form a complete basis of the solution.
The inverse problem can be straightforwardly solved for the driving function expansion coefficients by a comparison of the corresponding eigenvalues, as long as none of the expansion coefficients of the operator kernel is zero.
Otherwise the problem is termed \emph{ill-conditioned}.
Finally the explicit analytical solution is found for the driving function as an infinite sum of the weighted basis functions.
The method is often referred to as \emph{mode-matching} solutions, since the eigenfunctions of the given geometry are termed the \emph{modes}.

This solution utilizing the single layer potential is unique for general enclosures and also for the---strictly speaking---non-enclosing planar case as shown in \cite{Zotter2013:uniqueness} and \cite{Fazi2010} respectively. In contrary sound field control utilizing the Kirchhoff-Helmholtz formulation would be non-unique on the eigenfrequencies of the enclosure due to resonance phenomena.

The determination of the appropriate eigenfunctions for a general geometry is a tough challenge.
For spherical and circular geometries spherical and circular harmonics form the demanded basis functions. For a rigorous treatment for mode-matching SFS using spherical and circular SSDs see \cite{Ahrens2010phd,Zotter2009phd,Ahrens2012,Ahrens2009:circularSSD_mismatch,Ahrens2009:circular25D_SFR,Ahrens2008:Analytical_Circ_Spherical_SFS}.
In the present thesis only the planar and linear geometries are investigated in details.

\subsection{Planar SSD geometry}

For the planar geometry Equation \eqref{Eq:Theory:3D_planar_SFS} is termed a Fredholm-integral of the first kind. Due to the infinite integration limit such integrals are called \emph{singular integrals}, thus not forming a compact operator \cite[p.~921.]{MorseFeshbach1953}. 
In this case the infinite, non-denumerable eigenvalues of the problem form a continuous function \cite{MorseFeshbach1953,Schultz2014:Comparing_approaches}.
However, due to the reciprocity of the integration kernel the inverse problem can be solved applying the convolution theorem, utilizing that basically \eqref{Eq:Theory:3D_planar_SFS} describes a continuous convolution along the $y=0$ plane:
\begin{equation}
P(\vx,\omega) = D(x,z,\omega)\ast_{x} \ast_{z} G(x,y,z,\omega).
\end{equation}
Here $G(x,y,z,\omega)$ denotes the sound field of a secondary source element placed at the origin.

For the infinite planar geometry the orthogonal basis is given by the continuous set of exponentials, therefore the decomposition of the involved quantities is given by a double Fourier-transform \cite{Ahrens2012, Arfken2005,Schultz2014:Comparing_approaches}, with the physical interpretation of a plane wave decomposition:
Applying the convolution theorem to the angular spectrum representation the convolution may be transformed into a multiplication \cite{Girod2001}:
\begin{equation}
\tilde{P}(k_x,y,k_z, \omega) = \tilde{D}(k_x,k_z, \omega)  \tilde{G}(k_x,y,k_z, \omega).
\end{equation}
%
%For the infinite planar geometry the orthogonal basis is given by the continuous set of exponentials, therefore the expansion of the involved quantities is given by a double inverse %Fourier-transform \cite{Ahrens2012, Arfken2005,Schultz2014:Comparing_approaches}, with the physical interpretation of a plane wave decomposition:
%\begin{equation}
%G(\vx - \vxo,\omega) = \frac{1}{4\pi^2} \iint_{-\infty}^{\infty} \tilde{G}(k_x,y,k_z, \omega)  \te^{\ti (k_x x_0 + k_z z_0)} \te^{-\ti (k_x x + k_z z)} \td k_x \td k_z.
%\label{Eq:Theory:G_x_inverse_fourier}
%\end{equation}
%\begin{equation}
%P(\vx,\omega) = \frac{1}{4\pi^2} \iint_{-\infty}^{\infty} \tilde{P}(k_x,y,k_z, \omega) \te^{-\ti (k_x x + k_z z)} \td k_x \td k_z.
%\end{equation}
%In \eqref{Eq:Theory:G_x_inverse_fourier} the translation property of the Fourier-transform is applied.
%The expansion coefficients i.e. the angular spectrum of the involved sound fields may be obtained by a forward Fourier-transform.
%
%The series expansions---along with the expansion of driving function---may be substituted into Equation \eqref{Eq:Theory:3D_planar_SFS}. By changing the order of integration, utilizing the orthogonality of the exponental functions and exploiting the sifting property of the Dirac-delta one finally obtains
%\begin{equation}
%\tilde{P}(k_x,y,k_z, \omega) = \tilde{D}(k_x,k_z, \omega)  \tilde{G}(k_x,y,k_z, \omega),
%\end{equation}
%thus the convolution theorem for the Fourier-transform holds \cite{Girod2001}.
%
The expansion coefficient are therefore obtained by a comparison of spectral coefficients and the driving function takes the form:
\begin{equation}
\tilde{D}(k_x,k_z,\omega) = \frac{\tilde{P}(k_x,y,k_z, \omega)}{ \tilde{G}(k_x,y,k_z, \omega)} = 
\frac{\mathcal{F}\left\{ P(\vx,\omega) \right\} }
{  \mathcal{F}\left\{ G(\vx,\omega) \right\} },
\end{equation}
\begin{equation}
D(x_0,z_0,\omega) = \frac{1}{4\pi^2} \iint_{-\infty}^{\infty} \tilde{D}(k_x,k_z, \omega) \te^{-\ti (k_x x_0 + k_z z_0)} \td k_x \td k_z.
\label{Eq:Theory:Dkx_inverse_Fourier}
\end{equation}
Since the driving function spectrum is yielded by a division in the spectral domain the approach is termed as the \emph{Spectral Division Method} \cite{Ahrens2010a, Ahrens2012:Ambisonics_for_planar_linear, Ahrens2011:icassp, Ahrens2010:Ambisonics_w_planar_linear}.

It should be noted, that this method does not pose any constraint on the integral kernel. Theoretically an arbitrary transfer function may be assigned for the SSD elements: as long the problem is well-conditioned---i.e. the spectrum of the transfer function does not exhibit zeros---unique driving functions may be derived applying the above.

\vspace{3mm}
Generally the elements of the SSD are described by the 3D Green's function. The plane wave expansion of the 3D free field Green's function is termed as the Weyl's integral representation \cite{Williams1999, Lalor1969}:
\begin{equation}
G(\vx - \vxo,\omega ) = \frac{1}{4\pi^2} \iint_{-\infty}^{\infty} -\frac{\ti}{2}\frac{\te^{ -\ti k_y  | y - y_0 |  }}{ k_y }
\te^{\ti (k_x x_0 + k_z z_0)} \te^{-\ti (k_x x + k_z z)} \td k_x \td k_z.
\label{Eq:Theory:Weyls_integral}
\end{equation}
with $k_y$ defined as \eqref{eq:theory:k_y_definition}, thus the angular spectrum of the Green's function placed at the origin is given by
\begin{equation}
\tilde{G}(k_x,y,k_z,\omega) =-\frac{\ti}{2}\frac{\te^{ -\ti k_y  | y |  }}{ k_y },
\end{equation}
as it was already shown in table \ref{tab:theory:Greens_fun_representations}.
Applying equation \eqref{Eq:Theory:Wave_field_extrapolation} the target sound field spectrum on a fixed $(y=\mathrm{const})$ plane may be extrapolated from the field measured on $y=0$:
\begin{equation}
\tilde{P}(k_x,y,k_z,\omega) = \tilde{P}(k_x,0,k_z,\omega) \te^{- \ti k_y y}.
\label{Eq:Theory:Wave_field_extrapolation_2}
\end{equation}
By carrying out the spectral division the exponential pressure propagators cancel out, and the driving function becomes independent from the $y$-coordinate. The driving function in the wavenumber domain therefore reads
\begin{equation}
\tilde{D}(k_x,k_z,\omega) = 2\ti k_y \tilde{P}(k_x,0,k_z,\omega).
\label{Eq:Theory:Planar_explicit_driv_fun}
\end{equation}

\vspace{3mm}
In this case the spatial inverse Fourier-transform may be carried out analytically.
By taking the derivative of both sides of \eqref{Eq:Theory:Wave_field_extrapolation_2} one obtains
\begin{equation}
\frac{\partial}{\partial y}  \tilde{P}(k_x,y,k_z,\omega) = - \ti k_y  \tilde{P}(k_x,y,k_z,\omega) \te^{-\ti k_y y}.
\end{equation}
Comparison with \eqref{Eq:Theory:Planar_explicit_driv_fun} reveals, that 
\begin{equation}
\tilde{D}(k_x,k_z,\omega) = -2 \left. \frac{\partial}{\partial y} \tilde{P}(k_x,y,k_z,\omega) \right|_{y = 0}.
\label{Eq:Theory:Planar_explicit_driv_fun_spatial}
\end{equation}
Straightforwardly, the explicit expression of the driving function in the spatial domain is obtained by the corresponding inverse Fourier-transform according to \eqref{Eq:Theory:Dkx_inverse_Fourier}:
\begin{equation}
D(x_0,z_0,\omega) = -2 \left. \frac{\partial}{\partial y} P(\vx,\omega) \right|_{y = 0}.
\label{Eq:Theory:Planar_explicit_driv_fun_spatial}
\end{equation}

\subsection{Linear SSD geometry}

Similarly to the planar case the basis functions for a linear SSD are given by exponentials.
By realizing that equation \eqref{Eq:Theory:Linear_SFS} is a convolution along the $x$-axis,
the convolution is transformed into a multiplication by means of a forward Fourier-transform
\begin{equation}
\tilde{P}(k_x,y,z, \omega) = \tilde{D}(k_x,\omega)\tilde{G}(k_x,y,z, \omega).
\end{equation}
The driving function spectra is then obtained as a spectral ratio
\begin{equation}
\tilde{D}(k_x,\omega) = \frac{\tilde{P}(k_x,y,z, \omega)}{\tilde{G}(k_x,y,z, \omega)} = \frac{\mathcal{F}_x\left\{ P(\vx,\omega) \right\}}{\mathcal{F}_x\left\{ G(\vx,\omega) \right\}},
\end{equation}
and the frequency domain driving function therefore reads
\begin{equation}
D(x_0,\omega) = \frac{1}{2\pi} \int_{-\infty}^{\infty} \frac{\tilde{P}(k_x,y,z, \omega) }{\tilde{G}(k_x,y,z, \omega)} \te^{-\ti k_x x_0} \td k_x.
\label{Eq:Theory:LinearSDM1}
\end{equation}

Again, theoretically the transfer function may describe the field of an arbitrary sound source, as long as it does not exhibit zeros in order to keep the problem well-conditioned.
When applying 3D point sources as SSD elements the Fourier-transform coefficients of the Green's function is given in \ref{tab:theory:Greens_fun_representations}
\begin{equation}
\tilde{G}(k_x,y,z,\omega) = -\frac{\ti}{4} H_0^{(2)}\left( \sqrt{ \left( \frac{\omega}{c} \right)^2 - k_x^2 } \sqrt{ y^2 + z^2 } \right).
\end{equation}

\vspace{3mm}
Note, that unlike the planar case the present driving function contains both $y$ and $z$ positions, thus the driving function depends on the listener position: Equation \eqref{Eq:Theory:LinearSDM1} may be solved only for positions on the surface of a cylinder with fixed radius $d = \sqrt{y^2 + z^2}$ \cite[p.~60.]{Ahrens2010phd}.
Also since an infinite line source---i.e. the SSD---can only radiate wavefronts with cylindrical symmetry the following  dispersion relation must hold:
%
\begin{equation}
\left( \frac{\omega} {c}\right)^2 - k_x^2 = k_y^2 + k_z^2 = k_{\rho}^2,
\end{equation}
%
with $k_{\rho}$ being the radial wavenumber. This implies that for a fixed temporal frequency only component $k_x$ can be controlled individually using a linear SSD.

These restrictions will have the following consequence:
Since for a fixed $k_x$ the radial wavenumber and the propagation direction of the synthesized field is determined, phase correct synthesis may be assured only in a plane containing the SSD in which the radial wavenumber of the synthesized field and the target field coincide. Furthermore amplitude correct synthesis is assured in this plane at a distance $\dref = \sqrt{y^2 + y^2}$ for which driving functions are calculated.
%
\begin{figure} 
	\centering
	\begin{overpic}[width = .95\columnwidth]{Figures/SFS_theory/synthesis_w_linear_ssd.png}
	\footnotesize
	\put(0,2){(a)}
	\put(50,2){(b)}
	\put(17.5,30){$\dref$}
	\put(72,30){$\dref$}
	\end{overpic}
	\caption{Synthesis of a plane wave with the target sound field (a) and the synthesized field using a linear SSD (b). Since the synthesized field is cylindrically symmetric phase correct synthesis is restricted to the plane containing the SSD at $[x,\ 0,\ 0]^{\mathrm{T}}$ where the radial wavenumber of the plane wave equals to $\left( \frac{\omega}{c} \right)^2 - k_x^2$ (denoted by dotted line), while amplitude correct synthesis is restricted to a cylindrical surface with the radius of $\dref$ (denoted by dashed line)}
	\label{Fig:Theory:synthesis_w_linear_SSD}
\end{figure}

For practical applications we choose the horizontal plane $z=0$ for the plane of synthesis, and reference the driving functions to the \emph{reference line}, by setting $y = \yref$.
See Figure \ref{Fig:Theory:planar_linear_geometry} (b) for an illustration. The driving function thus reads
\begin{equation}
D(x_0,\omega) = \frac{1}{2\pi} \int_{-\infty}^{\infty} \frac{\tilde{P}(k_x,\yref,0, \omega) }{\tilde{G}(k_x,\yref,0, \omega)} \te^{-\ti k_x x_0} \td k_x.
\label{Eq:Theory:Linear_SDM}
\end{equation}
In this geometry amplitude correct synthesis is restricted to the reference line.
Furthermore, the propagation direction can be reconstructed only for those sound fields, where $k_z = 0$ in the plane $z=0$. Practically this means plane waves propagating along the horizontal plane, line sources perpendicular to the synthesis plane, or point sources located in the plane of synthesis.

Since the pressure of an arbitrary 3D sound field on the SSD does not determine completely the pressure measured on the reference line---and vice versa---therefore the explicit driving function for a linear array based on the target field measured on the SSD can not be expressed, 
as it was given by \eqref{Eq:Theory:Planar_explicit_driv_fun} for the planar case.

\vspace{3mm}
It is worth noting that the analytic Fourier-transform cofficients of the target sound field are available only for limited simple virtual source models. Even in these cases the inverse transform of the driving functions rarely can be evaluated analytically, therefore numerical transforms are needed.
For a practical and optimized implementation of the SDM for an arbitrary target sound field refer to \cite{ahrens2013a:efficientSDM}.

% To check: SDM w linear sources from the helical spectrum representation (eg. single layer potential, or scattering from a rigid line source)
%
% To check: Approximation of explicit linear SSD driving functions to by reduce it to the wavefield on the SSD (done for 2.5D synth)

% To check: why Frank writes, that no solution is known for (A12) in Schultz,Spors Analytical SFS... It is given is Fourier Acoustics (2.65)

\subsection{Application example}

As an example for the application of the SDM, the synthesis of a virtual 3D point source is presented. 

\begin{figure}
	\centering
	\begin{overpic}[width = 1\columnwidth]{Figures/SFS_theory/Planar_SDM.png}
	\footnotesize
	\put(0, 0){(a)}
	\put(45,0){(b)}
	\end{overpic}
\caption{Synthesis of a virtual point source using a planar SSD based on SDM driving functions. The SSD is located at $\vxo = [x_0,\ 0,\ z_0]^{\mathrm{T}}$, denoted by a solid black line. The virtual source is located at $\vxs = [0,\ -1,\ 0]^{\mathrm{T}}$ oscillating at $\omega_0 = 2\pi \cdot 1000 ~\mathrm{rad/sec}$. The figures depict the crossections at $z=0$ of the synthesized field $\mathcal{R}\left( P_{\mathrm{synth}}(x,y,0,\omega) \right)$ (a) and the deviation from the target sound field $20\mathrm{log}_{10}\left( P_{\mathrm{synth}}(x,y,0,\omega) - P(x,y,0,\omega) \right)$ (b). Using a planar SSD in $y>0$ a perfect synthesis can be achieved.}
	\label{Fig:Theory:monopole_synthesis_by_planar_SDM}
\end{figure}

\vspace{3mm}
In case of 3D synthesis the virtual source is located at $\vxs = [x_s,\ y_s,\ z_s]^{\mathrm{T}}$, with $y_s<0$, i.e. behind the SSD plane.
Assuming 3D point source SSD elements the wavenumber domain representation of the driving function is obtained by substituting the angular spectrum of the virtual point source---applying the Fourier-shift theorem---into either \eqref{Eq:Theory:Dkx_inverse_Fourier} or directly to \eqref{Eq:Theory:Planar_explicit_driv_fun}:
\begin{equation}
\tilde{D}(k_x,k_z,\omega) =  \frac{-\frac{\ti}{2} \frac{ \te^{-\ti k_y | y - y_s|} }{ k_y} \te^{\ti (k_x x_s +k_z z_s)} }{-\frac{\ti}{2} \te^{-\ti k_y | y |} / k_y   } = \te^{-\ti k_y |y_s|}\te^{\ti (k_x x_s +k_z z_s)},
\end{equation}
and the spatial driving function reads
\begin{equation}
D(x_0,z_0,\omega) = \frac{1}{4\pi^2} \iint_{-\infty}^{\infty} \te^{-\ti k_y |y_s|}\te^{\ti (k_x x_s +k_z z_s)} \te^{-\ti (k_x x_0 + k_z z_0)} \td k_x \td k_z.
\label{Eq:Theory:Monopole_SDM_planar_driv_fun}
\end{equation}

The double inverse Fourier-transform may be carried out analytically, by taking the $y$-derivative of the Weyl's integral \eqref{Eq:Theory:Weyls_integral} (See \cite[(2.65)]{Williams1999}):
\begin{equation}
\frac{\partial}{\partial y} G(\vxo - \vxs,\omega ) = 
\frac{1}{4\pi^2} \iint_{-\infty}^{\infty} -\frac{1}{2} \te^{ -\ti k_y  | y - y_s |  }
\te^{\ti (k_x x_s + k_z z_s)} \te^{-\ti (k_x x_0 + k_z z_0)} \td k_x \td k_z,
\label{Eq:Theory:Weyls_derivative}
\end{equation}
Comparing \eqref{Eq:Theory:Monopole_SDM_planar_driv_fun} and \eqref{Eq:Theory:Weyls_derivative} it is revealed, that the driving function in the spatial domain is given by
\begin{equation}
D(x_0,z_0,\omega) = -2 \frac{\partial}{\partial y} \left. G(\vxo - \vxs,\omega )\right|_{y = y_0 = 0},
\end{equation}
which is in agreement with equation \eqref{Eq:Theory:Planar_explicit_driv_fun_spatial}.

The result of synthesizing a steady-state point source is illustrated in Figure \ref{Fig:Theory:monopole_synthesis_by_planar_SDM}. In the target sound field perfect synthesis is achieved, as it is indicated in Figure \ref{Fig:Theory:monopole_synthesis_by_planar_SDM} (b) depicting the difference between the synthesized and the target sound field. Since in this case the SSD is a quasi-enclosing surface, the equivalent scattering interpretation of the synthesis---detailed in the next section---holds. The image of discrepancy therefore depicts the scattering of a point source from an infinite sound soft plane. 

\begin{figure}
	\centering
	\begin{overpic}[width = 1\columnwidth]{Figures/SFS_theory/Linear_SDM.png}
	\footnotesize
	\put(0, 0){(a)}
	\put(45,0){(b)}
	\end{overpic}
\caption{Synthesis of a virtual point source using a linear SSD applying the SDM driving functions.
The SSD is located at $\vxo = [x_0,\ 0,\ 0]^{\mathrm{T}}$, denoted by a solid black line. The virtual source is located at $\vxs = [0,\ -1,\ 0]^{\mathrm{T}}$ oscillating at $\omega_0 = 2\pi \cdot 1000 ~\mathrm{rad/sec}$. The reference line was set to $\yref = 1~\mathrm{m}$.
The figure depicts the synthesized field at the synthesis plane ($z = 0$):
$\mathcal{R}\left( P_{\mathrm{synth}}(x,y,0,\omega) \right)$ (a) and the deviation from the target sound field $20\mathrm{log}_{10}\left( P_{\mathrm{synth}}(x,y,0,\omega) - P(x,y,0,\omega) \right)$ (b).}
	\label{Fig:Theory:monopole_synthesis_by_linear_SDM}
\end{figure}

\vspace{3mm}
For the case of a linear SSD the target sound field of a 3D point source, positioned at $\vxs = [x_s,\ y_s,\ 0]^{\mathrm{T}}$, with $y_s<0$ is chosen. The explicit driving function for a linear SSD is given by \eqref{Eq:Theory:Linear_SDM}. Substituting the spectra of the virtual and the secondary point sources with applying the Fourier-shift theorem the driving function reads
\begin{equation}
\hat{D}(k_x,\omega) = 
\frac{  H_0^{(2)} \left( \sqrt{ \left(\frac{\omega}{c}\right)^2 - k_x^2} |\yref - y_s| \right)  }
     {  H_0^{(2)} \left( \sqrt{ \left(\frac{\omega}{c}\right)^2 - k_x^2} |\yref|       \right)  }
\te^{\ti k_x x_s},
\end{equation}
and in the spatial domain
\begin{equation}
D(x_0,\omega) = \frac{1}{2\pi} \int_{-\infty}^{\infty} 
\frac{  H_0^{(2)} \left( \sqrt{ \left(\frac{\omega}{c}\right)^2 - k_x^2} |\yref - y_s| \right)  }
     {  H_0^{(2)} \left( \sqrt{ \left(\frac{\omega}{c}\right)^2 - k_x^2} |\yref|       \right)  }
\te^{- \ti k_x (x_0 - x_s)}
\td k_x.
\end{equation}
The synthesized field using this driving function is depicted in \ref{Fig:Theory:monopole_synthesis_by_linear_SDM} (a). 
As it can be seen from Figure (b) displaying the deviation of the synthesized field from the target field, application of the explicit driving function ensures perfect synthesis on the reference line. In other parts of the space amplitude errors are present.
\newpage

\section{Implicit solution and Wave Field Synthesis}

The implicit solution for the general SFS problem aims at the reduction of the KHIE to a single layer potential instead of the explicit solution of the inverse problem, as treated in the previous subsection. 

As we could see the KHIE describes the sound field inside the enclosure in the form of the sum of a single and a double layer potential:
\begin{equation}
P(\vx,\omega) = 
\oint_{S} - \left( 
\frac{\partial P(\vxo,\omega)}{\partial \vni} G(\vx-\vxo,\omega)
-
P(\vxo,\omega)  \frac{\partial G(\vx-\vxo,\omega)}{\partial\vni} 
\right)   \td \partial \Omega( \vxo),
\end{equation}
with $\vx \in \Omega$,  $\Omega \subset \mathbb{R}^n$.
In order to let the double layer vanish two different approaches exist:
\begin{itemize}
\item Imposing homogeneous Dirichlet boundary conditions on the total field---which in the interior equals to the target sound field---in an equivalent scattering problem: 
\begin{equation}
P(\vxo,\omega) = 0.
\end{equation}
The resulting single layer potential is termed as the \emph{simple source formulation}.
\item Deriving the Neumann Green's function with vanishing normal derivative on the boundary:
\begin{equation}
\frac{\partial G(\vx|\vxo,\omega)}{\partial n}  = 0.
\end{equation}
\end{itemize}
In section \ref{Section:Theory:Rayleigh} the Rayleigh-integrals were introduced from the latter, Neumann Green's function approach, following \cite{Berkhout1984}. Since Rayleigh integrals represent a single layer potential applying the free field Green's function, this formulation can be used directly for SFS applying a planar SSD. 
This property however can not be generalized: in general geometries the resulting Neumann Green's function can not be expressed in terms of the free field Green's function, therefore the obtained formulation---although being a single layer potential---can not be realized in practice with real life sound sources \cite{Schultz2014:Comparing_approaches}.

For the sake of completeness it is shown here that the simple source formulation theoretically ensures realizable driving functions for an arbitrary geometry, leading to the same result for the planar case as the Neumann Green's function approach. The didactic importance of this solution along with the equivalent scattering interpretation stems from the further simplification of the resulting driving functions, based on the Kirchhoff-approximation lent from classic scattering theory. This latter forms the backbone of the extension of WFS for arbitrary SSD geometries.

%\begin{figure}
%	\centering
%	\begin{overpic}[width = 1\columnwidth]{Figures/SFS_theory/simple_source_formulation.png}
%	\put(0, 50){(a)}
%	\put(50,50){(b)}
%	\put(0,  0){(c)}
%	\put(50, 0){(d)}
%	%\put(22,70){$P_i$}
%	\put(33, 92){$P(\vx,\omega)$}
%	\put(83, 92){$P_e(\vx,\omega)$}
%	\put(33, 42){$P_{\mathrm{synth}}$}
%	\put(77, 42){$P_T = P - P_{\mathrm{synth}}$}
%	\put(27,80){$\Omega_i$}
%	\put(33,72){$\partial \Omega$}
%	%\put(60,60){$P_e$}
%	\put(85,87){$\Omega_e$}
%	\put(83,72){$\partial \Omega$}
%	\end{overpic}
%\caption{Illustration of simple source formulation in a 2D SFS problem ($\Omega \subset \mathbb{R}^2$) with a circular SSD at $R_0 = 1~\mathrm{m}$: The incident field, to be synthesized is the field of a 2D point source, described by $G_{\mathrm{2D}}(\vx-\vxo,\omega)$, $\vxo = [-1.5,\ 1.5]^{\mathrm{T}}~\mathrm{m}$, depicted in (a). For a circular SSD contour the interior and exterior fields may be calculated analytically using \cite[Eq.~4.57]{Williams1999} with $k_z=0$, and using $J_n()$ for the interior problem. The resulting exterior field is depicted in (b), while the interior solution is the target field in $\Omega_i$ in (a). Due to the separated variables normal derivatives in equation \eqref{Eq:Theory:Simple_source_HIE}---which means radial derivative in the present setup---are calculated analytically. The obtained strength function is $D(\varphi,\omega) = \sum_{n = 0}^{k} \te^{\ti n \phi} k C_n(\omega) \left( \frac{J'_n(k R_0)}{J_n(k R_0)} - \frac{H_n^{'(2)}(k R_0)}{H_n^{(2)}(k R_0)} \right),$ with $C_n(\omega)$ being the incident pressure spectrum on $\partial \Omega$. The synthesized field, depicted in (c), is calculated by evaluating \eqref{Eq:Theory:3D_SFS}. In $\Omega_i$ the synthesized field equals to the target sound field. The solution of the equivalent scattering problem is depicted in (d) constructed from the simple source approach.}
%	\label{Fig:Theory:simple_source_formulation}
%\end{figure}


\subsection{Simple Source Formulation/Equivalent Scattering Problem}
The simple source formulation is derived from the KHIE by the construction of a separate exterior and interior radiation problem with prescribing the same inhomogeneous Dirichlet boundary condition for both fields on the boundary surface $\partial \Omega$ \cite{Ahrens2012}.

Let's assume an exterior sound field $P_{\mathrm{e}}(\vx,\omega)$, satisfying the homogeneous Helmholtz equation at $\vx \in \Omega_{\mathrm{e}}$, i.e. that all sources are located within the enclosure. The exterior wave field is the combination of radiating, or diverging waves. On the other hand assume an interior sound field $P_{\mathrm{i}}(\vx,\omega)$ inside the enclosure $\vx \in \Omega$, induced by a sound source located outside the volume of investigation, thus the interior field also satisfies the homogeneous Helmholtz equation constructed by a set of incoming or converging waves.
The two spatially disjunct problems are connected through the following boundary condition written onto the boundary surface
\begin{equation}
P_{\mathrm{e}}(\vxo,\omega) = P_{\mathrm{i}}(\vxo,\omega), \hspace{15mm} \vxo \in \partial \Omega.
\end{equation}
Both fields may be expressed in terms of an exterior and an interior KHIE respectively, for the exterior KHIE with inward normals refer to \cite[eq. 8.30]{Williams1999}.
By adding the exterior and interior KHIEs, due to the coupled boundary condition terms, weighted by the pressure on the boundary vanish and the following integral expression is obtained \cite[p.~268.]{Williams1999}
\begin{equation}
\oint_{\partial \Omega} 
G(\vx|\vxo,\omega) 
\left(
\frac{\partial P_{\mathrm{e}}(\vxo,\omega)}{\partial \vni} - \frac{\partial P_{\mathrm{i}}(\vxo,\omega)}{\partial \vni} 
\right)
\td \partial \Omega ( \vxo)
= 
\begin{cases} 
P_{\mathrm{e}}(\vx,\omega)           & \hspace{1mm} \forall \hspace{5mm}  \vx \in \Omega_e  	   \\
P_{\mathrm{e}}=P_{\mathrm{i}} & \hspace{1mm} \forall \hspace{5mm}         \vx \in \partial \Omega  \\
P_{\mathrm{i}}(\vx,\omega) 			& \hspace{1mm} \forall \hspace{5mm}   \vx \in \Omega_i.
\end{cases}
\label{Eq:Theory:Simple_source_HIE}
\end{equation}
The equation states that either the interior or the exterior sound field, satisfying the homogeneous Helmholtz equation may be determined as a single layer potential, by constructing the corresponding exterior or interior problems, respectively.
The \emph{single layer strength function} is given in the integral \eqref{Eq:Theory:Simple_source_HIE} implicitly.
The discontinuity in the pressure gradient is termed as the \emph{jump relation}, expressing the fact that the sound field generated by the single layer potential continuous in pressure on the boundary $\partial \Omega$, while the gradient changes sign i.e. \emph{jumps}.

\begin{figure}[h!]
	\centering
	\begin{overpic}[width = 1\columnwidth ]{Figures/SFS_theory/simple_source_formulation_2.png}
	\footnotesize
	\put(2, 36){(a)}
	\put(52,36){(b)}
	\put(27, 0){(c)}
	\put(33, 64){$P(\vx,\omega)$}
	\put(77, 64){$P_\mathrm{e}(\vx,\omega) = -P_\mathrm{s}(\vx,\omega)$}
	\put(73, 53){$P_\mathrm{i}(\vx,\omega)$}
	\put(95,44){$\Omega_e$}
	\put(92,52){$\Omega_i$}
	\put(80,44){$\partial \Omega$}
	\put(45,29){$P_\mathrm{t}(\vx,\omega) = P(\vx,\omega) + P_\mathrm{s}(\vx,\omega)$}
	\put(70,8){$\Omega_e$}
	\put(68,16){$\Omega_i$}
	\put(55,7.5){$\partial \Omega$}
	\end{overpic}
\caption{Illustration of simple source formulation in a 2D SFS problem ($\Omega \subset \mathbb{R}^2$). Figures show the incident/target sound field (a), the field given by the simple source formulation (b) and the scattering of the incident field from a sound soft boundary (c). The incident field is the field of a 2D point source (i.e. a line source) at $\vxs = [-0.4,\ 2.5]^{\mathrm{T}}$. Equation \eqref{Eq:Theory:Simple_source_HIE} was evaluated numerically using an open source C++ Boundary Element software \cite{Fiala2014:BEM}. The figures demonstrate, how simple source formulation expresses the target field inside $\Omega_i$, and the $(-1)$ times the scattered field at $\Omega_e$ in an equivalent sound soft scattering problem. Figure (c) showing the difference between the incident field and the simple source field ((a)-(b)) therefore illustrates the total scattering in the exterior, and the error of synthesis in the interior, being 0 ideally.}
	\label{Fig:Theory:simple_source_formulation}
\end{figure}

In terms of sound field synthesis the interior sound field is the desired sound field itself. The simple source formulation therefore states that for an arbitrary geometry the SSD driving function is given by
\begin{equation}
D(\vxo,\omega) = 
\frac{\partial P_{\mathrm{e}}(\vxo,\omega)}{\partial \vni} - \frac{\partial P(\vxo,\omega)}{\partial \vni},
\label{Eq:Theory:Source_strength}
\end{equation}
where $P_{\mathrm{e}}(\vxo,\omega)$ is the corresponding exterior sound field, needed to be calculated in order to solve the SFS problem.

\vspace{3mm}
As pointed out in \cite{Fazi2013:Equivalent_scattering, Fazi2010, Schultz2014:Comparing_approaches, Zotter2013:uniqueness} the following physical interpretation can be assigned to the simple source formulation: we assume that the surface $\partial \Omega$ represents no longer an SSD, but the boundary of a sound soft scattering object. In acoustic scattering problems we consider an a-priori known \emph{incident sound field} $P(\vx,\omega)$ that is reflected by the scattering object, generating the \emph{scattered field} $P_{\mathrm{s}}(\vx,\omega)$.
The field measured in the presence of the obstacle is termed the \emph{total field} $P_{\mathrm{t}}(\vx,\omega)$, given by the sum of the incident and scattered fields.
The scattered field is the solution of the exterior radiation problem, so that the total field obeys homogeneous boundary conditions on the sound soft scatterer surface, i.e. $P_{\mathrm{s}}(\vxo,\omega) = -P(\vxo,\omega) = - P_{\mathrm{e}}(\vxo,\omega), \hspace{.2cm} \vxo \in \partial \Omega$.
In the aspect of SFS the incident field inside the theoretical scatterer is the target sound field itself.
Comparing this result with the simple source formulation it is clear, that the single layer driving function is the derivative of $(-1)$ times the total field on the SSD.
See Figure \ref{Fig:Theory:simple_source_formulation} for an illustration of the simple source formulation and for its interpretation as an equivalent scattering problem.
Another demonstration of this principle is shown in Figure \ref{Fig:Theory:monopole_synthesis_by_planar_SDM} (b) where the difference between the synthesized field and the target sound field is the total field in the equivalent scattering problem $P_{\mathrm{t}}(\vx,\omega) = P(\vx,\omega) - P_{\mathrm{e}}(\vx,\omega)$.
%\begin{equation}
%D(\vxo,\omega) = \frac{\partial P_T(\vxo,\omega)}{\partial n}
%=
%\frac{\partial P(\vxo,\omega)}{\partial n} + \frac{\partial P_s(\vxo,\omega)}{\partial n}.
%\label{Eq:Theory:Equivalent_scattering_driv_fun}
%\end{equation}

Simple source approach---and the equivalent scattering interpretation---gives the analytical driving function for an arbitrary SSD geometry implicitly. Unfortunately the exterior scattering solution is scarcely available analytically except for simple geometries. The general application therefore would require numerical computation method, e.g. BEM. Also, when compared to the explicit solution a further drawback is that simple source approach allows only point secondary sources.

\subsection{The Kirchhoff approximation}

Based on the equivalent scattering interpretation the simple source formulation may be simplified in the high-frequency region using the \emph{Kirchhoff/Physical optics approximation}, applied frequently to estimate scattering from random surfaces \cite{Voronich1999, Tsang2000}.

In order to approximate the scattered field---and its normal gradient on the scatterer surface-- two approximations are applied:
\begin{itemize}
\item Based on the \emph{geometrical optics} or \emph{ray acoustics} the scatterer surface is divided into an \emph{illuminated} and a \emph{shadow region}: only those parts of the scatterer surface contribute to the scattered field that are directly illuminated by the primary source, i.e. where the local propagation directions of the incident and the scattered field close are not reverse. This basically means the neglection of both diffracting waves around the scattering object and reflections from one part of the scatterer to an other \cite{Pignier2015}. Due to this latter restriction the Kirchhoff approximation may be applied only to convex surfaces, which geometry is free of these secondary reflections.
\item 
\end{itemize}

The mathematical basis stems from the equivalent scattering problem interpretation of the general SFS problem, discussed earlier. 
In scattering problems from sound soft bodies at high frequencies the \emph{Kirchhoff/Physical Optics approximation} is frequently used, approximating the scattered field as a single layer potential \cite{Fazi2013:Equivalent_scattering, ColtonKress1998}.
The general scattering problem is modeled with the scattering of plane waves from planar surfaces with a well-known analytical solution.
If the SSD is convex (ie. the scattering of the synthesized field is avoided) and the SSD dimensions are much higher, than the physical wavelength of the incident/synthesized field the smooth SSD boundary may be considered locally planar, ie. reflections may be modeled as that of local plane waves from locally planar surfaces.

To the synthesis however only those SSD elements contribute, which are \emph{illuminated} by the incident sound field, ie. those SSD elements, whose sound field propagate into the same direction in the synthesis plane as the target sound field. Otherwise the SSD element is in the \emph{shadow region}, that's contribution to the synthesized field is set identically to zero.
Mathematically this requirement is formulated by
\begin{equation}
w(\vxo) = \begin{cases}
                        1, \hspace{3mm} \forall \hspace{3mm} \langle \mathbf{k}(\vxo) \cdot \mathbf{n}_i(\vxo) \rangle > 0 \\
                        0  \hspace{3mm} \text{elsewhere},
                    \end{cases}
\end{equation}
where $\mathbf{k}(\vxo)$ denotes wavenumber vector of the incident sound field at $\vxo$, $ \mathbf{n}_i(\vxo)$ is the inward normal of the SSD elements.
% How is wavenumber vector defined?
In the context of WFS this windowing is termed as \emph{secondary source selection criterion} \cite{Spors2007, Spors2007:DAGA:SS_selection_criterion}, which chooses out the \emph{active secondary sources}.

\begin{figure}
	\centering
	\begin{overpic}[width = 1\columnwidth]{Figures/SFS_theory/KH_approx.png}
%	\put(0, 50){(a)}
%	\put(50,50){(b)}
%	\put(0,  0){(c)}
%	\put(50, 0){(d)}
%	%\put(22,70){$P_i$}
%	\put(33, 92){$P(\vx,\omega)$}
%	\put(83, 92){$P_e(\vx,\omega)$}
%	\put(33, 42){$P_{\mathrm{synth}}$}
%	\put(77, 42){$P_T = P - P_{\mathrm{synth}}$}
%	\put(27,80){$\Omega_i$}
%	\put(33,72){$\partial \Omega$}
%	%\put(60,60){$P_e$}
%	\put(85,87){$\Omega_e$}
%	\put(83,72){$\partial \Omega$}
	\end{overpic}
\caption{Illustration of Kirchhoff approximation in a 2D SFS problem ($\Omega \subset \mathbb{R}^2$).}
	\label{Fig:Theory:KH_approximation}
\end{figure}


With all these approximations the general WFS driving functions for an arbitrary SSD surface reads
\begin{equation}
D(\vxo,\omega) = -2 w(\vxo) \frac{\partial}{\partial \mathbf{n}_i} \left. P(\vx,\omega) \right|_{\vx = \vxo}.
\end{equation}
This result is not exclusively valid for WFS: theoretically any planar driving function may be applied for arbitrary SSD surfaces using the Kirchhoff-approximation \cite{Ahrens2012}, however only WFS provides an easily implementable approach.

A detailed discussion on application of these approximation is given in \cite[Sec. 3.8]{Ahrens2012}.

Exclusively in the context of WFS one may deduce the same results by utilizing the Fresnel-Kirchhoff diffraction formula in order to give a single-layer high frequency approximation of the KHIE \cite[p. 215.]{Pierce1991}. Using several further approximations as given in \cite{Zotter2013:uniqueness} the same driving functions are obtained as given above.

\subsection{2D/3D Wave Field Synthesis}

Comparison of the Rayleigh-integral \eqref{Eq:Theory:RayleighI} with the planar SFS equation \eqref{Eq:Theory:3D_planar_SFS} it is revealed, that Rayleigh-integral implicitly contains the driving functions for a planar SSD and the driving function is given by
\begin{equation}
D_{\mathrm{3D}}(\vxo,\omega) = -2\frac{\partial}{\partial y} P(\vxo,\omega) = - 2 \ti \rho_0 c k V_{\mathrm{n}}(\vxo,\omega).
\label{Eq:Theory:3D_WFS_driv_fun}
\end{equation}
This forms the basic equation of traditional \emph{Wave Field Synthesis}, thus equation \eqref{Eq:Theory:3D_WFS_driv_fun} is referred to as \emph{3D WFS driving function}.

Simple physical considerations lead us to the same result from the equivalent scattering problem:
An arbitrary sound field reflected from an infinite pressure release plane is simply obtained by mirroring the sound field to the infinite plane, with a phase change of $180^{\circ}$ in order to satisfy homogeneous boundary conditions. The phase change along with the mirroring cancel out in the normal derivative, thus the derivative of the total field equals two times the derivative of the incident field.

Comparison with the explicit solution \eqref{Eq:Theory:Planar_explicit_driv_fun_spatial} reveals also the equality of the implicit and the explicit solutions.
The equivalence of the explicit approach and the simple source formulation follows from the uniqueness of the solution for the inverse problem in the case of a planar geometry \cite{Fazi2010}.
It should be noted, that the traditional derivation of the Rayleigh-integral follows the Neumann Green's function approach, therefore the equivalence of all three approaches accidental, and is valid only for the planar geometry.

As a conclusion: all three approaches lead to the very same result for a planar SSD geometry: an arbitrary source free sound field may be perfectly synthesized by a set of point sources distributed along an infinite plane by driving the SSD with $(-2)$-times the normal derivative of the target sound field, measured on this plane.


\chapter{2.5D Wave Field Synthesis}
\label{sec:2_5D_WFS}
Traditional WFS theory applying a linear SSD, termed as 2.5D WFS---as first proposed by Berkhout et al.---applied the stationary phase approximation to the 3D Rayleigh integral formulation in order to derive the driving functions for a secondary dipole array, modeling electrostatic transducers \cite{Berkhout1988, Berkhout93}.
The theory focused on the reproduction of virtual 3D point sources, optimizing the synthesis on a reference line.
The concept was soon unified for secondary arrays consisting of 3D point sources \cite{Start1997:phd,Vogel1993:phd, Verheijen1997:phd, Bruijn2004, Hulsebos2004}, considering also curved SSDs \cite{start1996application}, directive SSD elements \cite{devries1996sound} or the variation of the reference curve by the manual adjustment of the driving function amplitude factor \cite{Sonke1998, Sonke2000:Phd}.

A recent 2.5D WFS generalization by Spors included arbitrary virtual source models applying arbitrary shaped SSDs, with a target field independent referencing factor aiming at the optimization of synthesis to a reference point \cite{Spors2008:WFSrevisited}.
The referencing scheme was further simplified by Ahrens \cite{Ahrens2012} using a constant reference factor.
The approaches account for the SSD dimension correction factors, but neglect the \emph{virtual source dimensionality} mismatch \cite{Voelk2012} in the case of reproducing a 3D point source, resulting in amplitude errors.

Instead of introducing 2.5D WFS theory via the existing approaches, this chapter presents a novel unified 2.5D WFS formulation that adapts the target wave field characteristics to the referencing function, and ensures optimized synthesis on an arbitrary reference curve applying arbitrary shaped SSDs.
The presented approach inherently includes former methods as special cases of the applied referencing schemes.

2.5D WFS theory relies heavily on the \emph{stationary phase approximation} (SPA) in order to approximate the 2D and 3D Neumann Rayleigh integrals.
The section starts with the introduction of the stationary phase method.
In the previous chapter, dealing with the Kirchhoff approximation the wavenumber vector was already introduced.
In the followings a formal definition is given, and by applying the wavenumber vector concept, an expressive physical interpretation is given for the SPA in the context of WFS.

%
\section{The stationary phase approximation}
\label{Section:25D_WFS:SPA}
%
\subsection{The integral approximation}
%
The SPA is a basic tool of asymptotic analysis, applied to approximate integrals around critical points in the integral path.
The method yields approximate solutions of integrals of the form
\begin{equation}
\label{Eq:SPAintegral_1d}
I_{1\mathrm{D}} = \int\limits_{-\infty}^{\infty} F(z) \, \te^{\ti \phi(z)} \, \td z
\end{equation}
in one dimension and with $\vxo \in \mathbb{R}^{n}$
\begin{equation}
\label{Eq:SPAintegral_nd}
I_{n\mathrm{D}} = \int\limits_{\dO} F(\vxo) \, \te^{\ti \phi(\vxo)} \, \td \dO(\vxo)
\end{equation}
in higher dimensions, when $\te^{\ti \phi(\vxo)}$ is highly oscillating and $F(\vxo)$ is comparably slowly varying.


For the 1D case a rigorous derivation of the SPA based on integration by parts is given in \cite{Bleistein1984, Blenstein1975, Williams1999}.
More informally the method relies on the second order truncated Taylor series of the exponent around $z^*$, where $\phi'(z^*) = 0$ and $\phi''(z^*) \neq 0$, with $\phi'(z)$ denoting the derivative with respect to $z$:
\begin{equation}
\phi(z) \approx \phi(z^*) + \frac{1}{2}\phi''(z^*)(z-z^*)^2.
\end{equation}
Point $z^*$ is termed the \emph{stationary point}.

Supposing that $F(z)$ is a slowly varying smooth function compared to $\phi(z)$, it is assumed, that where the phase varies, i.e.\ $\phi'(z) \neq 0$, the integral of rapid oscillation cancels out, thus the greatest contribution to the total integral comes from the immediate surroundings of the stationary point.
Moreover in the proximity of the stationary point $F(z)$ can be regarded as constant with the value $F(z^*)$.

With these considerations the integral becomes
\begin{align}
I_{1D} \approx F(z^*)\,\te^{+\ti\phi(z^*)} 
\int\limits_{-\infty}^{\infty} \te^{+\ti \frac{1}{2}\phi''(z^*)(z-z^*)^2} \, \td z.
\end{align}
The remaining integral can be evaluated and the SPA of \eqref{Eq:SPAintegral_1d} becomes \cite[Ch.\ 2.8]{Blenstein1975}
\begin{equation}
\label{Eq:SPAResult}
I_{1D} \approx \sqrt{\frac{2\pi}{| \phi''(z^*) |}} F(z^*) \, \te^{+\ti \phi(z^*) + \ti \frac{\pi}{4}\,\mathrm{sgn}\left(  \phi''(z^*) \right)}.
\end{equation}
\vspace{3mm}
Similarly, in higher dimensions the stationary point (or more precisely a \emph{simple stationary point}) is defined as
\begin{align}
\label{Eq:ndim_stat_point}
\begin{split}
\left.
\nabla \phi(\vxo)\right|_{\vxo = \vx^*} &= 0,
\\ \vspace{3mm} \\
\det A \neq 0,
\hspace{5mm} 
A &= \left[
\frac{\partial^2 \phi(\vx^*)}{\partial x_i \partial x_j} 
\right],
\hspace{5mm}
i,j = 1,2,...,n,
\end{split}
\end{align}
with $A$ being the Hessian matrix of the phase function.
The multidimensional formula for the integral value reads
\begin{equation}
\label{Eq:SPAResult_nd}
I_{nD} \approx \sqrt{\frac{(2\pi)^n}{|\det A|}} F(\vx^*) \te^{\ti \phi(\vx^*) + \ti \frac{\pi}{4}\,\mathrm{sgn}\left( A \right)},
\end{equation}
where $\mathrm{sgn}\left( A \right)$ is the signature of the Hessian (the number of positive eigenvalues minus the number of negative eigenvalues) \cite{Bleistein2000}.

\subsection{The local wavenumber vector}
Consider an arbitrary steady state harmonic sound field in $\vx \in \mathbb{R}^2 / \mathbb{R}^3$ written in a general polar form with $A(\vx,\omega)$, $\phi(\vx,\omega) \in \mathbb{R}$
\begin{equation}
P(\vx,\omega) = A(\vx,\omega)\te^{\ti \phi(\vx,\omega)},
\label{eq:25D_WFS:general_sf}
\end{equation}
%
with a suppressed temporal dependency $\te^{\ti \omega t}$.
The dynamics of wave propagation is described by the phase of the sound field.
Borrowed from ray-tracing/geometrical optics theory we introduce the following quantities \cite{Romer2005,Carozzi2004}:
%
\begin{equation}
\vk(\vx) = [k_x(\vx),\ k_y(\vx),\ k_z(\vx)]^{\mathrm{T}} = -\nabla \phi(\vx,\omega),
\end{equation}
%
\begin{equation}
k(\vx) =  \frac{1}{c} \frac{\partial}{\partial t} \phase{P(\vx,\omega)}  = \frac{1}{c} \left( \omega + \frac{\partial}{\partial t} \phi(\vx,\omega) \right),
\end{equation}
%
termed as the \emph{local wavenumber vector} and the \emph{instantaneous local wavenumber}, respectively, with $ \frac{\partial}{\partial t} \phase{P(\vx,\omega)} $ being the \emph{instantaneous local angular frequency}, where $\phase{f}$ denotes the phase of $f$.
The wavenumber vector, defined as the negative gradient of the phase function points in the direction of maximal phase advance, i.e.\ it is perpendicular to the wave front in an arbitrary position.
For an isotropic media, where the propagation speed is constant, the phase velocity the and group velocity coincide, and the wavenumber vector points in the direction of the wave's energy flow, i.e.\ in the local wave propagation direction \footnote{This statement holds exclusively for isotropic media.
Although the wavenumber vector is always perpendicular to the wavefront, in anisotropic media the energy of a wave not necessarily travels along the path as the wavefront normals\cite{Pollard1977}.}.
For stationary steady state sound fields the wavenumber vector does not depend on time and the instantaneous wavenumber is constant, given by 
\begin{equation}
k^2 = \left( \frac{\omega}{c} \right)^2 = |\vk(\vx)|^2 = k_x(\vx)^2 + k_y(\vx)^2 + k_z(\vx)^2,
\end{equation}
thus the simple \emph{local dispersion relation } holds.
Obviously, for 2D sound fields $k_z(\vx)=0$.
In the current chapter, dealing exclusively with stationary virtual sound fields these assumptions hold.

\begin{figure}[h!]
	\small
	\centering
	\begin{overpic}[width = .9\columnwidth]{Figures/WFS_theory/wavenumber_vector.png}
	\put(0,30){a)}
	\put(50,30){b)}
	\put(0,0){c)}
	\put(50,0){d)}
	\end{overpic}
	\caption{Illustration of the local wavenumber vector for a 2D acoustic point source (a,c) and a 2D plane wave (b,d).
(a-b) show an arbitrarily chosen contour of constant phase, along with the wavenumber vector on this contour.
(c-d) show the local normalized $\hat{k}_x(x,y_0)$ component along the line $y_0 = 0.5 ~\mathrm{m}$.
These latter components are termed as the \emph{Lagrange submanifolds} in the field of ray-tracing \cite{Tracy2014}, forming one component of the wavenumber vector distribution, termed the \emph{Lagrange manifold}.}
	\label{Fig:Theory:general_sfs_geometry}
\end{figure}

Finally, one may define the \emph{normalized wavenumber vector} as
\begin{equation}
\vhk(\vx) = \frac{\vk(\vx)}{|\vk(\vx)|} = \frac{\vk(\vx)}{k(\vx)} = \frac{\vk(\vx)}{\omega/c},
\end{equation}
being a vector of unit length, pointing in the local propagation direction of the sound field\footnote{The normalized phase change of wave fields, here termed as normalized wavenumber vector, is a massively used concept in short wavelength wave theory, dealing with asymptotic/local approximate solutions of the wave equation.
In the field of ray tracing, expression $\Gamma(\vx) = \frac{\phi(\vx,\omega)}{k}$ is termed as the \emph{eikonal}, whose gradient defines the local propagation direction of the wave field: $\nabla \Gamma(\vx) = \vhk(\vx)$.
Substituting \eqref{eq:25D_WFS:general_sf} into the Helmholtz equation in terms of the eikonal and applying high-frequency assumptions leads to the \emph{eikonal equation}, forming the basis of ray acoustic theory \cite{Kinsler2000}.
In the field of high-frequency geometrical optics the representation of wave fields in $\vx, \vk(\vx)$ is termed the phase space representation \cite{Arnold1995}.
Over the last decades also the phase space representation of acoustic fields has gained an increasing interest\cite{Steinberg1993, Teyssandier2005}.}.

For the interpretation of the local wavenumber one should express the first order Taylor-expansion of the phase function around an arbitrary point $\vxo$ in the space
\begin{equation}
\phi(\vx,\omega) \approx \phi(\vxo,\omega) + (\vx-\vxo) \nabla \phi(\vx,\omega).
\end{equation}
By substitution into \eqref{eq:25D_WFS:general_sf} in the proximity of $\vxo$ the sound field is approximated as
\begin{equation}
P(\vx,\omega) \approx A(\vx,\omega) \te^{\ti ( \phi_0(\vxo,\omega) - \vk(\vx)^{\mathrm{T}} \vx )},
\end{equation}
with $\phi_0(\vxo,\omega) = \phi(\vxo,\omega) + \vk(\vx)^{\mathrm{T}} \vxo $.
Therefore each point of an arbitrary sound field is approximated as a local elementary plane wave, with the wavenumber and angular frequency given by $\vk(\vx)$ and $\omega$ respectively.

\subsubsection{Relation of local wavenumber and angular spectrum representation}
Clearly, there is a strong relationship between the local wavenumber vector concept and the plane wave decomposition/angular spectrum of sound fields.
The relation is established by the SPA.
Consider the forward and inverse Fourier transform of a general polar form sound field $P(\vx,\omega)$ given by \eqref{eq:25D_WFS:general_sf}
\begin{equation}
\tilde{P}(k_x,y,k_z,\omega) = \iint_{-\infty}^{\infty} A(\vx,\omega)\te^{\ti \phi(\vx,\omega)} \te^{\ti k_x x} \te^{\ti k_z z} \td x \td z,
\label{eq:forward_transform}
\end{equation}
\begin{equation}
P(x,y,z,\omega) = \frac{1}{(2\pi)^2} \iint_{-\infty}^{\infty} \tilde{A}(k_x,y,k_z,\omega)\te^{\ti \tilde{\phi}(k_x,y,k_z,\omega)}  \te^{-\ti k_x x} \te^{-\ti k_z z} \td k_x \td k_z,
\label{eq:inverse_transform}
\end{equation}
with $\tilde{P}(k_x,y,k_z,\omega) =\tilde{A}(k_x,y,k_z,\omega)\te^{\ti \tilde{\phi}(k_x,y,k_z,\omega)}$.

Supposing, that the sound field fulfills the SPA requirements---i.e.\ high frequency assumptions---the forward transform \eqref{eq:forward_transform}
may be evaluated asymptotically applying the stationary phase method \cite{Arnold1995, Tinkelman2005}.
The stationary point is found for a given $k_x$, where the gradient of the exponent is zero, thus where
\begin{align}
\frac{\partial}{\partial x} \phi(\vx,\omega) + k_x &= 0 \hspace{3mm} \rightarrow \hspace{3mm} k_x(\vx) = k_x, \\
\frac{\partial}{\partial z} \phi(\vx,\omega) + k_z &= 0 \hspace{3mm} \rightarrow \hspace{3mm} k_z(\vx) = k_z
\end{align}
holds.
This finding states, that each point in the plane wave spectrum of a sound field is dominated by the parts of the space, where the local wavenumber vector coincides with the corresponding plane wave wavenumber.
The local wavenumber components therefore may be interpreted as the stationary points of \eqref{eq:forward_transform} as a function of space \footnote{For a stationary phase approximation of the forward and inverse Fourier-transforms see \cite[Eq.\ 5.20, 5.51]{Tracy2014}}.

The counterpart of this statement is that the greatest contribution to the inverse transform \eqref{eq:inverse_transform} is associated to those plane waves---the stationary phase of the inverse integral for given $\vx$---, whose wave number vector coincide with the local wavenumber components at $\vx$.

\subsubsection{High-frequency gradient approximation}
%Applying the local wavenumber concept the 2D/3D WFS theory may be further simplified.
In the previous section it was verified, that the high-frequency WFS driving function is (-2) times the normal component of the target field gradient measured on the SSD, as given by equation \eqref{eq:theory:gen_WFS}.
In case of a planar SSD this driving function ensures perfect reconstruction of the virtual field.


In the high-frequency domain the gradient of an arbitrary sound field may be expressed in a simplified form in terms of the local wavenumber vector.
By applying the product rule of differentiation, the gradient of an arbitrary sound field, described by \eqref{eq:25D_WFS:general_sf} reads
\begin{equation}
\nabla P(\vx,\omega) = \left(  \frac{\nabla A(\vx,\omega)}{A(\vx,\omega)} + \ti \nabla \phi(\vx,\omega) \right) P(\vx,\omega) =  \left(  \frac{\nabla A(\vx,\omega)}{A(\vx,\omega)} - \ti \vk(\vx) \right) P(\vx,\omega).
\end{equation}
A standard pre-assumption for the general WFS theory is that in the frequency domain of interest the target sound field's phase function varies rapidly compared to the envelope of the oscillation, which must hold both to apply the Kichhoff approximation and the stationary phase approximation in the followings.
In the high frequency region $|\vk(\vx)| = k \gg \left| \frac{ \nabla A(\vx,\omega)}{A(\vx,\omega)} \right|$ holds, thus the gradient is approximated as
\begin{equation}
\nabla P(\vx,\omega) \approx - \ti \vk(\vx) P(\vx,\omega).
\label{eq:25D_WFS:gradient_appr}
\end{equation}
Note, that it is a further local plane wave approximation of arbitrary sound fields.
The expression holds for an arbitrary plane wave with equality, and approximates a point-like source in its far-field, where the amplitude does not change rapidly.

\subsection{Physical interpretation}
When the SPA is applied for the Rayleigh integral an expressive physical interpretation can be given for the solution of the stationary position.
Since in the frequency domain of interest the Green's functions in the 2D and 3D Rayleigh integrals are rapidly oscillating functions, therefore the pre-requisitions for the application of the SPA hold.
Approximating the gradient with its high-frequency/plane wave form, given by equation \eqref{eq:25D_WFS:gradient_appr} 
the synthesized field using a planar SSD located at $\vxo = [x_0,\ 0,\ z_0]^{\mathrm{T}}$ (i.e.\ the Rayleigh integral) reads
\begin{equation}
P(\vx,\omega) = \iint_{-\infty}^{\infty} 2 \ti k_y(\vxo) P(\vxo,\omega) G(\vx-\vxo,\omega) \td z_0 \td x_0.
\end{equation}
By definition \eqref{Eq:ndim_stat_point}, the stationary point $\vxo^*$ is given by
\begin{multline}
\nabla_{x_0,z_0} \left.
\left( \phase{ \ti P(\vxo,\omega)} + \phase{ G(\vx-\vxo,\omega)} \right) \right|_{\vxo^*} =
\\
\nabla_{x_0,z_0} \left.
\left( \frac{\pi}{2}+\phase{ P(\vxo,\omega)} + \phase{ G(\vx-\vxo,\omega)} \right) \right|_{\vxo^*} = 0
\end{multline}
holds.
The gradient of the constant phase shift vanishes and the $x_0$ and $z_0$ derivatives of the target field and the Green's function phase can be recognized as the $k_x$ and $k_z$ component of the corresponding local wavenumber vectors.
For the stationary position therefore the following equation holds
\begin{align}
\begin{split}
-k_{x}^P(\vxo^*) &= k_{x}^G(\vxo^*), \\
-k_{z}^P(\vxo^*) &= k_{z}^G(\vxo^*),
\end{split}
\end{align}
with $k_{x/z}^{P}$ and $k_{x/z}^{G}$ denoting the wavenumber components of $P(\vxo)$ and $G(\vxo|\vx)$ respectively
%
\begin{figure}
	\centering
	\begin{overpic}[width = .45\columnwidth ]{Figures/WFS_theory/stationary_phase_a.png}
	\scriptsize
	\put(16,44){virtual source}
	\put(55,31){stationary position}
	\put(98,14.5){$x_0$}
	\put(98,49.2){$x_0$}
	\put(81,60){$\vx$}
	\put(45,64){$y$}
	\put(18,38){$\phase{\ti P(x_0) + G(x_0)}$}
	\put(59,14){$x_0^*$}
	\end{overpic}
	\hspace{10mm}
	\begin{overpic}[width = .4\columnwidth]{Figures/WFS_theory/stationary_phase_b.png}
	\scriptsize
	\put(98,28){$x_0$}
	\put(17, 75.5){$y$}
	\put(40, 28){$x_0^*$}
	\put(58, 63){$\vx$}
	\put(47, 37.5){$\vk^P(\vxo^*)$}
	\put(34, 18){$\vk^G(\vxo^*)$}
	\end{overpic}
	\caption{Illustration of the stationary phase approximation.
Analytically, the stationary phase position is found, where the phase of the integrand has a stationary point, i.e.\ where its gradient is zero (a).
In the aspect of the Rayleigh integral the stationary position can be interpreted as the point, where the local wavenumber vector of the Green's function $G(\vxo|\vx)$ equals (-1) times the local wavenumber vector of the virtual field measured on the SSD (b).}
	\label{Fig:Theory:stationary_position}
\end{figure}

Along with the \emph{local dispersion relation} \cite{Tracy2014} $\left( \frac{\omega}{c} \right)^2 = k_x(\vx)^2+k_y(\vx)^2+k_z(\vx)^2$ at a fixed angular frequency two components completely determine the wavenumber vector, therefore the stationary position on the SSD for a given receiver position $\vx$ satisfies
\begin{equation}
\vk^P(\vxo^*) = - \vk^G(\vxo^*).
\end{equation}
Obviously, the same derivation holds for the 2 dimensional case by applying the 1D SPA, where the stationary position is found, where the $k_x$ components of the target and secondary source field's are equal.

Hence, the SPA compares the propagation direction/wave fronts of the virtual field and the Green’s function (placed at the receiver position $\vx$) along the SSD.
The stationary position is then found, where these two directions coincide.
See Figure \ref{Fig:Theory:stationary_position} (b) for an illustration, with the example of a virtual point source.
Obviously, at the stationary position $\vxo^*$, $\vk^P(\vxo^*)$ coincides with $-\vk^G(\vxo^*)$ both in magnitude and direction.
This result is of primary importance herein.
It states, \emph{that for each receiver position $\vx$, the synthesized sound field is mostly influenced by that SSD element $\vxo^*$, from which the emerging spherical wave fronts locally coincide with the target sound field’s wave fronts, or with other words the propagation direction of a SSD element and the virtual sound field coincide.}\footnote{Since the general SPA equation \eqref{Eq:SPAintegral_nd} holds for an arbitrary surface geometry the same conclusion can be drawn from finding the stationary position in the general Kirchhoff approximation integral with the driving function given by \eqref{eq:25D_WFS:generalized_2d_3d_wfs} by comparing the local wavenumber components specified by the actual SSD geometry.}

The counterpart of this statement declares that \emph{every point $\vxo$ on the SSD highly dominates the total synthesized sound field along a straight line, pointing from $\vxo$ towards the direction of the wave number vector $\vk(\vxo)$ of the target sound field.}
%For the case of a virtual spherical/cylindrical wave, this point is found in the intersection of the vector x 􀀀 xs and the SSD, as seen in Figure 2 (a).
% This is a well-known result in WFS theory
%
\begin{figure}
	\centering
\begin{overpic}[width = 0.45\columnwidth ]{Figures/WFS_theory/Spherical_wave_Stationary_point.png}
	\scriptsize
	\put(-5,2){(a)}
	\put(99, 19){$x_0$}
	\put(16, 69){$y$}
	\put(7, 46){$y$}
	\put(66, 19){$x$}
	\put(40, 19){$x_0^*$}
	\put(16, 4.5){$\vxs$}
	\put(70, 45.5){$\vx$}
	\begin{turn}{37}
	\put(60,-3.5){$\vx - \vxo$}
	\put(25,-2){$\vxo - \vxs$}
	\end{turn}
	\end{overpic}		
	\hspace{10mm}
	%%	
	\begin{overpic}[width = 0.45\columnwidth ]{Figures/WFS_theory/plane_wave_stationary_point.png}
	\scriptsize
	\put(-5,2){(b)}
	\put(69.5,52){$\mathbf{k}_\text{pw}$}
	\put(100,19){$x_0$}
	\put(18, 69){$y$}
	\put(12, 48){$y$}
	\put(65, 19){$x$}
	\put(68, 44.5){$\vx$}
	\put(23, 19){$x_0^*$}
	\put(33, 24.5){$\varphi_\text{PW}$}
	\begin{turn}{30}
	\put(47,13){$\vx - \vxo$}
	\end{turn}
	\end{overpic}
\caption{Geometry for finding the stationary point for a) a virtual cylindrical wave and b) a virtual plane wave.}
	\label{Fig:Theory:Spherical_and_Plane_Wave_stationary_point}
\end{figure}

For the case of a virtual spherical/cylindrical wave, this point is found in the intersection of the vector $ \vx - \vxs $ and the SSD, as seen in Figure \ref{Fig:Theory:Spherical_and_Plane_Wave_stationary_point} (a).
In terms of the wavenumber vector, where
\begin{equation}
k \frac{\vxo^*-\vxs}{|\vxo^*-\vxs|} = k \frac{\vx-\vxo^*}{|\vx-\vxo^*|}
\end{equation}
holds.
This is a well-known result in traditional WFS theory \cite{Start1997:phd,Verheijen1997:phd}.
%
\subsubsection{Plane Wave Example}	
The principle is further demonstrated via the example of a plane wave synthesis, which has not been covered in the related literature so far.
Let's assume a virtual 2D plane wave field with propagating direction $\mathbf{k}_\text{pw}=[k_x,k_y,k_z]^{\mathrm{T}}=k\,[\cosfi_\text{PW},\sinfi_\text{PW},0]^{\mathrm{T}}$.
The $y$-derivative of such a sound field is given by
\begin{equation}
\left.
\frac{\partial\,\te^{-\ti k ( \cosfi_\text{PW}\cdot x_0 + \sinfi_\text{PW}\cdot y )} }{\partial y} \right|_{ y = 0 } = -\ti k \sinfi_\text{PW}\,\te^{-\ti k \cosfi_\text{PW}\cdot x_0}.
\end{equation}
Substituting this back into the Rayleigh integral at a fixed receiver position the synthesized field reads
\begin{equation}
P(x,y,z=0,\omega) = \int\limits_{-\infty}^{\infty} \sinfi_\text{PW} \sqrt{8\pi \ti k |\vx-\vxo|}\,
\te^{-\ti k \cosfi_\text{PW} \cdot x_0} \frac{1}{4\pi}\,\frac{\te^{-\ti k |\vx - \vxo|}}{|\vx - \vxo|} \, \td x_0.\nonumber
\end{equation}
The derivative of the phase function within the integral is given by
\begin{equation}
\label{Eq:Phix0_PlaneWaveExample}
\phi'(x_0) = - k \left( \cosfi_\text{PW} - \frac{x-x_0}{|\vx - \vxo|} \right).
\end{equation}
For a fixed $x$-position the stationary point on the SSD is found for $\phi'(x_0) = 0$, thus where $\cosfi_\text{PW} = \frac{x-x_0}{|\vx-\vxo|}$ holds.
By observing the geometry in Figure \ref{Fig:Theory:Spherical_and_Plane_Wave_stationary_point} (b) it is obvious, that the equation is satisfied, where $\vx-\vxo$ coincides with the propagation direction $\mathbf{k}_\text{pw}$ of the plane wave.


%
%\fscom{example directly based on the calculus above: with \eqref{Eq:k_vs_PhaseGradients_P_G} 
%\begin{align}
%k_x(x_0,0) = -\frac{ \partial }{\partial x_0}\phase{ P(x_0,0,\omega) } = \frac{ \partial }{\partial x_0} \phase{G_{\mathrm{3D}}(|\vxo-\vx|,\omega)}
%\end{align}
%and the plane wave
%\begin{align}
%&P(x,y,\omega)\big|_{x=x_0,y=0} =\\
%&\te^{-\ti\,k\,(\cosfi_\text{PW}\,x+\sinfi_\text{PW}\,y)}\big|_{x=x_0,y=0} =\\
%&\te^{-\ti\,k\,(\cosfi_\text{PW}\,x_0)}
%\end{align}
%follows: we derive its phase
%\begin{align}
%&\frac{\partial }{\partial x_0} [-k\,(\cosfi_\text{PW}\,x_0)] = -k\cosfi_\text{PW}\\
%&k_x(x_0,0) = -\frac{ \partial }{\partial x_0}\phase{ P(x_0,0,\omega) } = (-)(-k\cosfi_\text{PW})
%\end{align}
%now check phase of Green 3D
%\begin{align}
%\frac{\partial }{\partial x_0} [-k\,\sqrt{(x_0-x)^2+(y_0-y)^2+(z_0-z)^2}] = \frac{k(x-x_0)}{|\vx-\vxo|}
%\end{align}
%thus we must ensure that
%\begin{align}
%\frac{k(x-x_0)}{|\vx-\vxo|} == k\cosfi_\text{PW}\\
%\frac{(x-x_0)}{|\vx-\vxo|} == \cosfi_\text{PW}\\
%\end{align}
%which gives us that the vector directions of $\mathbf{k}_{pw}$ and $\vx-\vxo$ coincide, or with their unit vectors the dot product is one.
This is then also consistent with the result \eqref{Eq:Phix0_PlaneWaveExample}.
%}
%

\begin{figure} 
	\centering
	\begin{overpic}[width = .95\columnwidth]{Figures/WFS_theory/2D_unified_wfs.png}
	\footnotesize
	\put(0,2){(a)}
	\put(47,2){(b)}
	\end{overpic}
	\caption{Synthesis of a point (line) source at $\vxs = [0,\ -1]^{\mathrm{T}}$ oscillating at $\omega_0 = 2\pi \cdot 1000 ~\mathrm{rad/sec}$ in an entirely 2D SFS scenario, using the high-frequency WFS driving functions \eqref{eq:25D_WFS:generalized_2d_3d_wfs}.
 The SSD is located at $\vxo = [x_0,\ 0]^{\mathrm{T}}$.
The figures show the synthesized field $\mathcal{R}\left( P_{\mathrm{synth}}(x,y,\omega) \right)$ (a) and the deviation from the target sound field $20\mathrm{log}_{10}\left( P_{\mathrm{synth}}(x,y,\omega) - P(x,y,\omega) \right)$ (b).}
	\label{Fig:WFS_theory:2D_unified_WFS}
\end{figure}


\section{Unified 2D/3D WFS driving functions}
In the previous section the high-frequency gradient approximation has already been applied to simplify the Rayleigh integral.
The same approach may be used in order to obtain compact 2/3D WFS driving functions valid for an arbitrary SSD surface.
Since the application of the Kirchhoff-approximation requires high-frequency conditions, therefore the approximation of the gradient is feasible.
Substituting the high-frequency gradient expression \eqref{eq:25D_WFS:gradient_appr} into the general WFS driving function \eqref{eq:theory:gen_WFS}
\begin{equation}
D(\vxo,\omega) = 2\ti w(\vxo) \langle \vni(\vxo) \cdot \left.
\vk(\vx)\right|_{\vx = \vxo}\rangle P(\vxo,\omega)
\end{equation}
is yielded, with $\vx, \vxo \in \mathbb{R}^{2/3}$.
One should introduce the normal component of the wavenumber vector.
By comparing the definition of the windowing function \eqref{eq:theory:windowing_function} with the driving functions it becomes clear, that the secondary source selection criterion can be inherently included by defining the normal component of $\vk(\vx)$ only for positive values as\footnote{Spors defined the secondary source selection criterion based on the time average intensity vector \cite{Spors2007:DAGA:SS_selection_criterion, Spors2007}.
Having said that in isotropic media the local wavenumber vector coincides with the direction of the energy flow the previous and the proposed criteria are equivalent for stationary sound fields.
For the non-stationary case, e.g.
for moving sound sources time averaging is infeasible}
\begin{equation}
\kn(\vxo) = \begin{cases}
						\langle \vni(\vxo) \cdot \left.
\vk(\vx)\right|_{\vx = \vxo}\rangle, \hspace{3mm} \forall \hspace{3mm} \kn(\vxo) > 0 \\
						0  \hspace{3mm} \text{elsewhere},
					\end{cases}
\label{eq:25D_WFS:normal_k}
\end{equation}
and the general 2D/3D WFS driving functions can be expressed as
\begin{equation}
D(\vxo,\omega) = 2\ti\kn(\vxo)  P(\vxo,\omega).
\label{eq:25D_WFS:generalized_2d_3d_wfs}
\end{equation}
It is noted, that for the special case of a virtual 3D point source the resulting driving function is completely equivalent with that given in \cite[Eq.(20)]{Zotter2013:uniqueness}.	

The application of the driving functions is depicted in \ref{Fig:WFS_theory:2D_unified_WFS} for the case of a linear SSD in a 2D WFS scenario.
Obviously, the approximation of the gradient introduces synthesis errors in front of the SSD towards the local wavenumber vector of the target sound field, emerging from SSD parts, where $\langle \left.
\frac{\nabla A(\vx,\omega)}{A(\vx,\omega)} \right|_{\vx = \vxo}  \cdot \vni(\vxo) \rangle$ is considerable.

\section{2.5D WFS of 2D sound fields}
For didactic reasons the presented WFS theory is introduced starting out from the 2D and 3D Rayleigh integrals, i.e.\ supposing a linear/planar SSD.
The theory is later extended for arbitrary SSD shapes applying the unified driving functions \eqref{eq:25D_WFS:generalized_2d_3d_wfs}.

\subsection{The 2.5D Rayleigh integral}
Suppose a sound field, synthesized by a planar SSD located at $\vxo = [x_0,\ 0,\ z_0]^{\mathrm{T}}$, applying the planar WFS driving functions given implicitly by the \emph{3D Neumann Rayleigh integral}:
\begin{equation}
P(\vx,\omega) = -2\iint_{-\infty}^{\infty} \left.
\frac{\partial P(\vx,\omega)}{\partial y} \right|_{\vx = \vxo} G_{3\mathrm{D}}(\vx-\vxo) \td z_0 \td x_0.
\label{eq:25D_WFS:3D_neumann} 
\end{equation}
Assuming a target sound field, being independent from the $z$ direction---i.e.\ $k_z(\vx) = 0$---the integral degenerates into the \emph{2D Neumann Raylegh integral} by the direct application of \eqref{Eq:Wave_Theory:2D_Green}
\begin{equation}
P(\vx,\omega) = -2\int_{-\infty}^{\infty} \left.
\frac{\partial P(\vx,\omega)}{\partial y} \right|_{\vx = \vxo} \int_{-\infty}^{\infty}  G_{3\mathrm{D}}(\vx-\vxo) \td z_0 \td x_0.
\label{eq:25D_WFS:2D_neumann} 
\end{equation}


\vspace{3mm}
In the following we restrict our investigation to the synthesis plane $z=0$.
For the sake of brevity $\vx = [x,\ y,\ 0]^{\mathrm{T}}$ denotes in-plane positions.
Restricting the synthesis to the $z=0$ plane the integral of the 3D Green's function (i.e.\ the 2D Green's function) may be approximated applying the SPA: based on the foregoing if the receiver position is located at $z=0$ and the virtual sound field radiation is independent from the $z$-coordinate, the stationary position of integral \eqref{eq:25D_WFS:2D_neumann} is located trivially at $z_0=0$: $\vxo^* = [x_0,\ y_0,\ 0]$.
The involved quantities for the SPA read
\begin{equation}
\phi(z_0=0) = -k|\vx-\vxo^*|, \hspace{5mm} \left.
\frac{\partial^2}{\partial z_0^2} \phi(z_0)\right|_{z_0=0} = -\frac{k}{|\vx-\vxo^*|},
\end{equation}
thus the discrepancy between the 2D and 3D Green's function is described by $ \sqrt{\frac{1}{ |\phi''(z_0^*) | }}$.
%\begin{equation}
%\label{Eq:SPAintegral_1d}
%I_{1\mathrm{D}} = \int\limits_{-\infty}^{\infty} F(z) \, \te^{\ti \phi(z)} \, \td z
%\end{equation}
By accounting for the negative sign of the second derivative around the stationary point, the integral is approximated by the SPA (by substituting into \eqref{Eq:SPAResult}) as:
\begin{equation}
\int_{-\infty}^{\infty} \frac{1}{4\pi} \frac{\te^{-\ti k |\vx-\vxo| }}{|\vx-\vxo|}\td z_0 = -\frac{\ti}{4} H_0^{(2)}\left( k|\vx-\vxo^*| \right) \approx 
\sqrt{\frac{2\pi |\vx-\vxo^*|}{\ti k}} \frac{1}{4\pi}  \frac{\te^{-\ti k |\vx-\vxo^*| }}{|\vx-\vxo^*|}.
\label{Eq:25D_WFS:Hankel_approx}
\end{equation}
This approximation of the Hankel function---termed as the asymptotic expansion for large arguments \cite[(10.2.6)]{Nist2010}---is frequently applied in the field of SFS, valid withing the SPA assumptions, i.e.\ for high frequencies and in the farfield, where $k|\vx-\vxo|\gg 1$ holds.

In the followings also the SSD positions are restricted to the synthesis plane, therefore we denote the in plane SSD positions as $\vxo = [x_0,\ 0,\ 0]^{\mathrm{T}}$.
Substituting back the approximation of the 2D Green's function into the 2D Rayleigh integral expressed in terms of the 3D Green's function one obtains the \emph{2.5 Neumann Rayilegh integral}:
\begin{equation}
P(\vx,\omega) = -2\int_{-\infty}^{\infty} \left.
\frac{\partial P(\vx,\omega)}{\partial y} \right|_{\vx = \vxo}\sqrt{\frac{2\pi |\vx-\vxo|}{\ti k}} G_{3\mathrm{D}}(\vx-\vxo,\omega) \td x_0.
\label{eq:25D_WFS:25D_Rayleigh} 
\end{equation}
The 2.5D Neumann Rayleigh integral \eqref{eq:25D_WFS:25D_Rayleigh}  contains the implicit solution of the SFS problem for a virtual 2D source in terms of the unknown driving function $D(x_0, \omega)$
\begin{equation}
D(x_0,\omega) = 
- 2\,\sqrt{\frac{2\,\pi}{\ti k}} \sqrt{|\vx-\vxo|} \left.
\frac{\partial P(\vx,\omega)}{\partial y} \right|_{\vx=\vxo}.
\label{Eq:2_5D_driv_fun_implicit}
\end{equation}
The factor $\sqrt{\frac{2\pi |\vx-\vxo|}{\ti k}}$ corrects the \emph{secondary source dimensional mismatch}, which emerges from the fact, that 3D point sources are applied in a 2D SFS scenario: the correction factor approximates the frequency response and the attenuation factor of the 2D Green's function in terms of the 3D Green's function.
Obviously, attenuation correction may be optimized only in one dedicated position in space, therefore the driving function depends on the listening position $\vx$.
In the following it is verified, that by choosing a proper referencing scheme the driving function can be made independent from a single receiver position, still ensuring amplitude correct synthesis on an arbitrary convex curve in front of the SSD.

It is important realizing, that no 3D virtual sound field can be synthesized by \eqref{eq:25D_WFS:25D_Rayleigh}.
Even by applying ideal line sources, the correct synthesis of a 3D monopole is impossible: amplitude errors would be present in the synthesized field.
This phenomenon is referred to as \emph{virtual source dimensional mismatch}, and has to be accounted for, when the aiming for a 3D virtual sound field.
In the 2D case two types of virtual sound field models are considered in the present treatise: virtual plane waves, propagating in the $z=0$ plane (i.e.\ with $k_z = 0$) and 2D point(line) sources.

\subsection{Referencing schemes for 2D sound fields}

\subsubsection{Referencing Function}

Let's define a \emph{referencing function} $d(x_0)$ for each SSD element to generalize the WFS driving function \eqref{Eq:2_5D_driv_fun_implicit} towards
\begin{equation}
D(x_0,\omega) = 
- \sqrt{\frac{8\pi}{\ti k}} \sqrt{d(x_0)} \left.
\frac{\partial P(\vx,\omega)}{\partial y}\right|_{\vx = \vxo}.
\label{Eq:Gen_rayleigh_dx0}
\end{equation}

The principle of arbitrary referencing---stemming from the wavenumber vector interpretation of the SPA given in section \ref{Section:25D_WFS:SPA}---can be stated as follows:
\emph{Each receiver point $\vx$ is mainly contributed by one individual SSD element $x_0$, from which the wave number vector ${\bf k}$ of the field of the SSD element and that of the target wave field coincide.
This SSD element is termed the stationary secondary source/stationary SSD element.
Therefore, one may control the amplitude of the synthesized field along an arbitrary receiver curve, by controlling the amplitude of the corresponding stationary SSD elements.}

The referencing function for virtual 2D sound fields is, cf.\ \eqref{Eq:2_5D_driv_fun_implicit}, \eqref{Eq:Gen_rayleigh_dx0}
\begin{align}
d(x_0) = |\vx-\vxo|,
\label{dx0_2DSources}
\end{align}
i.e.\ given simply by the distance from the stationary SSD point, where the corrected 3D Green's function approximately equals the ideal 2D Green's function.

In the following it is shown how this function could be chosen in order to reference the synthesized field along an arbitrary receiver curve.
Furthermore, the analysis allows for the critical investigation of the referencing schemes applied in traditional WFS theory.

\subsubsection{Parametric Curves}

\begin{figure}
	\centering
	\begin{overpic}[width = 0.45\columnwidth ]{Figures/WFS_theory/Plane_wave_Stationary_point_2.png}
	\scriptsize
	\put(-5,2){(a)}
	\put(100,19){$x$}
	\put(18, 69){$y$}
	\put(33,19){$\vxo$}
	%\put(74,42){$d(x_0)$}
	\put(12, 50){$y_{\mathrm{ref}}$}
	\put(82, 19){$x_{\mathrm{ref}}$}
	\put(84, 52){$\mathbf{x}_{\mathrm{ref}}$}
	\put(41.5, 25){$\varphi_\text{PW}$}
	\put(39, 33){$\mathbf{k}(x_0)$}
	\put(47, 41){$k_x(x_0)$}
	\put(63, 30){$k_y(x_0)$}
	\end{overpic}
	\hspace{10mm}
	\begin{overpic}[width = 0.45\columnwidth ]{Figures/WFS_theory/Spherical_wave_Stationary_point_2.png}
	\scriptsize
	\put(-5,2){(b)}
	\put(99, 19){$x$}
	\put(16, 69){$y$}
	\put(8, 53){$y_{\mathrm{ref}}$}
	\put(82.5, 19){$x_{\mathrm{ref}}$}
	\put(38, 19){$\vxo$}
	\put(16, 4.5){$\vxs$}
	%\put(74,42){$d(x_0)$}
	\put(45.5, 25){$\varphi$}
	\put(43, 33){$\mathbf{k}(x_0)$}
	\put(50, 43){$k_x(x_0)$}
	\put(66, 30){$k_y(x_0)$}
	\put(85, 55){$\mathbf{x}_{\mathrm{ref}}$}
	\put(27, 18){$r_0$}
	\end{overpic}	
\caption{Geometry for finding the position of correct synthesis for case of a plane wave (a) and cylindrical wave (b)}
	\label{Fig:25D_WFS:Position of correct synthesis}
\end{figure}

In Section \ref{Section:25D_WFS:SPA} the link between the stationary point and the wavenumber vector of the 2D virtual source $\vk(x_0) = [k_x(x_0),\ k_y(x_0)]^{\mathrm{T}}$ given on the SSD was introduced.
The positions of amplitude correct synthesis lie in the direction of the normalized wavenumber vector $\vhk(x_0) = \vk(x_0)/k$ from each SSD element $x_0$ at a distance $d(x_0) = |\vx-\vxo|$, since this is the distance at which the compensated 3D Green's function coincides with the 2D one.
The coordinates of the \emph{points of correct synthesis (PCS)} are thus given by %$\mathbf{x}_{\mathrm{ref}}(x_0)$:
\begin{align}
\label{Eq:xRefyRef_2D_General}
\mathbf{x}_{\mathrm{ref}}(x_0)=
\colvec{2}{x_{\mathrm{ref}}(x_0)}{y_{\mathrm{ref}}(x_0)}
=
\vxo + \vhk(\vxo)|\vx-\vxo|
=
\colvec{2}
{x_0 + \frac{k_x(x_0)}{k} d(x_0)}
{\frac{k_y(x_0)}{k} d(x_0)}
%x_{\mathrm{ref}}(x_0)& = x_0 + \frac{k_x(x_0)}{k} d(x_0)\\
%y_{\mathrm{ref}}(x_0)& =\frac{k_y(x_0)}{k} d(x_0).
\end{align}
\emph{The positions of correct synthesis are therefore restricted to a parametric curve, with the free variable being the SSD position $x_0$, and the shape of the curve is given by the referencing function and the virtual source model trough $\mathbf{k}(x_0)$.} %See Figure \ref{Fig:25D_WFS:Position of correct synthesis} for an illustration.

\paragraph{For a virtual 2D plane wave} the wavenumber vector / propagation direction ($k_z=0$) is
\begin{equation}
\label{Eq:kx0_PW}
\vk_{\text{PW}}(x_0) = \colvec{2}{ k_x(x_0) }{ k_y(x_0)} =  k\colvec{2}{\cosfi_\text{PW}}{\sinfi_\text{PW}}
\end{equation}
and with \eqref{Eq:kx0_PW} into \eqref{Eq:xRefyRef_2D_General} the positions of correct synthesis are 
\begin{align}
\vx_{\mathrm{ref,PW}}(x_0)=
\colvec{2}
{x_0 + \cosfi_\text{PW} \, d(x_0)}
{\sinfi_\text{PW} \, d(x_0)}.
%x_{\mathrm{ref}}(x_0)& = x_0 + \cosfi_\text{PW} \, d(x_0) \\
%y_{\mathrm{ref}}(x_0)& = \sinfi_\text{PW} \, d(x_0).
\end{align}
For the illustration see Figure \ref{Fig:25D_WFS:Position of correct synthesis} (a).

\paragraph{For a virtual line source} consider the location at $\mathbf{x}_s = [ x_s,\ y_s ]^{\mathrm{T}}$ with $r_0 = | \vxo - \mathbf{x}_s |$.
For the sake of convenience $y_s<0$ restricts the virtual source behind the SSD.

The wavenumber vector is obtained by evaluating the derivative of the line source's phase function using the high-frequency/far-field approximation of the Hankel function \eqref{Eq:25D_WFS:Hankel_approx}
\begin{equation} 
\label{Eq:Line_source_correcty_synth}
\vk_{\text{LS}}(x_0)
=-\colvec{2}
{\left.
\frac{\partial(\cdot)}{\partial x} \right|_{x = x_0} 
}{
\left.
\frac{\partial(\cdot)}{\partial y} \right|_{y = 0} 
}
\left(-k\,\sqrt{(x-x_s)^2 + (y-y_s)^2}\right)
=k\colvec{2}
{\frac{x_0- x_s}{r_0}
}{
\frac{-y_s}{r_0}
}, 
\end{equation}
which then is inserted into \eqref{Eq:xRefyRef_2D_General} yielding
\begin{align}
\vx_{\mathrm{ref,LS}}(x_0)=
\colvec{2}
{x_0 + \frac{x_0- x_s}{r_0} d(x_0)}
{\frac{-y_s}{r_0} d(x_0)}
\end{align}
The same conclusion may be drawn from simple geometrical considerations, as shown in Figure \ref{Fig:25D_WFS:Position of correct synthesis} (b).

Now we are able to estimate the PCS of different parametric curves by varying $d(x_0)$.
The different referencing approaches are validated using the virtual source configurations shown in Figure \ref{Fig:Theory:Real_part}.
\begin{figure}[]
	\centering
	\begin{overpic}[width = .95\columnwidth ]{Figures/WFS_theory/real_part.png}
	\scriptsize
	\end{overpic}
\caption{Real part of the synthesized field for a virtual plane wave (a) and a virtual line source (b) used in the following examples: (a) a virtual plane wave with $\varphi_\text{PW} = 45^{\circ}$ and (b) a line source at $\mathbf{x}_s = [0,\ -1]^{\mathrm{T}}$ both oscillating at $\omega = 2\pi\cdot 1~\mathrm{krad/s}$.}
	\label{Fig:Theory:Real_part}
\end{figure}

\begin{figure}[h]
	\centering
	\begin{overpic}[width = .85\columnwidth ]{Figures/WFS_theory/fixed_referencing.png}
	\scriptsize
	\end{overpic}
\caption{Effects of a constant referencing function: The absolute value of the error measured between the synthesized sound field and the target sound field.
The referencing function is set to $\dref = 2~\mathrm{m}$.
For the virtual plane wave the amplitude is thus referenced on a line at $y = \sinfi_\text{PW}\,\dref = 1.41~\mathrm{m}$.
For a virtual line source the positions of amplitude correct synthesis is given by the curve \eqref{Eq:Fixed_referencing_LS}, denoted by white dashed line.
In front of the virtual source the position of amplitude correct synthesis is located at $\dref$.}
	\label{Fig:Theory:fixed_referencing}
\end{figure}
%
\subsubsection[Referencing with constant dref]{Referencing with constant $\dref$}
With $d(x_0) = \dref$ the distance from the stationary SSD element becomes fixed.
In this referencing scheme the PCS are given by
\begin{align}
\label{Eq:Fixed_referencing_PW}
\vx_{\mathrm{ref,PW}}(x_0) =  \colvec{2}
{ x_0 + \cosfi_\text{PW}\,\dref }{ \sinfi_\text{PW}\,\dref}\\
\label{Eq:Fixed_referencing_LS}
\vx_{\mathrm{ref,LS}}(x_0) =  \colvec{2}
{ x_0 + \frac{x_0-x_s}{r_0}\,\dref }{ \frac{-y_s}{r_0}\,\dref}
\end{align}
for a plane wave and a line source, respectively.
By that a plane wave will be referenced along a line that is parallel to the SSD at $y = \sinfi_\text{PW}\,\dref$, whereas the line source becomes synthesized with correct amplitude along a bell contour.
There the maximum distance $\dref$ corresponds directly to the frontal position of the virtual line source.
For an illustration see Figure \ref{Fig:Theory:fixed_referencing}.

\begin{figure}
	\centering
	\begin{overpic}[width = .85\columnwidth ]{Figures/WFS_theory/line_referencing.png}
	\scriptsize
	\end{overpic}
\caption{Effects of referencing on a line, parallel to the SSD: The absolute value of the error measured between the synthesized sound field and the target sound field for (a) a virtual plane wave and (b) line source is shown.
The referencing function is set to $d_{pw,line}(x_0) = 2 / \sinfi_\text{PW} ~\mathrm{m}$ for the plane wave, and $d_{ls,line}(x_0) = 2 r_0/y_s ~\mathrm{m}$ for the line source, resulting in an amplitude correct synthesis at $\yref = 2~\mathrm{m}$.}
	\label{Fig:Theory:line_referencing}
\end{figure}
\subsubsection{Referencing along a parallel line}
With the appropriate choice of $d(x_0)$ in \eqref{Eq:xRefyRef_2D_General} it is possible to reference amplitude correct synthesis along an arbitrary curve.
A feasible choice is to reference the synthesis to a line that is parallel to the SSD at $y=\yref>0$.
This can be done, by setting the $y$-coordinate of the parametric curve to a constant value.
For a virtual plane wave the correct choice is given by
\begin{equation}
d_\text{line,PW}(x_0) = \frac{\yref}{\sinfi_\text{PW}},
\end{equation}
and for a virtual line source by
\begin{equation}
d_\text{line,LS}(x_0) = \frac{\yref}{-y_s} r_0.
\end{equation}

By substituting back to the WFS driving function \eqref{Eq:Gen_rayleigh_dx0}, it is obtained, that the driving functions, that reference the synthesis of a plane wave and a cylindrical wave to a reference line is given by
\begin{multline}
\label{Eq:Plane_wave_ref_line}
D_\text{line,PW}(x_0,\omega)
=- \sqrt{\frac{8\pi}{\ti k}}\sqrt{\frac{\yref}{\sinfi_\text{PW}}} \left.
\frac{\partial \, \te^{-\ti k (\cosfi_\text{PW}\cdot x_0 + \sinfi_\text{PW}\cdot y)}}{\partial y} \right|_{y = 0}\\
=\sqrt{8\pi\,\ti\,k\,\sinfi_\text{PW}\,\yref\,}\,\te^{-\ti k \cosfi_\text{PW}\cdot x_0 }
=\sqrt{8\pi\,\ti\,k_y\,\yref}\,\te^{-\ti k_x x_0 }
\end{multline}
and
\begin{multline}
D_\text{line,LS}(x_0,\omega) 
=- \sqrt{\frac{8\pi}{\ti k}}\sqrt{\frac{\yref}{y_s} r_0} \left.
\frac{\partial\,G_\text{2D}(\vx -  \mathbf{x}_s,\omega)}{\partial y} \right|_{x=x_0, y = 0} \\
=- \sqrt{\frac{\ti\,k\,\pi\,\yref\,y_s}{2\,r_0}}  H_1^{(2)}( k r_0 )
\end{multline}
respectively.
For the case of a virtual plane wave this is a well-known result, equivalent with the explicit SDM solution \cite[(29)]{Ahrens2010a}, \cite{Schultz2016:DAGA, Ahrens2012}.
%This fact suggest that explicit solution for a linear SSD operates based on the same principle by matching the amplitude and phase based on the direction of the virtual source and the SSD elements analytically, in the wave number domain.

As a generalization one may express the driving function for an arbitrary 2D target sound field, referencing the synthesis on the reference line.

With the introduced high-frequency gradient approximation \eqref{eq:25D_WFS:gradient_appr} applyied for the $y$-derivative
\begin{align}
\frac{\partial P(x,y,\omega)}{\partial y}  \approx  -\ti k_y(x,y) P(x,y,\omega)
\end{align}
and the general parallel line's referencing function, cf.
\eqref{Eq:xRefyRef_2D_General}
\begin{align}
d(x_0) = \frac{k}{k_y(x_0)}\,\yref
\end{align}
the driving function \eqref{Eq:Gen_rayleigh_dx0} becomes
\begin{equation}
D(x_0,\omega) = \sqrt{8\pi\,\ti\,k_y(x_0)\,\yref } P(x_0,0,\omega).
\end{equation} 
This formulation requires only the pressure and the $y$-derivative of the phase of the virtual sound field.
%, and ensures amplitude correct synthesis on a reference line.

\subsubsection{Relation with the explicit solution}
Since plane waves give a full orthogonal basis for an arbitrary 2D sound field, therefore the driving function of an arbitrary 2D sound field may be constructed from the plane wave driving functions with the appropriate referencing function, expressed in terms of $k_x$ and $k_y$.
 As an example: the driving function referencing the synthesis on a reference line is obtained from the spatio-temporal spectrum/plane wave decomposition of the virtual sound field measured on the SSD utilizing \eqref{Eq:Plane_wave_ref_line}: %\fscom{$\hat{P}$ is not introduced}
\begin{equation}
D(x_0,\omega) = \frac{1}{2\pi} \int\limits_{-\infty}^{\infty} \hat{P}(k_x,0,\omega) \,\sqrt{8\pi\,\ti\,k_y\,\yref} 
\,\te^{-\ti k_x x_0}\,
 \td k_x,
	\label{Eq:General_WFS_in_kx_domain}
\end{equation}
where $\hat{P}(k_x,0,\omega)$ is the angular spectrum/spatial Fourier transform of the target sound field taken on the SSD.

\begin{figure} 
	\centering
	\raisebox{-0.5\height}{\begin{overpic}[width = .4\columnwidth ]{Figures/WFS_theory/circular_referencing.png}
	\scriptsize
	\put(100,22){$x$}
	\put(10, 61){$y$}
	\put(20, 30){$R_{\mathrm{ref}}$}
	\put(35, 11){$r_0$}
	\put(55, 34){$\dref$}
	\end{overpic}}
	\hspace{10mm}
	\raisebox{-0.5\height}{\begin{overpic}[width = .5\columnwidth ]{Figures/WFS_theory/circle_referencing.png}
	\scriptsize
	\end{overpic}}
\caption{Referencing a line source on a circle: Geometry for finding the corresponding referencing function (a), and the absolute value of the error measured between the synthesized sound field and the target sound field (b) for a line source at $\mathbf{x}_s = [0,\ -1]^{\mathrm{T}}$.
The synthesis is referenced on a circle around the line source, with a radius of $R_{\mathrm{ref}} = y_s + 2 ~\mathrm{m}$.
}
	\label{Fig:Theory:circle_referencing}
\end{figure}

\vspace{3mm}
As it was stated in section \ref{Sec:SFS_theory:linear_SDM} the explicit SDM driving function \eqref{Eq:Theory:LinearSDM1} can not be further simplified in case of a 3D target sound field.
However when a 2D target field is considered it can be expressed in terms of the target spectrum on the SSD by using 2D wave field extrapolation applying \eqref{Eq:Theory:Wave_field_extrapolation} with $k_z = 0$, i.e.\ %
 \begin{equation}
D_{\mathrm{SDM}}(x_0,\omega) = \frac{1}{2\pi} \int_{-\infty}^{\infty} \frac{ \hat{P}(k_x,0,\omega) \te^{-\ti k_y |\yref|}}{
-\frac{\ti}{4}H_0^{(2)}\left( k_y |\yref| \right) }
\te^{-\ti k_x x_0 }
\td k_x.
\label{Eq:25D_WFS:2D_sdm}
\end{equation}
holds, with $k_y = \sqrt{k^2-k_x^2}$.
The large-argument approximation of the Green's function spectrum is yielded by applying the stationary phase approximation to the Fourier integral 
\begin{equation}
-\frac{\ti}{4} H_0^{(2)} \left( \sqrt{k^2-k_x^2} |\vx| \right)  = \int_{-\infty}^{\infty} \frac{1}{4\pi} \frac{\te^{-\ti k|\vx|}}{|\vx|}
\te^{\ti k_x x} \td x
\approx  
\frac{\te^{-\ti k_y |\vx|}}{\sqrt{8\pi\ti k_y |\vx|}}.
\end{equation}
valid for $|\vx| \gg 1$ i.e.\ here in the far-field of the SSD.
Substituting the approximated spectrum into \eqref{Eq:25D_WFS:2D_sdm} the exponentials cancel out, and one obtains the WFS driving functions, given by \eqref{Eq:General_WFS_in_kx_domain}.
This remark indicates the fact, that WFS with a linear SSD and a linear reference curve is the general approximation of the explicit solution for the same setup.

\subsubsection{Referencing along a circle}
For a cylindrical virtual source it might be feasible in several applications to reference the synthesis on a circle around the center of the virtual source.
The example is presented in order to demonstrate the validity of the presented referencing approach.

Referencing to a circle with a radius of $R_{\mathrm{ref}} > r_0$ may be done by observing the problem geometry in Figure \ref{Fig:Theory:circle_referencing} (a).

In the stationary point $R_{\mathrm{ref}} = \dref + r_0$, thus
%
\begin{equation}
d_\text{circle,LS}(x_0) = R_{\mathrm{ref}} - r_0.
\label{Eq:dCircleLS}
\end{equation}
For the result of this type of referencing see Figure \ref{Fig:Theory:circle_referencing} (b).

\section{2.5D WFS of 3D sound fields}

All the foregoing holds for pure 2D sound fields, invariant to the vertical direction (here $z$). 
In many practical cases the virtual source model is a virtual point source with a flat frequency response and an attenuation factor of $\frac{1}{4\pi\,r}$ opposed to a line source. 
The different attenuation factor yields that---although the stationary point remains in the same position in the horizontal direction as for a line source due to the same phase functions in the plane of the synthesis---the same referencing function results in amplitude error in a  distance $\dref$ measured from a stationary SSD element. 
This phenomenon is referred to as \emph{virtual source dimension mismatch}, and has already been reported in e.g. \cite{Voelk2012}.

\subsection{2.5D Neumann Rayleigh Integral}
In order to find the PCS with the incorporation of the 3D nature of a point source the 3D Rayleigh integral \eqref{eq:25D_WFS:3D_neumann} is applied. 
Assuming a virtual 3D point source at $\vxs = [x_s,\ y_s,\ 0]^{\mathrm{T}}$  the Rayleigh integral reads
\begin{equation}
\label{Eq:Rayleigh3DPointSrc}
P(\vx,\omega) = -2 \iint\limits_{-\infty}^{\infty} \left. \frac{\partial}{\partial y} G_{3\text{D}}\left( \vxo-\vxs,\omega \right) \right|_{\vx = \vxo} G_{3\text{D}}\left( \vx-\vxo,\omega \right)\,\td x_0\,\td z_0.
\end{equation}
Again, the SPA is applied in order to approximate the integral along the $z$-dimension. Obviously, for any listener position in the plane of the synthesis the vertical stationary position is at $z_0^*=0$. This holds for an arbitrary 3D sound field, for which $k_z(x,y,0) = 0$. By applying the high-frequency gradient of the target field \eqref{eq:25D_WFS:gradient_appr} the phase function and its second derivative under consideration is now
\begin{equation}
\phi(z_0^*) = -k \left( |\vxo -\vxs| + |\vx - \vxo| \right) +\pi/2, \hspace{10mm} \phi''(z_0^*) = -k\left(   \frac{1}{| \vxo - \vxs|} + \frac{1}{ |\vx - \vxo |} \right).
\end{equation}
with $\vx = [x,\ y,\ 0]^{\mathrm{T}}$, $\vxo = [x_0,\ 0,\ 0]^{\mathrm{T}}$ being the in-plane distances. 
Note the difference of $|\phi''(z^*)| = k \frac{1}{ |\vx - \vxo| }$ for the virtual 2D line source and $|\phi''(z^*)| = k \left( \frac{1}{| \vx - \vxo|} + \frac{1}{ |\vxo - \mathbf{x}_s |} \right)$ for the virtual 3D point source. 
In the latter case, not only the secondary source, but also the virtual source attenuation needs to be corrected.

Evaluating the integral by substituting the corresponding expressions into \eqref{Eq:SPAResult}---with taking the negative sign of the second derivative into consideration---leads to the 2.5D Neumann Rayleigh integral for a 3D target field
\begin{equation}
\label{Eq:Rayleigh25DPointSrc}
P(\vx,\omega) = -2 \int\limits_{-\infty}^{\infty}\sqrt{ \frac{2\pi}{\ti k} }\sqrt{\frac{|\vxo-\vxs| |\vx-\vxo|}{|\vxo-\vxs| + |\vx-\vxo|}} \left. \frac{\partial}{\partial y}  G_{3\text{D}}\left( \vx-\vxs,\omega \right) \right|_{\vx = \vxo} G_{3\text{D}}\left( \vx-\vxo,\omega \right)\,\td x_0,
\end{equation}
or written in the more general form
\begin{equation}
\label{Eq:Rayleigh25DGeneral}
P(\vx,\omega) = -2 \int\limits_{-\infty}^{\infty}
\sqrt{\frac{2\pi}{\ti| \phi_P''(\vxo,\omega) + \phi_G''(\vxo,\omega)|}} 
\left. \frac{\partial}{\partial y}  P\left( \vx,\omega \right) \right|_{\vx = \vxo} 
G_{3\text{D}}\left( \vx-\vxo,\omega \right)\,\td x_0.
\end{equation}

\subsection{2.5D WFS driving function for a virtual 3D point source}
The 2.5D Neumann Rayleigh integral contains the implicit solution of the SFS problem for a virtual 3D point source in terms of the unknown driving function $D(x_0, \omega)$
\begin{equation}
D(x_0,\omega) = 
- 2\,\sqrt{\frac{2\pi}{\ti k}} \sqrt{\frac{| \vxo - \mathbf{x}_s |\cdot | \vx - \vxo|  }{| \vxo - \mathbf{x}_s | + | \vx - \vxo| }} \left. \frac{\partial G(\vx-\vxs,\omega)}{\partial y} \right|_{\vx = \vxo}.
\label{Eq:2_5D_point_source_implicit_df}
\end{equation}
Comparing this result with \eqref{Eq:2_5D_driv_fun_implicit} it is revealed, that the virtual source mismatch can be compensated with the correction factor $\sqrt{\frac{| \vxo - \mathbf{x}_s |  }{| \vxo - \mathbf{x}_s | + | \vx - \vxo| }}$ and therefore the 3D point source driving function consist of the following terms:
\begin{equation}
D(x_0,\omega) = 
\underbrace{\sqrt{\frac{2\pi}{\ti k}}}_{\substack{\text{SSD freq.}\\\text{compensation}}} 
\underbrace{\sqrt{ | \vx - \vxo|}}_{\substack{\text{SSD amp.}\\\text{compensation}}} 
\underbrace{\sqrt{\frac{| \vxo - \mathbf{x}_s |  }{| \vxo - \mathbf{x}_s | + | \vx - \vxo| }}  }_
{\substack{\text{Virt. source}\\\text{amp. comp.}}}\times
\underbrace{ \left. -2 \frac{\partial P(\vx,\omega)}{\partial y}  \right|_{\vx = \vxo}}_{\text{3D driving function}}.
\label{Eq:2_5D_point_source_implicit_df_Explanations}
\end{equation}
Also in the 3D case, the horizontal SPA holds, and the stationary SSD element is found in the same position as for the case of a virtual line source at $z=0$.
The correction factor gains physical meaning the horizontal stationary position ($\vk_P(\vxo) = \vk_G(\vxo)$), i.e. where $|\vxo-\vxs| + |\vx - \vxo| = |\vx-\vxs|$:
in the stationary position the numerator stands for the attenuation correction from the point source to the SSD, and the denominator for the correction from the point source to the listener position. 

Recently, a physical interpretation was given for the correction factors by Völk\cite{Voelk2012}, based on introducing manual correctional terms. 
Although giving a very specific solution for the general problem, the main idea is outlined here in order to give an insight into the 2.5D WFS driving functions. 
It is known, that 2D WFS is capable of the synthesis of 2D point sources using a linear array of vertical line sources, i.e.
\begin{equation}
G_{2D}(\vx-\vxs) = -2\int_{-\infty}^{\infty} \left. \frac{\partial}{\partial y}G_{2D}(\vx-\vxs)\right|_{\vx = \vxo} G_{2D}(\vx-\vxo)\td x_0.
\end{equation}
In order to adjust the equation for the synthesis of 3D point sources using a linear array of 3D point sources and applying the derivative of the 3D target field taken on the SSD three correction factors are needed
\begin{itemize}
\item in order to apply 3D point sources as SSD elements the secondary source correction is applied to the integral kernel, i.e. $G_{3D}(\vx-\vxo)/G_{2D}(\vx-\vxo)$.
\item to express the driving functions in term of the 3D point source driving function the adjustment by $\frac{\partial}{\partial y} G_{3D}(\vxo-\vxs)/\frac{\partial}{\partial y} G_{2D}(\vxo-\vxs)$ is needed.
\item these corrections would result in the synthesis of a 2D point source written in terms of 3D ones. 
Finally to let the 2D Rayleigh integral describe a 3D point source both sides should be multiplied by $G_{3D}(\vx-\vxs)/G_{2D}(\vx-\vxs)$.
These latter two constitute the virtual source compensation factor.
\end{itemize}
Applying the high-frequency approximations for the Green's functions \emph{around the vertical stationary point} the two approaches result in the very same driving functions, given by \eqref{Eq:2_5D_point_source_implicit_df_Explanations}.
This statement gives us an important insight into the WFS compensation factors, also reflecting, that the vertical and horizontal SPAs are inherently related, since the result of the vertical SPA can be physically interpreted only in the horizontal stationary point.


\subsection{Referencing schemes for 3D sound fields}

\subsubsection{Referencing function}
For the general WFS driving function \eqref{Eq:Gen_rayleigh_dx0}
\begin{equation*}
D(x_0,\omega) = 
- \sqrt{\frac{8\pi}{\ti k}} \sqrt{d(x_0)} \left. \frac{\partial P(\vx,\omega)}{\partial y}\right|_{\vx = \vxo},
\end{equation*}
the \emph{referencing function} $d(x_0)$ for each SSD element can be stated as
\begin{equation}
d(x_0) = \frac{ | \vxo - \mathbf{x}_s | \cdot | \vx - \vxo|  }{ | \vxo - \mathbf{x}_s | + | \vx - \vxo| }
\label{Eq:dx0_PointSource}
\end{equation}
from \eqref{Eq:2_5D_point_source_implicit_df} and \eqref{Eq:2_5D_point_source_implicit_df_Explanations}. For the sake of brevity $r_0 = | \vxo - \mathbf{x}_s |$ is used in the followings.

\subsubsection{Parametric Curves}
The synthesis may be referenced to an arbitrary curve. From \eqref{Eq:dx0_PointSource} the PCS measured from the stationary SSD point
\begin{equation}
|\vx - \vxo| = d(x_0)\frac{r_0}{r_0 - d(x_0)}
\label{Eq:dx0_PointSource_xx0}
\end{equation}
are derived. 
Comparing the referencing function \eqref{dx0_2DSources} for virtual 2D sound fields and \eqref{Eq:dx0_PointSource_xx0} for the virtual point source, the substitution
\begin{equation}
d(x_0) \rightarrow d(x_0)\frac{r_0}{r_0 - d(x_0)}
\label{Eq:dx0_mapping}
\end{equation}
allows for utilizing the parametric curve description \eqref{Eq:Line_source_correcty_synth} that was derived for the virtual line source, due to the equality of the thir wavenumber vectors in $z=0$. 
This yields the PCS
\begin{align}
\mathbf{x}_{\mathrm{ref,PS}}(x_0)=
\colvec{2}
{x_0 + (x_0-x_s) \frac{d(x_0)}{r_0 - d(x_0)} }{
-y_s \frac{d(x_0)}{r_0 - d(x_0)}}.
\label{Eq:3D_curve}
\end{align}

\begin{figure}
	\centering
	\begin{overpic}[width = 0.85\columnwidth ]{Figures/WFS_theory/point_source_referencing.png}
	\scriptsize
	\end{overpic}
\caption{Referencing the synthesis of a 3D point source with a constant referencing function (a), to a reference line (b) and to a circle around the virtual source (c). The virtual source is a point source located at $\mathbf{x}_s = [0,\ -3,\ 0]^\mathrm{T}$, oscillating at $\omega = 2\pi \cdot 1 ~\mathrm{krad/s}$. For the constant referencing $\dref = 1.5~\mathrm{m}$ is set. Note, that in front of the point source the amplitude correct synthesis is ensured at $\frac{y_s}{y_s/\dref - 1 } = 3~\mathrm{m}$. For the case of referencing to a line $\yref = 1.5~\mathrm{m}$, while for referencing to a circle $R_{\mathrm{ref}} = y_s + 1.5~\mathrm{m}$ was chosen.}
	\label{Fig:Theory:point_source_referencing}
\end{figure}

\subsubsection[Referencing with constant dref]{Referencing with constant $\dref$}
By fixing the referencing function to a constant value $d(x_0) = \dref$ in \eqref{Eq:3D_curve} the PSC are given by
\begin{equation}
\mathbf{x}_{\mathrm{ref}}(x_0)  =  
\colvec{2}
{x_0 + (x_0-x_s)\frac{1}{r_0/\dref - 1 } }
{-y_s \frac{1}{r_0/\dref - 1}}.
\label{Eq:3D_fixed_reference}
\end{equation}
This parametric curve is similar to that of the 2D line source case (cf. \eqref{Eq:Fixed_referencing_LS}, Figure \ref{Fig:Theory:fixed_referencing}), however with several important differences.

The $y$-coordinate exhibits a maximum in front of the virtual source, where $r_0 = |y_s|$. For any other position the PCS is closer to the SSD. 
Therefore, $r_0 = |y_s|$ is the critical point of the curve: a position of correct synthesis can only be found in the listening area if $|y_s| > \dref$, i.e. the virtual source must be further away from the SSD than the fixed reference distance. 
Actually, the synthesis gives a fair result close to the SSD only in case, $|y_s| \gg \dref$. 
As the virtual source approaches the SSD the position of correct synthesis tends to infinity.
This type of referencing therefore may suffer from serious amplitude errors in the vicinity of the SSD.

The result of synthesis using constant referencing is often investigated along a line, parallel with the SSD at $y=\yref$, i.e. on the reference line. 
In the previous section it was revealed, that for a virtual plane wave the constant referencing function differs from that, referencing the synthesis on the reference line in a factor of $\sqrt{\sinfi}$. 
This fact is also pointed out in \cite[(30)]{Ahrens2010a}, \cite[Ch. 3.9.4]{Ahrens2012}, \cite{Schultz2016:DAGA}. 
From this, it is clear, that using constant referencing the synthesized field on the reference line will read
\begin{equation}
P_{\mathrm{synth,PW}}(x,\yref,\omega) = \sqrt{\sinfi_\text{PW}} \, P_{\mathrm{ideal,PW}}(x,\yref,\omega) 
\end{equation}
%
which particular result has already been pointed out in \cite[3.9.4]{Ahrens2012}.
For a virtual line source located at $\mathbf{x}_s = [x_s,\ y_s,\ 0]$:
%
\begin{equation}
P_{\mathrm{synth,LS}}(x,\yref,\omega) = \sqrt{\cosfi_0(x)} \, P_{\mathrm{ideal,LS}}(x,\yref,\omega),
\end{equation}
where $\cosfi_0(x) = \langle\, \frac{\vx- \mathbf{x}_s}{|\vx- \mathbf{x}_s|}, \mathbf{n}_\text{SSD}\rangle$ with $\mathbf{n}_\text{SSD} = [0,\ 1,\ 0]^{\mathrm{T}}$ for the present setup.
For a virtual point source the error stemming from the virtual source dimensional mismatch is also present, therefore the synthesized field with constant referencing will read
%
\begin{equation}
P_{\mathrm{synth,PS}}(x,\yref,\omega) = \sqrt{\frac{ | \vxo - \mathbf{x}_s | + | \vx - \vxo| }{ | \vxo - \mathbf{x}_s |}} \times
\sqrt{\cosfi_0(x)} \, P_{\mathrm{ideal,PS}}(x,\yref,\omega).
\end{equation}
%
This type of error can be examined in \cite[Fig.5.13.]{Ahrens2012} with normalized amplitudes (i.e. the constant error factor from the virtual source dimensional mismatch is not shown in the figure.).

\subsubsection{Referencing along a parallel line}
Referencing along a line that is parallel to the SSD at distance $y=\yref>0$ is obtained by setting the $y$-coordinate of \eqref{Eq:3D_curve} to
\begin{equation}
-y_s \frac{d(x_0)}{r_0 - d(x_0)} = \yref.
\end{equation}
The resulting referencing function reads
\begin{equation}
d_\text{line,PS}(x_0) = r_0 \frac{\yref}{\yref-y_s}.
\end{equation}
Substituting it back to \eqref{Eq:2_5D_point_source_implicit_df} and applying the high-frequency gradient approximation, with $k_{y,\text{PS}}(\vxo) = k \frac{y_0-y_s}{r_0} $ one obtains
\begin{equation}
D(x_0,\omega) = 
- \sqrt{\frac{\ti k}{2\pi}} \sqrt{\frac{\yref}{\yref -y_s}}  y_s \frac{\te^{-\ti k r_0 }}{r_0^{3/2}},
\end{equation}
which exactly yields the well-known WFS driving function \cite[(2.27)]{Verheijen1997}, \cite[(3.16)\&(3.17)]{Start1997:phd} of a point source, that is equivalent to the farfield/high-frequency approximated explicit SDM solution \cite[(25)]{Spors10ahrens:analysis}.

\subsubsection{Referencing along a circle}
Referencing the synthesis along a circle around the point source in the synthesis plane is obtained by solving, cf. \eqref{Eq:dx0_mapping} and \eqref{Eq:dCircleLS}
\begin{equation}
d(x_0)\frac{r_0}{r_0 - d(x_0)} = R_{\mathrm{ref}} - r_0,
\end{equation}
resulting in
\begin{equation}
d_\text{circle,PS}(x_0)  = r_0 \frac{R_{\mathrm{ref}} - r_0}{R_{\mathrm{ref}}}.
\end{equation}
%
The performance of this referencing type is demonstrated in Figure \ref{Fig:Theory:point_source_referencing} (c).

\section{2.5D WFS for non-linear secondary source distributions}

\subsection{2.5D Neumann Rayleigh integral}
%
\begin{figure}
	\centering
	\begin{overpic}[width = 0.7\columnwidth ]{Figures/WFS_theory/wfs_nonlinear_Geometry.png}
	\scriptsize
	\put(17,14){$x$}	
	\put(12,28){$y$}	
	\put(1, 36){$z$}
	\put(33.5,20){$\vxo$}
	\put(46.6,31.5){$\vx$}
	\put(80,34){synthesis plane}
	\put(69,42){$\dO_{3\mathrm{D}}$}
	\put(69,23){$\dO_{2.5\mathrm{D}}$}
	\end{overpic}
\caption{General 3D WFS geometry for the derivation of 2.5D driving functions.
The SSD surface $\dO_{3\mathrm{D}} = f(x_0,y_0)$ is chosen to be independent of $z$-coordinate in order to be able to evaluate the integral with respect to $z_0$ using the SPA. If the virtual sound field is a 2D one, propagating in the direction parallel to the listening plane the the SSD can be interpreted as a continuous set of infinite vertical line sources along $\dO_{2.5\mathrm{D}}$ (described by the 2D Green's function), capable of the perfect synthesis of a virtual 2D field inside the enclosure.}
	\label{Fig:WFS_Theory:non_linear_geometry}
\end{figure}
%
In order to extend the 2.5D WFS theory for non-linear SSDs the geometry depicted in Figure \ref{Fig:WFS_Theory:non_linear_geometry} is considered. The SSD surface is chosen to be an enclosing contour in the $xy$-plane, being invariant along the vertical dimension.
Under high-frequency assumptions the general 2D/3D WFS driving functions, given by \eqref{eq:theory:gen_WFS} can be applied. In this special geometry the synthesized field inside the enclosure can be written in the form
\begin{equation}
P(\vx,\omega) = -2 \oint\displaylimits_{\dO_{2.5\mathrm{D}}} \int\displaylimits_{-\infty}^{\infty} w(\vxo) \frac{\partial P(\vxo,\omega)}{\partial \vni} G(\vx-\vxo,\omega) \, \td z_0 \, \td \dO_{2.5\mathrm{D}}(x_0,y_0).
\end{equation}
Similarly to the planar SSD case, the vertical integral may be approximated by the SPA. Again, $k_{z,P}(x,y,0)$ is assumed, i.e. virtual fields, propagating in the plane of synthesis parallel in $z=0$ direction. The stationary position for any $z=0$ receiver position therefore becomes trivially $z_0^*=0$, and the surface integral can be approximated by a contour integral, termed as the \emph{general 2.5D Neumann Rayleigh integral}
\begin{multline}
P(\vx,\omega) = -2 \oint\displaylimits_{\dO_{2.5\mathrm{D}}} \sqrt{\frac{2 \pi}{\ti \left| \phi''_{P,z,z}(\vxo,\omega) +  \phi''_{G,z,z}(\vx - \vxo,\omega) \right|}} \\
w(\vxo) \frac{\partial P(\vxo,\omega)}{\partial \vni} G(\vx-\vxo,\omega) \, \td \dO_{2.5\mathrm{D}}(x_0,y_0),
\end{multline}
with $\vx = [x,y,0]^{\mathrm{T}}$, $\vxo = [x_0,y_0, 0]^{\mathrm{T}}$ denoting in-plane positions.

The integral implicitly contains the---yet receiver position dependent---2.5D driving functions. Substituting the second derivative of the 3D SSD elements' field yields the general form 
\begin{equation}
D(\vxo,\omega) = -2 w(\vxo) \sqrt{\frac{2 \pi}{\ti k}}\sqrt{\dref(\vxo)}
\frac{\partial P(\vxo,\omega)}{\partial \vni},
\end{equation}
where the referencing function takes the same form as given by \eqref{dx0_2DSources} and \eqref{Eq:dx0_PointSource} for a 2D and a 3D point source target field respectively. Using the Kirchhoff approximation therefore the 2.5D driving functions are given by the 2.5D linear driving functions, completed by the already known windowing function.

In the followings it is demonstrated, that the referencing approach, described earlier may be applied without any modification for arbitrary SSD contours. 
Since both SPA and Kirchhoff-approximation are valid at high-frequency regions no further presumptions are required. The positions of correct synthesis are therefore given by
\begin{equation}
\vxref = \vxo + \vhk |\vx-\vxo|
\end{equation}
for an arbitrary SSD geometry.


\subsection{Referencing function and referencing schemes}

The validity of the referencing schemes are demonstrated via the example of a circular SSD with a radius of $R_0$. 
For such an SSD ensemble the normal derivative is given by the radial derivative and the WFS driving function reads
\begin{equation}
D(\beta,\omega) = + \sqrt{\frac{8\pi}{\ti k}}\,w(\beta)\,\sqrt{d(\beta)} \left. \frac{\partial P(r,\beta)}{\partial r}  \right|_{r = R_0}.
\end{equation}

\begin{figure}
	 \centering
	 \begin{overpic}[width = .95\columnwidth]{Figures/WFS_theory/real_part_circular.png}
	 \end{overpic}
 \caption{ The synthesis of a plane wave (a) and a line source (b), used in the following examples. The plane waves propagates in the $x$-direction. The line source is located at $\mathbf{x}_s = [-3,\ 0]^{\mathrm{T}}$. The SSD is a circular array with $R_0 = 2~\mathrm{m}$. The source frequency is set to $\omega_0 = 2\pi	\cdot 1 ~\mathrm{krad/s}$.}
	 \label{Fig:Theory:real_part_circular}
 \end{figure}
%
The referencing function $d(\beta)$ for an arbitrary curve can be found in the same manner as for a linear SSD using the SPA. 
In this case geometrical considerations give us a much simpler referencing method: the stationary secondary source can be found where the wavenumber vector intersects the SSD, which is a constant directed vector for a plane wave, and any vector directed radially from the source position for a line/point source. 
The position of reference position can be found in the same direction from any stationary secondary source in a distance of $d(\beta)$ in the plane wave and line source case, and $d(\beta)\frac{r_0}{r_0-d(\beta)}$ for a 3D point source.

Two referencing schemes are presented in order to demonstrate the validity of the presented approach: referencing the synthesis along a straight line, and along a concentric circle inside the circular SSD.

\subsubsection{Effects of constant referencing}

First, the effect of a constant referencing function is investigated. 
For both a plane wave and a line source the position of correct synthesis can be found based on the problem geometry.

\begin{figure}
	\centering
	\begin{overpic}[width = .85\columnwidth  ]{Figures/WFS_theory/fixed_referencing_circle.png}
	\scriptsize
	\put(6,36.2){$\varphi_\text{PW}$}
	\put(43, 17){$x$}
	\put(23, 39){$y$}
    \put(9.5,30){$\dref$}
    \put(3,  39){$\mathbf{k}$}
    \put(17.5, 19.2){$\beta$}
	\put(3.5,    27){$\vxo$}
%
    \put(63.5, 27.5){$\dref$}
    \put(100,17){$x$}
    \put(79, 38){$y$}
    \put(69, 22){$R_0$}
    \put(56.5,21.5){$r_0$}
    \put(58.5,19.5){$\alpha$}
    \put(68.5,19.5){$\beta$}
    \put(55,  17){$\mathbf{x}_s$}
	\end{overpic}
\caption{ Geometry for finding the position of amplitude correct reproduction of a plane wave (a) and a line source (b) }
	\label{Fig:Theory:fixed_referencing_circle}
\end{figure}

For the case of the synthesis of a plane wave it is obvious, that the points from each SSD element at a distance of $\dref$ describe a circle with the same radius as the SSD, translated in the direction of the plane wave by $\dref$. 
See Figure \ref{Fig:Theory:fixed_referencing_circle} (a). The PCS are therefore given by
\begin{eqnarray}
\mathbf{x}_{\mathrm{ref,PW}}(\vxo) = \colvec{2}{ \cos \beta \,R_0  + \cosfi_\text{PW}\, \dref }{  \sin \beta \,R_0  + \sinfi_\text{PW}\, \dref }
\label{Eq:Fixed_referencing_pw_circle}
\end{eqnarray}
with $\varphi_\text{PW}$ being the propagation angle of the plane wave measured from the $x$-axis.
\begin{figure}
	\centering
	\begin{overpic}[width = .95\columnwidth]{Figures/WFS_theory//fixed_referencing_circular.png}
	\end{overpic}
\caption{ Positions of amplitude correct synthesis using circular SSD for a virtual plane wave (a) and line source (b). The referencing function is set to $\dref = 1~\mathrm{m}$. }
	\label{Fig:Theory:fixed_referencing_circular}
\end{figure}

For a virtual line source the parametric curve can be given by the problem geometry, displayed in Figure \ref{Fig:Theory:fixed_referencing_circle} (b).
The PCS are given by
\begin{eqnarray}
\mathbf{x}_{\mathrm{ref,LS}}(\vxo) =  \colvec{2}{ x_s + \cos\alpha \, (\dref + r_0)}{ y_s + \sin\alpha\, (\dref+r_0)}.
\label{Eq:Fixed_referencing_ls_circle}
\end{eqnarray}
This curve is not easily accessible for analytic investigation, however two limiting curves can be described purely from the problem geometry. 
As the virtual source approaches the SSD the curve describes a circle around the virtual source with a radius of $\dref$. 
If the virtual source is very far from the SSD the incident field becomes a plane wave and the curve describes a circle with a radius same as the SSD, translated by $\dref$ into the direction, defined by the position vector of the virtual source. 
%In other virtual source position the curve is a mixture of these two circles. 
It is also ensured, that in front of the virtual source, the distance of amplitude correct synthesis from the corresponding SSD element is $\dref$. 
%This fact is in analogy with the linear SSD case.
Results of numeric simulation, using these referencing schemes are depicted in Figure \ref{Fig:Theory:fixed_referencing_circular}.

\begin{figure}
	\centering
	\begin{overpic}[width = .85\columnwidth]{Figures/WFS_theory/fixed_referencing_circular_ps.png}
	\end{overpic}
\caption{ Positions of amplitude correct synthesis using circular SSD with radius of $R_0 = 1.5~\mathrm{m}$ for a virtual point source with the virtual source positioned at $\mathbf{x}_s = [-2.5,\ 0,\ 0]^{\mathrm{T}}$ (a) and $\mathbf{x}_s = [-4,\ 0,\ 0]^{\mathrm{T}}$ (b). The referencing function is set to $\dref = 1~\mathrm{m}$. In (a) in front of the virtual source is $0.5~\mathrm{m}$, with $r_0< \dref$, therefore no position for correct synthesis can be found. As the virtual source gets further from the SSD (b), $r_0> \dref$ is satisfied, and the position of correct synthesis in front of the virtual source becomes $x_c = x_s + r_0\frac{r_0}{r_0-\dref} = 0$. }
	\label{Fig:Theory:fixed_referencing_circular_ps}
\end{figure}

In each case it should be noted, that the position of amplitude correct synthesis can be defined only inside the SSD. 
Moreover, due to high-frequency approximations for the arbitrary SSD extension the curve of correct synthesis is shorter, than it would be allowed by the previous limitation (i.e. does not limited by the intersections with the SSD). 
This can be explained by two reasons: in the ending of the curves the corresponding stationary SSD elements can not be modeled as flat surfaces in order to apply the Kirchhoff-approximation due to the large local angle of inclination of the incident field. On the other hand in these parts also the diffraction waves would have considerable contribution in the synthesized field, which are inherently omitted in the geometrical optics approximation (i.e. in these parts of the SSD both assumptions for the Kirchhoff approximation fail).
This phenomenon can be observed in both Figure \ref{Fig:Theory:fixed_referencing_circular} (a) and (b).

For the synthesis of a virtual point source the substitution of $\dref \rightarrow \dref \frac{r_0}{r_0 - \dref}$ w.r.t. the line source case in \eqref{Eq:Fixed_referencing_ls_circle} holds as discussed above. The PSC are then given as
\begin{eqnarray}
\mathbf{x}_{\mathrm{ref,PS}}(\vxo) =  \colvec{2}{ x_s + \cos\alpha \, r_0 \frac{r_0}{r_0 - d}  }{ y_s + \sin\alpha\, r_0 \frac{r_0}{r_0 - d} }.
\label{Eq:Fixed_referencing_ps_circle}
\end{eqnarray}
Similarly to the linear SSD case that means, that in front of the virtual source the position of correct synthesis measured from the SSD becomes $\dref \frac{r_0}{r_0- \dref}$, therefore if the point source is closer to the SSD than the reference distance, no position for amplitude correct synthesis can be found. This can be observed in Figure \ref{Fig:Theory:fixed_referencing_circular_ps}.

\subsubsection{Referencing to an arbitrary curve}
\begin{figure}
	\centering
	\begin{overpic}[width = .85\columnwidth]{Figures/WFS_theory/referencing_on_circle_and_line.png}
	\scriptsize
	\put(43, 17){$x$}
	\put(23, 39){$y$}
    \put(9.5,28.5){$d(\varphi)$}
    \put(5,  38){$\mathbf{k}$}
    \put(16, 20){$\beta$}
	\put(4.5,  28){$x_0$}
	\put(12,  22){$R_0$}
	\put(16.5,  25){$R_{\mathrm{ref}}$}
%
    \put(65, 27.5){$d(\varphi)$}
    \put(100,17){$x$}
    \put(80, 39){$y$}
    \put(69, 22){$R_0$}
    \put(56.5,21.5){$r_0$}
    \put(70.5,17){$x_0$}
    \put(68.5,20){$\beta$}
    \put(55,  17){$\mathbf{x}_s$}
	\end{overpic}
\caption{ Geometry for finding $d(\varphi)$ referencing the synthesis of a plane wave to a concentric circle (a) and the synthesis of a line source to a straight line (b). }
	\label{Fig:Theory:referencing_on_line_and_circle}
\end{figure}

Finally it is shown, how synthesis may be referenced to an arbitrary curve. 
Two examples are presented: referencing a plane wave on a circle, and a point source on a line.
Our aim is to find the referencing function $d(\beta)$, equaling the length of the corresponding vector, seen in Figure \ref{Fig:Theory:referencing_on_line_and_circle}. With geometrical considerations they can be expressed as
\begin{eqnarray}
d_\text{circle,PW}(\beta) = R_0\cos \beta - \sqrt{ R_{\mathrm{ref}}^2 - R_0^2\,\sin^2\varphi_\text{PW} }
\\
d_\text{line,LS}(\beta) = -r_0 \left( \frac{x_0 + R_0\cos \beta}{x_s + R_0\cos \beta} \right)
\end{eqnarray}
respectively. 
This latter would ensure the amplitude correct synthesis for a line source. In order to apply the formulation for a 3D point source  $\dref \rightarrow \dref \frac{r_0}{r_0 - \dref}$ substitution is needed, ending up in
\begin{equation}
d_{ps,line}(\beta) = r_0 \frac{x_0 + R_0\cos \beta}{x_0-x_s}.
\end{equation}
The validity of the referencing functions are demonstrated in Figure \ref{Fig:Theory:pw_to_circle_ps_to_line}.
\begin{figure}
	\centering
	\begin{overpic}[width = 1\columnwidth]{Figures/WFS_theory/pw_to_circle_ps_to_line.png}
	\end{overpic}
\caption{ Result of referencing the synthesis of a plane wave to a circle (a) and a 3D point source to a line. The plane wave propagates parallel to the $x$-axis and the point source is located at $\mathbf{x}_s = [-4,\ 0,\ 0]^{}\mathrm{T}$. The radius of the reference circle is $R_{\mathrm{ref}} = 1~\mathrm{m}$ at (a) and the line of referencing is located at $x_0 = -0.5~\mathrm{m}$ in (b).}
	\label{Fig:Theory:pw_to_circle_ps_to_line}
\end{figure}

\newpage
\section{Relation with the explicit solution}
\subsection{SDM driving functions in the spatial domain}
\subsection{Driving functions on the SSD}
	
%
% Referencing focused sources
% Synthesis applying directive SSD
% The WFS pre-fiters in time domain

%\begin{appendices}
%\chapter{Appendix A}
%\section{Definition and properties of Fourier transform and the Dirac delta}
\label{App:Fourier_def}

\paragraph{Temporal Fourier transform:}\mbox{} \\
Given a four dimensional function $f(\vx,t)$, depending on both time and the spatial position.
The forward and inverse temporal Fourier transform is defined as 
\begin{equation}
\label{eq:temporal_fourier_transform_def}
F(\vx,\omega) = \mathcal{F}_t \left\{ f(\vx,t) \right\} = \int\limits_ {-\infty}^{\infty} f(\vx,t) \te^{-\ti \omega t} \td t,
\end{equation}
\begin{equation}
\label{eq:temporal_inverse_fourier_transform_def}
f(\vx,t) = \mathcal{F}_{\omega}^{-1} \left\{ F(\vx,\omega) \right\} = \frac{1}{2\pi} \int_ {-\infty}^{\infty} F(\vx,\omega) \te^{ \ti \omega t} \td \omega.
\end{equation}
Note that capital letter indicates that the function is taken in the angular frequency domain.
%
\paragraph{Spatial Fourier transforms:}\mbox{} \\
Following the convention, given in e.g. \cite{Ahrens2012} the spatial Fourier transform is defined as follows:
\begin{itemize}
\item in one dimension:
\begin{equation}
\label{eq:spatial_fourier_transform_def}
\tilde{F}(k_x,y,z,\omega) = \mathcal{F}_x \left\{ F(\vx,\omega) \right\} = \int_ {-\infty}^{\infty} F(\vx,\omega) \, \te^{\ti k_x x} \, \td x,
\end{equation}
\begin{equation}
\label{eq:spatial_inverse_fourier_transform_def}
F(\vx,\omega) = \mathcal{F}_{k_x}^{-1} \left\{ \hat{F}(k_x,y,z,\omega) \right\} = \frac{1}{2\pi} \int_ {-\infty}^{\infty} \tilde{F}(k_x,y,z,\omega) \, \te^{ -\ti k_x x} \td k_x,
\end{equation}
\item in two dimensions:
\begin{equation}
\label{eq:spatial_fourier_transform_def_2d}
\tilde{F}(k_x,y,k_z,\omega) = \iint_ {-\infty}^{\infty} F(\vx,\omega) \te^{\ti \left( k_x x + k_z z \right) } \td x \, \td z,
\end{equation}
\begin{equation}
\label{eq:spatial_inverse_fourier_transform_def_2d}
F(\vx,\omega) = \frac{1}{\left( 2\pi \right)^2} \iint_ {-\infty}^{\infty} \tilde{F}(k_x,y,k_z,\omega) \, \te^{ -\ti \left( k_x x + k_z z \right)} \td k_x \, \td k_z,
\end{equation}
\item in three dimensions:
\begin{equation}
\label{eq:spatial_fourier_transform_def_3d}
\tilde{F}(\vk,\omega)= \iiint_ {-\infty}^{\infty} F(\vx,\omega) \te^{ \ti \left< \vk \cdot \vx \right>} \td x \,\td y\,\td z,
\end{equation}
\begin{equation}
\label{eq:spatial_inverse_fourier_transform_def_3d}
F(\vx,\omega) = \frac{1}{\left( 2\pi \right)^3} \iiint_{-\infty}^{\infty} \tilde{F}(\vk,\omega) \te^{ -\ti \left< \vk \cdot \vx \right>} \td k_x \, \td k_y \, \td k_z.
\end{equation}
\end{itemize}
Hence, hat over the function symbol indicates that the function is taken in the wavenumber domain.
Note that the exponent of the spatial Fourier transform is taken with a reversed sign, compared to the temporal transform.
Writing an arbitrary function in the form of a spatio-temporal inverse Fourier transform
\begin{equation}
f(\vx,t) = \frac{1}{\left( 2\pi \right)^4} \iiiint_{-\infty}^{\infty} \hat{F}(\vk,\omega) \te^{ \ti \left( \omega t - \left< \vk \cdot \vx \right> \right)} \td k_x \, \td k_y \, \td k_z \td \omega
\end{equation}
basically describes the expansion of an arbitrary function into the linear combination of plane waves, propagating to direction $\vk$.
The reversed sign therefore ensures that this simple physical interpretation can be assigned to the Fourier transform.

\paragraph{Fourier transform properties:}\mbox{} \\
Several important properties of the Fourier transform, applied frequently in the present thesis are the following.
\begin{itemize}
\item Shift theorem:
\begin{equation}
\int_{-\infty}^{\infty} f(x-x_0) \te^{\ti k_x x} \td x = \mathcal{F}_x \left\{ f(x-x_0) \right\} = \hat{F}(k_x) \te^{\ti k_x x_0}.
\end{equation}
In case of temporal Fourier transform the right side is with reversed exponent.
\item Convolution theorem:
\begin{equation}
\int_{-\infty}^{\infty} f(x-x_0) g(x_0) \td x_0 \, \te^{\ti k_x x} \td x = \mathcal{F}_x \left\{ f(x) \ast_x g(x) \right\} = \hat{F}(k_x) \cdot \hat{G}(k_x).
\end{equation}
\item Differentiation property:
\begin{equation}
\int_{-\infty}^{\infty} \frac{\partial}{\partial x} f(x) \te^{\ti k_x x} = \mathcal{F}_x \left\{ \frac{\partial}{\partial x} f(x) \right\} = 
-\ti k_x \hat{F}(k_x).
\end{equation}
In case of temporal Fourier transform the right side is with reversed sign.
\item Scaling property:
\begin{equation}
\int_{-\infty}^{\infty} f(a x) \te^{\ti k_x x} = \mathcal{F}_x \left\{ f( a x) \right\} = 
\frac{1}{|a|}\hat{F}(\frac{k_x}{a}).
\end{equation}
In case of temporal Fourier transform the right side is with reversed sign.
\end{itemize}

\paragraph{Properties of Dirac delta:}\mbox{} \\
The Dirac delta is a generalized function (or distribution) applied frequently in modeling acoustic phenomena, defined as
\begin{equation}
\delta(x) = 
\begin{cases}
\infty, & \hspace{1mm} x = 0\\
0, & \hspace{1mm}  x \neq 0
\end{cases},
\hspace{1cm}
\text{with}
\hspace{2cm}
\int_{-\infty}^{\infty} \delta(x) \td x = 1.
\end{equation}
Several important properties of the Dirac delta, applied frequently in the present thesis are the following.
\begin{itemize}
\item Forward Fourier transform:
\begin{equation}
\mathcal{F}_x \left\{\delta(x-x_0)\right\} = \int_{-\infty}^{\infty} \delta(x-x_0) \te^{\ti k_x x} \td x =   \te^{\ti k_x x_0}.
\end{equation}
\item Inverse Fourier transform:
\begin{equation}
\delta(x-x_0) = \frac{1}{2\pi} \int_{-\infty}^{\infty} \te^{-\ti k_x (x-x_0)} \td k_x =  \mathcal{F}^{-1}_{k_x} \left\{ \te^{\ti k_x x_0} \right\}.
\end{equation}
\item Sifting property:
\begin{equation}
\int_{-\infty}^{\infty} \delta(x-x_0) f(x) \td x = f(x_0).
\end{equation}
\item Generalized sifting property:
\begin{equation}
\int_{-\infty}^{\infty} f(x) \delta(g(x)) \td x = \sum_{i} \frac{f(x_i)}{\left| \frac{\partial}{\partial x} g(x) \right|_{x = x_i}}, \hspace{5mm} \text{where} \hspace{5mm} g(x_i) = 0.
\end{equation} 
\end{itemize}
%\end{appendices}

%\the\textwidth
%\\
%textwidth in cm: \printinunitsof{cm}\prntlen{\textwidth}

\bibliographystyle{plain}
\bibliography{dissertation}

\end{document}