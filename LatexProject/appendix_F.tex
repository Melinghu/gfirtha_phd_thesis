\section{Spectral representation of non-stationary convolutions}
\label{Sec:Non_stat_conv}

Assume the following non-stationary convolution
\begin{equation}
g(t) = \int_{-\infty}^{\infty} h(t_0,t-t_0) \, f(t_0) \, \td t_0,
\label{Eq:App:Non_stat_conv_td}
\end{equation}
with $h(t_0,t-t_0)$ denoting the time-variant convolution kernel.
Terms $f(t_0)$ and $h(t_0,t-t_0)$ can be expressed in terms of the inverse transform of their spectra, given as
\begin{align}
f(t_0) &= \frac{1}{2\pi} \int_{-\infty}^{\infty} F(\omega_0) \, \te^{\ti \omega_0 t_0}\, \td \omega_0,
\\
h(t_0,t-t_0) &= \frac{1}{2\pi}\int_{-\infty}^{\infty} H(t_0,\omega_1) \, \te^{\ti \omega_1(t-t_0)} \, \td \omega_1.
\end{align}
Note that in case of applying the non-stationary convolution in order to calculate the field of a moving source as presented in Chapter \ref{sec:moving_source_synthesis}, $\omega_0$ represents the source frequency, while $\omega_1$ will turn out to represent the perceived angular frequency.

By substituting the above formulations into \eqref{Eq:App:Non_stat_conv_td}, the non-stationary convolution can be rewritten into the form
\begin{equation}
g(t) = \frac{1}{2\pi} \iint_{-\infty}^{\infty}  H(t_0,\omega_1) \, \te^{\ti \omega_1(t-t_0)} \, \td \omega_1 f(t_0) \td t_0,
\end{equation}
\begin{equation}
g(t) = \frac{1}{(2\pi)^2} \iiint_{-\infty}^{\infty}  H(t_0,\omega_1) \te^{\ti \omega_1(t-t_0)} \, \td \omega_1  \, F(\omega_0) \, \te^{\ti \omega_0 t_0} \, \td \omega_0 \, \td t_0.
\end{equation}
Taking the temporal forward Fourier transform of the latter expression yields
\begin{equation}
G(\omega) =\frac{1}{(2\pi)^2} \iiint_{-\infty}^{\infty} H(t_0,\omega_1) \te^{\ti \omega_1(t-t_0)} \, \td \omega_1 \int_{-\infty}^{\infty} F(\omega_0) \, \te^{\ti \omega_0 t_0} \, \td \omega_0 \, \td t_0 \, \te^{-\ti \omega t} \, \td t.
\end{equation}
Reversing the order of integration and rearrangement results in
\begin{equation}
G(\omega) =\frac{1}{(2\pi)^2} \iiint_{-\infty}^{\infty} H(t_0,\omega_1) F(\omega_0)  \te^{-\ti (\omega_1-\omega_0) t_0}  
\underbrace{ \int_{-\infty}^{\infty}  \te^{-\ti (\omega-\omega_1) t}  \td t}_{2\pi \delta(\omega-\omega_1)}
 \, \td \omega_1 \,  \td \omega_0 \, \td t_0.
\end{equation}
Integration with respect to $\omega_1$ sifts out $\omega_1 = \omega$:
\begin{equation}
\label{Eq:App:Non_stat_conv_fd_1}
G(\omega) =  \iint_{-\infty}^{\infty} H(t_0,\omega) F(\omega_0)  \, \te^{-\ti (\omega-\omega_0) t_0}  \, \td \omega_0 \, \td t_0.
\end{equation}
Finally, it is exploited that in \eqref{Eq:App:Non_stat_conv_fd_1} 
\begin{equation}
\tilde{H}(\omega-\omega_0,\omega) =  \int_{-\infty}^{\infty}  H(t_0,\omega) \, \te^{-\ti (\omega-\omega_0) t_0} \, \td t_0,
\end{equation}
describes the 2D spectrum of the 2D impulse response, where $\tilde{H}(w_1,w_2) = \FT[t_1,t_2]{h(t_1,t_2)}$.
Therefore, as a final result
\begin{equation}
G(\omega) =  \int_{-\infty}^{\infty} H(\omega - \omega_0,\omega) F(\omega_0)  \td \omega_0
\label{Eq:App:Non_stat_conv_fd}
\end{equation}
is obtained.
Hence, a non-stationary convolution with respect to the second variable of the time domain yields a non-stationary convolution in the first variable of its spectral representation.