The present thesis discussed a unified Wave Field Synthesis theory, giving an asymptotic, ray-tracing solution for the general sound field synthesis problems.
As the central result of the present treatise, general 2.5 dimensional WFS loudspeaker driving functions are introduced, allowing the synthesis of arbitrary virtual sound field by applying an arbitrary shaped SSD contour, with ensureing an optimal synthesis along a prescribed reference curve.

The central concept of the presented results is the stationary phase method, yielding an asymptotic approximation of integral formulas.
Adapted from classic ray-tracing theory, the local wavenumber vector and the local wavefront curvature are introduced, as local attributes of sound fields.
These quantities allow a simple, elegant physical interpretation of asymptotic evaluation of integrals, describing radiation problems: it is demonstrated via numerous examples, how boundary integrals and spatial Fourier transforms can be evaluated around their stationary point, and how the stationary points for these problems can be found in a simple geometric manner.
All the involved concepts give a fair approximation for radiation problems under high frequency conditions.

By applying the introduced local wavenumber concept it is demonstrated that the stationary phase approximation of boundary integrals, with the integral kernel being the Green's function basically realizes a wavefront matching between the target field and the surface point sources regarding both propagation direction and wavefront curvature.
For a contour of secondary source elements this wavefront matching is possible in amplitude only at a single receiver point per secondary source element.
The ensemble of these individual reference positions form the reference curve, that's shape can be controlled with a simple amplitude term in the boundary integral kernel.
From the stationary phase evaluation of the Kirchhoff approximation the unified 2.5D Wave Field Synthesis driving functions are extracted, inherently containing the above referencing concept.
The introduced generalized theoretical framework contains previous WFS approaches as special cases, as it is demonstrated via examples.

Alternatively to WFS also an explicit solution exists for the general sound field synthesis scenario, solving the problem by mode-matching in the spectral domain.
For an infinite linear secondary source distribution the approach yields the driving functions in the form of a spectral integral, while for enclosing arrays (e.g. circular, spherical distributions) the solution is an infinite spectral sum.
By applying the stationary phase approximation in order to evaluate the explicit spectral driving function for a linear SSD a novel, purely spatial domain explicit driving function is introduced: the new driving function requires the description of the target sound field measured along the reference curve, opposed to the WFS solution, which requires the target field properties on the secondary source array.
It is verified that this new solution also realizes the wavefront matching of the target wavefront and the secondary sources' wavefronts of a prescribed reference curve.
Hence, the global mode-matching solution can be transformed into a local wavefront matching approach under high frequency conditions.
This solution may be generalized towards synthesis with arbitrary shaped secondary source distribution, as long as the loudspeaker array can be considered locally linear.
Finally it is also demonstrated that the explicit spatial driving functions are equivalent with the unified WFS solution.

The presented equivalence of the spatial explicit and implicit solutions can be exploited in order to discuss phenomena, concerning WFS in a unified manner.
As an example, the effects of the secondary source discretization---modeling real-life loudspeaker arrays--- can be analytically described in the wavenumber domain, by using the explicit solution.
By utilizing the introduced local wavenumber concept a simple antialiasing strategy, that can be realized by simple low-pass filtering of the driving signals is presented, in order to avoid the resulting aliasing wavefronts.
As a result antialiased synthesis may be achieved along certain direction over the listening region.

Finally, as a complex application example for the foregoing the synthesis of a moving point source is discussed within the context of the introduced unified WFS framework.
In this case the proper reproduction of the Doppler effect is of central importance, which is inherently ensured once an appropriate analytical model is used.
It is presented how the local attributes of wavefronts can be extended for moving sources and it is demonstrated how the introduced WFS framework can be adapted to this dynamic scenario.
For the special case of sources under uniform motion explicit driving functions are introduced in the wavenumber and in the spatial domain for 3D synthesis, applying a planar SSD and 2.5D synthesis, applying a linear SSD.
Similarly to the stationary case, the explicit driving function is found to coincide with the WFS solution under high frequency assumtions.
Based on the wavenumber domain representation the effects of secondary source disretization is discussed, and the antialiasing strategy introduced for stationary fields is extended for the synthesis of moving sources.

\vspace{3mm}
An aim of the present thesis is to give a complete, self-contained discussion on the questions concerning Wave Field Synthesis, however several aspects were out of the scope of the present treatise.
As the most important example throughout the foregoing only diverging fields were discussed, i.e. the target wavefronts would originate from a sound source, outside the listening region.
Under several restrictions Wave Field Synthesis is capable of synthesizing focused sources, in which case the synthesized wavefront converges towards a focal point inside the listening area.
The synthesis of such converging sound field needs the proper manipulation of the stationary phase approximation.
Although hints are given in the treatise, how adaptation of the unified framework to focused sources could be done, the exact study of this focused case---even involving focused moving sources--- is the subject of future work.