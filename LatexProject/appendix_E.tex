\section{Representation of sources, moving on straight trajectories}
\label{App:Moving_source_representations}

For the sound field, generated by sources moving uniformly along arbitrary directed straight trajectories closed form expression may be found either in the spatio-temporal, time-frequency or in the wavenumber domain.
All the formulations utilize the analytical expression for a source, moving parallel with the $x$-axis and a corresponding coordinate transform.

Assume a source, with the location time history given by $\vxs(t) = \posvec{3}{v\cdot t}{0}{0}$, i.e. moving along the $x$-axis, located at the origin at $t=0$.
The point source oscillates harmonically at the angular frequency $\omega_0$
The generated sound field in the spatio-temporal domain reads as
\begin{equation}
P_{\mathrm{m}}(\vx,t,\omega_0) = \frac{1}{4\pi} \frac{\te^{\ti \omega_0 \left( t - \tau(\vx,t) \right) )}}{\Delta(\vx,t)},
\end{equation}
with
\begin{eqnarray}
\Delta(\vx,t) = \sqrt{(x - v\cdot t)^2 + (y^2 + z^2) (1-M^2)} 
\\
\tau(\vx,t) = \frac{1}{c}\frac{M(x- v \cdot t) + \Delta(\vx,t)}{(1-M^2)}.
\end{eqnarray}
The angular frequency content of the radiated field is given by \eqref{Eq:Moving_sources:Freq_domain_representation}, reading
\begin{equation}
P_{\mathrm{m}}(\vx,\omega,\omega_0) =
\frac{1}{v}
\tilde{G}(\frac{\omega-\omega_0}{v}, y, z, \omega)
\te^{-\ti \frac{\omega-\omega_0}{v} x},
\end{equation}
%
\begin{figure}
\centering
	\begin{overpic}[width = 0.75\columnwidth ]{Figures/Appendices/MS_Coordinate_transform.png}
	\small
	\put(-5, 0){(a)}
	\put(47,0){(b)}	
	\end{overpic}   
    \caption{Receiver coordinate transform in order to derive the field of a point source moving along an arbitrary directed straight trajectory.
    }
\label{fig:App:ms_coo_tr}  
\end{figure}

These formulations can be applied in order to express the corresponding representations for sources moving parallel with the $z=0$ plane and arriving to the $x$-axis under an angle of inclination $\varphi$.
This is performed by a simple rotation of the receiver coordinate system, so that the new coordinate system $\mathbf{x'}= [x',\ y',\ z']^{\mathrm{T}}$ is defined by the transform
%
\begin{equation}
\vx'=
\begin{bmatrix} x' \\ y' \\ z' \end{bmatrix}
=
\begin{bmatrix}
\cosfi  &  \sinfi & 0\\
-\sinfi & \cosfi &0\\
0       &   0 &   1
\end{bmatrix}
\begin{bmatrix} x-x_\mathrm{s} \\ y-y_\mathrm{s} \\ z -z_s \end{bmatrix}.
\label{eq:transform_matrix}
\end{equation} 
%
As it is illustrated in Figure \ref{fig:App:ms_coo_tr} in the shifted and rotated coordinate system the source is located in the origin at $t = 0$ and travels parallel to the $x'$-axis, thus the radiated field is written in time time domain 0as
\begin{equation}
P_{\mathrm{m}}(\mathbf{x},t,\omega_0) = \frac{1}{4\pi} \frac{\te^{\ti \omega_0 \left(t - \tau(\mathbf{x}',t) \right)}}{\Delta(\mathbf{x}',t)}.
\label{eq:inclined_field}
\end{equation}
Note, that this description of sources with arbitrary linear trajectories is equivalent with the approach proposed in \cite{Ahrens2008moving}.
The corresponding frequency domain formulation is simply obtained as
\begin{equation}
P_{\mathrm{m}}(\vx,\omega,\omega_0) =
\frac{1}{v}
\tilde{G}(\frac{\omega-\omega_0}{v}, y', z', \omega)
\te^{-\ti \frac{\omega-\omega_0}{v} x'}.
\label{eq:inclined_field_omega}
\end{equation}

The wavenumber domain representation is obtained by the spatial Fourier transform of \eqref{eq:inclined_field_omega} along the $x$-axis.
According to the convolution theorem the wavenumber content is written as a spectral convolution
\begin{equation}
\tilde{P}_{\mathrm{m}}(k_x,y,z,\omega,\omega_0) =
\frac{1}{v}
\mathcal{F}_x \left\{
\tilde{G}(\frac{\omega-\omega_0}{v}, y', z', \omega) \right\}
\ast_{k_x}
\mathcal{F}_x \left\{
\te^{-\ti \frac{\omega-\omega_0}{v} x' }
 \right\},
\end{equation}
with the explicit dependencies with respect to the $x$-coordinate given by \eqref{eq:transform_matrix}.
For the sake of simplicity the spectrm is only investigated in the most relevant plane, at $z = 0$.
The first term on the right handside may be either evaluated by making use of the expression \cite[(6.677,9.)]{Gradshteyn2007} or applying directly the angular spectrum representation of the Green's function given in \eqref{tab:theory:Greens_fun_representations} together with applying the Fourier similarity and shift theorem
\begin{equation}
\mathcal{F}_x \left\{
\tilde{G}(\frac{\omega-\omega_0}{v}, y', z', \omega) \right\} =
\frac{1}{|\sinfi|} \tilde{G}( \frac{\omega-\omega_0}{v}, - \frac{k_x}{\sinfi} , z ) \te^{\ti k_x \frac{\sinfi x_s + \cosfi (y-y_s)}{\sinfi}}.
\end{equation}
The right term in the spectral convolution is again obtained from the spectrum of an exponential using the shifting theorem, yielding
\begin{equation}
\mathcal{F}_x \left\{
\te^{-\ti \frac{\omega-\omega_0}{v} x'(x)}
 \right\} = 2\pi \delta(k_x - \cosfi \frac{\omega-\omega_0}{v}) \te^{-\ti \frac{\omega-\omega_0}{v}\left( \sinfi(y-y_s) - \cosfi x_s \right)}.
\end{equation}
The convolution of the two expressions sifts out $k_x = k_x - \cosfi \frac{\omega-\omega_0}{v}$, resulting in the final expression for the wavenumber content of the moving source
\begin{multline}
\tilde{P}_{\mathrm{m}}(k_x,y,z,\omega,\omega_0) = \\ =
\frac{2\pi}{v|\sinfi|} \tilde{G}( \frac{\omega-\omega_0}{v}, \frac{\cosfi \frac{\omega-\omega_0}{v}- k_x }{\sinfi} , z )  
\te^{\ti k_x\frac{\sinfi x_s + \cosfi (y-y_s)}{\sinfi}}
\te^{-\ti \frac{\omega-\omega_0}{v} \frac{y-y_s}{\sinfi}},
\end{multline}
while obviously, the angular spectrum representation simply reads as a further Fourier transform of the Green's function with respect to the $z$-dimension
\begin{multline}
\tilde{P}_{\mathrm{m}}(k_x,y,k_z,\omega,\omega_0) = \\ =
\frac{2\pi}{v|\sinfi|} \tilde{G}( \frac{\omega-\omega_0}{v}, \frac{\cosfi \frac{\omega-\omega_0}{v}- k_x }{\sinfi} , k_z )  
\te^{\ti k_x\frac{\sinfi x_s + \cosfi (y-y_s)}{\sinfi}}
\te^{-\ti \frac{\omega-\omega_0}{v} \frac{y-y_s}{\sinfi}}.
\end{multline}
These latter two wavenumber representations can be proven to converge into a Dirac-delta distribution in $k_x$ as the inclination angle converges to zero $\varphi \rightarrow 0$, as given by \eqref{Eq:Moving_sources:MS_field_kx_z} and \eqref{Eq:Moving_sources:MS_field_kx_kz}.