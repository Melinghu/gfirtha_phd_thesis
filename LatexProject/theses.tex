\begin{thesisgroup}[Generalization of WFS theory]
I introduced a generalized WFS framework allowing one to synthesize 3D sound fields with arbitrary shaped convex loudspeaker ensembles (secondary source distribution (SSD)) and to optimize the synthesis on an arbitrary convex reference curve. 
The generalized framework inherently contains the existing WFS approaches as special cases \cite{Firtha2016}.
\begin{thesis}
I established a physical interpretation of the stationary phase approximation (SPA) of boundary integrals.
By defining the local wavenumber vector of a time-harmonic sound field, I showed that the SPA ensures wave front matching of the virtual field and the secondary sound fields at the receiver position.
\end{thesis}
\begin{thesis}
I derived WFS driving function for an arbitrary convex SSD contour based on the above physical interpretation, within the validity of the physical optics approximation of the Kirchhoff-Helmholtz integral.
\end{thesis}
\begin{thesis}
I derived analytical expression for the general \emph{reference curve} that connects the points in the synthesis plane where the amplitude error is minimal.
I presented how the shape of the reference curve can be controlled by applying a frequency independent amplitude correction term to the driving function.
I critically revised existing WFS solutions by the analytical characterization of their reference curves. 
\end{thesis}
\end{thesisgroup}

\begin{thesisgroup}[Spatial explicit driving functions and WFS equivalence]
Besides the implicit WFS technique---yielding the required driving functions as an implicit integral kernel in a reduced surface integral---explicit solutions exist obtaining the driving functions as a spectral integral.
For a linear SSD the explicit solution is termed as the \emph{Spectral Division Method (SDM)} yielding the linear driving functions in terms of an inverse spatial Fourier transform of the ratio of the target field spectrum and the Green's function spectrum measured along a reference line.
So far the connection between the implicit and explicit solutions has only been investigated for special target sound fields.
By applying the SPA to the SDM driving function, I derived it's asymptotical spatial approximation, and I highlighted the general equivalence of the explicit and implicit solutions in the high frequency region. \cite{Firtha2017:daga, Firtha2018:WFS_vs_SDM}
\begin{thesis}
I derived analytical SDM driving functions in the spatial domain by applying the SPA to the Fourier integral with establishing a physical interpretation of the stationary phase approximation of Fourier integrals.
Unlike WFS the new expicit driving functions express the SSD driving signals in terms of the target sound field measured along the convex reference curve \cite{Firtha2017:daga}.\end{thesis}
\begin{thesis}
I proved that under high frequency assumptions the explicit SDM and the implicit WFS driving functions are completely equivalent for an arbitrary target sound field \cite{Firtha2018:WFS_vs_SDM}.
The proof is performed by expressing the newly introduced driving function in terms of the target field's gradient measured on the SSD.
\end{thesis}
\begin{thesis}
I gave a simple asymptotic anti-aliasing criterion in order to suppress aliasing waves emerging due to the application of a discrete SSD in practical scenarios.
The derivation is based on the above equivalence of the WFS and SDM driving function.
The proposed approach can be implemented in practice by the temporal low-pass filtering of the loudspeaker driving signals \cite{Firtha2018_daga_a}.
\end{thesis}
\end{thesisgroup}

\begin{thesisgroup}[Wave Field Synthesis of moving point sources]
In the aspect of synthesizing dynamic sound scenes, the synthesis of moving sources is of primary importance.
I adapted the introduced WFS framework to the synthesis of sound fields generated by moving point sources.
\begin{thesis}
I adapted the generalized 3D WFS theory to the synthesis of the field of a point source moving along an a-priori known trajectory, and I defined driving function for an arbitrary convex SSD surface.
The solution takes the Doppler-effect inherently into account \cite{Firtha2015:daga, firtha2016wave, doi:10.1121/1.4996126}.\end{thesis}
\begin{thesis}
I derived 2.5D WFS driving functions for a 2D SSD contour in order to synthesize 3D point sources moving along an arbitrary trajectory in the plane of the SSD \cite{doi:10.1121/1.4996126}.
The derivation relies on the adaptation of the SPA to this dynamic scenario, allowing to optimize the amplitude correct synthesis to a convex reference curve. 
I verified that for the special case of a linear SSD and a parallel reference line the presented driving functions coincide with the traditional WFS driving functions with the stationary source position replaced by the source position at the emission time \cite{doi:10.1121/1.4996126}.
\end{thesis}
\begin{thesis}
I gave closed form WFS driving function for sources in uniform motion for which particular case the propagation time delay can be expressed explicitly \cite{firtha2016wave}.
\end{thesis}
\begin{thesis}
I derived frequency domain 2.5D WFS driving function for a linear SSD by applying the SPA directly to the frequency content of a point source under uniform motion \cite{firtha2015sound}.
\end{thesis}
\end{thesisgroup}

\begin{thesisgroup}[Synthesis of moving sources in the wavenumber domain]

I gave analytical expressions for the spatial Fourier transform of a source moving uniformly along an arbitrary directed straight trajectory. 
Since the SDM is not restricted to stationary sound fields therefore the obtained formulation can be used in order to derive explicit driving function for the synthesis of a moving source.
\begin{thesis} 
I gave the SDM driving functions for the synthesis of a source under uniform motion in the wavenumber domain.
For the special case of a source moving parallel to the secondary source distribution I derived analytical, closed form driving function in the spatial-frequency domain \cite{Firtha2014:daga, Firtha2014:isma, firtha2015sound}.
I showed that similarly to the stationary case, the WFS solution is the high-frequency/farfield approximation of the presented explicit driving function for moving sources \cite{firtha2015sound}. 
\end{thesis}
\begin{thesis} 
I presented an analytical investigation of the spatial aliasing artifacts emerging from the discretization of the SSD based on the wavenumber description.
I connected the phenomena of frequency distortion with the poles in the secondary sources wavenumber representation and I analytically expressed the aliasing frequency components.
I showed that the artifact can be avoided by applying an SSD that does not exhibit poles on the receiver curve, satisfied optimally by a circular SSD \cite{firtha2016:daga}.
\end{thesis}
\begin{thesis} 
I extended the spatial anti-aliasing criterion in order to include the synthesis of dynamic sound fields.
By using the introduced formulation spatial aliasing may be eliminated by simple low-pass filtering of the loudspeaker driving signals \cite{Firtha2018_daga_moving_source}.
\end{thesis}
\end{thesisgroup}
