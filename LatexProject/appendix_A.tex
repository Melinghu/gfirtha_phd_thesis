\section{Definition and properties of the Fourier transform and the Dirac delta}
\label{App:Fourier_def}

\paragraph{Temporal Fourier transform:} 
Given a four dimensional function $f(\vx,t)$, depending on both time and the spatial position.
The forward and inverse temporal Fourier transform, leading to the angular frequency domain, is defined as 
\begin{equation}
\label{eq:temporal_fourier_transform_def}
F(\vx,\omega) = \FT[t]{f(\vx,t)} = \int_{-\infty}^{\infty} f(\vx,t) \, \te^{-\ti \omega t} \td t,
\end{equation}
\begin{equation}
\label{eq:temporal_inverse_fourier_transform_def}
f(\vx,t) = \IFT[\omega]{F(\vx,\omega)} = \frac{1}{2\pi} \int_{-\infty}^{\infty} F(\vx,\omega) \, \te^{ \ti \omega t} \, \td \omega.
\end{equation}
Note that capital letter indicates that the function is taken in the angular frequency domain, and the subscript of $\mathcal{F}$ indicates the transform variable.
%
\paragraph{Spatial Fourier transforms:} 
Following the convention, given in e.g. \cite{Ahrens2012} the spatial Fourier transform, leading to the wavenumber domain, is defined as follows:
\begin{itemize}
\item In one dimension:
\begin{equation}
\label{eq:spatial_fourier_transform_def}
\tilde{F}(k_x,y,z,\omega) = \FT[x]{F(\vx,\omega)} = \int_{-\infty}^{\infty} F(\vx,\omega) \, \te^{\ti k_x x} \, \td x,
\end{equation}
\begin{equation}
\label{eq:spatial_inverse_fourier_transform_def}
F(\vx,\omega) = \IFT[k_x]{\tilde{F}(k_x,y,z,\omega)} = \frac{1}{2\pi} \int_{-\infty}^{\infty} \tilde{F}(k_x,y,z,\omega) \, \te^{ -\ti k_x x} \, \td k_x.
\end{equation}
\item In two dimensions:
\begin{equation}
\label{eq:spatial_fourier_transform_def_2d}
\tilde{F}(k_x,y,k_z,\omega) = \iint_{-\infty}^{\infty} F(\vx,\omega) \te^{\ti \left( k_x x + k_z z \right) } \, \td x \, \td z,
\end{equation}
\begin{equation}
\label{eq:spatial_inverse_fourier_transform_def_2d}
F(\vx,\omega) = \frac{1}{\left( 2\pi \right)^2} \iint_ {-\infty}^{\infty} \tilde{F}(k_x,y,k_z,\omega) \, \te^{ -\ti \left( k_x x + k_z z \right)} \, \td k_x \, \td k_z.
\end{equation}
\item In three dimensions:
\begin{equation}
\label{eq:spatial_fourier_transform_def_3d}
\tilde{F}(\vk,\omega)= \iiint_ {-\infty}^{\infty} F(\vx,\omega) \, \te^{ \ti \left< \vk \cdot \vx \right>}\,  \td x \,\td y\,\td z,
\end{equation}
\begin{equation}
\label{eq:spatial_inverse_fourier_transform_def_3d}
F(\vx,\omega) = \frac{1}{\left( 2\pi \right)^3} \iiint_{-\infty}^{\infty} \tilde{F}(\vk,\omega) \, \te^{ -\ti \left< \vk \cdot \vx \right>} \, \td k_x \, \td k_y \, \td k_z.
\end{equation}
\end{itemize}
Hence, tilde over the function symbol indicates that the function is taken in the wavenumber domain.
Note that the exponent of the spatial Fourier transform is taken with a reversed sign, compared to the temporal transform.
Writing an arbitrary function in the form of a spatio-temporal inverse Fourier transform
\begin{equation}
f(\vx,t) = \frac{1}{\left( 2\pi \right)^4} \iiiint_{-\infty}^{\infty} \tilde{F}(\vk,\omega) \, \te^{ \ti \left( \omega t - \left< \vk \cdot \vx \right> \right)} \, \td k_x \, \td k_y \, \td k_z \, \td \omega
\end{equation}
describes the expansion of an arbitrary function into the linear combination of harmonic plane waves, propagating into direction $\vk$.
The reversed sign therefore allows this simple physical interpretation of the Fourier transform.

\paragraph{Fourier transform properties:}
Several important properties of the Fourier transform, applied frequently in the present thesis are the following:
\begin{itemize}
\item Shift theorem:
\begin{equation}
\int_{-\infty}^{\infty} f(x-x_0) \, \te^{\ti k_x x} \, \td x = \FT[x]{ f(x-x_0) } = \tilde{F}(k_x) \, \te^{\ti k_x x_0}.
\end{equation}
In case of temporal Fourier transform the right side is with reversed exponent.
\item Convolution theorem:
\begin{equation}
\iint_{-\infty}^{\infty} f(x-x_0) \, g(x_0) \, \td x_0 \, \te^{\ti k_x x} \, \td x = \FT[x]{ f(x) \ast_x g(x) } = \tilde{F}(k_x) \cdot \tilde{G}(k_x).
\end{equation}
\item Differentiation property:
\begin{equation}
\int_{-\infty}^{\infty} \frac{\partial}{\partial x} f(x) \, \te^{\ti k_x x} \, \td x = \FT[x]{ \frac{\partial}{\partial x} f(x) } = 
-\ti k_x \, \tilde{F}(k_x).
\end{equation}
In case of temporal Fourier transform the right side is with reversed sign.
\item Scaling property:
\begin{equation}
\int_{-\infty}^{\infty} f(a x) \, \te^{\ti k_x x} \, \td x = \FT[x]{ f( a x ) } = 
\frac{1}{|a|} \, \tilde{F}(\frac{k_x}{a}).
\end{equation}
In case of temporal Fourier transform the right side is with reversed sign.
\end{itemize}

\paragraph{Properties of Dirac delta:}
The Dirac delta is a generalized function (or distribution), used frequently in order to model acoustic phenomena, defined as
\begin{equation}
\delta(x) = 
\begin{cases}
\hspace{1mm} \infty, & \hspace{1mm} x = 0\\
\hspace{1mm} 0, & \hspace{1mm}  x \neq 0
\end{cases},
\hspace{1cm}
\text{with}
\hspace{2cm}
\int_{-\infty}^{\infty} \, \delta(x) \, \td x = 1.
\end{equation}
Several important properties of the Dirac delta, exploited in the present thesis are the following:
\begin{itemize}
\item Forward Fourier transform:
\begin{equation}
\FT[x]{\delta(x-x_0)} = \int_{-\infty}^{\infty} \delta(x-x_0) \, \te^{\ti k_x x} \, \td x =   \te^{\ti k_x x_0}.
\end{equation}
\item Inverse Fourier transform:
\begin{equation}
\delta(x-x_0) = \frac{1}{2\pi} \int_{-\infty}^{\infty} \te^{-\ti k_x (x-x_0)} \, \td k_x =  \IFT[k_x]{ \te^{\ti k_x x_0} }.
\end{equation}
\item Sifting property:
\begin{equation}
\int_{-\infty}^{\infty} \delta(x-x_0) \, f(x) \, \td x = f(x_0).
\end{equation}
\item Generalized sifting property:
\begin{equation}
\int_{-\infty}^{\infty} f(x) \, \delta(g(x)) \, \td x = \sum_{i} \frac{f(x_i)}{\left| \frac{\partial}{\partial x} g(x) \right|_{x = x_i}}, \hspace{5mm} \text{where} \hspace{5mm} g(x_i) = 0.
\end{equation} 
\end{itemize}