\section{Functions and distributions}
\section{Definition and properties of Fourier-transforms}

The temporal Fourier transform (or frequency content) of a function $f(t)$ is defined in the same manner as in the related literature e.g. \cite{Ahrens2012, Ahrens2010a}:
%
\begin{align}
\mathcal{F}_{t} \left\{  f(t) \right\}  = F(\omega) = \int_{-\infty}^{\infty} f(t)  \te^{-\ti \omega  t}  \td t, \\
\mathcal{F}^{-1}_{t} \left\{  F(\omega) \right\}  = \frac{1}{2\pi} \int_{-\infty}^{\infty} F(\omega)\, \te^{\ti \omega  t} \, \td \omega.
\label{Eq:Math:Temp_Fourier}
\end{align}
%
Similarly, the spatial Fourier transform (or wavenumber content) is defined as
%
\begin{align}
\mathcal{F}_{x} \left\{  f(x) \right\}  = \tilde{f}(k_x) = \int_{-\infty}^{\infty} f(x) \te^{\ti k_x x}\td x, \\
\mathcal{F}^{-1}_{x} \left\{  \tilde{f}(k_x) \right\}  = \frac{1}{2\pi} \int_{-\infty}^{\infty} \tilde{f}(k_x)\, \te^{-\ti k_x x} \td k_x.
\end{align}

Several important properties of Fourier-transform, used frequently trough the thesis are:
\begin{itemize}
\item Shift theorem
%
\item Convolution theorem
%
\item Differentiation theorem
\begin{equation}
\mathcal{F}_{t} \left\{ \frac{\partial}{\partial t} f(t) \right\}  = \ti \omega F(\omega).
\label{Eq:Math:Fourier_tr_diff}
\end{equation}
%
\item Similarity theorem
\begin{equation}
\mathcal{F}_{t} \left\{ f(a t) \right\}  = \frac{1}{|a|} F(\frac{\omega}{a}).
\label{Eq:Math:Fourier_tr_similarity}
\end{equation}
\end{itemize}
  

\section{Stochastic signal theory basics}

