\section{Asymptotic approximation of the explicit driving function}
\label{App:SDM_SPA}

\paragraph{Asymptotic spectral driving function:}
The derivation starts from the 2.5D explicit driving function in the wavenumber domain given by \eqref{Eq:SFS_Theory:LinearSDM_spectral},
\begin{equation}
\tilde{D}(k_x,y,0,\omega) = \frac{\tilde{P}(k_x,y,0, \omega)}{\tilde{G}(k_x,y,0, \omega)},
\end{equation}
ensuring perfect synthesis along fixed line.
In the following, for the sake of brevity and transparency $\omega$ and $z$ dependencies are suppressed, the latter since the driving function is defined at $z=0$.
By definition, the wavenumber content of the involved quantities are obtained via a forward Fourier transform, with the involved sound fields expressed by their polar form reading as
\begin{align} 
\tilde{P}(k_x,y) = \int\limits_{-\infty}^{\infty} A^P(x,y) \, \te^{\ti \phi^P(x,y)} \, \te^{\ti k_x x} \td x, \\
\tilde{G}(k_x,y) = \int\limits_{-\infty}^{\infty} A^G(x,y) \, \te^{\ti \phi^G(x,y)} \, \te^{\ti k_x x} \td x.
\end{align}
The spectra can be approximated by using the SPA:
Under high frequency assumptions the Fourier integrals may be approximated by evaluation around their stationary point $x^*_P(k_x)$ and $x^*_G(k_x)$ where their phase derivatives vanish. 
The stationary positions are defined as
\begin{align}	
\label{eq:xP_xG_in_spatial_domain}
\left. \frac{\partial}{\partial x} \left(  \phi^P(x,y) + k_x x \right )\right|_{x = x^*_P(k_x)} = 0
\hspace{3mm} &\rightarrow \hspace{3mm}
k_x^P(x^*_P(k_x),y) = k_x \nonumber
\\ 
\left. \frac{\partial}{\partial x} \left( \phi^G(x,y) + k_x x \right )\right|_{x = x^*_G(k_x)} = 0
\hspace{3mm} &\rightarrow \hspace{3mm}
k_x^G(x^*_G(k_x),y) = k_x.
\end{align}
Since it is assumed that $k_z^P(x,y) = k_z^G(x,y) = k_z = 0$ holds in the plane of investigation, therefore, the stationary positions are found where the local propagation direction of the virtual field and the Green's function matches to that of a plane wave, defined by $k_x$.
The properties of the involved wavefields at these positions will dominate the corresponding Fourier integrals.
Hence, the forward transform defines two particular positions in the space, linked together via the common spectral wavenumber.

Having defined the stationary positions, the forward Fourier transforms can be evaluated by the SPA. 
With accounting for the negative second phase-derivatives---since both the virtual sound field and the Green's function are diverging---their spectra can be approximated as \cite[Ch. 5]{Tracy2014}
\begin{align}
\tilde{P}(k_x,y) \approx& \sqrt{\frac{2\pi}{\ti \, |\phiPxx(x^*_P(k_x),y)|}} A^P(x^*_P(k_x),y) \, \te^{\ti \phi^P(x^*_P(k_x),y)} \, \te^{\ti k_x \cdot x^*_P(k_x)},\\
\tilde{G}(k_x,y) \approx& \sqrt{\frac{2\pi}{\ti \, |\phiGxx(x^*_G(k_x),y)|}} A^G(x^*_G(k_x),y) \, \te^{\ti \phi^G(x^*_G(k_x),y)} \, \te^{\ti k_x \cdot x^*_G(k_x)},
\end{align}
and the asymptotic approximation of the explicit driving function on a given spectral component reads as
\importanteq{Asymptotic spectral driving function}{
\label{eq:hfapproxspectra}
\tilde{D}(k_x,y)
\approx  
\sqrt{\frac{\phiGxx(x^*_G(k_x),y)}{\phiPxx(x^*_P(k_x),y)}}
\, \frac{P(x^*_P(k_x),y)}{G(x^*_G(k_x),y)}
\, \te^{\ti k_x  \left( x^*_P(k_x) - x^*_G(k_x)\right)}.
}
Hence, the spectrum can be be expressed by evaluating the target pressure and the Green's function at evaluation points where the local propagation direction of the involved fields coincides with that of the actual spectral plane wave.


\paragraph{SPA of the inverse Fourier transform:}
In order to express the driving function in the spatial domain the inverse Fourier transform of the asymptotic spectrum \eqref{eq:hfapproxspectra}, reading
%
\begin{equation}
\label{Eq:inverse_transform_def}
D(x_0,y)
=  \frac{1}{2\pi} \int\limits_{-\infty}^{\infty}
\sqrt{\frac{\phiGxx(x^*_G(k_x),y)}{\phiPxx(x^*_P(k_x),y)}} 
\frac{P(x^*_P(k_x),y)}{G(x^*_G(k_x),y)}
\,\te^{\ti k_x \left( x^*_P(k_x) - x^*_G(k_x)\right)}
\,\te^{-\ti k_x x_0} \td k_x
\end{equation}
is approximated by the stationary phase method, with the phase function under investigation given by
\begin{equation}
\label{Eq:inverse_transform_phase_function}
{\Phi}(k_x) = \phi^P(x^*_P(k_x),y) -  \phi^G(x^*_G(k_x),y) +  k_x \, x^*_P(k_x) - k_x\, x^*_G(k_x) -  k_x x_0.
\end{equation}
%
As it was discussed in Section \ref{Sec:SPA_for_Fourier}, in the spatial inverse Fourier transform of an arbitrary wavefield spectrum $\tilde{P}$ each wavenumber component $k_x$ will dominate one spatial position $x_0$, where the actual wavenumber component $k_x(x_0)$ coincides with the local wavenumber of the sound field $k_x^P(x_0)$.
For the present case this wavenumber is found as the stationary phase wavenumber $k_x^*(x_0)$ of the integral \eqref{Eq:inverse_transform_def} \cite{Tracy2014}.

The derivative of the spectral phase function \eqref{Eq:inverse_transform_phase_function} can be evaluated by applying the chain rule, resulting in
\begin{multline}
\label{eq:spectral_phase_first_derivative}
\frac{\partial}{\partial k_x}\Phi(k_x) =  
  x^{*'}_{P,k_x}(k_x) \underbrace{ \left( \phi^{P'}_x(x^*_P(k_x),y)  + k_x \right)}_{ = 0} - \\	
- x^{*'}_{G,k_x}(k_x) \underbrace{ \left( \phi^{G'}_x(x^*_G(k_x),y)  + k_x \right) }_{ = 0} 
+x^*_P(k_x)- x^*_G(k_x) -  x_0,
\end{multline}
where $x^{*'}_{k_x}(k_x)$ is the rate of change of the forward transform stationary positions with respect to the change of the spectral wavenumber.
The bracketed terms cancel out due to the definition of the stationary points for the forward transform \eqref{eq:xP_xG_in_spatial_domain}.
The stationary wavenumber $k_x^*(x_0)$ is then found where
\begin{equation}
\label{eq:xP_xG_in_spectral_domain}
\left. \frac{\partial}{\partial k_x}\Phi(k_x) \right|_{k_x=k_x^*(x_0)} = x^*_P(k_x^*(x_0))- x^*_G(k_x^*(x_0)) -  x_0 = 0
\end{equation}
holds.
This definition relates the evaluation points $x^*_P$ and $x^*_G$ directly to the actual SSD coordinate $x_0$, therefore, the dependency on the intermediate stationary wavenumber $k_x^*$ may be omitted (i.e., $x^*_P(k_x^*(x_0)) \rightarrow x^*_P(x_0)$ and $x^*_G(k_x^*(x_0)) \rightarrow x^*_G(x_0)$ may be written). 

\begin{figure}[t!]
\small
  \begin{minipage}[c]{0.54\textwidth}
%  \hspace{1cm}
	\small
%	\centering
%	\hspace{-30mm}
	\begin{overpic}[width = \textwidth ]{Figures/Appendices/explicit_sol_stationary_point.png}
	\put(96,30){$x$}
	\put(15,80){$y$}
	\put(60,29.5){$x_0$}
	\put(51,72){\parbox{6.5em}{$\vk^P(x_P^*,y) = \vk^G(x_P^*-x_0,y)$}}
	\put(75,29.5){$x_P^*(x_0)$}
	\end{overpic}  \end{minipage}\hfill
	\begin{minipage}[c]{0.4\textwidth}
    \caption{
       Illustration of the evaluation position $x_P^*(x_0)$ (and $x_G^*(x_0)$) as the function of $x_0$. 
	   For a given SSD position $x_0$ the stationary position is found on a given reference line $y = \text{const}$ where the virtual field propagation direction coincides with that of the Green's function, translated into $x_0$.
	   Furthermore, at $x_P^*(x_0)$ the local principle radii of the Green's function positioned at $y=0$, is always smaller, than that of a field, generated by a source distribution at $y<0$, i.e. that of the virtual field.} 
       \label{fig:SFS_theroy:explicit_sol_stationary_points}
  \end{minipage}
\end{figure}

The definitions for the forward and inverse transform stationary points completely define the evaluation points $x^*_P$, $x^*_G$ for a given SSD position $x_0$ independently of the spectral wavenumber:
combining \eqref{eq:xP_xG_in_spatial_domain} with \eqref{eq:xP_xG_in_spectral_domain}, for an arbitrary position $x_0$ the evaluation points along a fixed $y$ are found, where
%
\importanteq{2.5D SDM evaluation position}{
\label{Eq:stationary_evaluation_points_app}
k_{x}^P(x^*_P(x_0),y) = k_{x}^G(x^*_P(x_0) - x_0,y)
}
is satisfied.

This result states that for a given SSD coordinate $x_0$ the evaluation point $x^*_P$ is found on the reference line where the local propagation direction of the target field $P$ coincides with that of a point source positioned at $\posvec{3}{x_0}{0}{0}$. 
This principle is depicted in Figure \ref{fig:SFS_theroy:explicit_sol_stationary_points}.
%

Having found the stationary position for \eqref{Eq:inverse_transform_def} one still needs the phase function's second derivative and its sign around the stationary position in order to apply the SPA.
The second derivative is obtained by a further differentiation of \eqref{eq:spectral_phase_first_derivative} with respect to $k_x$
\begin{multline}
\frac{\partial^2}{\partial k_x^2}\Phi(k_x) = \\
  x^{*''}_{P,k_x k_x}(k_x) \cdot \underbrace{ \left( \phi^{P'}_x(x^*_P(k_x),y)  + k_x \right)}_{ = 0} + 
  x^{*'}_{P,k_x}(k_x) \cdot  \left( x^{*'}_{P,k_x}(k_x) \, \phi^{P''}_{xx}(x^*_P(k_x),y)  +2  \right)  -\\
  x^{*''}_{G,k_x k_x}(k_x) \cdot  \underbrace{ \left( \phi^{G'}_x(x^*_G(k_x),y)  + k_x \right) }_{ = 0} 
- x^{*'}_{G,k_x}(k_x) \cdot \left( x^{*'}_{G,k_x}(k_x) \,  \phi^{G''}_{xx}(x^*_G(k_x),y)  + 2\right).
\end{multline}
The required rate of change of the stationary positions $x^{*'}_{P,k_x}(k_x)$ and $x^{*'}_{G,k_x}(k_x)$ can be obtained by differentiating their definition \eqref{eq:xP_xG_in_spatial_domain} with respect to $k_x$, yielding
\begin{equation}
x^{*'}_{P,k_x}(k_x) = -\frac{1}{\phi^{P''}_{xx}(x^*_P(k_x),y)}, \hspace{5mm} x^{*'}_{G,k_x}(k_x) = -\frac{1}{\phi^{G''}_{xx}(x^*_G(k_x),y)}.
\end{equation}
Thus the required second derivative is given by
\begin{equation}
\label{eq:SFS_theory:second_Derivative_2}
\small
\frac{\partial^2}{\partial k_x^2}\Phi(k_x) = 
\frac{\phi^{P''}_{xx}(x^*_P(k_x),y) - \phi^{G''}_{xx}(x^*_G(k_x),y)}{\phi^{P''}_{xx}(x^*_P(k_x),y) \cdot \phi^{G''}_{xx}(x^*_G(k_x),y)} =
\frac{\Rh^P(x^*_P(k_x),y) - \Rh^G(x^*_G(k_x),y) }{k \, \hat{k}^{P 2}_y(x^*_P(k_x),y)}
,
\end{equation}
with $\Rh^P$ and $\Rh^G$ being the horizontal principal radii of the target field and the Green's function, expressed by applying \eqref{Eq:App:Hessian_inplane}.
Obviously, for any source distribution behind the SSD its horizontal principal radius is larger at $y>0$ than that of the stationary secondary point source and the sign of \eqref{eq:SFS_theory:second_Derivative_2} is positive. 

These results now may be substituted back into the SPA \eqref{Eq:SPAResult} of the inverse transform \eqref{Eq:inverse_transform_def}.
For the sake of brevity in the following the evaluation point is denoted by $x^*_P \rightarrow x^*$. 
Denoting the stationary position by $\vxref(\vxo) = \posvec{3}{x^*(x_0)}{y}{0}$ the resulting driving function is formulated as
\begin{equation}
D(x_0,\omega) \approx 
\sqrt{\frac{ \left| \phiGxx(\vxref(\vxo)-\vxo,\omega )\right|^2}{\left| \phiPxx(\vxref(\vxo),\omega) - \phiGxx(\vxref(\vxo)-\vxo,\omega)\right|}}
\sqrt{\frac{\ti}{2\pi}} 
\frac{P(\vxref(\vxo),\omega)}{G(\vxref(\vxo)-\vxo,\omega)},
\end{equation}
where $k_x^P(\vxref(\vxo)) = k_x^G(\vxref(\vxo)-\vxo)$ holds.
