This dissertation concludes the work of 9 years which I spent in the Laboratory of Acoustics and Studio Technologies starting with my MSc studies in 2010.
To being from that point, I'm thankful for my BSc supervisor Krisztián Gulyás to draw my attention to the technology called Wave Field Synthesis, and to advise me to seek the help of a co-supervisor who could better help me out with all the underlying math:

My greatest gratitude goes to my MSc and PhD supervisor, Péter Fiala for supporting me with his friendship and strong mathematical and theoretical skills throughout the years.
His contribution was indispensable for the birth and the quality of the present writing.\footnote{as well as in co-founding the first hungarian Winnie the Pooh club.}
Also, my thanks goes to all my colleagues for welcoming me in the Laboratory of Acoustics and for the years we spent together: Péter Rucz, Attila Balázs Nagy, Fülöp Augusztinovicz, Ferenc Márki, Dóra Jenei-Kiss, all of whose company and support I greatly appreciated throughout the years.

Over the years I had the luck to connect my work with fellow researchers.
I am very grateful to meet and work together with Fiete Winter, Nara Hahn, Franz Zotter, Jens Ahrens and Sascha Spors, whose work was a starting point for my research.
I am especially thankful for Frank Schultz for the friendly discussions, all the collaboration and for his careful proof-reading of the present dissertation.

Also, from the last years I am grateful for Csaba Huszty and his team for the co-operation in which I could employ my knowledge from the field of sound field reproduction.

Last, but certainly not least, I would like to thank my family---my parents, my brother and my sisters---all my friends and my girlfriend to not only accept but also to support me throughout the years I spent with my research.