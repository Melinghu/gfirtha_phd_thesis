This dissertation concludes the work of 9 years which I spent in the Laboratory of Acoustics and Studio Technologies starting with my MSc studies in 2010.
I'm thankful for my BSc supervisor Krisztián Gulyás who drew my attention to an interesting technology called Wave Field Synthesis, and who advised me to seek the help of Péter who could better help me with the technique's theoretical basics.

My greatest gratitude goes to my MSc and PhD supervisor Péter Fiala.
His contribution, his strong mathematical and theoretical skills, his help regarding all the underlying math and his friendship were indispensable for the birth of the presented results and for the quality of the present writing, as well as for the co-foundation of the first hungarian „Winnie the Pooh” club.

Also, I thank to all my colleagues for welcoming me in the Laboratory of Acoustics and for the years we spent together: Péter Rucz, Attila Balázs Nagy, Fülöp Augusztinovicz, Ferenc Márki, Dóra Jenei-Kulcsár and Tamás Mócsai, all of whose company and support I greatly appreciated throughout the years.

Over the years I had the luck to connect my work with fellow researchers.
I am very grateful to meet and work together with Fiete Winter, Nara Hahn, Franz Zotter, Jens Ahrens and Sascha Spors, whose work was a starting point for my research.
I am especially thankful for Frank Schultz for the friendly and fruitful discussions, all the collaboration and for his careful proofreading of the present dissertation.

Also, from the last years I am grateful for Csaba Huszty and his team for the cooperation in which I could employ my knowledge from the field of sound field reproduction.

Last, but certainly not least, I would like to thank my family, all my friends and my girlfriend for not only accepting but also for supporting me throughout the years I spent with my research.