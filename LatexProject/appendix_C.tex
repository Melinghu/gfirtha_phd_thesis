\section{Notes on the Hessian of the phase function}
\label{App:Hessian}

\subsection{Definition of principal curvatures and principal directions}
Assume a wave field, described by the general polar form $P(\vx,\omega) = A^P(\vx,\omega)\te^{\ti \phi^P (\vx,\omega)}$!
Supposing, that the amplitude changes slowly, compared to the phase function the local dispersion relation $| \nabla \phi(\vx,\omega) |= k$ holds and the equation describing an arbitrary wavefront i.e. $\phi^P(\vx,\omega) - C = 0$ is the \emph{normalform} of the given surface \cite{Hartmann1999}.
The Hessian matrix of the function is given by the symmetric matrix 
\begin{equation}
H = - \begin{bmatrix} 
\phi^{P''}_{xx}(\vx,\omega) & \phi^{P''}_{xy}(\vx,\omega) & \phi^{P''}_{xz}(\vx,\omega) \\[.7em]
\phi^{P''}_{xy}(\vx,\omega) & \phi^{P''}_{yy}(\vx,\omega) & \phi^{P''}_{yz}(\vx,\omega) \\[.7em]
\phi^{P''}_{xz}(\vx,\omega) & \phi^{P''}_{yz}(\vx,\omega) & \phi^{P''}_{zz}(\vx,\omega) \\[0.5em]    \end{bmatrix},
\end{equation}
with the eigenvalues $\lambda_1, \lambda_2, \lambda_3$ and the corresponding eigenvectors $\mathbf{v}_1, \mathbf{v}_2, \mathbf{v}_3$.
Since the function under consideration is a normalform, therefore the following properties hold
\begin{itemize}
\item $\lambda_3 = 0$ with the eigenvector $\mathbf{v}_3 = - \nabla \phi^P(\vx,\omega) /k = \hat{\vk}^P(\vx)$, i.e. being the normal of the wavefront
\item $\lambda_1 = \kappa^P_1(\vx)$, $\lambda_2 = \kappa^P_2(\vx)$ being the \emph{main or principal curvatures} of the wavefront and $\rho^P_1(\vx) = 1/\kappa^P_1(\vx)$, $\rho^P_2(\vx) = 1/\kappa^P_2(\vx)$ being the \emph{principal radii}.
The principal curvatures and radii are defined as the following:
Consider all the planes containing the normal of the surface at the point of investigation, defined by the surface normal and vector $\mathbf{v}$, being the tangent of the surface.
The curvature is defined as the quadratic form 
\begin{equation}
\kappa = \mathbf{v} H \mathbf{v}^{\mathrm{T}}.
\label{Eq:App:curvature_def}
\end{equation}
The main curvatures are then defined as the minimum and maximum values of curvature, i.e. the reciprocal of the osculating circles radii (the principal radii).
The corresponding eigenvectors $\mathbf{v}_2$ and $\mathbf{v}_3$ are the tangential, orthogonal unit vectors pointing into the direction of the maximal and minimal curvatures.
\end{itemize}

Note, that for the special case of the three-dimensional Green's function positioned at $\vxs$, only eigenvector $\lambda_3$ is well-defined over the spherical /\emph{umbilical} wavefront, being a radial directed normal vector.
$\lambda_2$ and $\lambda_3$ may be arbitrary orthogonal vectors in the plane, tangent to the surface in the point of investigation $\vxo$, with the corresponding principal curvatures being $\kappa_1 = \kappa_2 = 1/|\vxo - \vxs| $.


In the present treatise when dealing with 2.5D problems it is a standard prerequisition, that in the plane of investigation ($z = 0$) all the involved wave fields propagate along the horizontal direction ($k_z(x,y,0) \equiv 0$), thus the Hessian of the phase function becomes
\begin{equation}
H = - \begin{bmatrix} 
\phi^{P''}_{xx}(\vx,\omega) & \phi^{P''}_{xy}(\vx,\omega) & 0 \\[.7em]
\phi^{P''}_{xy}(\vx,\omega) & \phi^{P''}_{yy}(\vx,\omega) & 0 \\[.7em]
0 & 0 & \phi^{P''}_{zz}(\vx,\omega) \\[0.5em]    \end{bmatrix},
\end{equation}
with the trivial eigenvector/principal direction $\mathbf{v_2} = \posvec{3}{0}{0}{1}$ and the corresponding principal curvature $\kappa_2^P(\vx) =  \phi^{P''}_{zz}(\vx,\omega)$.
Furthermore, considering, that the eigenvector with a zero eigenvalue is given by $\mathbf{v}_3 = \hat{\vk}^P(\vx) = \posvec{3}{\hat{k}_x^P(\vx)}{\hat{k}_y^P(\vx)}{0}$ and $\mathbf{v}_1$ is orthogonal to  $\mathbf{v}_2$ and $\mathbf{v}_3$, therefore $\mathbf{v}_1 = \posvec{3}{\hat{k}_y^P(\vx)}{\hat{k}_x^P(\vx)}{0}$ holds.
From simple geometrical considerations and applying \eqref{Eq:App:curvature_def} the elements of the Hessian matrix can be expressed as
\begin{equation}
H = - \begin{bmatrix} 
\hat{k}_y^{P}(\vx)^2 \kappa_1^P(\vx) & \hat{k}_x^{P}(\vx)\hat{k}_y^{P}(\vx)\kappa_1^P(\vx) & 0 \\[.7em]
\hat{k}_x^{P}(\vx)\hat{k}_y^{P}(\vx) \kappa_1^P(\vx) & \hat{k}_x^{P}(\vx)^2\kappa_1^P(\vx) & 0 \\[.7em]
0 & 0 & \kappa_2^P(\vx) \\[0.5em]    \end{bmatrix}.
\end{equation}

\vspace{3mm}
In the aspect of the present treatise the signature and the determinant of the Hessian in the stationary position is of importance.
In the followings these properties will be discussed when the SPA is applied for the Rayleigh integral.


\subsection{Hessian of the SPA applied for the Rayleigh-integral}

Assume the Rayleigh-integral, written onto the plane $y = 0$, with the high-frequency gradient approximation applied:
\begin{equation}
P(\vx,\omega) = 2 \int_{-\infty}^{\infty} \ti k_y^P(\vxo) P(\vxo, \omega) G(\vx-\vxo,\omega) \td x_0  \td z_0.
\end{equation}
The phase function under consideration and hence the Hessian of the phase is given by the sum
\begin{equation}
\phi(\vx,\omega) = \phi^P(\vxo,\omega) + \phi^G(\vx-\vxo,\omega), \hspace{1cm} H = H^P+ H^G.
\end{equation}
The Hessian is investigated in the stationary position.
The stationary position is found where the normals of the involved wavefronts coincide on the Rayleigh plane, i.e. where the wavefront of $P$ is tangential with the spherical wavefront of the Green's function.
Furthermore, since the principal directions of the Green's function's wavefront are not well-defined, they can be chosen to coincide with the principal directions of $P$.
Therefore at the stationary point the eigenvectors of $H^P$ and $H^G$ coincide and their eigenvalues are additive.
The eigenvalues of the resultant matrix are therefore simply given as
\begin{align}
\lambda_1 &= \lambda_1^P + \lambda_1^G = \kappa_1^P(\vxo) + \kappa_1^G(\vx-\vxo), \\
\lambda_2 &= \lambda_2^P + \lambda_2^G = \kappa_2^P(\vxo) + \kappa_2^G(\vx-\vxo). \\
\lambda_3 &= \lambda_3^P + \lambda_3^G = 0.
\end{align}

\vspace{3mm}
In the case of approximating the integral with respect to both $x$ and $z$ direction the Hessian of the SPA may be formed from the 3x3 Hessian by removing the rows and columns containing the $y$ derivatives, hence by forming its 2x2 principal submatrix.
The eigenvalues of this submatrix cannot be expressed in a general way, however the signature of the Hessian can be determined, by applying the \emph{interlacing inequalities of principal submatrices}:
In case of a hermitian matrix with the real eigenvalues in an increasing order, $\lambda_1, \lambda_2, ..., \lambda_n$, than for the $k-$th eigenvalue $\lambda'_k$ of its principal submatrix
\begin{equation}
\lambda_k \leq \lambda'_k \leq \lambda_{k+1}
\end{equation}
holds.
This inequality ensures that the eigenvalues of the 2x2 Hessian of the SPA has the same sign as $\lambda_2$ and $\lambda_3$, and the signature of the Hessian is 
\begin{itemize}
\item assuming a divergent sound field, the eigenvalues of the Hessian are positive and the signature is given by 2.
\item assuming a convergent sound field with both principal curvature being negative \emph{on the Rayleigh plane}, the signature of the Hessian depends on the evaluation position $\vx$.
On the parts of the space where the curvature of wavefront $P$ is greater in magnitude than that of the Green's function the eigenvalues of the Hessian are negative and it signature is -2.
On other parts of the space the signature is 2.
\end{itemize}
In practice with a simple example it means, that if the Rayleigh integral describes a sound field propagating toward a point, than the signature for an evaluation point between the Rayleigh plane and the focus point is given by -2, and in other parts of the space, where the waves already diverge after passing the focus point , the signature is 2.

In the special case of the derivation of the 2.5D Rayleigh integral only the integration along the $z$-dimension is approximated and the Hessian is simply given by $\phi''_{zz}(\vxo) =\phi^{P''}_{zz}(\vxo) + \phi^{G''}_{zz}(\vx-\vxo)$.
Requiring, that in the horizontal plane of investigation $k_z(\vx) \equiv 0$ guarantees, that the second derivative is the principal curvature itself (see below), thus
\begin{equation}
\phi''_{zz}(\vxo) = \kappa_2^P(\vxo) + \kappa_2^G(\vx-\vxo).
\end{equation}
Obviously for the sign of the second derivative the description given for the 2D SPA case holds.

Note, that the description above is not limited for the Rayleigh integral: if the Kirchhoff-Helmholtz integral is written onto a smooth convex surface with the surface's curvature being significantly smaller than the wavefront curvature, than the surface can be considered locally plane, and the above given description holds.
