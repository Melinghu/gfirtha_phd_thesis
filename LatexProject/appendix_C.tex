\section{Wavenumber vector of a point source pair}
\label{App:stereophony}

Given a point source pair, positioned symmetrically to the $y$-axis $\vx_1 = \posvec{3}{x_1}{y_1}{0}$, $\vx_2 = \posvec{3}{-x_1}{y_1}{0}$, the radiated field reads as
\begin{equation}
P(\vx,\omega) = 
\frac{A_1}{4\pi}\frac{\te^{-\ti \frac{\omega}{c}|\vx - \vx_1|} }{|\vx - \vx_1|} + 
\frac{A_2}{4\pi}\frac{\te^{-\ti \frac{\omega}{c}|\vx - \vx_2|} }{|\vx - \vx_2|},
\end{equation}
with the amplitude factors denoted by $A_1$ and $A_2$.
Applying Euler's formula the phase function of the resultant field becomes
\begin{equation}
-\phi^P(\vx,\omega) = \arctan \frac{ 
\frac{A_1}{r_1}\sin \left( k r_1 \right) +  
\frac{A_1}{r_2}\sin \left( k r_2 \right)  }
{
\frac{A_1}{r_1}\cos \left( k r_1 \right) +
\frac{A_2}{r_1}\cos \left( k r_2 \right)  
},
\end{equation}
with $r_1(\vx) = |\vx-\vx_1|$ and $r_2(\vx) = |\vx-\vx_2|$ and $k = \frac{\omega}{c}$.
The arguments of the distance functions are suppressed, for the sake of brevity.

The gradient of the expression, i.e. the local wavenumber vector, can be calculated by using that
\begin{equation}
\left( \arctan \frac{f}{g} \right)' = \frac{f'g - f g'}{f^2+g^2},
\end{equation}
with derivatives given here by
\begin{align}
f' & = A_1 r'_1 \frac{k r_1 \cos \left(k  r_1 \right) - \sin \left(  k r_1 \right) }{r_1^2} + 
A_2 r'_2 \frac{k r_2 \cos \left(k r_2 \right) - \sin \left( k r_2 \right) }{r_2^2}
\\
g' & = A_1 r'_1 \frac{-k r_1 \sin \left( k r_1 \right) - \cos \left(  k r_1 \right) }{r_1^2} + 
A_2 r'_2 \frac{-k r_2 \sin \left(k r_2 \right) - \cos \left( k r_2\right) }{r_2^2},
\end{align}
Note that for sake of transparency, $f'$ denotes the gradient operator, expressing both the $x$- and $y$-derivative of the phase field.
Several simplifications lead to the gradient expression
\begin{multline}
-\Dx \phi^{P}(\vx) = \vk^P(\vx) =
\\
\frac{
k \left(
r_1' A_1^2 r_2^2 + r_2' A_2^2 r_1^2 + A_1 A_2 r_1 r_2 (r_1'+r_2') \cos\left( k \Delta r \right)
\right) 
-A_1 A_2 (r_1' r_2 - r_2' r_1) \sin \left( k \Delta r \right)
 }
 {r_1^2 r_2^2 {A^{P}}(\vx,\omega)^2},
\end{multline}
with $\Delta r = r_1 - r_2$ and $A^P(\vx,\omega)$ denoting the amplitude of the resulting field, with omitting the normalizing factor $1/4\pi$ reading as
\begin{equation}
A^P(\vx,\omega) = \frac{\sqrt{ A_1^2 r_2^2 + A_2^2 r_1^2 + 2 A_1 A_2 r_1 r_2 \cos\left( k \Delta r \right) } } {r_1 r_2}.
\end{equation}

These expressions describe the interference pattern of the point source pair over the entire listening area.
Note that the local wavenumber vector contains the amplitude of the field in its denominator.
Due to destructive interference the amplitude of the field vanishes over particular spatial locations.
At these positions the phase changes rapidly, resulting in the increasing of the local wavenumber vector length.
%
\begin{figure}
\small
  \begin{minipage}[c]{0.68\textwidth}
	\begin{overpic}[width = 1\columnwidth]{Figures/Appendices/stereo_wn.png}
	\end{overpic}   \end{minipage}\hfill	
	\begin{minipage}[c]{0.3\textwidth}
    \caption{The length of the local wavenumber vector of a stereo point source pair, normalized by $k = \frac{\omega}{c}$.
    The point sources are positioned with a base angle of $\phi_0 = 30^\circ$, with their distances from the origin being $R_0 = 2.5~\mathrm{m}$.
The gain factors $A_1, A_2$ were selected, so that the angle of the local wavenumber vector at the origin would equal to $\phi_p = 10^\circ$. }
\label{fig:App:stereo_wn}   \end{minipage}
\end{figure} 
Also, between the two point sources the standing waves are present.
The numerator of the local wavenumber vector expression vanishes at one particular position, where only a standing wave component is present.
Around that position the length of the wavenumber vector decreases to zero.
This phenomena can be investigated in Figure \ref{fig:App:stereo_wn}, depicting the normalized local wavenumber vector length, in a standard stereo setup.

From the aspect of stereophonic applications, only the stereo axis is of interest, where $r_1 = r_2$ holds.
Due to the symmetry of the geometry, here $\frac{\partial}{\partial x} r_1 = - \frac{\partial}{\partial x} r_2$, $\frac{\partial}{\partial y} r_1 = \frac{\partial}{\partial y} r_2$ and $\frac{\partial}{\partial z} r_1 = \frac{\partial}{\partial z} r_2$ hold. 
finally, the derivatives of the phase function are given along $y = 0$ as
\begin{align}
-\phi^{P'}_x(0,y,0,\omega) &= k r_x' \frac{ A_1  - A_2 
 }{ A_1 + A_2} \\
-\phi^{P'}_y(0,y,0,\omega) &= k r_y',
\end{align}
and the amplitude factor reads
\begin{equation}
A^P(0,y,0,\omega) = \frac{A_1 +  A_2}{r}.
\label{Eq:AppB:stereo_amplitude}
\end{equation}