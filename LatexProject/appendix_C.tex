\section{Spectral representation of non-stationary convolutions}
\label{Sec:Non_stat_conv}

Assume the following non-stationary convolution
\begin{equation}
g(t) = \int_{-\infty}^{\infty} h(t_0,t-t_0) f(t_0) \td t_0,
\label{Eq:App:Non_stat_conv_td}
\end{equation}
with $h(t_0,t-t_0)$ denoting the time-variant convolution kernel.
Terms $f(t_0)$ and $h(t_0,t-t_0)$ can be expressed by means of the inverse transform of its spectrum given as
\begin{align}
f(t_0) &= \frac{1}{2\pi} \int_{-\infty}^{\infty} F(\omega_0) \te^{\ti \omega_0 t_0}\td \omega_0
\\
h(t_0,t-t_0) &= \frac{1}{2\pi}\int_{-\infty}^{\infty} H(t_0,\omega_1) \te^{\ti \omega_1(t-t_0)} \td \omega_1,
\end{align}
Applying the above formulations the non-stationary convolution can be cast into the form
\begin{equation}
g(t) = \frac{1}{2\pi^2} \iint_{-\infty}^{\infty}  H(t_0,\omega_1) \te^{\ti \omega_1(t-t_0)} \td \omega_1 f(t_0) \td t_0,
\end{equation}
\begin{equation}
g(t) = \frac{1}{(2\pi)^2} \iiint_{-\infty}^{\infty}  H(t_0,\omega_1) \te^{\ti \omega_1(t-t_0)} \td \omega_1  F(\omega_0) \te^{\ti \omega_0 t_0}\td \omega_0 \td t_0,
\end{equation}
Taking the Fourier transform of the latter expression:
\begin{equation}
G(\omega) =\frac{1}{(2\pi)^2} \iiint_{-\infty}^{\infty} H(t_0,\omega_1) \te^{\ti \omega_1(t-t_0)} \td \omega_1 \int_{-\infty}^{\infty} F(\omega_0) \te^{\ti \omega_0 t_0}\td \omega_0 \td t_0 \te^{-\ti \omega t} \td t,
\end{equation}
Reversing the order of integration and rearrangement results in
\begin{equation}
G(\omega) =\frac{1}{(2\pi)^2} \iiint_{-\infty}^{\infty} H(t_0,\omega_1) F(\omega_0)  \te^{-\ti (\omega_1-\omega_0) t_0}  
\underbrace{ \int_{-\infty}^{\infty}  \te^{-\ti (\omega-\omega_1) t}  \td t}_{2\pi \delta(\omega-\omega_1)}
 \, \td \omega_1 \,  \td \omega_0 \, \td t_0.
\end{equation}
Integration along $\omega_1$ sifts out $\omega_1 = \omega$:
\begin{equation}
G(\omega) =  \iint_{-\infty}^{\infty} H(t_0,\omega) F(\omega_0)  \te^{-\ti (\omega-\omega_0) t_0}   \td \omega_0 \, \td t_0.
\end{equation}
Finally it can be exploited that
\begin{equation}
\tilde{H}(\omega-\omega_0) =  \int_{-\infty}^{\infty}  H(t_0,\omega)  \te^{-\ti (\omega-\omega_0) t_0}  \td t_0
\end{equation}
holds, yielding the final result
\begin{equation}
G(\omega) =  \int_{-\infty}^{\infty} H(\omega - \omega_0,\omega) F(\omega_0)  \td \omega_0.
\label{Eq:App:Non_stat_conv_fd}
\end{equation}
Hence, a non-stationary convolution in the second variable of a time domain function yields a non-stationary convolution in the first variable of its spectral representation.