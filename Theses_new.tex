\documentclass[a4paper,10pt]{article}
%\usepackage{a4wide}
\usepackage{fullpage}
\usepackage{enumerate}
\usepackage{t1enc}
\usepackage[latin2]{inputenc}
\usepackage[T1]{fontenc}
\usepackage{ae,aecompl}
\usepackage[magyar]{babel}
\usepackage{indentfirst}

\frenchspacing 
\pagestyle{empty}
\title{Theses}
\date{}
\begin{document}
\begin{center}
  \textbf{\normalsize Theses for the dissertation \\
   		  \Large Reproduction of moving sound sources applying a unified WFS framework}\\[0.5cm]
\end{center}

\begin{itemize}
\item \emph{Generalization of WFS theory} \\ 
I introduced a unified WFS framework, allowing one to synthesize arbitrary sound fields with arbitrary shaped loudspeaker ensembles (secondary source distribution (SSD)), and to optimize the synthesis on an arbitrary reference curve. 
The unified framework inherently contains the existing WFS approaches as special cases.
\begin{itemize}
\item By defining the local wavenumber vector of a time-harmonic sound field I established the physical interpretation of the stationary phase approximation (SPA):
The SPA ensures wave front matching of the virtual sound field and the wave fronts emerging from the SSD elements at the receiver position
\item Using the local wavenumber vector I derived 3D WFS driving function for an arbitrary shaped SSD curve, emerging from the Physical Optics approximation of the Kirchhoff-Helmholtz integral equation.
\item It had been an already known fact, that---according to the SPA---the synthesized field at an arbitrary position is dominated by its stationary SSD element, that's emitted wavefront matches with the target sound field at the receiver position.
By utilizing this concept and the local wavenumber vector I derived an analytical expression for the \emph{reference curve}. The reference curve connects the points, where ampltide error between the synthesized field and the target field is minimal.
\item I defined the concept of the \emph{referencing function}.
% For 2D virtual sound fields the referencing function gives the distance between the reference curve and the corresponding stationary SSD element.
% For 3D virtual sound fields, the referencing function compensates the \emph{virtual source dimensionality mismatch}.
I verified, that by controlling the referencing function one may reference the synthesis on an arbitrary reference curve.
As a special case I showed that application of the general framework for a linear SSD with a parallel reference line provides the traditional WFS driving functions.
\item Using the framework I proved that 2.5D WFS is the high-frequency approximation of the explicit spectral solution, termed the \emph{Spectral Division Method} for an arbitrary 2D virtual sound field.
\item I showed, how the framework may be used for the analysis of the existing referencing approaches, by providing the reference curve for previous WFS techniques.
\end{itemize}
%
\item \emph{Wave Field Synthesis of moving point sources}\\
In the aspect of synthesizing dynamic sound scenes the synthesis of moving sources is of primary importance.
In order to give the analytically correct driving function I adapted the unified WFS framework to the description of moving point sources.
\begin{itemize}
\item Considering a source, with a known position time history the radiated field can be given analytically, which takes the Doppler-effect inherently into account.
I adapted the unified 3D WFS theory to this dynamic description in order to define driving functions for an arbitrary SSD surface.
\item For the case of a 3D moving source synthesized with a 2D SSD contour the definition of the compensation factor for the virtual source dimensionality mismatch is not straightforward.
I adapted the stationary phase method in order to derive the correct compensation factors, required for the definition of the referencing function. 
\item Applying the obtained referencing function I gave the general 2.5D WFS driving functions for the synthesis of a moving source using arbitrary shaped SSD contour.
\item The solution with an arbitrary source trajectory requires the a-priori determination of the propagation time delay---the time it takes for the wavefront emitted by the moving source to travel from the source position at the emission time to the receiver position---, which leads to the solution of a non-linear equation for each SSD element at each time instant.
In order to avoid the great computational complexity of the approach a finite difference scheme is presented, relying on the first order Taylor's approximation of the propagation time delay function with respect to time.
The proposed solution requires the solution of the non-linear equation only at the time origin.
In later time instant the field of a moving source, and thus the driving functions can be calculated iteratively.
\item I verified, that applying a linear SSD with a reference line the moving source driving functions coincide with the traditional WSF driving functions for a stationary point source, with the stationary source position replaced with the source position at the emission time.
\item For a source, moving uniformly on a straight trajectory the propagation time delay can be expressed explicitly.
I utilized this solution to give explicit WFS driving functions for sources in uniform motion.
\item For a uniformly moving source the frequency content of the radiated field may be expressed analytically.
I adapted the SPA to this formulation in order to derive 2.5D WFS driving functions for a linear SSD in the frequency domain.
\end{itemize}

\item \emph{Synthesis of moving sources in the wavenumber domain\\}
For the case of a linear SSD and a parallel reference line the spatial Fourier-transform of the involved functions (i.e. the target field, the SSD's sound field and the driving function) may be defined.
In the wavenumber domain the synthesis problem can be formulated as a spectral multiplication, and the driving functions are obtained via a spectral division, hence the technique is termed as the \emph{Spectral Division Method (SDM)}.
For the case of virtual sources moving uniformly along a straight trajectory the required transforms can be evaluated analytically, resulting in the explicit, reference solution.
	\begin{itemize}
	\item I gave analytical expression for the wavenumber-frequency content of a sound field, generated by a moving acoustic point source with an arbitrary straight trajectory
	\item I verified, that the derived analytic expression converges weakly to a Dirac-delta distribution for a sound source, moving along the direction of the Fourier-transformation.
	\item Based on the expression for the wavenumber-content I gave the SDM driving functions for the synthesis of a moving point source in the wavenumber domain
	\item For the special case of a source moving parallel to the secondary source distribution I gave analytical, closed form driving function in the spatial-frequency domain
	\item For the case of a source with a trajectory parallel to the SSD I explicitly showed, that the WFS solution is the high-frequency/farfield approximation the SDM, which had been proved for a stationary point source in the related literature.
	\item The wavenumber representation of the synthesis problem gives a simple tool for the investigation of the artifacts, emerging from the discretization of the SSD, which inherently present in case of real-life applications.
	I gave an analytical description of the resulting spatial aliasing artifacts.
	For the dynamic case the inverse spatial Fourier transform could be carried out analytically, resulting in the temporal-frequency representation of the aliased synthesized field.
	This description allows us to investigate the aliasing components individually which are present in an additive manner in the synthesized field.
I showed, that in this dynamic scenario two main artifacts will be present in the synthesized field:
	\begin{itemize}
	\item \emph{Amplitude ringing}, due to the temporally varying interference pattern of the ideal field and the aliasing components.
	For the case of a source with a harmonic excitation history the artifact appears as an unwanted variation in the envelope of the synthesized field, measured at a fixed receiver position.
	\item \emph{Frequency ringing}, which occurs as undesired frequency components in the synthesized field time history.
	I gave the analytical description of these frequency components, arising due to the altered propagation direction of the wave fronts with different time frequencies in the aliased components.
	I showed, that mathematically speaking, these artifacts are present in the synthesized field due to poles in the secondary sources wavenumber representation (hence the terminology \emph{ringing}).
	The frequency ringing therefore can be avoided by applying any SSD geometry, that's transfer function does not exhibit poles on the receiver curve, which holds for any smooth, enclosing SSD.
	I verified via numerical simulations, that for the case of a circular SSD the frequency ringing artifacts are eliminated. 
	\end{itemize}
	\end{itemize}
\end{itemize}
\end{document}
